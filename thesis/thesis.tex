\documentclass[12pt]{report}
\usepackage{standalone}
%include polycode.fmt
\usepackage[T1]{fontenc}
\usepackage{calc}
%% \usepackage{fourier}
\usepackage{palatino}
\renewcommand\ttdefault{lmtt}
\usepackage{helvet}
%% \usepackage{inconsolata}
\usepackage{comment}
\usepackage{calc}
\usepackage{xspace}
\usepackage{verbatim}
\usepackage{url}
\usepackage{fancyvrb}
\usepackage{setspace}
\usepackage{amsmath}
\usepackage{booktabs}
\usepackage[margin=\parindent, format=hang,labelfont=bf]{caption}
%% \usepackage[subrefformat=parens]{subcaption}
%% The following makes sure we get parentheses around
%% subreferences. The newest version of the subcaption
%% package has an option for this, but that's not available
%% widely.
%%
%% From http://tex.stackexchange.com/questions/25644
\usepackage[labelformat=simple]{subcaption}
\makeatletter
  \def\thesubfigure{(\alph{subfigure})}
  \providecommand\thefigsubsep{~}
  \def\p@subfigure{\@nameuse{thefigure}\thefigsubsep}
\makeatother

\usepackage{ifthen}
\usepackage{stmaryrd}
\usepackage{longtable}
\usepackage{afterpage}
\usepackage{xifthen}
\usepackage{mathtools}
\usepackage{xparse}
\usepackage[natbib=true,style=authoryear,backend=bibtex8]{biblatex}
\setlength{\bibitemsep}{\bigskipamount}
\addbibresource{thesis.bib}
\usepackage{microtype}

\usepackage{tikz}
\usetikzlibrary{arrows,automata,positioning,calc}
%% Used for CFGs.
\tikzset{
  >=stealth, 
  node distance=.5in,
  stmt/.style={rectangle,
    draw=black, thick,        
    minimum height=2em,
    %% inner sep=2pt,
    %% text centered,
    %% node distance=.5in,
  },
  entex/.style={
    minimum height=2em,
    %% inner sep=2pt,
    %% text centered,
  },
  labelfor/.style={circle, 
    draw=black, thin,
    font={\footnotesize},
    inner sep=0,
    fill=white,
    above right=-1.5mm and -1.5mm of #1,
  },
  fact/.style={overlay},
  %% Invisible node
  invis/.style={inner sep=0pt, 
    minimum height=0em}, 
  table/.style={circle, fill=white,height=2mm}
}

%% GSO margins.
\usepackage[left=1.5in, right=1in, top=1in, bottom=1in]{geometry}
\usepackage{abstract}

%% GSO requires 12 pt font for all headings
\usepackage[bf,sf,tiny,compact]{titlesec}
\titleformat{\chapter}[display]
            {}% format
            {\sffamily\bfseries\chaptertitlename\ \thechapter}
            {\baselineskip}
            {\sffamily\bfseries}
            {}

\hyphenation{data-flow mo-na-dic} 

\newboolean{lhs2tex}
\setboolean{lhs2tex}{true}

% Used by included files to know they
% are NOT standalone
\newboolean{standaloneFlag}
\setboolean{standaloneFlag}{true}

\newlength{\rulefigmargin}
\setlength{\rulefigmargin}{2\parindent}

\newcommand\figbegin{\rule{\linewidth-\rulefigmargin}{0.4pt}\\\vspace{12pt}}
\newcommand\figend{\rule{\linewidth-\rulefigmargin}{0.4pt}}

%\providecommand{\citep}[1]{(\emph{#1})\xspace}
%\renewcommand{\cite}[1]{\emph{#1}\xspace}

%% Functional languages chapter commands
\newcommand{\lamA}{\ensuremath{\lambda}-calculus\xspace}
\newcommand{\LamA}{\ensuremath{\lambda}-Calculus\xspace}
\newcommand{\lamAbs}[2]{\ensuremath{\lambda#1.\ #2}}
\newcommand{\lamApp}[2]{\ensuremath{#1\ #2}}
\newcommand{\lamPApp}[2]{\ensuremath{(#1\ #2)}}
\newcommand{\lamAPp}[2]{\ensuremath{(#1)\ #2}}
\newcommand{\lamApP}[2]{\ensuremath{#1\ (#2)}}
\newcommand{\lamAPP}[2]{\ensuremath{(#1)\ (#2)}}
\let\lamApPp=\lamApP
\let\lamAppP=\lamAPp

\newcommand{\lamId}{\lamAbs{x}{x}}
\newcommand{\lamCompose}{\lamAbs{f}{\lamAbs{g}{\lamAbs{x}{\lamApp{f}{(\lamApp{g}{x})}}}}}
\newcommand{\machLam}{\ensuremath{M_\lambda}\xspace}
\newcommand{\compMach}[1]{\ensuremath{\left\llbracket #1 \right\rrbracket}}
\newcommand{\compRho}[1]{\ensuremath{\rho(#1)}}
\newcommand{\verSub}[2]{\ensuremath{#1_{#2}}}
\newcommand{\verSup}[2]{\ensuremath{#1^{#2}}}
\newcommand{\lamC}{\ensuremath{\lambda_C}\xspace}
\newcommand{\lamPlus}{\lamAbs{m}{\lamAbs{n}{\lamAbs{s}{\lamAbs{z}{\lamApp{m}{\lamApPp{s}{\lamApp{n}{\lamApp{s}{z}}}}}}}}}
%% Substitution notation -- [#1 -> #2]
\newcommand{\lamSubst}[2]{\ensuremath{[#1 \mapsto #2]}}
%% End functional languages chapter

%% Dataflow chapter commands
\newcounter{nodeCounter}[figure]
\newcommand{\inE}{\ensuremath{\mathit{in}}\xspace}
\newcommand{\out}{\ensuremath{\mathit{out}}\xspace}
\newcommand{\In}{\ensuremath{\mathit{In}}\xspace}
\newcommand{\InBa}{\ensuremath{\mathit{In}(B)}\xspace}
\newcommand{\Out}{\ensuremath{\mathit{Out}}\xspace}
%% Out(B_x) -- fact function for an named block.
\newcommand{\OutB}[1]{\ensuremath{\mathit{Out}(B_{\ref{#1}})}\xspace}
\newcommand{\OutBa}{\ensuremath{\mathit{Out}(B)}\xspace}
%% in(B) -- fact function for an anonymous block.
\newcommand{\inBa}{\ensuremath{\mathit{in}(B)}\xspace}
%% in(X) -- fact function for an anonymous block, but using a different variable.
\newcommand{\inXa}[1]{\ensuremath{\mathit{in}(#1)}\xspace}
%% in(B,v) -- fact function for an anonymous block and some variable.
\newcommand{\inBav}[1]{\ensuremath{\mathit{in}(B, #1)}\xspace}
%% in(B_x) -- fact function for an named block.
\newcommand{\inB}[1]{\ensuremath{\mathit{in}(B_{\ref{#1}})}\xspace}
%% in(B_x,v) -- fact function for an named block and some variable.
\newcommand{\inBv}[2]{\ensuremath{\mathit{in}(B_{\ref{#1}}, #2)}\xspace}
%% out(B) -- fact function for an anonymous block.
\newcommand{\outBa}{\ensuremath{\mathit{out}(B)}\xspace}
%% out(X) -- fact function for an anonymous block, but using a different variable.
\newcommand{\outXa}[1]{\ensuremath{\mathit{out}(#1)}\xspace}
%% out(B,v) -- fact function for an anonymous block and some variable.
\newcommand{\outBav}[1]{\ensuremath{\mathit{out}(B, #1)}\xspace}
%% out(B_x) -- fact function for an named block.
\newcommand{\outB}[1]{\ensuremath{\mathit{out}(B_{\ref{#1}})}\xspace}
%% out(B_x,v) -- fact function for an named block and some variable.
\newcommand{\outBv}[2]{\ensuremath{\mathit{out}(B_{\ref{#1}}, #2)}\xspace}
\newcommand{\entryN}{\emph{E}\xspace}
\newcommand{\exitN}{\emph{X}\xspace}
\newcommand{\refNode}[1]{\ensuremath{B_{\ref{#1}}}\xspace}
\newcommand{\labelNode}[1]{\refstepcounter{nodeCounter}\label{#1}}
\newcommand{\setL}[1]{\textsc{#1}\xspace}
\newcommand{\setLC}{\setL{Const}}

%% Formats a list of facts
%% Argument should be like \facts{a/1, b/2, foobar/\bot, baz/\top}.
%% 
\newcounter{factctr}
\newtoks\varVal
\newtoks\varName
\newcommand{\facts}[1]{\begingroup%%
  %% Test if the argument given contains a forward slash (/). Expands
  %% slashTest with argument such that if a slash is NOT present the 
  %% token \noSlash will be given as argument 2 to slashTest. Otherwise
  %% there must be slash.
  \def\hasSlash##1{\expandafter\slashTest##1/\noslash\endslash}%%
  \def\slashTest##1/##2##3\endslash{\ifx\noslash##2 N\else Y\fi}%%
  \def\getArgs##1/##2{\varName={##1}%%
    \varVal={##2}}
  \ensuremath{%%
    \setcounter{factctr}{0}%%
    \foreach \var in {#1}{%%
      %% Separate list with a comma
      \ifthenelse{\value{factctr}>0}{,\allowbreak}{}%%
      %% \tracingmacros=1%%
      %% If key/val arguments, use first form. Otherwise
      %% use second.
      \ifthenelse{\equal{\hasSlash{\var}}{Y}}%%
                  {\expandafter\getArgs\var \factC{\the\varName}{\the\varVal}}%%
                  {\var}%%
      %% \tracingmacros=0%%
      \stepcounter{factctr}%%
    }}%%
\endgroup}
\newcommand{\factC}[2]{{\ensuremath{(\mathit{#1},#2)}}}
\newcommand{\doFacts}[4]{\ensuremath{#3{#1}: %%
    \left\{ %%
    \begin{minipage}[c]{#4}%%
      \facts{#2} %%
  \end{minipage}\kern -0.23em\right\}}}

\ExplSyntaxOn
\DeclareDocumentCommand \inFactsM {m m m} {\doFacts{#1}{#2}{\inB}{#3}}
\DeclareDocumentCommand \inFacts {m m O{1in}} {\doFacts{#1}{#2}{\inB}{#3}}
\DeclareDocumentCommand \outFactsM {m m m} {\doFacts{#1}{#2}{\outB}{#3}}
\DeclareDocumentCommand \outFacts {m m O{1in}} {\doFacts{#1}{#2}{\outB}{#3}}
\ExplSyntaxOff

\newcommand{\lub}{\ifthenelse{\boolean{mmode}}{\sqcap}{\raisebox{.1em}{\ensuremath{\sqcap}}}\xspace}
\newcommand{\sqlt}{\ensuremath{\sqsubset}\xspace}
\newcommand{\sqlte}{\ensuremath{\sqsubseteq}\xspace}

%% End dataflow

%% MIL Chapter
\newcommand{\compMILE}[1]{\ensuremath{\left\llbracket #1 \right\rrbracket}}
\newcommand{\compMILV}[1]{\ensuremath{\left\llbracket #1 \right\rrbracket}}
\newcommand{\compMILQ}[2]{\ensuremath{\left\llbracket #2 \right\rrbracket}}
\newcommand{\milCtx}[1]{\ensuremath{\llfloor}#1\ensuremath{\rrfloor}}
%% End MIL chapter

\newenvironment{myfig}[1][tbh]{\begin{figure}[#1]%%
\centering%%
\figbegin}{\figend%%
\end{figure}}

%% Produce a sub-caption and label it.
\newcommand{\scap}[2][1in]{\begin{minipage}{#1}%%
\subcaption{}\label{#2}\end{minipage}}

%% Produce a sub-caption with text.
\newcommand{\lscap}[3][1in]{\begin{minipage}{#1}%%
\subcaption{#3}\label{#2}\end{minipage}}

% single-argument comment. Do not put
% a space before the command when used
% or the file will have two spaces.
\newcommand{\rem}[1]{}

%% A verbatim environment with active charactesr
%% so we can use math shortcuts and macros
\DefineVerbatimEnvironment{AVerb}{Verbatim}{commandchars=\\\{\},%% 
  codes={\catcode`\_8\catcode`\$3\catcode`\^7},%%
  numberblanklines=false}

%% Turn on line numbers for Haskell code, 
%% and reset the line number counter.
\newcommand{\hsNumOn}{\numberson\numbersreset}
\newcommand{\hsNumOff}{\numbersoff}
%% Turn on line numbering in Haskell code within
%% the environment, then turn it off.
\newenvironment{withHsNum}{\numberson\numbersreset}{\numbersoff}

%% Paragraph run-in
\newcommand{\runin}[1]{\begingroup\noindent\sffamily\textbf{#1}\qquad\endgroup}

%% Chapter bibliographies
\newcommand{\standaloneBib}{%%
  \ifthenelse{\boolean{standaloneFlag}}%%
             {\begin{singlespace}
                \printbibliography
             \end{singlespace}}{}}

%% Adds an equation number on demand.
\newcommand\addtag{\refstepcounter{equation}\tag{\theequation}}

%% For typesetting set definitions like {x | x \in f(y)}
\newcommand\setdef[2]{\ensuremath{\{#1\ |\ #2\}}}

%% For typesetting function names like dom(f) or out(b).
\newcommand\mfun[1]{\ensuremath{\mathit{#1}}}

%% Marginal notes
\newcommand\margin[2]{\marginpar{\begin{singlespace}\emph{\footnotesize #2}\end{singlespace}}\relax #1}

%% Describe intent of a passage
\newcommand\intent[1]{{\leftskip = -1in\begin{singlespace}\emph{\noindent\footnotesize Intent: #1}\end{singlespace}}}


% Used by included files to know they
% are NOT standalone
\setboolean{standaloneFlag}{false}

\begin{document}

\VerbatimFootnotes
\DefineShortVerb{\#}
\doublespacing
             
\documentclass[12pt]{article}
\usepackage{standalone}
\usepackage{comment}
%include polycode.fmt
\usepackage{palatino}
\usepackage[scaled=0.92]{helvet}
\usepackage{xspace}
\usepackage{verbatim}
\renewcommand\ttdefault{cmtt}
\usepackage{url}
\usepackage{fancyvrb}
\usepackage{setspace}
\usepackage{cmds}

\usepackage{ifthen}
\newboolean{lhs2tex}
\setboolean{lhs2tex}{true}
%if False
% lhs2tex ignores this section
\setboolean{lhs2tex}{false}
\newcommand{\authorEmail}{\url{justinb@cs.pdx.edu}}
%else
% LaTeX ignores this section, unless pre-processed with lhs2Text
\ifthenelse{\boolean{lhs2tex}}%
           {\newcommand{\authorEmail}{\url{justinb@@cs.pdx.edu}}}%
           {}
%endif

% For compiling sub-packages
\newboolean{standaloneFlag}
\setboolean{standaloneFlag}{true}

%% GSO margins.
\usepackage[left=1.5in, right=1in, top=1in, bottom=1in]{geometry}
\usepackage{abstract}

%% GSO requires 12 pt font for all headings
\usepackage[sf, bf, tiny]{titlesec}
\titleformat{\chapter}[hang]{}% format
 {\sffamily\bfseries\thechapter}
 {1em}
 {\sffamily\bfseries}
\titlespacing{\chapter}{}{}{2ex}

\hyphenation{data-flow}
\begin{document}
\ifthenelse{\boolean{standaloneFlag}}
           {\date{}
             \author{Justin Bailey \\ \authorEmail}
             \title{Using Dataflow Optimization Techniques with a Monadic Intermediate Language}

             \VerbatimFootnotes
             \DefineShortVerb{\#}
             \doublespacing}{}

\maketitle

\renewcommand{\abstractnamefont}{\normalfont\small\sffamily\bfseries}
\begin{abstract}
  Dataflow analysis of programs represented as control-flow graphs
  (CFGs) of basic blocks underlies many optimizations implemented by
  imperative language compilers. Functional language compilers, in
  contrast, traditionally optimize by rewriting programs according to
  algebraic laws. We show that, by compiling to a \emph{monadic
    intermediate language}, we can treat our functional programs as
  CFGs of basic blocks. Doing so enables us to re-use the rich body of
  dataflow techniques developed for imperative language compilers. We
  first implement dead code elimination, an optimization common to
  both functional and imperative compilers. We then demonstrate an
  optimization specific to functional language compilers: elimination
  of intermediate closures. Next, we show how two optimizations,
  inlining and copy-propagation, derive from the \emph{monad laws} in
  our representation. Finally, we implement \emph{Lazy Code Motion}
  (LCM), one of the most complicated dataflow-based optimizations. To
  our knowledge this is the first implementation of LCM over a monadic
  intermediate language. Throughout, we use the \emph{Hoopl} library
  to implement our optimizations, making our work a case-study for the
  library as well.
\end{abstract}


\end{document}


\documentclass[12pt]{report}
%include polycode.fmt
\usepackage[T1]{fontenc}
\usepackage{calc}
%% \usepackage{fourier}
\usepackage{palatino}
\renewcommand\ttdefault{lmtt}
\usepackage{helvet}
%% \usepackage{inconsolata}
\usepackage{comment}
\usepackage{calc}
\usepackage{xspace}
\usepackage{verbatim}
\usepackage{url}
\usepackage{fancyvrb}
\usepackage{setspace}
\usepackage{amsmath}
\usepackage{booktabs}
\usepackage[margin=\parindent, format=hang,labelfont=bf]{caption}
%% \usepackage[subrefformat=parens]{subcaption}
%% The following makes sure we get parentheses around
%% subreferences. The newest version of the subcaption
%% package has an option for this, but that's not available
%% widely.
%%
%% From http://tex.stackexchange.com/questions/25644
\usepackage[labelformat=simple]{subcaption}
\makeatletter
  \def\thesubfigure{(\alph{subfigure})}
  \providecommand\thefigsubsep{~}
  \def\p@subfigure{\@nameuse{thefigure}\thefigsubsep}
\makeatother

\usepackage{ifthen}
\usepackage{stmaryrd}
\usepackage{longtable}
\usepackage{afterpage}
\usepackage{xifthen}
\usepackage{mathtools}
\usepackage{xparse}
\usepackage[natbib=true,style=authoryear,backend=bibtex8]{biblatex}
\setlength{\bibitemsep}{\bigskipamount}
\addbibresource{thesis.bib}
\usepackage{microtype}

\usepackage{tikz}
\usetikzlibrary{arrows,automata,positioning,calc}
%% Used for CFGs.
\tikzset{
  >=stealth, 
  node distance=.5in,
  stmt/.style={rectangle,
    draw=black, thick,        
    minimum height=2em,
    %% inner sep=2pt,
    %% text centered,
    %% node distance=.5in,
  },
  entex/.style={
    minimum height=2em,
    %% inner sep=2pt,
    %% text centered,
  },
  labelfor/.style={circle, 
    draw=black, thin,
    font={\footnotesize},
    inner sep=0,
    fill=white,
    above right=-1.5mm and -1.5mm of #1,
  },
  fact/.style={overlay},
  %% Invisible node
  invis/.style={inner sep=0pt, 
    minimum height=0em}, 
  table/.style={circle, fill=white,height=2mm}
}

%% GSO margins.
\usepackage[left=1.5in, right=1in, top=1in, bottom=1in]{geometry}
\usepackage{abstract}

%% GSO requires 12 pt font for all headings
\usepackage[bf,sf,tiny,compact]{titlesec}
\titleformat{\chapter}[display]
            {}% format
            {\sffamily\bfseries\chaptertitlename\ \thechapter}
            {\baselineskip}
            {\sffamily\bfseries}
            {}

\hyphenation{data-flow mo-na-dic} 

\newboolean{lhs2tex}
\setboolean{lhs2tex}{true}

% Used by included files to know they
% are NOT standalone
\newboolean{standaloneFlag}
\setboolean{standaloneFlag}{true}

\newlength{\rulefigmargin}
\setlength{\rulefigmargin}{2\parindent}

\newcommand\figbegin{\rule{\linewidth-\rulefigmargin}{0.4pt}\\\vspace{12pt}}
\newcommand\figend{\rule{\linewidth-\rulefigmargin}{0.4pt}}

%\providecommand{\citep}[1]{(\emph{#1})\xspace}
%\renewcommand{\cite}[1]{\emph{#1}\xspace}

%% Functional languages chapter commands
\newcommand{\lamA}{\ensuremath{\lambda}-calculus\xspace}
\newcommand{\LamA}{\ensuremath{\lambda}-Calculus\xspace}
\newcommand{\lamAbs}[2]{\ensuremath{\lambda#1.\ #2}}
\newcommand{\lamApp}[2]{\ensuremath{#1\ #2}}
\newcommand{\lamPApp}[2]{\ensuremath{(#1\ #2)}}
\newcommand{\lamAPp}[2]{\ensuremath{(#1)\ #2}}
\newcommand{\lamApP}[2]{\ensuremath{#1\ (#2)}}
\newcommand{\lamAPP}[2]{\ensuremath{(#1)\ (#2)}}
\let\lamApPp=\lamApP
\let\lamAppP=\lamAPp

\newcommand{\lamId}{\lamAbs{x}{x}}
\newcommand{\lamCompose}{\lamAbs{f}{\lamAbs{g}{\lamAbs{x}{\lamApp{f}{(\lamApp{g}{x})}}}}}
\newcommand{\machLam}{\ensuremath{M_\lambda}\xspace}
\newcommand{\compMach}[1]{\ensuremath{\left\llbracket #1 \right\rrbracket}}
\newcommand{\compRho}[1]{\ensuremath{\rho(#1)}}
\newcommand{\verSub}[2]{\ensuremath{#1_{#2}}}
\newcommand{\verSup}[2]{\ensuremath{#1^{#2}}}
\newcommand{\lamC}{\ensuremath{\lambda_C}\xspace}
\newcommand{\lamPlus}{\lamAbs{m}{\lamAbs{n}{\lamAbs{s}{\lamAbs{z}{\lamApp{m}{\lamApPp{s}{\lamApp{n}{\lamApp{s}{z}}}}}}}}}
%% Substitution notation -- [#1 -> #2]
\newcommand{\lamSubst}[2]{\ensuremath{[#1 \mapsto #2]}}
%% End functional languages chapter

%% Dataflow chapter commands
\newcounter{nodeCounter}[figure]
\newcommand{\inE}{\ensuremath{\mathit{in}}\xspace}
\newcommand{\out}{\ensuremath{\mathit{out}}\xspace}
\newcommand{\In}{\ensuremath{\mathit{In}}\xspace}
\newcommand{\InBa}{\ensuremath{\mathit{In}(B)}\xspace}
\newcommand{\Out}{\ensuremath{\mathit{Out}}\xspace}
%% Out(B_x) -- fact function for an named block.
\newcommand{\OutB}[1]{\ensuremath{\mathit{Out}(B_{\ref{#1}})}\xspace}
\newcommand{\OutBa}{\ensuremath{\mathit{Out}(B)}\xspace}
%% in(B) -- fact function for an anonymous block.
\newcommand{\inBa}{\ensuremath{\mathit{in}(B)}\xspace}
%% in(X) -- fact function for an anonymous block, but using a different variable.
\newcommand{\inXa}[1]{\ensuremath{\mathit{in}(#1)}\xspace}
%% in(B,v) -- fact function for an anonymous block and some variable.
\newcommand{\inBav}[1]{\ensuremath{\mathit{in}(B, #1)}\xspace}
%% in(B_x) -- fact function for an named block.
\newcommand{\inB}[1]{\ensuremath{\mathit{in}(B_{\ref{#1}})}\xspace}
%% in(B_x,v) -- fact function for an named block and some variable.
\newcommand{\inBv}[2]{\ensuremath{\mathit{in}(B_{\ref{#1}}, #2)}\xspace}
%% out(B) -- fact function for an anonymous block.
\newcommand{\outBa}{\ensuremath{\mathit{out}(B)}\xspace}
%% out(X) -- fact function for an anonymous block, but using a different variable.
\newcommand{\outXa}[1]{\ensuremath{\mathit{out}(#1)}\xspace}
%% out(B,v) -- fact function for an anonymous block and some variable.
\newcommand{\outBav}[1]{\ensuremath{\mathit{out}(B, #1)}\xspace}
%% out(B_x) -- fact function for an named block.
\newcommand{\outB}[1]{\ensuremath{\mathit{out}(B_{\ref{#1}})}\xspace}
%% out(B_x,v) -- fact function for an named block and some variable.
\newcommand{\outBv}[2]{\ensuremath{\mathit{out}(B_{\ref{#1}}, #2)}\xspace}
\newcommand{\entryN}{\emph{E}\xspace}
\newcommand{\exitN}{\emph{X}\xspace}
\newcommand{\refNode}[1]{\ensuremath{B_{\ref{#1}}}\xspace}
\newcommand{\labelNode}[1]{\refstepcounter{nodeCounter}\label{#1}}
\newcommand{\setL}[1]{\textsc{#1}\xspace}
\newcommand{\setLC}{\setL{Const}}

%% Formats a list of facts
%% Argument should be like \facts{a/1, b/2, foobar/\bot, baz/\top}.
%% 
\newcounter{factctr}
\newtoks\varVal
\newtoks\varName
\newcommand{\facts}[1]{\begingroup%%
  %% Test if the argument given contains a forward slash (/). Expands
  %% slashTest with argument such that if a slash is NOT present the 
  %% token \noSlash will be given as argument 2 to slashTest. Otherwise
  %% there must be slash.
  \def\hasSlash##1{\expandafter\slashTest##1/\noslash\endslash}%%
  \def\slashTest##1/##2##3\endslash{\ifx\noslash##2 N\else Y\fi}%%
  \def\getArgs##1/##2{\varName={##1}%%
    \varVal={##2}}
  \ensuremath{%%
    \setcounter{factctr}{0}%%
    \foreach \var in {#1}{%%
      %% Separate list with a comma
      \ifthenelse{\value{factctr}>0}{,\allowbreak}{}%%
      %% \tracingmacros=1%%
      %% If key/val arguments, use first form. Otherwise
      %% use second.
      \ifthenelse{\equal{\hasSlash{\var}}{Y}}%%
                  {\expandafter\getArgs\var \factC{\the\varName}{\the\varVal}}%%
                  {\var}%%
      %% \tracingmacros=0%%
      \stepcounter{factctr}%%
    }}%%
\endgroup}
\newcommand{\factC}[2]{{\ensuremath{(\mathit{#1},#2)}}}
\newcommand{\doFacts}[4]{\ensuremath{#3{#1}: %%
    \left\{ %%
    \begin{minipage}[c]{#4}%%
      \facts{#2} %%
  \end{minipage}\kern -0.23em\right\}}}

\ExplSyntaxOn
\DeclareDocumentCommand \inFactsM {m m m} {\doFacts{#1}{#2}{\inB}{#3}}
\DeclareDocumentCommand \inFacts {m m O{1in}} {\doFacts{#1}{#2}{\inB}{#3}}
\DeclareDocumentCommand \outFactsM {m m m} {\doFacts{#1}{#2}{\outB}{#3}}
\DeclareDocumentCommand \outFacts {m m O{1in}} {\doFacts{#1}{#2}{\outB}{#3}}
\ExplSyntaxOff

\newcommand{\lub}{\ifthenelse{\boolean{mmode}}{\sqcap}{\raisebox{.1em}{\ensuremath{\sqcap}}}\xspace}
\newcommand{\sqlt}{\ensuremath{\sqsubset}\xspace}
\newcommand{\sqlte}{\ensuremath{\sqsubseteq}\xspace}

%% End dataflow

%% MIL Chapter
\newcommand{\compMILE}[1]{\ensuremath{\left\llbracket #1 \right\rrbracket}}
\newcommand{\compMILV}[1]{\ensuremath{\left\llbracket #1 \right\rrbracket}}
\newcommand{\compMILQ}[2]{\ensuremath{\left\llbracket #2 \right\rrbracket}}
\newcommand{\milCtx}[1]{\ensuremath{\llfloor}#1\ensuremath{\rrfloor}}
%% End MIL chapter

\newenvironment{myfig}[1][tbh]{\begin{figure}[#1]%%
\centering%%
\figbegin}{\figend%%
\end{figure}}

%% Produce a sub-caption and label it.
\newcommand{\scap}[2][1in]{\begin{minipage}{#1}%%
\subcaption{}\label{#2}\end{minipage}}

%% Produce a sub-caption with text.
\newcommand{\lscap}[3][1in]{\begin{minipage}{#1}%%
\subcaption{#3}\label{#2}\end{minipage}}

% single-argument comment. Do not put
% a space before the command when used
% or the file will have two spaces.
\newcommand{\rem}[1]{}

%% A verbatim environment with active charactesr
%% so we can use math shortcuts and macros
\DefineVerbatimEnvironment{AVerb}{Verbatim}{commandchars=\\\{\},%% 
  codes={\catcode`\_8\catcode`\$3\catcode`\^7},%%
  numberblanklines=false}

%% Turn on line numbers for Haskell code, 
%% and reset the line number counter.
\newcommand{\hsNumOn}{\numberson\numbersreset}
\newcommand{\hsNumOff}{\numbersoff}
%% Turn on line numbering in Haskell code within
%% the environment, then turn it off.
\newenvironment{withHsNum}{\numberson\numbersreset}{\numbersoff}

%% Paragraph run-in
\newcommand{\runin}[1]{\begingroup\noindent\sffamily\textbf{#1}\qquad\endgroup}

%% Chapter bibliographies
\newcommand{\standaloneBib}{%%
  \ifthenelse{\boolean{standaloneFlag}}%%
             {\begin{singlespace}
                \printbibliography
             \end{singlespace}}{}}

%% Adds an equation number on demand.
\newcommand\addtag{\refstepcounter{equation}\tag{\theequation}}

%% For typesetting set definitions like {x | x \in f(y)}
\newcommand\setdef[2]{\ensuremath{\{#1\ |\ #2\}}}

%% For typesetting function names like dom(f) or out(b).
\newcommand\mfun[1]{\ensuremath{\mathit{#1}}}

%% Marginal notes
\newcommand\margin[2]{\marginpar{\begin{singlespace}\emph{\footnotesize #2}\end{singlespace}}\relax #1}

%% Describe intent of a passage
\newcommand\intent[1]{{\leftskip = -1in\begin{singlespace}\emph{\noindent\footnotesize Intent: #1}\end{singlespace}}}

\begin{document}
\ifthenelse{\boolean{standaloneFlag}}
           {\VerbatimFootnotes
             \DefineShortVerb{\#}
             \setcounter{chapter}{0}}{}

%% Default float parameters. For case when
%% multiple chapters are included and
%% only one needs custom float settings.
\renewcommand{\textfraction}{0.2}
\renewcommand{\textfraction}{0.2}
\renewcommand{\topfraction}{0.9}


\chapter{Introduction}

Compilers for imperative languages implement many optimizations using
\emph{dataflow analysis}. This method treats the program as a graph,
where edges represent execution paths and nodes represent statements
in the program. Dataflow analysis computes facts about each node and
then transforms the graph into an equivalent, yet faster (or smaller,
or more efficient, etc.) program. Optimizations which use dataflow
analysis include constant propagation, dead-code elimination,
common-subexpression elimination and many others.

Dataflow analysis on imperative programs arises naturally due to the
explicit flow-of-control from statement to statement. For pure
functional languages, with flow-of-control determined by evaluation
order, the fit seems more awkward. However, call-by-value, pure,
\emph{monadic} functional programs embody the best of both
styles: expression-based evaluation \emph{and} explicit
control-flow. 

Our work, then, defines a monadic language and optimizes programs in
it using dataflow analysis. We implement a number of optimizations
common to imperative and functional languages, including constant
propagation and dead-code elimination. We implement an optimization
which eliminates intermediate closure construction, showing that this
technique can be used for optimizations specific to functional
languages. Finally, we implement \emph{lazy code motion}, which to our
knowledge has not been applied to programs in a monadic language
before.

We use the Hoopl library\rem{reference} to implement our
optimizations. Besides showing that it is possible (and even
desirable) to use dataflow analysis in this context, our work also
serves as a case-study for advanced uses of Hoopl.

\end{document}


\documentclass[12pt]{report}
%include polycode.fmt
\usepackage[T1]{fontenc}
\usepackage{calc}
%% \usepackage{fourier}
\usepackage{palatino}
\renewcommand\ttdefault{lmtt}
\usepackage{helvet}
%% \usepackage{inconsolata}
\usepackage{comment}
\usepackage{calc}
\usepackage{xspace}
\usepackage{verbatim}
\usepackage{url}
\usepackage{fancyvrb}
\usepackage{setspace}
\usepackage{amsmath}
\usepackage{booktabs}
\usepackage[margin=\parindent, format=hang,labelfont=bf]{caption}
%% \usepackage[subrefformat=parens]{subcaption}
%% The following makes sure we get parentheses around
%% subreferences. The newest version of the subcaption
%% package has an option for this, but that's not available
%% widely.
%%
%% From http://tex.stackexchange.com/questions/25644
\usepackage[labelformat=simple]{subcaption}
\makeatletter
  \def\thesubfigure{(\alph{subfigure})}
  \providecommand\thefigsubsep{~}
  \def\p@subfigure{\@nameuse{thefigure}\thefigsubsep}
\makeatother

\usepackage{ifthen}
\usepackage{stmaryrd}
\usepackage{longtable}
\usepackage{afterpage}
\usepackage{xifthen}
\usepackage{mathtools}
\usepackage{xparse}
\usepackage[natbib=true,style=authoryear,backend=bibtex8]{biblatex}
\setlength{\bibitemsep}{\bigskipamount}
\addbibresource{thesis.bib}
\usepackage{microtype}

\usepackage{tikz}
\usetikzlibrary{arrows,automata,positioning,calc}
%% Used for CFGs.
\tikzset{
  >=stealth, 
  node distance=.5in,
  stmt/.style={rectangle,
    draw=black, thick,        
    minimum height=2em,
    %% inner sep=2pt,
    %% text centered,
    %% node distance=.5in,
  },
  entex/.style={
    minimum height=2em,
    %% inner sep=2pt,
    %% text centered,
  },
  labelfor/.style={circle, 
    draw=black, thin,
    font={\footnotesize},
    inner sep=0,
    fill=white,
    above right=-1.5mm and -1.5mm of #1,
  },
  fact/.style={overlay},
  %% Invisible node
  invis/.style={inner sep=0pt, 
    minimum height=0em}, 
  table/.style={circle, fill=white,height=2mm}
}

%% GSO margins.
\usepackage[left=1.5in, right=1in, top=1in, bottom=1in]{geometry}
\usepackage{abstract}

%% GSO requires 12 pt font for all headings
\usepackage[bf,sf,tiny,compact]{titlesec}
\titleformat{\chapter}[display]
            {}% format
            {\sffamily\bfseries\chaptertitlename\ \thechapter}
            {\baselineskip}
            {\sffamily\bfseries}
            {}

\hyphenation{data-flow mo-na-dic} 

\newboolean{lhs2tex}
\setboolean{lhs2tex}{true}

% Used by included files to know they
% are NOT standalone
\newboolean{standaloneFlag}
\setboolean{standaloneFlag}{true}

\newlength{\rulefigmargin}
\setlength{\rulefigmargin}{2\parindent}

\newcommand\figbegin{\rule{\linewidth-\rulefigmargin}{0.4pt}\\\vspace{12pt}}
\newcommand\figend{\rule{\linewidth-\rulefigmargin}{0.4pt}}

%\providecommand{\citep}[1]{(\emph{#1})\xspace}
%\renewcommand{\cite}[1]{\emph{#1}\xspace}

%% Functional languages chapter commands
\newcommand{\lamA}{\ensuremath{\lambda}-calculus\xspace}
\newcommand{\LamA}{\ensuremath{\lambda}-Calculus\xspace}
\newcommand{\lamAbs}[2]{\ensuremath{\lambda#1.\ #2}}
\newcommand{\lamApp}[2]{\ensuremath{#1\ #2}}
\newcommand{\lamPApp}[2]{\ensuremath{(#1\ #2)}}
\newcommand{\lamAPp}[2]{\ensuremath{(#1)\ #2}}
\newcommand{\lamApP}[2]{\ensuremath{#1\ (#2)}}
\newcommand{\lamAPP}[2]{\ensuremath{(#1)\ (#2)}}
\let\lamApPp=\lamApP
\let\lamAppP=\lamAPp

\newcommand{\lamId}{\lamAbs{x}{x}}
\newcommand{\lamCompose}{\lamAbs{f}{\lamAbs{g}{\lamAbs{x}{\lamApp{f}{(\lamApp{g}{x})}}}}}
\newcommand{\machLam}{\ensuremath{M_\lambda}\xspace}
\newcommand{\compMach}[1]{\ensuremath{\left\llbracket #1 \right\rrbracket}}
\newcommand{\compRho}[1]{\ensuremath{\rho(#1)}}
\newcommand{\verSub}[2]{\ensuremath{#1_{#2}}}
\newcommand{\verSup}[2]{\ensuremath{#1^{#2}}}
\newcommand{\lamC}{\ensuremath{\lambda_C}\xspace}
\newcommand{\lamPlus}{\lamAbs{m}{\lamAbs{n}{\lamAbs{s}{\lamAbs{z}{\lamApp{m}{\lamApPp{s}{\lamApp{n}{\lamApp{s}{z}}}}}}}}}
%% Substitution notation -- [#1 -> #2]
\newcommand{\lamSubst}[2]{\ensuremath{[#1 \mapsto #2]}}
%% End functional languages chapter

%% Dataflow chapter commands
\newcounter{nodeCounter}[figure]
\newcommand{\inE}{\ensuremath{\mathit{in}}\xspace}
\newcommand{\out}{\ensuremath{\mathit{out}}\xspace}
\newcommand{\In}{\ensuremath{\mathit{In}}\xspace}
\newcommand{\InBa}{\ensuremath{\mathit{In}(B)}\xspace}
\newcommand{\Out}{\ensuremath{\mathit{Out}}\xspace}
%% Out(B_x) -- fact function for an named block.
\newcommand{\OutB}[1]{\ensuremath{\mathit{Out}(B_{\ref{#1}})}\xspace}
\newcommand{\OutBa}{\ensuremath{\mathit{Out}(B)}\xspace}
%% in(B) -- fact function for an anonymous block.
\newcommand{\inBa}{\ensuremath{\mathit{in}(B)}\xspace}
%% in(X) -- fact function for an anonymous block, but using a different variable.
\newcommand{\inXa}[1]{\ensuremath{\mathit{in}(#1)}\xspace}
%% in(B,v) -- fact function for an anonymous block and some variable.
\newcommand{\inBav}[1]{\ensuremath{\mathit{in}(B, #1)}\xspace}
%% in(B_x) -- fact function for an named block.
\newcommand{\inB}[1]{\ensuremath{\mathit{in}(B_{\ref{#1}})}\xspace}
%% in(B_x,v) -- fact function for an named block and some variable.
\newcommand{\inBv}[2]{\ensuremath{\mathit{in}(B_{\ref{#1}}, #2)}\xspace}
%% out(B) -- fact function for an anonymous block.
\newcommand{\outBa}{\ensuremath{\mathit{out}(B)}\xspace}
%% out(X) -- fact function for an anonymous block, but using a different variable.
\newcommand{\outXa}[1]{\ensuremath{\mathit{out}(#1)}\xspace}
%% out(B,v) -- fact function for an anonymous block and some variable.
\newcommand{\outBav}[1]{\ensuremath{\mathit{out}(B, #1)}\xspace}
%% out(B_x) -- fact function for an named block.
\newcommand{\outB}[1]{\ensuremath{\mathit{out}(B_{\ref{#1}})}\xspace}
%% out(B_x,v) -- fact function for an named block and some variable.
\newcommand{\outBv}[2]{\ensuremath{\mathit{out}(B_{\ref{#1}}, #2)}\xspace}
\newcommand{\entryN}{\emph{E}\xspace}
\newcommand{\exitN}{\emph{X}\xspace}
\newcommand{\refNode}[1]{\ensuremath{B_{\ref{#1}}}\xspace}
\newcommand{\labelNode}[1]{\refstepcounter{nodeCounter}\label{#1}}
\newcommand{\setL}[1]{\textsc{#1}\xspace}
\newcommand{\setLC}{\setL{Const}}

%% Formats a list of facts
%% Argument should be like \facts{a/1, b/2, foobar/\bot, baz/\top}.
%% 
\newcounter{factctr}
\newtoks\varVal
\newtoks\varName
\newcommand{\facts}[1]{\begingroup%%
  %% Test if the argument given contains a forward slash (/). Expands
  %% slashTest with argument such that if a slash is NOT present the 
  %% token \noSlash will be given as argument 2 to slashTest. Otherwise
  %% there must be slash.
  \def\hasSlash##1{\expandafter\slashTest##1/\noslash\endslash}%%
  \def\slashTest##1/##2##3\endslash{\ifx\noslash##2 N\else Y\fi}%%
  \def\getArgs##1/##2{\varName={##1}%%
    \varVal={##2}}
  \ensuremath{%%
    \setcounter{factctr}{0}%%
    \foreach \var in {#1}{%%
      %% Separate list with a comma
      \ifthenelse{\value{factctr}>0}{,\allowbreak}{}%%
      %% \tracingmacros=1%%
      %% If key/val arguments, use first form. Otherwise
      %% use second.
      \ifthenelse{\equal{\hasSlash{\var}}{Y}}%%
                  {\expandafter\getArgs\var \factC{\the\varName}{\the\varVal}}%%
                  {\var}%%
      %% \tracingmacros=0%%
      \stepcounter{factctr}%%
    }}%%
\endgroup}
\newcommand{\factC}[2]{{\ensuremath{(\mathit{#1},#2)}}}
\newcommand{\doFacts}[4]{\ensuremath{#3{#1}: %%
    \left\{ %%
    \begin{minipage}[c]{#4}%%
      \facts{#2} %%
  \end{minipage}\kern -0.23em\right\}}}

\ExplSyntaxOn
\DeclareDocumentCommand \inFactsM {m m m} {\doFacts{#1}{#2}{\inB}{#3}}
\DeclareDocumentCommand \inFacts {m m O{1in}} {\doFacts{#1}{#2}{\inB}{#3}}
\DeclareDocumentCommand \outFactsM {m m m} {\doFacts{#1}{#2}{\outB}{#3}}
\DeclareDocumentCommand \outFacts {m m O{1in}} {\doFacts{#1}{#2}{\outB}{#3}}
\ExplSyntaxOff

\newcommand{\lub}{\ifthenelse{\boolean{mmode}}{\sqcap}{\raisebox{.1em}{\ensuremath{\sqcap}}}\xspace}
\newcommand{\sqlt}{\ensuremath{\sqsubset}\xspace}
\newcommand{\sqlte}{\ensuremath{\sqsubseteq}\xspace}

%% End dataflow

%% MIL Chapter
\newcommand{\compMILE}[1]{\ensuremath{\left\llbracket #1 \right\rrbracket}}
\newcommand{\compMILV}[1]{\ensuremath{\left\llbracket #1 \right\rrbracket}}
\newcommand{\compMILQ}[2]{\ensuremath{\left\llbracket #2 \right\rrbracket}}
\newcommand{\milCtx}[1]{\ensuremath{\llfloor}#1\ensuremath{\rrfloor}}
%% End MIL chapter

\newenvironment{myfig}[1][tbh]{\begin{figure}[#1]%%
\centering%%
\figbegin}{\figend%%
\end{figure}}

%% Produce a sub-caption and label it.
\newcommand{\scap}[2][1in]{\begin{minipage}{#1}%%
\subcaption{}\label{#2}\end{minipage}}

%% Produce a sub-caption with text.
\newcommand{\lscap}[3][1in]{\begin{minipage}{#1}%%
\subcaption{#3}\label{#2}\end{minipage}}

% single-argument comment. Do not put
% a space before the command when used
% or the file will have two spaces.
\newcommand{\rem}[1]{}

%% A verbatim environment with active charactesr
%% so we can use math shortcuts and macros
\DefineVerbatimEnvironment{AVerb}{Verbatim}{commandchars=\\\{\},%% 
  codes={\catcode`\_8\catcode`\$3\catcode`\^7},%%
  numberblanklines=false}

%% Turn on line numbers for Haskell code, 
%% and reset the line number counter.
\newcommand{\hsNumOn}{\numberson\numbersreset}
\newcommand{\hsNumOff}{\numbersoff}
%% Turn on line numbering in Haskell code within
%% the environment, then turn it off.
\newenvironment{withHsNum}{\numberson\numbersreset}{\numbersoff}

%% Paragraph run-in
\newcommand{\runin}[1]{\begingroup\noindent\sffamily\textbf{#1}\qquad\endgroup}

%% Chapter bibliographies
\newcommand{\standaloneBib}{%%
  \ifthenelse{\boolean{standaloneFlag}}%%
             {\begin{singlespace}
                \printbibliography
             \end{singlespace}}{}}

%% Adds an equation number on demand.
\newcommand\addtag{\refstepcounter{equation}\tag{\theequation}}

%% For typesetting set definitions like {x | x \in f(y)}
\newcommand\setdef[2]{\ensuremath{\{#1\ |\ #2\}}}

%% For typesetting function names like dom(f) or out(b).
\newcommand\mfun[1]{\ensuremath{\mathit{#1}}}

%% Marginal notes
\newcommand\margin[2]{\marginpar{\begin{singlespace}\emph{\footnotesize #2}\end{singlespace}}\relax #1}

%% Describe intent of a passage
\newcommand\intent[1]{{\leftskip = -1in\begin{singlespace}\emph{\noindent\footnotesize Intent: #1}\end{singlespace}}}

\begin{document}
\ifthenelse{\boolean{standaloneFlag}}
           {\VerbatimFootnotes
             \DefineShortVerb{\#}
             \setcounter{chapter}{0}}{}

%% Default float parameters. For case when
%% multiple chapters are included and
%% only one needs custom float settings.
\renewcommand{\textfraction}{0.2}
\renewcommand{\textfraction}{0.2}
\renewcommand{\topfraction}{0.9}

\renewcommand{\textfraction}{0.1}
\renewcommand{\topfraction}{0.9}

\chapter{Dataflow Optimization}
\label{ref_chapter_background}

%% A short section giving the history of dataflow optimization techniques
%% and basic concepts.

The \emph{dataflow algorithm}, as described by Gary Kildall
\citep{Kildall1973}, provides a framework analyzing and optimizing
programs.  The algorithm \emph{iteratively} traverses the
\emph{control-flow graph} for a program either \emph{forwards} or
\emph{backwards}, computing \emph{facts} at each node using a
\emph{transfer function}, until the facts reach a \emph{fixed
  point}. The choice of facts, transfer function, and direction depend
on the particular analysis performed. In essence, the dataflow
\emph{algorithm} is parameterized by these choices; a dataflow
\emph{analysis} is a specific instance of the algorithm.

%% This chapter describes the concepts necessary to understand the
%% dataflow algorithm and gives an extended example demonstrating the use
%% of \emph{liveness analysis} to eliminate
%% \emph{dead-code}. Section~\ref{sec_back1} describes control-flow
%% graphs. We discuss facts, direction, the transfer function and the
%% meet operator in Section \ref{sec_back4}. We illustrate why dataflow
%% must be an iterative algorithm (and define what a fixed point means in
%% this context) in Section~\ref{sec_back6}. We treat rewriting in
%% Section~\ref{sec_back7}. To demonstrate these concepts together, we
%% give an extended example of \emph{dead-code elimination} in
%% Section~\ref{sec_back2}.

\section{Control-Flow Graphs}
\label{sec_back1}

%% Begin by placing the specific concept in the overall context of
%% dataflow. Give a small example highlighting the concept. Point
%% out fine points or subtleties that occur when generalizing the concept. End
%% by summarizing how the concept fits into dataflow (in a bit larger
%% sense than the first summary).

Figure~\ref{fig_back1} shows a simple C program and its
\emph{control-flow graph} (CFG). Each \emph{node} in
\subref{fig_back1_b} represents a statement or expression in the
original program. For example, \refNode{lst_back2_assigna} and
\refNode{lst_back2_assignb} represent the assignment statements on
line \ref{lst_back1_assign}. Notice that the declaration of #c# does
not appear in a corresponding node; because the declaration does not
cause a runtime effect, we do not represent it in the graph.  Nodes
\entryN and \exitN designate where program execution \emph{enters} and
\emph{leaves} the graph. If the graph represented the entire program,
we would say execution \emph{begins} at \entryN and \emph{terminates}
at \exitN. However, the CFG may be embedded in a larger program, for
which reason we say \emph{enters} and \emph{leaves}.

\begin{myfig}[th]
\begin{tabular}{cc}
\subfloat{\begin{minipage}[t]{2.5in}
\begin{AVerb}[numbers=left]
int a = 1, b = 2, c; \label{lst_back1_assign}
if(a > b) \label{lst_back1_test}
  c = 4; \label{lst_back1_test_true}
else     
  c = a + 3; \label{lst_back1_test_false}

printf(c); \label{lst_back1_print}
\end{AVerb}
\end{minipage}
%%
  \label{fig_back1_a}} \vline & 
\subfloat{\begin{minipage}[t]{2in}
\begin{Verbatim}
     E (1)
     |
     V
   -----(2)
  |a = 1|
  |b = 2|
   -----
     |
     V
 ---------(3)    -----(4)
|if(a > b)|---->|c = 4|
 ---------       -----
     |             |
     V             V
 ---------(5)  ---------(6)
|c = a + 3|-->|printf(c)|
 ---------     ---------
                   |
                   V
                   |
                   X (7)
\end{Verbatim}
\end{minipage}





%%
  \label{fig_back1_b}} \\
\subref{fig_back1_a} & \subref{fig_back1_b} 
\end{tabular}
\caption{\subref{fig_back1_a} A C-language program fragment. \subref{fig_back1_b} The
  \emph{control-flow graph} (CFG) for the program.}
\label{fig_back1}
\end{myfig}

Directed edges show the order in which nodes execute. The edges
leaving \refNode{lst_back2_test} (representing the test
``\verb=if(a > b)='' on line \ref{lst_back1_test}) show that
execution can branch to either \refNode{lst_back2_true} (when
$a > b$) or \refNode{lst_back2_false} (when
$a \leq b$). A node followed by multiple successors
(i.e., where multiple edges leave the node) represents a \emph{branch}
or \emph{conditional} statement. Any one of the successor nodes may
execute following the conditional statement.

In this particular example, it is obvious that
\refNode{lst_back2_false} will always execute after
\refNode{lst_back2_test}, because the test will always fail. However,
control-flow graphs show \emph{possible} execution paths. They do not
take into account the actual runtime values in a given graph. While in
this case it is easy to determine how the program will behave, in
general we cannot predict behavior without running the program. 

The dataflow algorithm approximates a program's runtime state by
analyzing the control flow of the program. Control-flow graphs show
the order in which expressions and statements in a program are
evaluated. It is the job of our \emph{dataflow analysis} to determine
how to make the program more efficient.

\section{Basic Blocks}
\label{sec_back3}

%% %% Begin by placing the specific concept in the overall context of
%% %% dataflow. Give a small example highlighting the concept. Point
%% %% out fine points or subtleties that occur when generalizing the concept. End
%% %% by summarizing how the concept fits into dataflow (in a bit larger
%% %% sense than the first summary).

%% Basic blocks
Consider the C-language fragment and control-flow graphs (CFG) in
Figure~\ref{fig_back4}.  Part~\subref{fig_back4_b} shows the CFG for
Part~\subref{fig_back4_a}: a long, straight sequence of nodes, one
after another. Part~\subref{fig_back4_c} represents the assignment statements on
lines~\ref{lst_back3_start} -- \ref{lst_back3_end} as a \emph{basic
  block}: a sequence of statements with one entry, one exit, and no
branches in-between. Execution cannot start in the ``middle'' of the
block, nor can it branch anywhere but at the end of the block. In fact,
Part~\ref{fig_back4_b} also shows four basic blocks -- they just happen
to consist of one statement each.

\begin{myfig}
\begin{tabular}{m{1.5in}m{1.5in}m{1.5in}}
  \begin{center}
    \subfloat{\begin{minipage}[t]{1.5in}
\begin{AVerb}[numbers=left]
int a = 1; \label{lst_back3_start}
int b = 2; 
int c = 3; 
int d = 4; \label{lst_back3_end}

if(a + d > b + c)
  \dots
else
  \dots
\end{AVerb}
\end{minipage}
\label{fig_back4_a}}
  \end{center} & %%
  \begin{center}
    \subfloat{\begin{tikzpicture}[>=stealth, node distance=.5in]
  \withmd{\node[invis] (entry) {};

  \node[stmt, below of=entry] (assigna) {!+a = 1+!\labelNode{lst_back4_assigna}};
  \node[labelfor=assigna] () {\refNode{lst_back4_assigna}};

  \node[stmt, below of=assigna] (assignb) {!+b = 2+!\labelNode{lst_back4_assignb}};
  \node[labelfor=assignb] () {\refNode{lst_back4_assignb}};

  \node[stmt, below of=assignb] (assignc) {!+c = 3+!\labelNode{lst_back4_assignc}};
  \node[labelfor=assignc] () {\refNode{lst_back4_assignc}};

  \node[stmt, below of=assignc] (assignd) {!+d = 4+!\labelNode{lst_back4_assignd}};
  \node[labelfor=assignd] () {\refNode{lst_back4_assignd}};

  \node[invis, below of=assignd] (exit) {};

  \draw [->>] (entry) to (assigna);
  \draw [->] (assigna) to (assignb);
  \draw [->] (assignb) to (assignc);
  \draw [->] (assignc) to (assignd);
  \draw [->>] (assignd) to (exit);}

\end{tikzpicture}
\label{fig_back4_b}}
  \end{center}
  & %%
  \begin{center}
    \subfloat{\begin{minipage}[t]{2in}
\begin{AVerb}
         E (1)
         |
         V
       -----(2)
      |a = 1|
      |b = 2|
      |c = 3|
      |d = 4|
       -----
         |
         V
 -----------------(3)
|if(a + d > b + c)|--|
 -----------------   V
         |          ---(4)
         V         |\dots|
     ---(5)         ---
    |\dots|--> X <---|
     ---
\end{AVerb}
\end{minipage}





\label{fig_back4_c}}
  \end{center} \\
  \vtop{\centering\subref{fig_back4_a}} & \vtop{\centering\subref{fig_back4_b}} & \vtop{\centering\subref{fig_back4_c}} \\
\end{tabular}
\caption{\subref{fig_back4_a}: A C-language fragment to illustrate
  \emph{basic blocks}.  \subref{fig_back4_b}: The CFG for
  \subref{fig_back4_a} without basic blocks. \subref{fig_back4_c}: The
  CFG for \subref{fig_back4_c} using basic blocks.}
\label{fig_back4}
\end{myfig}

The representation given in Part~\subref{fig_back4_c} has a number of
advantages. It tends to reduce both the number of nodes and the number
of edges in the graph. The dataflow algorithm maintains two sets of
\emph{facts} for every node -- reducing the number of nodes obviously
reduces the number of facts stored. The algorithm also iteratively
propagates facts along edges -- so reducing the number of edges
reduces the amount of work we need to do. When rewriting, blocks allow
us to move larger amounts of the program at once. It also can be shown
(see \citep{Aho2006}) that we do not lose any information by collapsing
statements into blocks. For efficiency and brevity, we will work with
basic blocks rather than statements from here forward.

\section{The Dataflow Algorithm}

Kildall's dataflow algorithm provides a general-purpose mechanism for
analyzing control-flow graphs of programs. The algorithm itself does
not mandate a specific analysis. Rather, it is parameterized by the
choice of \emph{facts}, \emph{lattice}, \emph{meet operator},
\emph{transfer function}, and \emph{direction}. Facts, the lattice,
and the meet operator are used to approximate some property of the
program that we wish to analyze. The transfer function creates facts
that mimic the flow of information in the control-flow graph. The
direction is dictacted by the type of analysis -- each particular
analysis runs \emph{forwards} or \emph{backwards}.

%% Constant-propagation
Consider Figure~\ref{fig_back7}, Part~\subref{fig_back7_initial}, which
shows a C function containing a loop that multiplies the argument by 10
some number of times. Line~\ref{fig_back7_m} declares #m# and assigns
it the value 10. The function uses #m# in the loop body on
Line~\ref{fig_back7_loop} to multiply the value passed in
repeatedly. 

\begin{myfig}[tbh]
  \begin{tabular}{cc}
    \subfloat{\begin{minipage}{2in}
\begin{AVerb}[gobble=2,numbers=left]
  int mult10(int cnt, int val) \{ 
    int m = 10, n = 1; \label{fig_back7_m}
    for(int i = 0; i < cnt; i++)
      n = val * m; \label{fig_back7_loop}

    return n;
  \}
\end{AVerb}
\end{minipage}
\label{fig_back7_initial}} & %%
    \subfloat{\begin{minipage}{\hsize/2-0.1in}\raggedright\disableoverfull
\begin{AVerb}[gobble=2,numbers=left]
  int mult10(int cnt, int val) \{ 
    int m = 10, n = 1; 
    for(int i = 0; i < cnt; i++)
      n += val * 10; \label{fig_back7_opt_loop}

    return n;
  \}
\end{AVerb}
\end{minipage}
\label{fig_back7_opt}} \\

    \subref{fig_back7_initial} & \subref{fig_back7_opt} 
  \end{tabular}
  \caption{A C program which multiplies its argument, \texttt{val}, by
    10 \texttt{cnt} times. Part~\subref{fig_back7_initial} shows the
    original program. In Part~\subref{fig_back7_opt}, we have used
    \emph{constant propagation} to replace the use of \texttt{m} in
    the loop body with 10.}
  \label{fig_back7}
\end{myfig}

The program in Figure~\ref{fig_back7_initial} can be improved by
replacing the variable #m# with 10 in the loop body. We can use a
dataflow analysis known as \emph{constant propagation} to implement
this optimization. The constant propagation analysis recognizes when a
variable's value does not change in some context and then replaces
references to the variable with the constant
value. Figure~\ref{fig_back7}, Part~\subref{fig_back7_opt} shows the
optimized program, replacing #m# with 10 on
Line~\ref{fig_back7_opt_loop}.

\subsection{Facts and Lattices} 
\label{back_subsec_facts}

Constant propagation analyzes how each variable's value changes during
execution. The analysis \emph{approximates} the actual values of the
variables, as we cannot in general determine their exact value. We
will place the value of each variable into one of three categories at
each point in the control-flow graph: \emph{unknown}, a \emph{known
  constant}, or \emph{indeterminate}. \emph{Unknown}, represented by
$\bot$ (``bottom''), is the initial value for all variables in our
analysis. A \emph{known constant}, $C$, means our analysis identified
that the variable was assigned a specific value. \emph{Indeterminate},
indicated by $\top$ (``top''), means our analysis found that the
variable might have more than one value at a given point. The values
$\bot$, $C$, and $\top$ form a set which we will denote as \setLC.

\begin{myfig}[bth]
  \begin{tikzpicture}[>=stealth, node distance=0.75in and 1in]
  \newbox\fboxA
  \newbox\fboxB
  \begin{pgfinterruptpicture}
    \global\setbox\fboxA=\hbox{\facts{i/0}}
    \global\setbox\fboxB=\hbox{\facts{i/\top}}
  \end{pgfinterruptpicture}
  \def\prefix{lst_back17_}
  \pgfkeys{/tikz/incr/.style={node distance=0.25in},
    /tikz/return/.style={node distance=0.25in}}
  \withmd{\pgfkeysifdefined{/tikz/incr}{}{\pgfkeys{/tikz/incr/.append style={}}}
\pgfkeysifdefined{/tikz/return}{}{\pgfkeys{/tikz/return/.append style={}}}
\pgfkeysifdefined{/tikz/assign}{}{\pgfkeys{/tikz/assign/.append style={}}}
\pgfkeysifdefined{/tikz/test}{}{\pgfkeys{/tikz/test/.append style={}}}
\pgfkeysifdefined{/tikz/mult}{}{\pgfkeys{/tikz/mult/.append style={}}}

  \node[invis] (entry) {};

  \node[stmt, assign, below=0.2in of entry] (assign) {\begin{minipage}{0.5in}
      \begin{AVerb} 
m = 10
n = 1
i = 0
      \end{AVerb}
    \end{minipage}
    \labelNode{\prefix assign}};
  \node[labelfor=assign] () {\refNode{\prefix assign}};

  \node[stmt, test, below=of assign] (test) {!+i < cnt+!\labelNode{\prefix test}};
  \node[labelfor=test] () {\refNode{\prefix test}};

  \node[stmt, mult, right=of test] (mult) {!+n += val * m+!\labelNode{\prefix mult}};
  \node[labelfor=mult] () {\refNode{\prefix mult}};

  \node[stmt, incr, below=of mult] (incr) {!+i+++!\labelNode{\prefix incr}};
  \node[labelfor=incr] () {\refNode{\prefix incr}};

  \node[stmt, return, below=of test] (return) {!+return n+!\labelNode{\prefix return}};
  \node[labelfor=return] () {\refNode{\prefix return}};

  \node[invis, below=0.2in of return] (exit) {};

  \draw [->>] (entry) to (assign);
  \draw [->] (assign) to (test);
  \draw [->] (test) to (mult);
  \draw [->] (mult) to (incr);
  \pausemd{\draw [->] (incr) -| ($(mult.east) + (5mm,0mm)$) |- ($(test.north)!.5!(assign.south)$) to (test.north);}
  \draw [->] (test) to (return);
  \draw [->>] (return) to (exit);
}
 
  \node[fact, below=5mm of assign, anchor=west] () {\outFactsM{lst_back17_assign}{i/0}{\wd\fboxA}};
  \node[fact, below=3mm of incr, anchor=west] () {\outFactsM{lst_back17_incr}{i/\top}{\wd\fboxB}};
  \node[fact, above=5mm of test, anchor=west] () {\inFactsM{lst_back17_test}{i/\top}{\wd\fboxB}};
\end{tikzpicture}

  \caption{Our program, annotated with facts partway through the
    analysis. Notice that \outB{lst_back17_assign} and
    \outB{lst_back17_incr} give differing values to $i$. We use a \emph{meet
      operator} when combining these two values while calculating
    \inB{lst_back17_test}.}
  \label{fig_back11}
\end{myfig}

Figure~\ref{fig_back11} shows the control-flow graph of our program
annotated with \emph{facts} before (\inE) and after (\out) select
program points. Each \emph{fact} indicates our knowledge of a
variable's value at that point in the program. For this analysis, our
facts are sets of pairs, $(a,x)$, where $a$ is the name of a variable
and $x$ a value in \setLC. 

Constant propagation is a \emph{forwards} analysis, meaning the values
for each \inE set are calculated based on the \out values of its
predecessors. Figure~\ref{fig_back11} shows the facts computed partway
through this analysis, concentrating on the nodes that reference $i$:
\refNode{lst_back17_assign}, \refNode{lst_back17_test} and
\refNode{lst_back17_incr}. \refNode{lst_back17_test} has two
predecessors: \refNode{lst_back17_assign} and
\refNode{lst_back17_incr}. Their \out sets,
\outB{lst_back17_assign} and \outB{lst_back17_incr}, give differing
values to $i$: 0 and $\top$. To combine these values when computing
\inB{lst_back17_test}, we use a \emph{meet operator}.

The \emph{meet operator}, \lub (pronounced ``least upper bound'' or
``lub''), defines how we will combine values in
\setLC. Table~\ref{tbl_back4} gives the definition of \lub. For any
value of $x \in \setLC$, $\bot \lub x$ results in $x$. Conversely, $x
\lub \top$ results in $\top$. Two differing constants, $C_1$ and
$C_2$, result in $\top$, while equal constants give the same constant. 

\begin{myfig}
  \begin{math}
    \begin{array}{cccc}
      v_1 & v_2 & v_1 \lub v_2 \\
      \cmidrule(r){1-1}\cmidrule(r){2-2}\cmidrule(r){3-3}
      \bot & x & x \\
      x & \top & \top & \\ 
      C_1 & C_2 & \top & \text{($C_1 \neq C_2$)} \\
      C_1 & C_2 & C_1 & \text{($C_1 = C_2$)}
    \end{array}
  \end{math}
  \caption{Definition of the \emph{meet operator}, lub, for the
    lattice used in our constant propagation analysis. $v_1$ and $v_2$
    are values in \setLC. The table shows how lub combines any two
    values.}
  \label{tbl_back4}
\end{myfig}

We have defined \lub on elements in \setLC, but our facts are set of
pairs $(a,x)$ where $a$ is a variable and $x$ a value in \setLC. To
compute \inBa, we apply \lub set-wise to the values for matching
variables in each \out set of $B$'s predecessors. We use the $\wedge$
(``wedge'') operator to represent our meet over sets and pairs:
\begin{align}\allowdisplaybreaks[0]
    \inBa &= \bigwedge\limits_{\mathclap{P \in \mathit{pred}(B)}} \outXa{P} \label{eqn_back8}\\ 
    B \bigwedge\ \,\mathclap{\emptyset}\phantom{C} &= B \notag\\*
    B \bigwedge\ \,\mathclap{C}\phantom{C} &= \bigcup\limits_{(a,x) \in B}
    \left(\bigcup\limits_{(b,y) \in C} (a,x) \wedge (b,y)\right) \label{eqn_back6}\\
  (a,x) \wedge (b,y) &= 
  \begin{cases}
    (a,x \lub y) & \text{when}\ (a,x) \in B, (b,y) \in C,\ \text{\lub as in Table~\ref{tbl_back4}.}\\
    (a,x) & \text{when}\ b \not\in \mathit{var}(B). \\
    (b,y) & \text{when}\ a \not\in \mathit{var}(C). \\
  \end{cases} \label{eqn_back7}
\end{align}


Using these equations, we can show how the \inB{lst_back17_test}
set in Figure~\ref{fig_back11} is derived:
\begin{flalign*}\allowdisplaybreaks[0]
    \inB{lst_back17_test} &= \bigwedge\limits_{\mathclap{P \in \mathit{pred}(\refNode{lst_back17_test})}} \outXa{P} 
    \intertext{Predecessors of \refNode{lst_back17_test}; Equation~\eqref{eqn_back8}.} 
    \inB{lst_back17_test} &= \outB{lst_back17_assign} \bigwedge \outB{lst_back17_incr} 
    \intertext{Equation~\eqref{eqn_back6}.}
    \inB{lst_back17_test} &= \bigcup\limits_{(a,x) \in \outB{lst_back17_assign}}
      \left(\bigcup\limits_{(b,y) \in \outB{lst_back17_incr}} (a,x) \wedge (b,y)\right) 
    \intertext{Definition of \outB{lst_back17_assign}, and \outB{lst_back17_incr} in Figure~\ref{fig_back11}.}
    \inB{lst_back17_test} &= \bigcup\limits_{(a,x) \in \{\factC{i}{0}\}}
      \left(\bigcup\limits_{(b,y) \in \{\factC{i}{\top}\}} (a,x) \wedge (b,y)\right) 
    \intertext{Equation~\eqref{eqn_back6}.}
    \inB{lst_back17_test} &= \{\factC{i}{0}\}\ \wedge \{\factC{i}{\top}\} 
    \intertext{Equation~\eqref{eqn_back7}.}
    \inB{lst_back17_test} &= \{\factC{i}{0 \lub \top}\}
    \intertext{Definition of \lub from Table~\ref{tbl_back4}.} 
    \inB{lst_back17_test} &= \{\factC{i}{\top}\} 
    \intertext{Definition of \inB{lst_back17_test}.}
    \{\factC{i}{\top}\} &= \{\factC{i}{\top}\}.
\end{flalign*}

Together, \setLC and \lub form a \emph{lattice}.\footnote{The lattice
  can also have a \emph{join} operator, but for our purposes we solely
  use the meet.}  The lattice precisely defines the facts computed in
our analysis. In this case, the lattice represents
knowledge about a variable's value. Each specific dataflow analysis
computes different facts, but those facts are always represented by a
lattice.

We have established that our analysis computes \emph{facts} at each
node in our programs control-flow graph. The facts are defined using a
\emph{lattice}. The lattice lets us assign the value $\bot$, $C$ (a
constant), or $\top$ to each variable in the program, at each node in
the graph. The lattice's \emph{meet operator} is used to combing
conflicting values when computing \inBa from the \out sets of $B$'s
predecessors. We next explore how \out facts are computed for each node.

%% Figure~\ref{fig_back11} demonstrates how the lattice computes facts
%% for constant propagation. The set \out{lst_back17_assign} indicates,
%% among other things, that $i$ is assigned 0: \factC{i}{0}. However,
%% \out{lst_back17_incr} indicates that the value of $i$ is indeterminate: 
%% \factC{i}{\top}. 

%% First, the values
%% computed for variables at each program point are in \setLC. Second,
%% the meet operator, \lub, is used to combine facts  The
%% lattice defines our facts. That is, the values in \setLC The lattice
%% defines how our facts will indicate if a variable has a constant
%% value.

%% Figure~\ref{fig_back10} shows our program with the final facts
%% computed. The sets \inB{lst_back13_mult} and \outB{lst_back13_mult}
%% show that #m# has the value 10 when block \refNode{lst_back13_mult}
%% executes. The variables #n# and #i# have the value $\top$, indicating
%% they could one of many different values. However, #cnt# and #val#
%% still have $\bot$, because their values are not modified anywhere in
%% the control-flow graph.

%% \begin{myfig}
%%   \begin{tikzpicture}[>=stealth, node distance=.75in]
  \def\prefix{lst_back14_}
  \withmd{\pgfkeysifdefined{/tikz/incr}{}{\pgfkeys{/tikz/incr/.append style={}}}
\pgfkeysifdefined{/tikz/return}{}{\pgfkeys{/tikz/return/.append style={}}}
\pgfkeysifdefined{/tikz/assign}{}{\pgfkeys{/tikz/assign/.append style={}}}
\pgfkeysifdefined{/tikz/test}{}{\pgfkeys{/tikz/test/.append style={}}}
\pgfkeysifdefined{/tikz/mult}{}{\pgfkeys{/tikz/mult/.append style={}}}

  \node[invis] (entry) {};

  \node[stmt, assign, below=0.2in of entry] (assign) {\begin{minipage}{0.5in}
      \begin{AVerb} 
m = 10
n = 1
i = 0
      \end{AVerb}
    \end{minipage}
    \labelNode{\prefix assign}};
  \node[labelfor=assign] () {\refNode{\prefix assign}};

  \node[stmt, test, below=of assign] (test) {!+i < cnt+!\labelNode{\prefix test}};
  \node[labelfor=test] () {\refNode{\prefix test}};

  \node[stmt, mult, right=of test] (mult) {!+n += val * m+!\labelNode{\prefix mult}};
  \node[labelfor=mult] () {\refNode{\prefix mult}};

  \node[stmt, incr, below=of mult] (incr) {!+i+++!\labelNode{\prefix incr}};
  \node[labelfor=incr] () {\refNode{\prefix incr}};

  \node[stmt, return, below=of test] (return) {!+return n+!\labelNode{\prefix return}};
  \node[labelfor=return] () {\refNode{\prefix return}};

  \node[invis, below=0.2in of return] (exit) {};

  \draw [->>] (entry) to (assign);
  \draw [->] (assign) to (test);
  \draw [->] (test) to (mult);
  \draw [->] (mult) to (incr);
  \pausemd{\draw [->] (incr) -| ($(mult.east) + (5mm,0mm)$) |- ($(test.north)!.5!(assign.south)$) to (test.north);}
  \draw [->] (test) to (return);
  \draw [->>] (return) to (exit);
}

  \node[fact, above=5mm of mult, anchor=west] () {\inFacts{lst_back14_mult}{m/{10},n/\top,i/\top,val/\top,cnt/\top}};
  \node[fact, below=3mm of mult, anchor=west] () {\outFacts{lst_back14_mult}{m/{10},n/\top,i/\top,val/\top,cnt/\top}};
\end{tikzpicture}

%%   \label{fig_back10}
%%   \caption{Our program, annotated with the final facts computed by the
%%     constant propagation analysis. Notice the \inB{lst_back14_mult}
%%     and \outB{lst_back14_mult} indicate that \texttt{m} has the value 10
%%     while \refNode{lst_back14_mult} executes.}
%% \end{myfig}

%% values. 
%% values $\bot$, an integer value, and $\top$ can be ordered such
%% that $\bot \le x \le \top$, for all integers $x$. The \emph{meet
%% operator} defines this ordering

%% Before
%% and after each block we will determine 

%% To track
%% the value of each variable, we use a \emph{lattice}. This particular
%% lattice encodes three types of values: 
%% Even so,
%% for this analysis all we care to know is if the value remains
%% the same or changes. 

%% approximate by determining, at each point in the control-flow graph,
%% whether a variable has one of three values: an \emph{unknown}, a
%% \emph{constant}, or

\subsection{Transfer Functions}
\label{back_subsec_transfer}

The dataflow algorithm calculates new facts using a \emph{transfer
  function}. The transfer function is specific to both the analysis
performed and the semantics of the programming language used to write
the program analyzed. Theoretically, each node in the graph can
have its own transfer function, but in practice the function is 
defined by cases for each statement or expression in the language. 

In a \emph{forwards} analysis, the transfer function computes the \out
set for a given node. In a backwards analysis, the transfer function
computes the \inE set. That is, a forwards analysis computes facts
that hold \emph{after} a node executes. A backwards analysis computes
facts that were true \emph{before} a node executed.  In both cases,
the transfer function also considers known facts (i.e., \inE facts for
forwards, \out for backwards) as well as the statements in the node.

Our constant propagation analysis determines, for each node in the
control-flow graph, what value a given variable has: $\bot$, $C$ (an
integer), or $\top$. We choose to use pairs $(a,x)$, where $a$ is
variable in the program and $x$ a value in \setLC, as the facts for
our analysis. We define our transfer function in terms of a single 
pair (i.e., fact):
\begin{equation}
  f (a,x) = 
  \begin{cases}
    (a,x \lub C) & \text{when \texttt{a = \emph{C}}, where \texttt{\emph{C}} is an integer.} \\
    (a,\top) & \text{when \texttt{a} updated}. \\
    (a,x) & \text{otherwise}. \\
  \end{cases}
  \label{eqn_back4}
\end{equation}

Equation~\eqref{eqn_back4} considers two kinds of statements: constant
and non-constant updates. If the statement #a = C# appears in the
node, our new fact is $(a,x \lub C)$, indicating we combine our
previous knowledge about $x$ with the new constant assigned. We create
the fact $(a,\top)$ for any other update.\footnote{In practice this
  analysis is much more sophisticated, capable of analyzing
  complicated arithmetic expressions. Algebraic properties such as
  associativity and distributivity are also usually considered.}
Finally, if neither type of statement appears in the node, we just
pass $x$ through unchanged.

Though we have defined Equation~\eqref{eqn_back4} in terms of a single
fact, we can easily extend it to sets of facts. To create \outBa for
some node $B$, we apply Equation~\eqref{eqn_back4} element-wise to
each fact in \inBa and combine the results into a set:
\begin{equation}
  \outBa = \bigcup\limits_{\mathclap{(a,x) \in \inBa}} f(a,x).
  \label{eqn_back5}
\end{equation}

Figure~\ref{fig_back9}, Part \subref{fig_back9_initial}, shows our
program, annotated with initial \inE and \out
facts. Figure~\ref{fig_back9}, Part \subref{fig_back9_transfer}, shows
the same graph with annotations updated using
Equation~\eqref{eqn_back4}. The assignments in
\refNode{lst_back18_assign} create the facts \factC{m}{10},
\factC{n}{1}, and \factC{i}{0} in \outB{lst_back18_assign}. The
multiplication in \refNode{lst_back18_mult} is a non-constant update,
so \outB{lst_back18_mult} contains \factC{n}{\top}. However,
\outB{lst_back18_mult} also shows that #m# is not modified in
\refNode{lst_back18_mult}; the value from \inB{lst_back18_mult} is
just copied to \outB{lst_back18_mult}.

\afterpage{\clearpage\begin{myfig}
  \begin{tabular}{c}
    \subfloat{\begin{tikzpicture}[>=stealth, node distance=.75in and 2in]
  \def\prefix{lst_back15_}
  \begin{pgfinterruptpicture}
    %% Box with width of a single fact.
    \newbox\sfbox 
    \global\setbox\sfbox=\hbox{\facts{X/\top}}
  \end{pgfinterruptpicture}

  \input{lst_back16}
  \setfacts{m/\bot,n/\bot,i/\bot}
  \node[fact, above=6mm of assign, anchor=west] () 
       {\inFactsM{\prefix assign}{\facttoks}{\wd\factbox}};

  \setfacts{m/\bot,n/\bot,i/\bot}
  \node[fact, below=3mm of assign, anchor=west] () 
       {\outFactsM{\prefix assign}{\facttoks}{\wd\factbox}};

  \node[fact, above=8mm of test, anchor=east] () 
       {\inFactsM{\prefix  test}{m/\bot,n/\bot,i/\bot}{\wd\sfbox}};

  \node[fact, below=8mm of test, anchor=east] () 
       {\outFactsM{\prefix test}{m/\bot,n/\bot,i/\bot}{\wd\sfbox}};

  \setfacts{m/\bot,n/\bot,i/\bot}
  \node[fact, above=5mm of mult, anchor=west] () 
       {\inFactsM{\prefix mult}{\facttoks}{\wd\factbox}};

  \setfacts{m/\bot,n/\bot,i/\bot}
  \node[fact, below=3mm of mult, anchor=west] () 
       {\outFactsM{\prefix mult}{\facttoks}{\wd\factbox}};

  \setfacts{m/\bot,n/\bot,i/\bot}
  \node[fact, above=5mm of incr, anchor=west] () 
       {\inFactsM{\prefix incr}{\facttoks}{\wd\factbox}};

  \setfacts{m/\bot,n/\bot,i/\bot}
  \node[fact, below=3mm of incr, anchor=west] () 
       {\outFactsM{\prefix incr}{\facttoks}{\wd\factbox}};

  \setfacts{m/\bot,n/\bot,i/\bot}
  \node[fact, above=6mm of return, anchor=west] () 
       {\inFactsM{\prefix return}{\facttoks}{\wd\factbox}};

  \setfacts{m/\bot,n/\bot,i/\bot}
  \node[fact, below=3mm of return, anchor=west] () 
       {\outFactsM{\prefix return}{\facttoks}{\wd\factbox}};

\end{tikzpicture}
\label{fig_back9_initial}} \\
    \subref{fig_back9_initial} \\
    \subfloat{\setcounter{nodeCounter}{0}\begin{tikzpicture}[>=stealth, node distance=.75in and 2in]
  \def\prefix{lst_back18_}
  \input{lst_back16}
  
  \setfacts{m/{10},n/0,i/0}
  \node[fact, below=3mm of assign, anchor=west] ()
       {\outFactsM{lst_back18_assign}{m/{10},n/0,i/0}{\wd\factbox}};

  \setfacts{m/\bot,n/\bot,i/\bot}
  \node[fact, above=6mm of test, anchor=west] () 
       {\inFactsM{\prefix  test}{m/\bot,n/\bot,i/\bot}{\wd\factbox}};

  \setfacts{m/\bot,n/\bot,i/\bot}
  \node[fact, above right=2mm and -11.75mm of mult] () 
       {\inFactsM{lst_back18_mult}{m/\bot,n/\bot,i/\bot}{\wd\factbox}};

  \setfacts{m/\bot,n/\top,i/\bot}
  \node[fact, below right=3mm and -14mm of mult,anchor=west] () 
       {\outFactsM{lst_back18_mult}{m/\bot,n/\top,i/\bot}{\wd\factbox}};

  \setfacts{m/\bot,n/\bot,i/\bot}
  \node[fact, above right=6mm and -2mm of incr, anchor=west] () 
       {\inFactsM{\prefix incr}{m/\bot,n/\bot,i/\bot}{\wd\factbox}};

  \setfacts{m/\bot,n/\bot,i/\bot}
  \node[fact, below right=3mm and -4.25mm of incr, anchor=west] () 
       {\outFactsM{\prefix incr}{m/\bot,n/\bot,i/\top}{\wd\factbox}};
\end{tikzpicture}
\label{fig_back9_transfer}} \\
    \subref{fig_back9_transfer} \\
  \end{tabular}
  \caption{Part~\subref{fig_back9_initial} shows our program annotated
    with initial facts. In Part~\subref{fig_back9_transfer}, we have
    updated each \outBa set using Equation~\eqref{eqn_back4}, our
    transfer function.}
  \label{fig_back9}
\end{myfig}
\clearpage}

Every dataflow analysis defines a \emph{transfer function} for
creating (or updating) facts. The function is specific to both the
analysis performed and the semantics of the underlying
program. Equation~\eqref{eqn_back4}, the transfer function for our
constant propagation example, defines how we derive information about
a variable's value after the statements in the given node execute.
Equation~\eqref{eqn_back5} extended Equation~\eqref{eqn_back4} to sets,
showing how we can create \outBa from \inBa. In the next section, we
iteratively apply our transfer function and lattice to the
control-flow graph. 

\subsection{Iteration \& Fixed Points}
\label{back_subsec_iter}

Figure~\ref{fig_back9} hints that facts
develop over time during analysis. In fact, the transfer function is
applied to each node in turn, creating new facts from old, until the
facts stop changing. In other words, the control-flow graph is
analyzed \emph{iteratively} until all \out (in the case of a forwards
analysis) or \inE (in the case of a backwards analysis) sets reach a
\emph{fixed point}.

Figure~\ref{fig_back13}, Part~\subref{fig_back13_tbl} shows how the
\inE and \out sets for each node change during our
analysis. The ``zeroth'' iteration corresponds to the initial value
for all facts: everything is $\bot$, i.e., unknown. Reading from
left-to-right gives the \inE and \out facts for a given node at each
iteration of the analysis. The control-flow graph is reproduced in
Part~\subref{fig_back13_cfg}. Following the control-flow between nodes
shows which \out sets are used to calcuate \inE sets.

\begin{myfig}[tb]
  \setlength{\tabcolsep}{2pt}
  \hbox to \textwidth{\hss
  \begin{tabular}{cc}
    \subfloat{\begin{minipage}{2.25in}
\begin{math}
  \begin{array}{l@{\phantom{X}}cccccc}
    \textit{Iteration:} &
    0 &
    1 &
    2 &
    3 &
    4 \\\midrule

    \inB{lst_back15_assign} & %%
    \bot & %% 0 
    \bot & %% 1 
    \bot & %% 2
    \bot & %% 3
    \bot \\ %% 4
    \outB{lst_back15_assign} & %%
    \bot & %% 0 
    0 & %% 1 
    0 & %% 2
    0 & %% 3
    0 \\\\ %% 4

    \inB{lst_back15_test} & %%
    \bot & %% 0 
    \bot & %% 1 
    \top & %% 2
    \top & %% 3
    \top\\ %% 4
    \outB{lst_back15_test} & %%
    \bot & %% 0 
    \bot & %% 1 
    \top & %% 2
    \top & %% 3
    \top \\\\ %% 4

    \inB{lst_back15_mult} & %%
    \bot & %% 0 
    \bot & %% 1 
    \bot & %% 2
    \top & %% 3
    \top \\ %%4  
    \outB{lst_back15_mult} & %%
    \bot & %% 0 
    \bot & %% 1 
    \bot & %% 2
    \top & %% 3
    \top \\\\ %% 4

    \inB{lst_back15_incr} & %%
    \bot & %% 0 
    \bot & %% 1 
    \bot & %% 2
    \bot & %% 3
    \top \\ %% 4
    \outB{lst_back15_incr} & %%
    \bot & %% 0 
    \top & %% 1 
    \top & %% 2
    \top & %% 3
    \top \\\\ %% 4

    \inB{lst_back15_return} & %%
    \bot & %% 0 
    \bot & %% 1 
    \bot & %% 2
    \top & %% 3
    \top \\ %% 4
    \outB{lst_back15_return} & %%
    \bot & %% 0 
    \bot & %% 1 
    \bot & %% 2
    \top & %% 3
    \top \\\\ %% 4
  \end{array}
\end{math}
\end{minipage}
\label{fig_back13_tbl}} & 
    \subfloat{\input{fig_back12_cfg}\label{fig_back13_cfg}} \\
    \subref{fig_back13_tbl} & \subref{fig_back13_cfg}
  \end{tabular}\hss}
  \caption{This figure shows the values for $i$ calculated by all nodes in our
    example program. Part~\subref{fig_back13_tbl} shows the \inE and
    \out facts associated with each node, for variable $i$. Part~\subref{fig_back13_cfg}
    reproduces the control-flow graph for our program. After 4
    iterations the facts reach a fixed point (i.e., they stop
    changing).}
  \label{fig_back13}
\end{myfig}

Consider the value for $i$ in \inB{lst_back15_test}, the node that
tests the condition #i < cnt#. In the first iteration,
\inB{lst_back15_test} still assigns $\bot$ to
$i$. Equation~\eqref{eqn_back8} states that \inB{lst_back15_test} is
derived from the \out sets of \refNode{lst_back15_test}'s
predecessors: \refNode{lst_back15_assign} and
\refNode{lst_back15_incr}. By Equations~\eqref{eqn_back8},
\eqref{eqn_back6}, and \eqref{eqn_back7} we can calculate the value of
$i$ in \inB{lst_back15_test}. Crucially, the \out set used comes from
the \emph{previous} iteration of the analysis, which we emphasize by
attaching the iteration number to each set:
\begin{align*}
  \inB{lst_back15_test}^1 &= \outB{lst_back15_assign}^0 \bigwedge \outB{lst_back15_incr}^0 \\
  &= {\factC{i}{\bot}} \wedge {\factC{i}{\bot}} \\
  &= {\factC{i}{\bot \lub \bot}} \\
  {\factC{i}{\bot}} &= {\factC{i}{\bot}}.
\end{align*}
Now consider the second iteration, where \inB{lst_back15_test} assigns
$\top$ to $i$. \outB{lst_back15_assign} gives $i$ the value
0 (by #i = 0#). However, \outB{lst_back15_incr} assigns $i$ the value $\top$,
because #i++# is a non-constant update. We can see why by
Equations~\eqref{eqn_back8}, \eqref{eqn_back6}, and
\eqref{eqn_back7}. Again we attach the iteration number to each set,
emphasizing its origin:
\begin{align*}
  \inB{lst_back15_test}^2 &= \outB{lst_back15_assign}^1 \bigwedge \outB{lst_back15_incr}^1 \\
  &= {\factC{i}{0}} \wedge {\factC{i}{\top}} \\
  &= {\factC{i}{0 \lub \top}} \\
  {\factC{i}{\top}} &= {\factC{i}{\top}}.
\end{align*}
Notice how the conflicting values for $i$ are resolved with the \lub
operator. The value of $i$ in \outB{lst_back15_test} has reached a fixed point
with this iteration; it will no longer change.

The above example raises an important question: how do we know we have
reached a fixed point? How do we know we have gotten the best possible
answer?  Both of these questsion can be answered if our lattice has
\emph{finite height} and the transfer function is \emph{monotone}.

Let us begin with the lattice. Consider again the meet operator, \lub,
defined in Table~\ref{tbl_back4} and our set of values, \setLC:
\begin{align*}
  \setLC &= \bot, \dots \mathit{all\ integers} \dots, \top. \\\\
  \bot \lub x &= x. \\
  x \lub \top &= \top.\\
  C_1 \lub C_2 &= \top, \text{where}\ C_1 \neq C_2.\\
  C_1 \lub C_1 &= C_1.
\end{align*}
The definition of \lub imposes a \emph{partial order} on the values
in \setLC. That is, we can define an operator, \sqlte, such that
for all $x$ and $y$ in \setLC:
\begin{equation}
  x \sqlte y\ \text{if and only if}\ x \lub y = y.
  \label{eqn_back9}
\end{equation}
That is, $x$ is ``less than or equal to'' $y$ only when $x \lub y$ equals
$y$.

We now define the \emph{height} of our lattice as the
longest possible ordering of values in \setLC such that:
\begin{equation}
  x_1 \sqlte x_2 \dots \sqlte x_n, \text{where}\ x_1 \neq x_2 \neq \dots \neq x_n.
  \label{eqn_back10}
\end{equation}
That is, the height is the longest possible path from the ``lowest''
to ``highest'' element of the lattice where we do not repeat any
values and where the \sqlte relation holds among all values.

We can more succinctly define the height using a strict ``less than''
ordering. First, the \sqlt relation:
\begin{equation}
  x \sqlt y\ \text{if and only if}\ x \lub y = y\ \text{and}\ x \neq y.
\end{equation}
And now we can redefine height as the largest $n$ such that:
\begin{equation}
  x_1 \sqlt x_2 \dots \sqlt x_n.
  \label{eqn_back17}
\end{equation}

We can show by contradiction that the height of our lattice is
3. Given $x_1 \sqlt x_2 \sqlt \dots \sqlt x_n$, such that n = 4. If
$x_4$ is $\top$, then $x_3$ must be an integer or $\bot$. If $x_3$ is
$\bot$, by Equation~\eqref{eqn_back17}, there is no such $x_2$ such
that $x_2 \sqlt \bot$. Therefore, $x_3$ cannot be $\bot$. If $x_3$ is
an integer, again by Equation~\eqref{eqn_back17}, $x_2$ must be
$\bot$. In turn, there is no such $x_1$ such that $x_1 \sqlt
\bot$. Therefore, $x_4$ cannot be $\top$ and in fact, by similar
arguments, it cannot exist. Using similar logic, we can show by cases
that $n$ (and the height of our lattice) must be 3.

Now let us address the transfer function. A \emph{monotone} function 
has the following property:
\begin{equation}
  f(x) \sqlte f(y)\ \text{whenever}\ x \sqlte y.
  \label{eqn_back11}
\end{equation}
That is, a monotone function does not change the ordering between its
inputs. If $x$ is ``less than or equal to'' $y$, $f(x)$ will also be
``less than or equal to'' $f(y)$. 

The transfer function moves our facts along the lattice. A monotone
transfer function will never ``decrease'' its argument -- $f$ will
always produce a value that is at the same ``height'' or ``higher'' in
the lattice. The lattice represents the information we have gathered
during our analysis. In turn, the ordering of values represents ``how
much'' we know. That is, when a variable is assigned $\bot$, we do not
know anything about it. If it is assigned $\top$, we have seen ``too
many'' assignments (or some other update).  This means our transfer
function always increases (or does not change) the information we
have. 

To show that our transfer function (from Equation~\eqref{eqn_back4}) is
monotone, consider some fact $(v,x)$ and some block $B$. $v$ is a
variable in the program; $x$ is a value in \setLC; and $B$ contains
some number of statements.  We analyze the fact, $(v,x')$ produced
applying our transfer function $f$ to $B$. If $B$ does not contain an
assignment affecting $v$, then $x = x'$, and we already know that $x \sqlte
x'$. If $B$ makes a non-constant update to $v$, then $x' = \top$ and we
know $x \sqlte \top$ for all $x$ by the definition of \lub. Finally,
if $B$ assigns some constant $C$ to $x$, then $x' = x \lub C$, which
again satisfies our relation. Therefore, Equation~\eqref{eqn_back4} is
monotone and, by a simple extension to sets, so is is
Equation~\eqref{eqn_back5}.

%% If our lattice has finite height, we can be sure that our
%% algorithm will terminate -- our transfer function will not oscillate
%% up and down the lattice!
%% By requiring the our analysis uses a \emph{lattice} with \emph{finite
%%   height} and a \emph{monotone} transfer function, we know our
%% analysis will eventually reach a fixed point and terminate. We will
%% now discuss how these same properties guarantee we have reached the
%% \emph{maximum fixpoint} and thus have the best possible answer.



%% Our goal is to find the ``largest amount'' of ``smallest'' information
%% about each variable. That is, if we assigned $\top$ to all variables,
%% we have the ``largest'' amount of information, but we do not know
%% anything useful. Assigning $\bot$ puts is in the opposite situation,
%% where we know the ``least.'' We want something in the middle, where we
%% have gathered the greatest amount of useful information. 



%% have this
%% property. 
%% In terms of \setLC and
%% Equation~\eqref{eqn_back4} (our transfer function), that means the
%% value for a variable will either stay the same or get ``bigger,'' as
%% defined by the lattice. For example, if \inBa (for some $B$) says $i$ is 10,
%% then $i$ in \outBa will be either 10 or $\top$. If $i$ could become $\bot$ (or
%% some other constant), Equation~\eqref{eqn_back4} would not be monotone. 


%% Trivially holds for our transfer function because it is defined in terms of \lub.

%% Together, guarantees we terminate. How do we get the ``best'' answer?


%% Monotonic
%% Finite height lattice
%% Partial orders
%% Maximum fixed point

%% \begin{myfig}
%%   \begin{tabular}[c]
%%     \subfloat{\begin{tikzpicture}[>=stealth, node distance=.75in and 2in]
  \def\prefix{lst_back15_}
  \begin{pgfinterruptpicture}
    %% Box with width of a single fact.
    \newbox\sfbox 
    \global\setbox\sfbox=\hbox{\facts{X/\top}}
  \end{pgfinterruptpicture}

  \withmd{\pgfkeysifdefined{/tikz/incr}{}{\pgfkeys{/tikz/incr/.append style={}}}
\pgfkeysifdefined{/tikz/return}{}{\pgfkeys{/tikz/return/.append style={}}}
\pgfkeysifdefined{/tikz/assign}{}{\pgfkeys{/tikz/assign/.append style={}}}
\pgfkeysifdefined{/tikz/test}{}{\pgfkeys{/tikz/test/.append style={}}}
\pgfkeysifdefined{/tikz/mult}{}{\pgfkeys{/tikz/mult/.append style={}}}

  \node[invis] (entry) {};

  \node[stmt, assign, below=0.2in of entry] (assign) {\begin{minipage}{0.5in}
      \begin{AVerb} 
m = 10
n = 1
i = 0
      \end{AVerb}
    \end{minipage}
    \labelNode{\prefix assign}};
  \node[labelfor=assign] () {\refNode{\prefix assign}};

  \node[stmt, test, below=of assign] (test) {!+i < cnt+!\labelNode{\prefix test}};
  \node[labelfor=test] () {\refNode{\prefix test}};

  \node[stmt, mult, right=of test] (mult) {!+n += val * m+!\labelNode{\prefix mult}};
  \node[labelfor=mult] () {\refNode{\prefix mult}};

  \node[stmt, incr, below=of mult] (incr) {!+i+++!\labelNode{\prefix incr}};
  \node[labelfor=incr] () {\refNode{\prefix incr}};

  \node[stmt, return, below=of test] (return) {!+return n+!\labelNode{\prefix return}};
  \node[labelfor=return] () {\refNode{\prefix return}};

  \node[invis, below=0.2in of return] (exit) {};

  \draw [->>] (entry) to (assign);
  \draw [->] (assign) to (test);
  \draw [->] (test) to (mult);
  \draw [->] (mult) to (incr);
  \pausemd{\draw [->] (incr) -| ($(mult.east) + (5mm,0mm)$) |- ($(test.north)!.5!(assign.south)$) to (test.north);}
  \draw [->] (test) to (return);
  \draw [->>] (return) to (exit);
}

  \setfacts{m/\bot,n/\bot,i/\bot}
  \node[fact, above=6mm of assign, anchor=west] () 
       {\inFactsM{\prefix assign}{\facttoks}{\wd\factbox}};

  \setfacts{m/\bot,n/\bot,i/\bot}
  \node[fact, below=3mm of assign, anchor=west] () 
       {\outFactsM{\prefix assign}{\facttoks}{\wd\factbox}};

  \node[fact, above=8mm of test, anchor=east] () 
       {\inFactsM{\prefix  test}{m/\bot,n/\bot,i/\bot}{\wd\sfbox}};

  \node[fact, below=8mm of test, anchor=east] () 
       {\outFactsM{\prefix test}{m/\bot,n/\bot,i/\bot}{\wd\sfbox}};

  \setfacts{m/\bot,n/\bot,i/\bot}
  \node[fact, above=5mm of mult, anchor=west] () 
       {\inFactsM{\prefix mult}{\facttoks}{\wd\factbox}};

  \setfacts{m/\bot,n/\bot,i/\bot}
  \node[fact, below=3mm of mult, anchor=west] () 
       {\outFactsM{\prefix mult}{\facttoks}{\wd\factbox}};

  \setfacts{m/\bot,n/\bot,i/\bot}
  \node[fact, above=5mm of incr, anchor=west] () 
       {\inFactsM{\prefix incr}{\facttoks}{\wd\factbox}};

  \setfacts{m/\bot,n/\bot,i/\bot}
  \node[fact, below=3mm of incr, anchor=west] () 
       {\outFactsM{\prefix incr}{\facttoks}{\wd\factbox}};

  \setfacts{m/\bot,n/\bot,i/\bot}
  \node[fact, above=6mm of return, anchor=west] () 
       {\inFactsM{\prefix return}{\facttoks}{\wd\factbox}};

  \setfacts{m/\bot,n/\bot,i/\bot}
  \node[fact, below=3mm of return, anchor=west] () 
       {\outFactsM{\prefix return}{\facttoks}{\wd\factbox}};

\end{tikzpicture}
\label{fig_back8_initial}} \\
%%     \subref{fig_back8_initial} \\
%%     \subfloat{\begin{tikzpicture}[>=stealth, node distance=.75in]
  \def\prefix{lst_back13_}
  \withmd{\pgfkeysifdefined{/tikz/incr}{}{\pgfkeys{/tikz/incr/.append style={}}}
\pgfkeysifdefined{/tikz/return}{}{\pgfkeys{/tikz/return/.append style={}}}
\pgfkeysifdefined{/tikz/assign}{}{\pgfkeys{/tikz/assign/.append style={}}}
\pgfkeysifdefined{/tikz/test}{}{\pgfkeys{/tikz/test/.append style={}}}
\pgfkeysifdefined{/tikz/mult}{}{\pgfkeys{/tikz/mult/.append style={}}}

  \node[invis] (entry) {};

  \node[stmt, assign, below=0.2in of entry] (assign) {\begin{minipage}{0.5in}
      \begin{AVerb} 
m = 10
n = 1
i = 0
      \end{AVerb}
    \end{minipage}
    \labelNode{\prefix assign}};
  \node[labelfor=assign] () {\refNode{\prefix assign}};

  \node[stmt, test, below=of assign] (test) {!+i < cnt+!\labelNode{\prefix test}};
  \node[labelfor=test] () {\refNode{\prefix test}};

  \node[stmt, mult, right=of test] (mult) {!+n += val * m+!\labelNode{\prefix mult}};
  \node[labelfor=mult] () {\refNode{\prefix mult}};

  \node[stmt, incr, below=of mult] (incr) {!+i+++!\labelNode{\prefix incr}};
  \node[labelfor=incr] () {\refNode{\prefix incr}};

  \node[stmt, return, below=of test] (return) {!+return n+!\labelNode{\prefix return}};
  \node[labelfor=return] () {\refNode{\prefix return}};

  \node[invis, below=0.2in of return] (exit) {};

  \draw [->>] (entry) to (assign);
  \draw [->] (assign) to (test);
  \draw [->] (test) to (mult);
  \draw [->] (mult) to (incr);
  \pausemd{\draw [->] (incr) -| ($(mult.east) + (5mm,0mm)$) |- ($(test.north)!.5!(assign.south)$) to (test.north);}
  \draw [->] (test) to (return);
  \draw [->>] (return) to (exit);
}

\end{tikzpicture}
\label{fig_back8_final}} \\
%%     \subref{fig_back8_final} 
%%   \end{tabular}
%%   \caption{The control flow graph for our program. Part~\subref{fig_back8_initial}
%%     shows the initial facts associated with each node. Part~\subref{fig_back8_final}
%%     shows the final facts computed by our constant propagation analysis.}
%%   \label{fig_back8}
%% \end{myfig}


\section{Dataflow Equations}
\label{back_subsec_eq}

As we stated in the beginning of this chapter, the dataflow algorithm
is \emph{parameterized} by four items: facts, meet operator,
direction, and transfer function. The prior section presented each
parameter for the constant propagation analysis
separately. Figure~\ref{fig_back10} presents all of them together, as
a set of \emph{dataflow equations}. Pairs of elements from the sets
\setLC and \setL{Var} define our facts. Equations~\eqref{eqn_back12}
and \eqref{eqn_back13} define our meet operator; together with the set
\setLC, they define our lattice. Equations~\eqref{eqn_back3} and
\eqref{eqn_back16} together defieshow we are defining a \emph{forwards}
  analysis. Equations~\eqref{eqn_back15} and \eqref{eqn_back16} define
our transfer function.

\begin{myfig}
  \begin{math}
    \begin{array}{rlr}

      \multicolumn{3}{c}{\emph{Facts}} \\

      \setLC &= \{\bot, \top\} \cup \ZZ.\\
      \setL{Var} &= \text{Set of all variables.} \\
      \setL{Fact} &= \setL{Var} \times \setLC. \\\\

      \multicolumn{3}{c}{\emph{Meet}} \\

      F_1 \wedge\ F_2 &= \begin{array}{rl}
        \{(a, x \lub y)\ | & a \in \dom(F_1), a \in \dom(F_2)\}\ \cup \\
        \{(a, y)\ | & a \in \dom(F_1), a \not\in \dom(F_2)\ \text{or} \\
                     & a \not\in \dom(F_1), a \in \dom(F_2)\},
      \end{array} \labeleq{eqn_back13} \labeleq{eqn_back12} & \eqref{eqn_back12} \\
      & \text{where\ } F_1, F_2 \in \setL{Fact}, \lub\ \text{as in Table~\ref{tbl_back4}.} \\\\

      \multicolumn{3}{c}{\emph{Transfer}} \\
      t (F, a\ \text{\tt =}\ C) &= \{(a, x \lub C), \text{when\ } (a, x) \in F\ \text{or} \\
                     & \phantom{= \{}(a, C), \text{when\ } a \not\in \dom(F)\}\ \cup \\
                     & \phantom{=} F\ \backslash\ \mfun{uses}(F, a),\\
          & \text{where\ } F \in \setL{Fact}, C \in \ZZ. \\
      t (F, a\text{\tt ++}) &= \{(a, \top)\} \cup (F\ \backslash\ \mfun{uses}(F, a)), \\
      & \text{where\ } F \in \setL{Fact}. \labeleq{eqn_back14} & \eqref{eqn_back14} \\\\
      \mfun{uses}(F, a) &= \{(a, x)\ |\ a \in \dom(F)\}, \\
      & \text{where\ } F \in \setL{Fact}, a \in \setL{Var}. \\\\

      \multicolumn{3}{c}{\emph{Direction}} \\

      \outBa &= t(\inBa, s), \labeleq{eqn_back3} & \eqref{eqn_back3} \\
      & \text{where $s$ a statement in block\ } B.\\
      \inBa &= \bigwedge\limits_{\mathclap{P \in \mathit{pred}(B)}} \outXa{P} \labeleq{eqn_back16} & \eqref{eqn_back16} \\\\ 
      \mfun{pred}(B) &= \text{Predecessors of block }\ B.
    \end{array}
  \end{math}
  \caption{The transfer function and associated definitions for the constant
  propagation analysis. Equation~\eqref{eqn_back3} shows how \out facts are
  created from \inE facts. \InBa facts, for some block $B$, are created from
  the \outBa facts of its predecessors. Facts are combined using the set-wise
  $\bigwedge$ operator.}
\label{fig_back10}
\end{myfig}


We can now define any dataflow analysis in terms of these four
parameters:
\begin{align*}
  \setL{Facts} & \qquad & \text{\it Our set of facts}. \\
  \wedge & \qquad & \text{\it Our \emph{meet} operator.} \\
  D & \qquad & \text{\it Direction, \emph{forwards} or \emph{backwards}}. \\
  \mathit{transfer}(v) & \qquad & \text{\it Our transfer function, with $v \in \setL{Facts}$.}
\end{align*}

In turn, we can define the iterative dataflow algorithm for the
\emph{forwards} direction. Figure~\ref{fig_back14} gives the
algorithm.\footnote{The \emph{backwards} algorithm is almost
  identical.} The algorithm initializes all \out sets to $\bot$, or
some suitable initial value from \setL{Facts}. The entry node gets
special treatment in some cases, so we set \outXa{Entry} to its own
initial value (though in many cases \outXa{Entry} is set to the same
value as other \out sets). We now enter the main loop of the
algorithm. The body of the loop always executes at least once. We
first calculate all \inE sets from their predecessors' \out sets on
Line~\ref{fig_back14_in}. On the next line, the new \inE sets are used
to calculate \out sets for each node.  Line~\ref{fig_back14_loop}
gives our termination condition: when \out sets stop changing, we are
done. The superscript on each \out set represents the
iteration. $\outBa^{i}$ means the $i$-th iteration, and $\outBa^{i-1}$
the previous ($i - 1$) iteration. The loop repeats if any \out set has
changed since the last iteration. Otherwise, the algorithm terminates.

\begin{myfig}
  \begin{minipage}{3in}
    \begin{AVerb}[numbers=left,gobble=6]
      Out(\emph{Entry}) = \emph{initial}, \emph{initial} $\in \setL{Facts}$.
      $\text{In(B)}^0$ = $\bot$, for all blocks $B$.
      \textbf{do} \{
        In(B) = $\bigwedge_{P \in pred(B)} \text{Out}(P)$. \label{fig_back14_in}
        Out(B) = \emph{transfer}(In(B)).  \label{fig_back14_out}
      \} \textbf{until}($\text{Out(B)}^{i} == \text{Out(B)}^{i - 1}$, for all $B$)\label{fig_back14_loop}
    \end{AVerb}
  \end{minipage}
  \caption{The dataflow algorithm, using parameters for facts, the meet operator,
    direction, and the transfer function.}
  \label{fig_back14}
\end{myfig}

We have presented the iterative, forwards dataflow algorithm and shown
how the algorithm can be parameterized for a particular analysis. We
gave the parameterization for our constant propagation analysis in
Figure~\ref{fig_back10}. We know the algorithm will terminate if
our transfer function is \emph{monotone} and we have defined lattice
with \emph{finite} height. However, we have not discussed how to measure
the results our analysis gives us -- how do we know that they are the 
best possible? We will address that question in the next section.

\section{Quality of Solutions to the Dataflow Equations}

Aho \citep{Aho2006} shows that for a dataflow analysis defined with a
finite lattice and monotone transfer function, Figure~\ref{fig_back14}
will compute a \emph{maximum fixed point}. A maximum fixed point means
that for any $F = \outBa$ computed by the algorithm, no other $F'$ can
be computed such that $F \sqlt F'$. In other words,
Figure~\ref{fig_back14} will compute \out facts with the best
possible information that our algorithm is capable of.

The maximum fixed point solution differs from the \emph{ideal}
solution in that the maximum fixed point solution may make more
conservative estimates than necessary. In particular,
the algorithm does not consider branches that will never be taken. For
example, the C program from Figure~\ref{fig_back1_a} will never
execute Line~\ref{lst_back19_test_true}, because the test #if(a > b)#
is always false:
\begin{center}
\begin{minipage}{2in}\singlespacing\vskip-\baselineskip
\begin{AVerb}[numbers=left]
int a = 1, b = 2, c; \label{lst_back19_assign}
if(a > b) \label{lst_back19_test}
  c = 4; \label{lst_back19_test_true}
else     
  c = 3; \label{lst_back19_test_false}
\dots
\end{AVerb}
\end{minipage}
\end{center}

Our algorithm, however, does not take such conditions into
account. The ideal solution is called the \emph{meet over paths}
solution because it takes into account only the paths that will taken
by the program. Determining the actual paths taken is an undecidable
problem -- thus we settle for the maximum fixed point. Fortunately,
the algorithm is conservative --- it never ignores (or adds) paths ---
so we can be sure that its analysis will never be wrong, just that it
probably will not be as good as the ideal.

\section{Applying Results}

Figure~\ref{fig_back7}, Part~\subref{fig_back7_initial} gave a sample
program which we wished to optimize using a \emph{constant propagation}
dataflow analysis. Figure~\ref{fig_back7}, Part~\subref{fig_back7_opt} gave
the result, replacing all occurrences of #m# with 10. Now knowing
the dataflow algorithm and the equations for constant propagation, we
can derive how that transformation is made.

Table~\ref{fig_back12} gives the facts calculated for all nodes in our
program, during each iteration of the analysis. The first iteration
calculates that \outB{lst_back15_assign} assigns #m# the value 10, due
to the assignment #m = 10# on Line~\ref{fig_back7_m}. The second
iteration propagates this value to \inB{lst_back15_test} and in turn
to \outB{lst_back15_test}, because the test on
Line~\ref{fig_back7_test} does not affect #m#. In the third iteration,
we see the same with \inB{lst_back15_mult} and \outB{lst_back15_mult}
on Line~\ref{fig_back7_loop}. The analysis continues for two more
iterations as other values propagate, but at this point we have all
the information we need to optimize the program. Once the analysis
reaches a fixed point, we can safely replace all occurrences of #m#
with #10#, resulting in the optimized program given in
Figure~\ref{fig_back7}, Part~\subref{fig_back7_opt}.

\begin{myfig}
  \setlength{\tabcolsep}{2pt}
  \hbox to \textwidth{\hss
  \begin{tabular}{cc}
    \subfloat{\begin{minipage}{5in}
\begin{math}
  \setlength{\arraycolsep}{2pt}
  \begin{array}{l@{\phantom{X}}ccc@{\phantom{X}}ccc@{\phantom{X}}ccc@{\phantom{X}}ccc@{\phantom{X}}ccc@{\phantom{X}}ccc}
    \rlap{\hfil\phantom{X}\textit{Iteration:}} &
    & 0 & & 
    & 1 & & 
    & 2 & & 
    & 3 & &
    & 4 & &
    & 5 & \\ 

    & %%
    m & n & i & %% 0
    m & n & i & %% 1
    m & n & i & %% 2
    m & n & i & %% 3
    m & n & i & %% 4
    m & n & i \\ %% 5

    \inB{lst_back15_assign} & %%
    \bot & \bot & \bot & %% 0 
    \bot & \bot & \bot & %% 1 
    \bot & \bot & \bot & %% 2
    \bot & \bot & \bot & %% 3
    \bot & \bot & \bot & %% 4
    \bot & \bot & \bot \\ %% 5
    \outB{lst_back15_assign} & %%
    \bot & \bot & \bot & %% 0 
    10 & 0 & 0 & %% 1 
    10 & 0 & 0 & %% 2
    10 & 0 & 0 & %% 3
    10 & 0 & 0 & %% 4
    10 & 0 & 0 \\\\ %% 5

    \inB{lst_back15_test} & %%
    \bot & \bot & \bot & %% 0 
    \bot & \bot & \bot & %% 1 
    10 & 0 & \top & %% 2
    10 & \top & \top & %% 3
    10 & \top & \top & %% 4
    10 & \top & \top\\ %% 5
    \outB{lst_back15_test} & %%
    \bot & \bot & \bot & %% 0 
    \bot & \bot & \bot & %% 1 
    10 & 0 & \top & %% 2
    10 & \top & \top & %% 3
    10 & \top & \top & %% 4
    10 & \top & \top \\\\ %% 5

    \inB{lst_back15_mult} & %%
    \bot & \bot & \bot & %% 0 
    \bot & \bot & \bot & %% 1 
    \bot & \bot & \bot & %% 2
    10 & 0 & \top & %% 3
    10 & \top & \top & %% 4
    10 & \top & \top \\ %%5  
    \outB{lst_back15_mult} & %%
    \bot & \bot & \bot & %% 0 
    \bot & \top & \bot & %% 1 
    \bot & \top & \bot & %% 2
    10 & \top & \top & %% 3
    10 & \top & \top & %% 4
    10 & \top & \top \\\\ %% 5

    \inB{lst_back15_incr} & %%
    \bot & \bot & \bot & %% 0 
    \bot & \bot & \bot & %% 1 
    \bot & \top & \bot & %% 2
    \bot & \top & \bot & %% 3
    10 & \top & \top & %% 4
    10 & \top & \top \\ %% 5
    \outB{lst_back15_incr} & %%
    \bot & \bot & \bot & %% 0 
    \bot & \bot & \top & %% 1 
    \bot & \top & \top & %% 2
    \bot & \top & \top & %% 3
    10 & \top & \top & %% 4
    10 & \top & \top \\\\ %% 5

    \inB{lst_back15_return} & %%
    \bot & \bot & \bot & %% 0 
    \bot & \bot & \bot & %% 1 
    \bot & \bot & \bot & %% 2
    10 & 0 & \top & %% 3
    10 & \top & \top & %% 4
    10 & \top & \top \\ %% 5
    \outB{lst_back15_return} & %%
    \bot & \bot & \bot & %% 0 
    \bot & \bot & \bot & %% 1 
    \bot & \bot & \bot & %% 2
    10 & 0 & \top & %% 3
    10 & \top & \top & %% 4
    10 & \top & \top \\\\ %% 5

  \end{array}
\end{math}
\end{minipage}
\label{fig_back12_tbl}} & 
    \subfloat{\input{fig_back12_cfg}\label{fig_back12_cfg}} \\
    \subref{fig_back12_tbl} & \subref{fig_back12_cfg}
  \end{tabular}\hss}
  \caption{This figure shows the facts calculated for all nodes in our
    example program. Part~\subref{fig_back12_tbl} shows the \inE and
    \out facts associated with each node. Part~\subref{fig_back12_cfg}
    reproduces the control-flow graph for our program. After 5
    iterations the facts reach a fixed point (i.e., they stop
    changing) and we can see that \inB{lst_back15_mult} shows that $m$
    is always 10, proving we can rewrite the multiplication safely. }
  \label{fig_back12}
\end{myfig}

We have now seen how we can use constant propagation to optimize a
simple program. Typically many more optimizations will be run over the
same code, each (hopefully) improving it a little more. For example,
we could use an optimziation called \emph{dead-code elimination} to
remove the declaration of #m# altogether from our optimized program,
as it is no longer used. 

\section{Summary}
\label{sec_back9}

This chapter gave an overview of \emph{dataflow optimization}. The
dataflow \emph{algorithm} gives a general technique for applying an
\emph{optimizing function} to the \emph{control flow graph} (CFG)
representing a given program. The optimizing function computes
\emph{facts} about each node in the graph, using a \emph{transfer}
function. A given analysis can proceed \emph{forwards} (where \inBa
facts produce \outBa facts) or \emph{backwards} (where \outBa facts
produce \inBa facts). Each optimization defines a specific \emph{meet
  operator} that combines facts for nodes with multiple predecessors
(for forwards analysis) or successors (for backwards). We compute
facts\emph{iteratively}, stopping when they reach a \emph{fixed
  point}. Finally, we \emph{rewrite} the CFG using the facts computed. The 
meaning of our program does not change, but its behavior will be ``better,'' 
whatever that means for the particular optimization applied.



%% \runin{Transfer Function} Our transfer function creates 

%% \section{Facts, Transfer Functions, Direction \& The Meet Operator}
%% \label{sec_back4}

%% Begin by placing the specific concept in the overall context of
%% dataflow. Give a small example highlighting the concept. Point
%% out fine points or subtleties that occur when generalizing the concept. End
%% by summarizing how the concept fits into dataflow (in a bit larger
%% sense than the first summary).
%% \begin{myfig}[th]
%% \centering
%% \begin{tikzpicture}
  \node[entex] (entry) {};

  \node[stmt,
    below of=entry] (assign) {
    \begin{minipage}{.5in}
      \begin{AVerb}
a = 1
b = 1
      \end{AVerb}
    \end{minipage}\labelNode{lst_back7_assign}};
  \node[labelfor=assign] {\refNode{lst_back7_assign}};
  \node[fact, above=5mm of assign, anchor=west] {\inFacts{lst_back7_assign}{a/\top, b/\top, c/\top}};
  \node[fact, below=3mm of assign, anchor=west] {\outFacts{lst_back7_assign}{a/1, b/2, c/\top}};

  \node[stmt,
    below=.75in of assign] (test) {#if(rnd() + a > b)#\labelNode{lst_back7_test}};
  \node[labelfor=test] {\refNode{lst_back7_test}};
  \node[fact, above=3mm of test, anchor=east] {\inFacts{lst_back7_test}{a/1, b/2, c/\top}};
  \node[fact, below=3mm of test, anchor=east] {\outFacts{lst_back7_test}{a/1, b/2, c/\top}};
  \node[fact, below right=0mm and 0mm of test, anchor=west] {\outFacts{lst_back7_test}{a/1, b/2, c/\top}};
  
  \node[stmt,
    right=2in of test] (true) {#c = 4#\labelNode{lst_back7_true}};
  \node[labelfor=true] {\refNode{lst_back7_true}};
  \node[fact, above=3mm of true, anchor=east] {\inFacts{lst_back7_true}{a/1, b/2, c/\top}};
  \node[fact, below=3mm of true, anchor=west] {\outFacts{lst_back7_true}{a/1, b/2, c/4}};

  \node[stmt,
    below=.75in of test] (false) {#c = 3#\labelNode{lst_back7_false}};
  \node[labelfor=false] {\refNode{lst_back7_false}};
  \node[fact, above=5mm of false, anchor=west] {\inFacts{lst_back7_false}{a/1, b/2, c/\top}};
  \node[fact, below=3mm of false, anchor=west] {\outFacts{lst_back7_false}{a/1, b/2, c/3}};

  \node[stmt,
    below=.75in of false] (print) {#print(c)#\labelNode{lst_back7_print}};
  \node[labelfor=print] {\refNode{lst_back7_print}};
  \node[fact, above=3mm of print, anchor=east] {\inFacts{lst_back7_print}{a/1, b/2, c/\top}};
  \node[fact, below=3mm of print, anchor=east] {\outFacts{lst_back7_print}{a/1, b/2, c/\top}};

  \node[entex, below=.5in of print] (exit) {};

  \draw [->>] (entry) to (assign);
  \draw [->] (assign) to (test);
  \draw [->] (test) to (true);
  \draw [->] (test) to (false);
  \draw [->] (true) |- (print);
  \draw [->] (false) to (print);
  \draw [->>] (print) to (exit);

\end{tikzpicture}

%% \caption{The CFG for the C-language fragment from
%%   Figure~\ref{fig_back1_a}, annotated with \emph{facts} about the
%%   value of \texttt{a}, \texttt{b}, and \texttt{c} before (``\inBa'') and
%%   after (``\outBa'') each node.}
%% \label{fig_back5}
%% \end{myfig}

%% The dataflow algorithm computes two sets of \emph{facts} for every
%% node in the CFG. Facts are a data structure that describe the state of
%% the program before and after execution of the block represented by the
%% node. Figure~\ref{fig_back5} annotates the program fragment in
%% Figure~\ref{fig_back1} with facts about #a#, #b#, and #c# (the only
%% state we care about in this program). Each \inBa gives the variables'
%% values just prior to executing block $B$, while each \outBa gives
%% their values just after $B$ has executed.  

%% Figure~\ref{fig_back5} shows a \emph{forwards} analysis, where \outBa
%% is computed from \inBa, for each block. Facts are created by a
%% \emph{transfer function} that inspects the statements in each node and
%% updates values assigned to variables, if any. Sometimes a dataflow
%% analysis will run \emph{backwards}, computing \inBa from
%% \outBa. Section \ref{sec_back2} gives a detailed example illustrating
%% a \emph{backwards} analysis. In general, the transfer function and
%% direction vary depending on the particular analysis performed.

%% To help define our transfer function, we define |valueOf|,
%% which either returns the value assigned to a variable, or its value
%% from \inBa:
%% \begin{equation} |valueOf|(v) = 
%%   \begin{cases}
%%     |assign|(v) & \text{when $v$ is assigned a value in the node,} \\
%%     \text{\inBa}(v) & \text{when $v$ is not assigned.} 
%%   \end{cases}
%% \label{eqn_back2}
%% \end{equation}
%% In the above, $v$ represents a variable; |assign| retrieves the value
%% assigned to that variable, if any.  Our transfer function just needs
%% to apply |valueOf| to all variables in \inBa, as well as all
%% variable assignments in the node itself. If |assigned| is the set of
%% all assigned variables in the node, we can define how our transfer
%% function relates \inBa and \outBa using set notation:
%% \begin{equation}
%%   \text{\outBa} = [|valueOf|(v) || v \in (\text{\inBa} \cup |assigned|)].
%% \end{equation}

%% Our initial fact, \inB{lst_back7_assign}:~\facts{a/\top, b/\top,
%%   c/\top}, assigns the value ``$\top$'' (``top'') to all variables,
%% indicating that we do not know the value for the given variable. Our
%% transfer function determines that \outB{lst_back7_assign} should be
%% \facts{a/1, b/2, c/\top}, showing that we know #a# is 1, #b# is 2, and
%% that we still do not know the value of #c#. At each block we perform a
%% similar analysis, except \refNode{lst_back7_print}, where we need to
%%   take special action.

%% When a node has multiple predecessors, like \refNode{lst_back7_print},
%% we must combine multiple \outBa values into a single \inBa. The value
%% for #c# in \outB{lst_back7_true} is 4, while in \outB{lst_back7_false}
%% #c# is 3. We have two distinct values for #c# and no way to determine
%% which will be the case when \refNode{lst_back7_print} executes. We
%% must be conservative, so we assign the value $\top$ to #c# in
%% \inB{lst_back7_print}.

%% \begin{table}[tbh]
%%   \centering
%%   \figbegin
%%   \begin{math}
%%     \begin{array}{ccccc}
%%       & v_1 & v_2 & v_1 \lub v_2 \\
%%       \cmidrule(r){2-2}\cmidrule(r){3-3}\cmidrule(r){4-4}
%%       1. & \top & v_2 & \top & \\ 
%%       2. & v_1 & \top & \top & \\
%%       3. & v_1 & v_2 & \top & \text{($v_1 \neq v_2$)}\\
%%       4. & v_1 & v_1 & v_1 
%%     \end{array}
%%   \end{math}
%%   \caption{How the meet operator used in Figure \ref{fig_back5}
%%     combines facts. $v_1$ and $v_2$ are values given by separate
%%     \outBa facts to the same variable. The table shows how they are
%%     combined.}
%%   \label{tbl_back2}
%%   \figend
%% \end{table}

%% A \emph{meet operator} defines how we combine facts when values
%% conflict. Table~\ref{tbl_back2} defines ``\lub'' (``least upper
%% bound'' or ``lub''), which combines values as we did for
%% \outB{lst_back7_true} and \outB{lst_back7_false}. $v_1$~and $v_2$
%% represent values given to the same variable by different
%% facts. Lines~1 and 2 show that when either value is $\top$, the result
%% is $\top$. When values differ, as in Line~3, again the result is
%% $\top$. Only when values are equal, as shown in the last line, do we
%% preserve the value.

%% Facts define the state of the program that we are analyzing. The
%% transfer function transforms input facts into output facts. In a
%% forwards analysis, input facts come from predecessor nodes and output
%% facts flow to successors. For a backwards analysis, the opposite
%% occurs. When multiple facts need to be combined, we use a meet
%% operator. Each of these elements will vary depending on the specific
%% analysis performed.

%% \section{Iterative Analysis}
%% \label{sec_back6}
%% Begin by placing the specific concept in the overall context of
%% dataflow. Give a small example highlighting the concept. Point
%% out fine points or subtleties that occur when generalizing the concept. End
%% by summarizing how the concept fits into dataflow (in a bit larger
%% sense than the first summary).

%% \begin{equation}
%%   \begin{split}
%%     B \bigwedge\ \emptyset\ &= B \\
%%     B \bigwedge\ C &= [\{a=v_b\} \wedge\ \{a=v_c\} || \{a=v_b\}\ \in\ B, \{a=v_c\}\ \in\ C] \\%%
%%                    &\; \cup\ [\{b=v_b\} || \{b=v_b\}\ \in\ B, \{b=v_c\}\ \not\in\ C] \\%%
%%                    &\; \cup\ [\{c=v_c\} || \{c=v_b\}\ \not\in\ B, \{c=v_c\}\ \in\ C] \\
%%     \{a=\bot\} \wedge\ \{a=v\} &= \{a=v\} \\
%%     \{a=\top\} \wedge\ \{a=v\} &= \{a=\top\} \\
%%     \{a=v\} \wedge\ \{a=u\} &= \{a=\top\} (u \neq v) \\
%%     \{a=v\} \wedge\ \{a=v\} &= \{a=v\} \\
%%   \end{split}
%% \end{equation}

%% \begin{equation}
%%   \begin{split}
%%     f_B(In) &= [\mathit{assign}(v) || v \in\ In], \text{where $B$ is a block in the CFG} \\
%%     assign(v) &= %%
%%     \begin{cases}
%%       c & \text{when $v$ assigned $c$ in B.} \\
%%       In(v) & \text{when v not assigned in B.}
%%     \end{split}
%%   \end{align}
%% \end{equation}

%% \begin{tabular}{ll}
%%   \textbf{Lattice} & $\bot$, 0, 1, \ldots, and $\top$. \\
%%   \textbf{Meet} &  As above. \\
%%   \textbf{Transfer} & As above. \\
%%   \textbf{Direction} & Forward.
%% \end{tabular}


%% As we saw in Figure \ref{fig_back5}, facts can conflict when nodes
%% have multiple predecessors. Even more complicated situations arise
%% when a program contains loops. Consider the fragment in
%% \ref{fig_back6}. To compute \inB{lst_back9_test}, we need
%% \outB{lst_back9_assign} and and \outB{lst_back9_incr}. To compute
%% \inB{lst_back9_incr} (in order to find \outB{lst_back9_incr}, we need
%% \outB{lst_back9_test}. But to compute \outB{lst_back9_test} we need
%% \inB{lst_back9_test}.  How do we apply our |valueOf| function
%% (Equation \ref{eqn_back2}) to a \refNode{lst_back9_test} when
%% \inB{lst_back9_test} depends on \outB{lst_back9_test}?

%% \begin{myfig}
%% \begin{tabular}{cc}
%%   \subfloat{\begin{minipage}[t]{2in}
\begin{AVerb}[numbers=left]
int c = 0;
while(c < 10)
  c += 1;
print(c);
\end{AVerb}
\end{minipage}
%%
%%     \label{fig_back6_a}} \vline &%%
%%   \subfloat{\begin{minipage}[t]{2in}
\begin{AVerb}

\end{AVerb}
\end{minipage}
%%
%%     \label{fig_back6_b}} \\ 
%%   \subref{fig_back1_a} & \subref{fig_back1_b}
%% \end{tabular}
%% \caption{\subref{fig_back6_a}: A simple C-language program with a loop. \subref{fig_back6_b}: The CFG 
%% for the fragment.}
%% \label{fig_back6}
%% \end{myfig}

%% We solve this problem by applying our transfer function
%% \emph{iteratively}. In the case of Figure \ref{fig_back6}, we first
%% initialize each all \inBa and \outBa facts to some default. We then use
%% |valueOf| to compute each \outBa. Of course, facts will change over
%% the course of iteration -- especially in the case of node
%% \ref{lst_back9_test}. We keep iterating until we reach a \emph{fixed
%%   point}, meaning the facts stop changing.

%% \begin{table}
%%   \centering
%%   \begin{math}
%%     \begin{array}{lcccc}
%%       \mathit{Iteration} & \outB{lst_back9_assign} & \outB{lst_back9_incr} & \inB{lst_back9_test} & \outB{lst_back9_test} \\
%%       \cmidrule(r){1-1}\cmidrule(r){2-5} 
%%       0 & \bot & \bot & \bot & \bot  \\
%%       1 & 0 & 10 & \bot & \bot \\
%%       2 & 0 & 10 & \bot & \bot \\
%%       \multicolumn{5}{c}{\dots} \\
%%       \multicolumn{5}{l}{\inB{lst_back9_test} = \outB{lst_back9_assign} \lub \outB{lst_back9_incr}} \\
%%     \end{array}
%%   \end{math}
%%   \caption{Iterative analysis of the CFG from Figure
%%     \ref{fig_back6}. We how the inputs used to calculate
%%     \outB{lst_back9_test} change in one iteration. The zeroth
%%     iteration represents the initial values given to \inBa and \outBa
%%     for all nodes.}  
%%   \figend
%%   \label{tbl_back3}
%% \end{table}

%% Table \ref{tbl_back3} shows \inE and \out for
%% \refNode{lst_back9_test}. To compute \inB{lst_back9_test}, we combine
%% \outB{lst_back9_assign} and \outB{list_back9_incr} using the meet
%% operator from Section~\ref{sec_back4}:
%% \begin{equation}
%%   \inB{lst_back9_test} = \outB{lst_back9_assign} \lub \outB{lst_back9_incr}.
%% \end{equation}
%% The zeroth iteration shows the initial
%% value for all sets. On the first iteration, we can see \inB{lst_back9_test} is $\bot$:
%% \begin{equation}
%%   \begin{split}
%%     \inB{lst_back9_test} &= \outB{lst_back9_assign} \lub \outB{lst_back9_incr} \\
%%     &= \bot \lub \bot \\
%%     &= \bot.
%%   \end{split}
%% \end{equation}
%% When computing \inBa, we always use \outBa from the
%% \emph{previous} iteration. In the above we use $\bot$ for \outB{lst_back9_incr} and
%% \outB{lst_back9_assign}. 

%% When computing \inB{lst_back9_test} in the second iteration,
%% \outB{lst_back9_incr} is 10 and \outB{lst_back9_assign} is
%% 0. According to our meet operator, \inB{lst_back9_test} still equals
%% $\bot$:
%% \begin{equation}
%%   \begin{split}
%%     \inB{lst_back9_test} &= \outB{lst_back9_assign} \lub \outB{lst_back9_incr} \\
%%     &= 0 \lub 10 \\
%%     &= \bot.
%%   \end{split}
%% \end{equation}
%% At this point, our facts have stopped changing so we stop
%% iterating. Our final result $\bot$ for #c# in \outB{lst_back9_test}.

%% \section{Rewriting}
%% \label{sec_back7}

%% Begin by placing the specific concept in the overall context of
%% dataflow. Give a small example highlighting the concept. Point
%% out fine points or subtleties that occur when generalizing the concept. End
%% by summarizing how the concept fits into dataflow (in a bit larger
%% sense than the first summary).

%% Direction, the meet operator, facts, and the transfer function
%% together define a particular dataflow analysis. We can use the result
%% of the analysis to alter, or ``rewrite,'' the CFG of the program. The
%% meaning of the program should not change, but it should behave
%% differently: execute faster, use less memory, or whatever
%% characteristic the optimization should improve.  We do not have to
%% rewrite, of course. In some cases, we use the analysis to drive later
%% optimizations, or to report errors to the programmer. For example, a
%% \emph{reaching definitions} \citep{AhoXX} analysis can warn if
%% variables are used without being initialized. However, in most cases
%% we do want to rewrite the CFG.

%% \section{Example: Dead-Code Elimination}
%% \label{sec_back2}

%% Begin by placing the specific concept in the overall context of
%% dataflow. Give a small example highlighting the concept. Point
%% out fine points or subtleties that occur when generalizing the concept. End
%% by summarizing how the concept fits into dataflow (in a bit larger
%% sense than the first summary).

%% Consider Figure \ref{fig_back2}, again showing a C-language fragment.
%% The assignment to #b# on line~\ref{fig_back2_dead_line} has no visible
%% effect and can be removed without affecting the meaning of the
%% program. We call this optimization \emph{dead-code elimination}.

%% \begin{myfig}[ht]
%% \begin{minipage}{1in}
%%   \begin{AVerb}[numbers=left]
%%     a = 1;
%%     b = a + 1;\label{fig_back2_dead_line}
%%     return a + 1;
%%   \end{AVerb}
%% \end{minipage}
%% \caption{A C-language fragment illustrating \emph{dead code}. After
%% assignment on line \ref{fig_back2_dead_line}, \verb=b= is not used
%% and can be considered ``dead.''}
%% \label{fig_back2}
%% \end{myfig}

%% Of course, programmers do not normally write programs with such
%% obviously useless statements, but other compiler optimizations can
%% leave behind many such statements. \emph{Uncurrying}, described in
%% Chapter~\ref{ref_chapter_uncurrying}, in fact depends on dead-code
%% elimination.

%% The assignment on line~\ref{fig_back2_dead_line} can be eliminated
%% because #b# does not get referenced again. If a variable is referenced
%% after assignment, we say it is ``live.'' We call the dataflow
%% analysis that finds all live variables a ``liveness'' analysis. 

%% \begin{myfig}[th]
%% \begin{minipage}{2in}
%% %% \begin{AVerb}[commandchars=\\\{\}]
%%        E
%%        ||      
%%        v
%%      -----
%%     ||a = 1||    \emph{live:}  \ensuremath{\emptyset}
%%      -----
%%        ||      
%%        V
%%    ---------
%%   ||b = a + 1||  \emph{live:} \{a\}  
%%    ---------
%%        ||      
%%        V
%%   ------------
%%  ||return a + 1|| \emph{live:} \{a\}
%%   ------------
%%        ||      
%%        X          \emph{live:}  \ensuremath{\emptyset}
%% \end{AVerb}
\begin{tikzpicture}
  \node[entex] (entry) {};

  \node[stmt, below of=entry] (assigna) {#a = 1#\labelNode{lst_back10_assigna}};
  \node[labelfor=entry] () {\refNode{lst_back10_assigna}};

  \node[stmt, below of=assigna] (test) {#if(rnd() > 1)#\labelNode{lst_back10_test}};
  \node[labelfor=test] () {\refNode{lst_back10_test}};

  \node[stmt, right of=test] (true) {#b = a + 1#\labelNode{lst_back10_true}};
  \node[labelfor=true] () {\refNode{lst_back10_lst_back10_true}};

  \node[stmt, below of=test] (false) {#b = a - 1#\labelNode{lst_back10_false}};
  \node[labelfor=false] () {\refNode{lst_back10_lst_back10_false}};

  \node[stmt, below of=false] (return) {#return a + 1#\labelNode{lst_back10_return}};
  \node[labelfor=return] () {\refNode{lst_back10_lst_back10_return}};

  \node[entex, below of=return] (exit) {};

  \draw [->>] (entry) to (assigna);
  \draw [->] (assigna) to (test);
  \draw [->] (test) to (true);
  \draw [->] (test) to (false);
  \draw [->] (true) |- (return);
  \draw [->] (false) to (return);
  \draw [->>] (return) to (exit);

\end{tikzpicture}

%% \end{minipage}
%% \caption{The CFG for our example program, annotated with the live
%% set for each node.}
%% \label{fig_back3}
%% \end{myfig}

%% Figure \ref{fig_back3} shows the CFG for this example. The annotations
%% on edges show the facts computed for this analysis: the set of live
%% variables. Recall from Section~\ref{sec_back4} that we can choose to
%% traverse the CFG forwards or backwards. Going forwards would require
%% us to track each assignment and then, after traversing the CFG,
%% determine if the variable was referenced again. The backwards case
%% requires that we add each variable reference to the live set. When an
%% assignment occurs, we can check if the variable appears in the live
%% set. If not, we know the variable was never referenced and is not
%% live. For simplicity and efficiency, we choose to go backwards rather
%% than forwards.

%% We define our transfer function such that, for some statement $B$,
%% \inBa represents the variables that are live in the statement. A live
%% variable is referenced (i.e., \emph{used}) in $B$ or one
%% of its successors. A variable that appears in \outBa but not \inBa must
%% not be live. The only way to remove a variable from \outBa is if it is 
%% assigned (i.e. or \emph{defined}) in $B$. We can then define our transfer
%% function from \outBa to \inBa in terms of the \emph{use} and \emph{def} sets:
%% \begin{align}
%%   & \inBa = (\outBa \cup |use|(B)) - |def|(B), \label{eqn_back1} \\
%% \text{where} \notag\\
%%   & B     & \text{Statement considered.} \notag\\
%%   & |use|(B) & \text{Variables referenced in $B$}. \notag\\
%%   & |def|(B) & \text{Variables assigned $B$}. \notag\\
%% \end{align}

%% Table \ref{tbl_back1} shows the |use|, |def|, \inBa and \outBa sets for
%% each statement. We include the exit node (``\exitN'') in the table to
%% show the initial value of \outBa for the last statement -- $\emptyset$,
%% the empty set. Our analysis then works backwards through the
%% program. Each \inBa becomes the input, \outBa, for the statement's
%% predecessor. If our program (and its CFG) contained any loops, we
%% would need to run this algorithm multiple times, until the live set
%% for each statement reached a fixed point.

%% \begin{table}
%%   \centering
%%   \begin{math}
%%     \begin{array}{lcccc}
%%       & |use|(B) & |def|(B) & \outBa & \inBa \\
%%       \cmidrule(r){2-5} %%\cmidrule(r){1-1}\cmidrule(r){2-2}\cmidrule(r){3-3}\cmidrule(r){4-4}\cmidrule(r){5-5}
%%       \exitN & \emptyset & \emptyset & \emptyset & \emptyset \\
%%       #return a + 1# & \{a\} & \emptyset & \emptyset & \{a\} \\
%%       #b = a + 1# & \{a\} & \{b\} & \{a\} & \{a\} \\
%%       #a = 1# & \emptyset & \{a\} & \{a\} & \emptyset \\
%%     \end{array}
%%   \end{math}
%%   \caption{The $|use|$, $|def|$, \inBa and \outBa sets computed using
%%     equation \ref{eqn_back1} for our example program.}
%%   \label{tbl_back1}
%%   \figend
%% \end{table}

%% With the live set computed for each statement, our analysis can now
%% determine which statements to eliminate. Only
%% \refNode{lst_back10_assigna} and \refNode{lst_back10_assignb} in
%% Figure~\ref{fig_back3} perform an assignment. The live set for
%% \refNode{lst_back10_assigna} (``#a = 1#'') contains #a#, so we do not
%% eliminate it. For \refNode{lst_back10_assignb} (``#b = a + 1#''), the
%% live set does \emph{not} contain #b#. Therefore, we can eliminate
%% \refNode{lst_back10_assignb}, giving us the new program in
%% Figure~\ref{fig_back6}.

%% \begin{myfig}[th]
%%   \centering
%%   \begin{minipage}{1in}
%%   \begin{AVerb}[numbers=left]
%% a = 1;
%% return a + 1;
%%   \end{AVerb}
%%   \end{minipage}
%%   \caption{The program from Figure~\ref{fig_back3} with the useless assignment to
%%     \verb=b= eliminated.}
%% \end{myfig}

%% In the
%% forwards case, we must track each assignment and determine, when we
%% exit the CFG, if the variable was ever used. We would need to track
%% every assignment until our traversal finished. However, if we traverse
%% backwards, we only need to add each reference to our live set. When we
%% see an assignment to a variable \emph{not} in our live set, we know it
%% has never been referenced.

%% \emph{live set} The \inE and \out sets show the facts computed for this
%% analysis. The computed show the live variables for that program point
%% \emph{live set}, annotates edges between each statement. The live set
%% is the \emph{fact} we compute for this analysis.

%% Annotations
%% show the facts we will compute
%% Recall from Section~\ref{sec_back4} that a dataflow analysis can
%% go \emph{forwards} or \emph{backwards}. 

%% To eliminate the assignment like that on
%% line~\ref{fig_back2_dead_line}, we need to determine which variables
%% are ``live'' -- that is, variables referenced after assignment. Such variables are ``live''; if a
%% variable is \emph{not} live, then it is dead. We use this ``liveness''
%% analysis to determine if a particular assignment is dead.

%% To determine if a variable is live, we need to know if it is
%% referenced after assignment. Such variables make up a \emph{live set}
%% that we can compute between each statement. To compute the live set,
%% we can choose to traverse the CFG for the program forwards or
%% backwards.  In the forwards case, we must track each assignment and
%% determine, when we exit the CFG, if the variable was ever used. We
%% would need to track every assignment until our traversal
%% finished. However, if we traverse backwards, we only need to add each
%% reference to our live set. When we see an assignment to a variable
%% \emph{not} in our live set, we know it has never been
%% referenced. Therefore we compute ``liveness'' using a backwards
%% traversal over the CFG.

%% \begin{myfig}[th]
%% \begin{minipage}{2in}
%% %% \begin{AVerb}[commandchars=\\\{\}]
%%        E
%%        ||      
%%        v
%%      -----
%%     ||a = 1||    \emph{live:}  \ensuremath{\emptyset}
%%      -----
%%        ||      
%%        V
%%    ---------
%%   ||b = a + 1||  \emph{live:} \{a\}  
%%    ---------
%%        ||      
%%        V
%%   ------------
%%  ||return a + 1|| \emph{live:} \{a\}
%%   ------------
%%        ||      
%%        X          \emph{live:}  \ensuremath{\emptyset}
%% \end{AVerb}
\begin{tikzpicture}
  \node[entex] (entry) {};

  \node[stmt, below of=entry] (assigna) {#a = 1#\labelNode{lst_back10_assigna}};
  \node[labelfor=entry] () {\refNode{lst_back10_assigna}};

  \node[stmt, below of=assigna] (test) {#if(rnd() > 1)#\labelNode{lst_back10_test}};
  \node[labelfor=test] () {\refNode{lst_back10_test}};

  \node[stmt, right of=test] (true) {#b = a + 1#\labelNode{lst_back10_true}};
  \node[labelfor=true] () {\refNode{lst_back10_lst_back10_true}};

  \node[stmt, below of=test] (false) {#b = a - 1#\labelNode{lst_back10_false}};
  \node[labelfor=false] () {\refNode{lst_back10_lst_back10_false}};

  \node[stmt, below of=false] (return) {#return a + 1#\labelNode{lst_back10_return}};
  \node[labelfor=return] () {\refNode{lst_back10_lst_back10_return}};

  \node[entex, below of=return] (exit) {};

  \draw [->>] (entry) to (assigna);
  \draw [->] (assigna) to (test);
  \draw [->] (test) to (true);
  \draw [->] (test) to (false);
  \draw [->] (true) |- (return);
  \draw [->] (false) to (return);
  \draw [->>] (return) to (exit);

\end{tikzpicture}

%% \end{minipage}
%% \caption{The CFG for our example program, annotated with the live
%% set for each node.}
%% \label{fig_back3}
%% \end{myfig}

%% Figure \ref{fig_back3} shows the CFG for this example. Annotations
%% show the facts we will compute: the live set before and after. Though
%% execution follows the arrows in the CFG, our analysis proceeds
%% backwards. For example, the input to node 2 is the live set computed
%% for node 3 (``$\{a\}$'' in this case).

%% Our transfer function computes the live set based on \emph{uses} and
%% \emph{definitions} in a statement. Any reference (or use) of a
%% variable goes into the live set. Any assignment (or definition) of a
%% variable removes it from the live set. We can then define our transfer
%% function, |live|, for a statement as:

%% \begin{align}
%%   & |live|(s) = (\Varid{in}(s) \cup |use|(s)) - |def|(s), \label{eqn_back1} \\
%% \intertext{where}
%%   & s     & \text{Statement considered.} \notag\\
%%   & |use|(s) &  \text{Set of variables used in $s$}. \notag\\
%%   & |def|(s) & \text{Variable assigned to in $s$ (a singleton set)}. \notag\\
%%   & \Varid{in}(s) & \text{Live variables computed for $s$' successor}. \notag
%% \end{align}

%% Table \ref{tbl_back1} shows the |use| and |def| sets for each
%% statement. The live set computed, |live|, becomes the input, $\Varid{in}$, for
%% the statement's predecessor. We include the exit node (``#X#'') in the
%% table to show the initial value of $\Varid{in}$ for the last statement --
%% $\emptyset$, the empty set. Our analysis then works backwards through the
%% program. If our program (and its CFG) contained any loops, we would
%% need to run this algorithm multiple times, until the live set for each
%% statement reached a fixed point.

%% \begin{table}
%%   \centering
%%   \begin{tabular}{lcccc}
%%     $s$ & $|use|(s)$ & $|def|(s)$ & $\Varid{in}(s)$ &  $|live|(s)$ \\
%%     \cmidrule(r){1-1}\cmidrule(r){2-2}\cmidrule(r){3-3}\cmidrule(r){4-4}\cmidrule(r){5-5}
%%     #X# & & & & $\emptyset$ \\
%%     #return a + 1# & $\{a\}$ & $\emptyset$ & $\emptyset$ & $\{a\}$ \\
%%     #b = a + 1# & $\{a\}$ & $\{b\}$ & $\{a\}$ & $\{a\}$ \\
%%     #a = 1# & $\emptyset$ & $\{a\}$ & $\{a\}$ & $\emptyset$ \\
%%     \bottomrule
%%   \end{tabular}
%%   \caption{The $|use|$, $|def|$ and $|live|$ sets computed using equation \ref{eqn_back1} for our example program.}
%%   \label{tbl_back1}
%% \end{table}

%% With the live set computed for each statement, our analysis can now
%% determine which statements to eliminate. Only nodes 1 and 2 in Figure
%% \ref{fig_back3} perform an assignment. The live set for node 1 (``#a = 1#'')
%% contains #a#, so we do not eliminate it. In node 2 (``#b = a + 1#''),
%% the live set does \emph{not} contain #b#. Therefore, we can eliminate
%% node 2, giving us a new program without any dead code:

%% \begin{Verbatim}
%% a = 1;
%% return a + 1;
%% \end{Verbatim}

%% \subsection{Basic Blocks and Control-Flow Graphs}

%% A dataflow optimization operates over a ``control-flow graph'' (CFG)
%% of the program -- a directed graph where edges encode branches or
%% jumps and nodes represent statements. Programs run by entering a node
%% from a predecessor, executing the statements in turn, and exiting the
%% node to a successor. Multiple successors imply a conditional branch,
%% though the program can only choose one. A special ``entry'' node, with
%% no predecssors, exists to give the program a starting point.

%% The statements in each node must define a ``basic block,'' which means
%% there can only be one entry and one exit to the node. Each
%% predeccessor starts at the same statement; execution cannot start in
%% the ``middle'' of the statements in the node. Each successor also
%% leaves from the same instruction, so only one ``branch'' can exist in
%% each node.

%% For example, consider the ``fall-through'' implied by the use of #case#
%% statements in this C-language program fragment:

%% \begin{verbatim}
%%   switch(i) {
%%   case 1:
%%     printf("1");
%%     break;
%%   case 2:
%%     printf("2");
%%   case 3:
%%     printf("3");
%%   }
%% \end{verbatim}

%% \begin{figure}[h]
%% \begin{verbatim}
%%    A
%%   switch   ----<-
%%   | |  |  |      |
%%   | |  |  v C    ^
%%   | |   ->case 3 |
%%   | |     |      |
%%   | |      ->----_--
%%   | | B          |  |
%%   |  ->case 2 ->-   v
%%   |                 |
%%   |   D       ----<-
%%    ->case 1  |
%%      |       v
%%      v       |
%%    --+-----<-
%%   |
%%    -> ...
%% \end{verbatim}
%% \caption{CFG illustrating \emph{fall-through} allowed by the
%%   C-language \texttt{switch} statement.}
%% \label{switchCfgEg}
%% \end{figure}

%% Figure \ref{switchCfgEg} shows a CFG for this fragment. Execution
%% begins at node A. Node C has two predeccessors: A and B. The edge
%% between Node B and C represents fall-through from the second to third
%% case. They cannot be combined because the node would need two distinct
%% entry points. Encoding a program into basic blocks usually involves
%% inserting similar branches. The CFG makes explicit control--flow that
%% exists by implication in the source program.

%% \subsection{Direction, Facts and Rewrites}

%% \subsection{Example: Bind/Return Collapse}

%% Dataflow optimizations transform the CFG representation of a program,
%% with the goal of making a faster (or smaller, or more efficient, etc.)
%% program. Dataflow computes a set of ``entry'' assumptions and ``exit''
%% facts for each node in the graph. Facts for one node become
%% assumptions for the nodes' successors (thus the term
%% ``dataflow''). The algorithm iteratves over the entire graph until a
%% fixed point is reached -- that is, facts and assumptions no longer
%% change. The computed facts can then be used to transform the graph.

%% \emph{Constant propagation example -- or something more functional?}

%% \emph{Introduce forward and backwards dataflow.}

% What does dataflow mean?

% How do you use it?

% Example

\ifthenelse{\boolean{standaloneFlag}}
           {\bibliography{dataflow}}{}

\end{document}

% LocalWords:  Dataflow dataflow CFG printf variable's CFGs ccc Uncurrying lst
% LocalWords:  liveness Kildall AhoXX assigna assignb runtime valueOf ccccc lub
% LocalWords:  incr lcccc


\documentclass[12pt]{report}
%include polycode.fmt
%include lineno\lineno.fmt
\usepackage[T1]{fontenc}
\usepackage{calc}
%% \usepackage{fourier}
\usepackage{palatino}
\renewcommand\ttdefault{lmtt}
\usepackage{helvet}
%% \usepackage{inconsolata}
\usepackage{comment}
\usepackage{calc}
\usepackage{xspace}
\usepackage{verbatim}
\usepackage{url}
\usepackage{fancyvrb}
\usepackage{setspace}
\usepackage{amsmath}
\usepackage{booktabs}
\usepackage[margin=\parindent, format=hang,labelfont=bf]{caption}
%% \usepackage[subrefformat=parens]{subcaption}
%% The following makes sure we get parentheses around
%% subreferences. The newest version of the subcaption
%% package has an option for this, but that's not available
%% widely.
%%
%% From http://tex.stackexchange.com/questions/25644
\usepackage[labelformat=simple]{subcaption}
\makeatletter
  \def\thesubfigure{(\alph{subfigure})}
  \providecommand\thefigsubsep{~}
  \def\p@subfigure{\@nameuse{thefigure}\thefigsubsep}
\makeatother

\usepackage{ifthen}
\usepackage{stmaryrd}
\usepackage{longtable}
\usepackage{afterpage}
\usepackage{xifthen}
\usepackage{mathtools}
\usepackage{xparse}
\usepackage[natbib=true,style=authoryear,backend=bibtex8]{biblatex}
\setlength{\bibitemsep}{\bigskipamount}
\addbibresource{thesis.bib}
\usepackage{microtype}

\usepackage{tikz}
\usetikzlibrary{arrows,automata,positioning,calc}
%% Used for CFGs.
\tikzset{
  >=stealth, 
  node distance=.5in,
  stmt/.style={rectangle,
    draw=black, thick,        
    minimum height=2em,
    %% inner sep=2pt,
    %% text centered,
    %% node distance=.5in,
  },
  entex/.style={
    minimum height=2em,
    %% inner sep=2pt,
    %% text centered,
  },
  labelfor/.style={circle, 
    draw=black, thin,
    font={\footnotesize},
    inner sep=0,
    fill=white,
    above right=-1.5mm and -1.5mm of #1,
  },
  fact/.style={overlay},
  %% Invisible node
  invis/.style={inner sep=0pt, 
    minimum height=0em}, 
  table/.style={circle, fill=white,height=2mm}
}

%% GSO margins.
\usepackage[left=1.5in, right=1in, top=1in, bottom=1in]{geometry}
\usepackage{abstract}

%% GSO requires 12 pt font for all headings
\usepackage[bf,sf,tiny,compact]{titlesec}
\titleformat{\chapter}[display]
            {}% format
            {\sffamily\bfseries\chaptertitlename\ \thechapter}
            {\baselineskip}
            {\sffamily\bfseries}
            {}

\hyphenation{data-flow mo-na-dic} 

\newboolean{lhs2tex}
\setboolean{lhs2tex}{true}

% Used by included files to know they
% are NOT standalone
\newboolean{standaloneFlag}
\setboolean{standaloneFlag}{true}

\newlength{\rulefigmargin}
\setlength{\rulefigmargin}{2\parindent}

\newcommand\figbegin{\rule{\linewidth-\rulefigmargin}{0.4pt}\\\vspace{12pt}}
\newcommand\figend{\rule{\linewidth-\rulefigmargin}{0.4pt}}

%\providecommand{\citep}[1]{(\emph{#1})\xspace}
%\renewcommand{\cite}[1]{\emph{#1}\xspace}

%% Functional languages chapter commands
\newcommand{\lamA}{\ensuremath{\lambda}-calculus\xspace}
\newcommand{\LamA}{\ensuremath{\lambda}-Calculus\xspace}
\newcommand{\lamAbs}[2]{\ensuremath{\lambda#1.\ #2}}
\newcommand{\lamApp}[2]{\ensuremath{#1\ #2}}
\newcommand{\lamPApp}[2]{\ensuremath{(#1\ #2)}}
\newcommand{\lamAPp}[2]{\ensuremath{(#1)\ #2}}
\newcommand{\lamApP}[2]{\ensuremath{#1\ (#2)}}
\newcommand{\lamAPP}[2]{\ensuremath{(#1)\ (#2)}}
\let\lamApPp=\lamApP
\let\lamAppP=\lamAPp

\newcommand{\lamId}{\lamAbs{x}{x}}
\newcommand{\lamCompose}{\lamAbs{f}{\lamAbs{g}{\lamAbs{x}{\lamApp{f}{(\lamApp{g}{x})}}}}}
\newcommand{\machLam}{\ensuremath{M_\lambda}\xspace}
\newcommand{\compMach}[1]{\ensuremath{\left\llbracket #1 \right\rrbracket}}
\newcommand{\compRho}[1]{\ensuremath{\rho(#1)}}
\newcommand{\verSub}[2]{\ensuremath{#1_{#2}}}
\newcommand{\verSup}[2]{\ensuremath{#1^{#2}}}
\newcommand{\lamC}{\ensuremath{\lambda_C}\xspace}
\newcommand{\lamPlus}{\lamAbs{m}{\lamAbs{n}{\lamAbs{s}{\lamAbs{z}{\lamApp{m}{\lamApPp{s}{\lamApp{n}{\lamApp{s}{z}}}}}}}}}
%% Substitution notation -- [#1 -> #2]
\newcommand{\lamSubst}[2]{\ensuremath{[#1 \mapsto #2]}}
%% End functional languages chapter

%% Dataflow chapter commands
\newcounter{nodeCounter}[figure]
\newcommand{\inE}{\ensuremath{\mathit{in}}\xspace}
\newcommand{\out}{\ensuremath{\mathit{out}}\xspace}
\newcommand{\In}{\ensuremath{\mathit{In}}\xspace}
\newcommand{\InBa}{\ensuremath{\mathit{In}(B)}\xspace}
\newcommand{\Out}{\ensuremath{\mathit{Out}}\xspace}
%% Out(B_x) -- fact function for an named block.
\newcommand{\OutB}[1]{\ensuremath{\mathit{Out}(B_{\ref{#1}})}\xspace}
\newcommand{\OutBa}{\ensuremath{\mathit{Out}(B)}\xspace}
%% in(B) -- fact function for an anonymous block.
\newcommand{\inBa}{\ensuremath{\mathit{in}(B)}\xspace}
%% in(X) -- fact function for an anonymous block, but using a different variable.
\newcommand{\inXa}[1]{\ensuremath{\mathit{in}(#1)}\xspace}
%% in(B,v) -- fact function for an anonymous block and some variable.
\newcommand{\inBav}[1]{\ensuremath{\mathit{in}(B, #1)}\xspace}
%% in(B_x) -- fact function for an named block.
\newcommand{\inB}[1]{\ensuremath{\mathit{in}(B_{\ref{#1}})}\xspace}
%% in(B_x,v) -- fact function for an named block and some variable.
\newcommand{\inBv}[2]{\ensuremath{\mathit{in}(B_{\ref{#1}}, #2)}\xspace}
%% out(B) -- fact function for an anonymous block.
\newcommand{\outBa}{\ensuremath{\mathit{out}(B)}\xspace}
%% out(X) -- fact function for an anonymous block, but using a different variable.
\newcommand{\outXa}[1]{\ensuremath{\mathit{out}(#1)}\xspace}
%% out(B,v) -- fact function for an anonymous block and some variable.
\newcommand{\outBav}[1]{\ensuremath{\mathit{out}(B, #1)}\xspace}
%% out(B_x) -- fact function for an named block.
\newcommand{\outB}[1]{\ensuremath{\mathit{out}(B_{\ref{#1}})}\xspace}
%% out(B_x,v) -- fact function for an named block and some variable.
\newcommand{\outBv}[2]{\ensuremath{\mathit{out}(B_{\ref{#1}}, #2)}\xspace}
\newcommand{\entryN}{\emph{E}\xspace}
\newcommand{\exitN}{\emph{X}\xspace}
\newcommand{\refNode}[1]{\ensuremath{B_{\ref{#1}}}\xspace}
\newcommand{\labelNode}[1]{\refstepcounter{nodeCounter}\label{#1}}
\newcommand{\setL}[1]{\textsc{#1}\xspace}
\newcommand{\setLC}{\setL{Const}}

%% Formats a list of facts
%% Argument should be like \facts{a/1, b/2, foobar/\bot, baz/\top}.
%% 
\newcounter{factctr}
\newtoks\varVal
\newtoks\varName
\newcommand{\facts}[1]{\begingroup%%
  %% Test if the argument given contains a forward slash (/). Expands
  %% slashTest with argument such that if a slash is NOT present the 
  %% token \noSlash will be given as argument 2 to slashTest. Otherwise
  %% there must be slash.
  \def\hasSlash##1{\expandafter\slashTest##1/\noslash\endslash}%%
  \def\slashTest##1/##2##3\endslash{\ifx\noslash##2 N\else Y\fi}%%
  \def\getArgs##1/##2{\varName={##1}%%
    \varVal={##2}}
  \ensuremath{%%
    \setcounter{factctr}{0}%%
    \foreach \var in {#1}{%%
      %% Separate list with a comma
      \ifthenelse{\value{factctr}>0}{,\allowbreak}{}%%
      %% \tracingmacros=1%%
      %% If key/val arguments, use first form. Otherwise
      %% use second.
      \ifthenelse{\equal{\hasSlash{\var}}{Y}}%%
                  {\expandafter\getArgs\var \factC{\the\varName}{\the\varVal}}%%
                  {\var}%%
      %% \tracingmacros=0%%
      \stepcounter{factctr}%%
    }}%%
\endgroup}
\newcommand{\factC}[2]{{\ensuremath{(\mathit{#1},#2)}}}
\newcommand{\doFacts}[4]{\ensuremath{#3{#1}: %%
    \left\{ %%
    \begin{minipage}[c]{#4}%%
      \facts{#2} %%
  \end{minipage}\kern -0.23em\right\}}}

\ExplSyntaxOn
\DeclareDocumentCommand \inFactsM {m m m} {\doFacts{#1}{#2}{\inB}{#3}}
\DeclareDocumentCommand \inFacts {m m O{1in}} {\doFacts{#1}{#2}{\inB}{#3}}
\DeclareDocumentCommand \outFactsM {m m m} {\doFacts{#1}{#2}{\outB}{#3}}
\DeclareDocumentCommand \outFacts {m m O{1in}} {\doFacts{#1}{#2}{\outB}{#3}}
\ExplSyntaxOff

\newcommand{\lub}{\ifthenelse{\boolean{mmode}}{\sqcap}{\raisebox{.1em}{\ensuremath{\sqcap}}}\xspace}
\newcommand{\sqlt}{\ensuremath{\sqsubset}\xspace}
\newcommand{\sqlte}{\ensuremath{\sqsubseteq}\xspace}

%% End dataflow

%% MIL Chapter
\newcommand{\compMILE}[1]{\ensuremath{\left\llbracket #1 \right\rrbracket}}
\newcommand{\compMILV}[1]{\ensuremath{\left\llbracket #1 \right\rrbracket}}
\newcommand{\compMILQ}[2]{\ensuremath{\left\llbracket #2 \right\rrbracket}}
\newcommand{\milCtx}[1]{\ensuremath{\llfloor}#1\ensuremath{\rrfloor}}
%% End MIL chapter

\newenvironment{myfig}[1][tbh]{\begin{figure}[#1]%%
\centering%%
\figbegin}{\figend%%
\end{figure}}

%% Produce a sub-caption and label it.
\newcommand{\scap}[2][1in]{\begin{minipage}{#1}%%
\subcaption{}\label{#2}\end{minipage}}

%% Produce a sub-caption with text.
\newcommand{\lscap}[3][1in]{\begin{minipage}{#1}%%
\subcaption{#3}\label{#2}\end{minipage}}

% single-argument comment. Do not put
% a space before the command when used
% or the file will have two spaces.
\newcommand{\rem}[1]{}

%% A verbatim environment with active charactesr
%% so we can use math shortcuts and macros
\DefineVerbatimEnvironment{AVerb}{Verbatim}{commandchars=\\\{\},%% 
  codes={\catcode`\_8\catcode`\$3\catcode`\^7},%%
  numberblanklines=false}

%% Turn on line numbers for Haskell code, 
%% and reset the line number counter.
\newcommand{\hsNumOn}{\numberson\numbersreset}
\newcommand{\hsNumOff}{\numbersoff}
%% Turn on line numbering in Haskell code within
%% the environment, then turn it off.
\newenvironment{withHsNum}{\numberson\numbersreset}{\numbersoff}

%% Paragraph run-in
\newcommand{\runin}[1]{\begingroup\noindent\sffamily\textbf{#1}\qquad\endgroup}

%% Chapter bibliographies
\newcommand{\standaloneBib}{%%
  \ifthenelse{\boolean{standaloneFlag}}%%
             {\begin{singlespace}
                \printbibliography
             \end{singlespace}}{}}

%% Adds an equation number on demand.
\newcommand\addtag{\refstepcounter{equation}\tag{\theequation}}

%% For typesetting set definitions like {x | x \in f(y)}
\newcommand\setdef[2]{\ensuremath{\{#1\ |\ #2\}}}

%% For typesetting function names like dom(f) or out(b).
\newcommand\mfun[1]{\ensuremath{\mathit{#1}}}

%% Marginal notes
\newcommand\margin[2]{\marginpar{\begin{singlespace}\emph{\footnotesize #2}\end{singlespace}}\relax #1}

%% Describe intent of a passage
\newcommand\intent[1]{{\leftskip = -1in\begin{singlespace}\emph{\noindent\footnotesize Intent: #1}\end{singlespace}}}

\begin{document}
\ifthenelse{\boolean{standaloneFlag}}
           {\VerbatimFootnotes
             \DefineShortVerb{\#}
             \setcounter{chapter}{0}}{}

%% Default float parameters. For case when
%% multiple chapters are included and
%% only one needs custom float settings.
\renewcommand{\textfraction}{0.2}
\renewcommand{\textfraction}{0.2}
\renewcommand{\topfraction}{0.9}


\chapter{The \LamA \& Functional Languages}
\label{ref_chapter_languages}

%% Overall: Justify why the lambda-calculus matters
%%   * Give syntax and evaluation rules for lambda-case
%%   * Set the foundation for MIL-to-LC later

John McCarthy created LISP, the first ``functional'' language, in 196X
\citep{McCarthyXX}. Other functional languages created since then
include Scheme, ML, Haskell, and many more. While syntax, semantics
and capabilities differ widely between all these languages, they 
share a common characteristic: \emph{the ability to manipulate
  functions as first-class values}.

A function that returns a function can be hard to get used to, so some
examples may help. First, we look at a function that just computes
a value -- it does not return a function or anything fancy.
Figure~\ref{lang_fig1} shows ``#mag#'', a function that doubles its value,
written in four different functional languages. 

\begin{myfig}[bth]
  \begin{tabular}{cc}
  \subfloat{%%
    \begin{minipage}{2in}\begin{withHsNum}%%
> mag :: Float -> Float {-"\label{lang_fig1_haskell_sig}"-}
> mag a = 2 * a {-"\label{lang_fig1_haskell_impl}"-}
    \end{withHsNum}\end{minipage}%%
    \label{lang_fig1_haskell}} & %%
  \subfloat{\begin{minipage}{2in}
  \begin{AVerb}[gobble=4,numbers=left]
    mag : float -> float \label{lang_fig1_ml_sig}
    mag a = 2 *. a \label{lang_fig1_ml_impl}
  \end{AVerb}
\end{minipage}
\label{lang_fig1_ml}} \\

  \subref{lang_fig1_haskell} & \subref{lang_fig1_ml} \\

  \subfloat{\begin{minipage}{2in}
  \begin{AVerb}[gobble=4,numbers=left]
    (def double a (* 2 a))
  \end{AVerb}
\end{minipage}
\label{lang_fig1_scheme}} & %%
  \subfloat{\begin{minipage}{2in}
  \begin{AVerb}[gobble=4,numbers=left]
    function mag(a) \{ \label{lang_fig1_js_def}
      return a * 2; \label{fig_lang1_js_impl}
    \}
  \end{AVerb}
\end{minipage}
\label{lang_fig1_js}} \\

  \subref{lang_fig1_scheme} & \subref{lang_fig1_js} 
  \end{tabular}
  \caption{Definitions of a function that doubles its argument in
    \subref{lang_fig1_haskell} Haskell, \subref{lang_fig1_ml} ML, 
    \subref{lang_fig1_scheme} Scheme, and \subref{lang_fig1_js} JavaScript.}
  \label{lang_fig1}
\end{myfig}

Part \subref{lang_fig1_haskell} gives the Haskell version. Haskell is
a statically typed language, so we begin with a type signature on
Line~\ref{lang_fig1_haskell_sig}: ``|mag :: Float -> Float|.'' This
signature indicates that |mag| takes an argument of type |Float| and
returns a result, also of type |Float|, where |Float| represents a
floating-point number
\citep{HaskellReportXX}. Line~\ref{lang_fig1_haskell_impl} implements
|mag|. The function name comes first, followed by the argument
(``|a|''). The right-hand side of the equals sign (``|=|'') defines
the \emph{body} of the function: ``|2 * a|.'' The body is evaluated
when the function is applied to an argument and a result must be
computed.\footnote{Haskell is a \emph{lazy} language, meaning no
  computations are performed until \emph{demanded}. Therefore, we say
  the body is evaluated only when a ``result must be computed.''}

Figure~\ref{lang_fig1}, Part~\subref{lang_fig1_ml}, gives the ML
implementation. ML is also statically typed, so we start with the type
signature on Line~\ref{lang_fig1_ml_sig}: ``#float -> float#.'' This
signature has much the same meaning as the Haskell
version. Line~\ref{lang_fig1_ml_impl} gives the implementation of
#mag#. The ``#*.#'' operator represents floating-point
multiplication. Otherwise, the implementation is much the same as the
Haskell version.

Figure~\ref{lang_fig1}, Part \subref{lang_fig1_scheme}, gives the
Scheme definition. Scheme is a dynamically typed language, so no
signature can be given -- just the implementation. The expression
``\texttt{define mag}\ \emph{expr}'' associates \emph{expr} with the
name ``#mag#.'' The expression in this case, ``\texttt{lambda (a)
  \ldots)},'' creates a new function that takes one argument, ``#a#.''
The body of the function, ``#(* 2 a)#,'' shows that the argument will
be doubled when the function is applied.

%% ``#define#'' keyword
%% associates a name with a value. The
%% ``#lambda#'' keyword indicates that a function will be created. The
%% funciton defined takes only one argument, designated ``#a#.'' The
%% postfix expression, ``#(* 2 a)#,'' defines the body of the function
%% and will be evaluated when the function is applied.

Figure~\ref{lang_fig1}, Part \subref{lang_fig1_js}, shows the
JavaScript version. Line~\ref{lang_fig1_js_def} gives the
\emph{signature} of the function -- the name of the function and any
named arguments. JavaScript is also a dynamically typed language, so
this is \emph{not} a type signature, but rather a specification of how
to call the function.\footnote{In JavaScript, functions can take more
  arguments than are declared, for which reason we say ``named''
  arguments here.}  Function definitions always start with the
``#function#'' keyword, followed by the function name and any
named arguments in parentheses: ``#mag (a)#''. The body of the function,
on Line~\ref{fig_lang1_js_impl}, uses the ``#return#'' keyword to
indicate the function doubles its argument and returns the resulting
value: ``#return 2 * a;#.''

The functions defined in Figure~\ref{lang_fig1} all have one thing in
common: they are limited to doubling their argument. If we want to
triple our argument, halve it, zero it or perform any other
multiplication, then we need to write a new function.

Of course, we can write a function that takes two arguments, the
multiplier and the argument. For example, in JavaScript:
\begin{AVerb}
function magBy(multiple, a) \{
  return multiple * a;
\}
\end{AVerb}
But this limits us from re-using ``#magBy#'' in certain ways. 

Imagine a function that wants to apply ``#magBy#'' to all items in a
list:\footnote{In this fragment, #items# is an array of values,
  accessed by index. We enumerate it using a #for# loop much like the
  that found in the C language.}
\begin{AVerb}
function magAll(items, multiple) \{
  for(var i = 0; i < items.length(); i++)
    items[i] = magBy(multiple, items[i]);
\}
\end{AVerb}
This definition creates two problems:
\begin{enumerate}
\item Every call to ``#magAll#'' requires us to specify a value for
  ``#multiple#.'' 
\item Our function is limited to using ``#magBy#.'' If ``#magBy#''
  isn't appropriate for some situation, we need to write a new
  ``#magAll#'' that uses a different version.
\end{enumerate}
We solve these two problems by making ``#magBy#'' a \emph{parameter}
to ``#magAll#.'' In pictures, we create a ``hole'' in ``#magAll#''
that we fill with code passed in:
\begin{AVerb}
function magAll(items, \emph{<code>}) \{
  for(var i = 0; i < items.length(); i++)
    items[i] = \emph{<code>};
\}
\end{AVerb}
The ``\emph{<code>}'' argument precisely illustrates how 
functional languages treat ``functions as values.''

Figure~\ref{lang_fig2} shows the definition of |multiplier| in the
same four languages as Figure~\ref{lang_fig1}, as well as a re-definition
of ``|mag|'' in terms of ``|multiplier|.'' When |multiplier| is
evaluated, it produces a value like any function; that value just
happens to be a function itself! The function returned takes an
argument and multiplies it by the original multiple given to
|multiplier|.

\begin{myfig}
  \begin{tabular}{cc}
    \subfloat{\begin{minipage}{3.5in}\begin{withHsNum} %%
> multiplier :: Float -> (Float -> Float)
> multiplier multiple = 
>   \a -> multiple * a {-"\label{lang_fig2_hs_fun}"-}
>
> mag :: Float -> Float
> mag = multiplier 2 {-"\label{lang_fig2_hs_mag}"-}
        \end{withHsNum}
      \end{minipage}\label{lang_fig2_hs}} & %%
    \subfloat{\begin{minipage}{3in}
  \begin{AVerb}[gobble=4,numbers=left]
    multiplier : float -> 
      (float -> float)
    multiplier multiple = 
      let f a = a *. multiple \label{lang_fig2_ml_fun}
      in f

    mag : float -> float
    mag = multiplier 2 \label{lang_fig2_ml_mag}
  \end{AVerb}
\end{minipage}
\label{lang_fig2_ml}} \\

    \subref{lang_fig2_hs} & \subref{lang_fig2_ml} \\

    \subfloat{\begin{minipage}{3in}
  \begin{AVerb}[gobble=4,numbers=left]
    (define multiplier \label{lang_fig2_scheme_fun}
      (lambda (multiple)  
        (lambda (a) (* multiple a))))

    (define mag \label{lang_fig2_scheme_mag}
      (multiplier 2))
  \end{AVerb}
\end{minipage}
\label{lang_fig2_scheme}} & %%
    \subfloat{\begin{minipage}{3in}
  \begin{AVerb}[gobble=4,numbers=left]
    function multiplier(multiple) \{
      return function(a) \{ \label{lang_fig2_js_fun}
        return multiple * a;
      \};
    \}
    
    var double = multiplier(2); \label{lang_fig2_js_double}
  \end{AVerb}
\end{minipage}
\label{lang_fig2_js}} \\

    \subref{lang_fig2_scheme} & \subref{lang_fig2_js} \\

  \end{tabular}
  \caption{The |multiplier| function and how it can be used to define
    |mag|. When evaluated, |multiplier| returns a function that
    will multiply its argument by |multiple|. We give
    \subref{lang_fig2_hs} Haskell, \subref{lang_fig2_ml} ML,
    \subref{lang_fig2_scheme} Scheme, and \subref{lang_fig2_js}
    JavaScript versions.}
  \label{lang_fig2}
\end{myfig}

Figure~\ref{lang_fig2}, Part~\subref{lang_fig2_hs} gives the Haskell
version of ``|multiplier|.'' The signature, ``|Float -> (Float ->
Float)|,'' shows that ``|multiplier|'' takes one argument, a ``|Float|''
value, and returns a function (``(|Float ->
Float)|''). Line~\ref{lang_fig2_hs_fun} creates an \emph{anonymous}
function: 

> \a -> multiple * a

The anonymous function is introduced with the ``|\|'' (``lambda'')
symbol, followed by one argument, ``|a|.'' The body of the function
follows the arrow (``|->|''). 
Notice that ``|multiple|'' is \emph{not} an argument to this
function. Instead, it is an argument to |multiplier|. We say
|multiple| is \emph{captured} by the anonymous function. The anonymous
function is the value returned by |multiplier|. When that value is
itself applied to an argument, it will use the value of |multiple|
originally given to |multiplier|.

On Line~\ref{lang_fig2_hs_mag} we use |multiplier| to define the
|mag| function from Figure~\ref{lang_fig2_hs}. The function has
the same signature, ``|Float -> Float|,'' but no argument:

> mag :: Float -> Float
> mag = multiplier 2

If we substitute the definition of
|multiplier| in |mag|, we can see the function |mag| represents:

\begin{math}
  \begin{array}{cc}
    |mag| &= |multiplier 2| \\
    &= |\a -> 2 * a | 
  \end{array}
\end{math}

Notice that the argument |a| appears on the right-hand
side here, for which reason |mag| does not specify an argument
in Figure~\ref{lang_fig2_hs}.

Figure~\ref{lang_fig2}, Part \subref{lang_fig2_ml} shows the ML
definition for #multiplier# and #mag#. \texttt{multiplier} returns
the value #f#, which is again a function. Line~\ref{lang_fig2_ml_fun}
defines #f# as a local, named function:

\begin{AVerb}
  let f a = a *. multiple
  in f
\end{AVerb}

Again, we capture the value of #multiple# when defining #f#. When #f#
is evaluated, it will multiply its argument by the #multiple#
given. The definition of #mag# on Line~\ref{lang_fig2_ml_mag} in terms
of #multiplier# looks almost exactly the same as the Haskell version.

In Figure~\ref{lang_fig2}, Part \subref{lang_fig2_scheme}, we give the
Scheme version of #multiplier# and #mag#. As in
Figure~\ref{lang_fig1_scheme}, the body of #multiplier# is a function,
defined using #lambda#. However, this function returns a function,
again defined with #lambda#:

\begin{AVerb}
  (lambda (a) (* multiple a))))
\end{AVerb}

As in the Haskell and ML versions, the inner function captures the
value of #multiple# given to the outer function. On
Line~\ref{lang_fig2_scheme_mag} we evaluate the expression
\texttt{({multiplier} 2)} and assign the result to #mag#:
\begin{AVerb}
  (define mag 
    (multiplier 2))
\end{AVerb}

Figure~\ref{lang_fig2}, Part \subref{lang_fig2_js} shows the JavaScript
version of #multiplier#. The body of #multiplier# returns an anonmous
function, defined using the #function# keyword without a function name:
\begin{AVerb}
  return function(a) \{ 
    return multiple * a;
  \};
\end{AVerb}

Once again, the #multiple# argument is captured and used by the
returned function. Line~\ref{lang_fig2_js_mag} shows how #mag#
is defined in terms of #multiplier#:

\begin{AVerb}
  var mag*.te = multiplier(2);
\end{AVerb}

The #var# keyword introduces an
identifier, to which we assign the function returned by
#multiplier(2)#. In some ways this syntax makes it most obvious that
we are treating functions as values.

Returning to #magAll#, we can redefine it to take a function argument:
\begin{AVerb}
function magAll(items, magnifier) \{
  for(var i = 0; i < items.length(); i++)
    items[i] = magnifier(i);
\}
\end{AVerb}
Here, #magnifier# is a function, passed as an argument. If 
we wish to double the items in the array, we just pass #double#
to #magAll#:
\begin{AVerb}
  magAll(items, double);
\end{AVerb}
To multiply the items however we need, we just create appropraite
\emph{function values} and pass them to #magAll#:
\begin{AVerb}
  var halve = multiplier(0.5);
  var quadruple = multiplier(4);
  magAll(items, quadruple);
  magAll(items, halve);
\end{AVerb}
We can even pass an \emph{anonymous function} directly to 
#magAll#, as here where we halve the values again:
\begin{AVerb}
  magAll(items, function (i) \{ return i * 0.5; \});
\end{AVerb}

%%  However, that value is itself a 
%% function. |multilp
%% evaluated, creates a new function. in our four languages

\section{Why \LamA?}
\label{lang_sec1_}

The \lamA really is the essential functional programming language.  
That does not mean it is the programming language anyone
should ever use outside research or theoretical settings. What it
means is the \lamA has everything you need to model the behavior
of other programming languages. 

Of course, you could say the same about any Turing-complete
language. What makes the \lamA special is its simplicity. Nothing is
built-in, not even the natural numbers. There is almost nothing to get
in the way of studying your particular domain, rather than fiddling
with the programming language.

Again, that could be said about any sufficiently simple
Turing-complete programming language. However, the \lamA is inherently
functional -- it can only define functions, and all values computed
are themselves functional. Translating from the \lamA to
another functional language is usually straightforward (or maybe your
target language isn't very functional). 

Its power, simplicity, and functional nature make the \lamA so popular
in functional language research, and motivate our own choice in using
it. We do not show how to compile a ``real-world'' functional language
to our intermediate language, but by showing how to compile the 
\lamA to our language, we show our technique could be used by ``real'' 
functional languages (with some adaptions, of course).

%% We will be demonstrating a number of dataflow optimizations over
%% our intermediate language programs, but all of our source programs will
%% be written in a variant of the \lamA. Any variante

%% A compilation technique demonstrated for
%% some variant of the \lamA can be translated into any other functional
%% programming language. Making the translation work well with the syntax
%% and semantics of the target language is still hard work, but
%% absolutely possible -- a result developed for the \lamA really is
%% universal (as far as you want to make use of that result, of course). 

%% It is these three reasons that make the \lamA such a popular
%% language for showing theoretic (or practical) results

%% This chapter introduces the \lamA calcus, giving its 

%% Most importantly, results obtained
%% using the \lamA are guaranteed to translate to other Turing-complete
%% languages -- and usually with better syntax! 

%% Being Turing-complete, the \lamA is capable of exeucting any program
%% you could write on a modern computer.  

%%  . What it does mean is that anything possible in the \lamA Being
%% Turing-complete, it can be

%% \emph{\ldots transition
%%   \ldots}. Most importantly, the \lamA serves as the \emph{lingua
%%   franca} for functional programmers. It provides a way to translate
%% between functional programming languages, and a way to carry
%% developments from one language to another. 

%% so its capabilities are
%% as powerful as any other Turing-complete programming language. 

%% Being Turing-complete, it can emulate any other
%% Turing-complete language. Its direct support for manipulating
%% functions-as-values makes it a good choice for emulating higher-level
%% functional languages.

%% It is not a language that you want to write many 
%% programs in, 

%% Its sparse syntax
%% and straightforward evaluation rules means their is less to worry about
%% when trying out new design ideas or theories.  . Most importantly, though, 


%% Figure~\ref{lang_fig1} defines
%% a function for adding two numbers in Scheme, ML, Haskell and JavaScript,
%% some of the more prominent functional languages in use today. 




 
%% 1. Relate lambda-calculus to functional langauges in general
%% 2. Define the lambda-calculus
%%    * syntax, semantics, evaluation rules
%% 3. Compliling - or better to move that to MIL chapter?

%% Informally, a \emph{function} is a mathematical definition that takes
%% arguments and computes some result. For example, \emph{plus1} just adds 1 
%% to its argument, using the normal rules of arithmetic:
%% \begin{equation}
%%   |plus1|\ x = x + 1.
%% \end{equation}
%% We are not limited to defining functions that add 1. Functions can
%% also be \emph{values} -- just as 1 or $x$ are in the expression
%% above. Here, we define a function, that returns a function, that always
%% adds 1:
%% \begin{equation}
%%   |adder1| = \lambda\ x = x + 1.
%% \end{equation}
%% In |adder1|, ``$\lambda\ x$'' indicates we return a function that takes one
%% argument. We can go further and define a function that, given an
%% argument, returns a function which always adds that amount:
%% \begin{equation}
%%   |adder|\ n = \lambda\ x = x + n.
%% \end{equation}
%% Notice how the outer argument $n$ gets ``captured'' by the body $x +
%% n$. Using $|adder|$, we can now re-define $|adder1|$ above:
%% \begin{equation}
%%   |adder1| = |adder|\ 1.
%% \end{equation}

%% Other functional languages created since then include 
%% Scheme, ML, Haskell, and many more. While syntax, semantics and capabilities
%% differ widely between all these languages, they all share the characteristic
%% shown above: \emph{the ability to manipulate functions as first-class values}.

\section{The \LamA}
%%\emph{Why is it important}

%%\emph{What is the \lamA}

Alonzo Church defined his \lamA (``lambda calculus'') in 19XX
\citep{ChurchXX} to study systems of recursive equations. Being
Turing-complete, it can be used to model the behavior of any
computational system. However, it is particularly useful for modeling
functional programming languages. %% why?

Figure \ref{lang_fig2} gives a syntax for the ``pure''
\lamA. ``Abstraction'' defines a new function, while ``application''
gives a particular argument to a function. ``Variables'' are defined
when mentioned. 

\begin{myfig}[ht]
  \begin{equation*}
    \begin{aligned}
      \mathit{term} &= a, b, \ldots & \text{\emph{(Variables})} \\
      &= \lamAbs{x}{t} & \text{\emph{(Abstraction)}} \\
      &= \lamApp{t}{t} & \text{\emph{(Application)}}
    \end{aligned}
  \end{equation*}
  \caption{The \lamA' syntax. A term can be a \emph{variable}, \emph{abstraction},
  or \emph{application}. When $t$ appears, it means a term can be substituted. Other
  letters stand for variables. In the case of \emph{abstraction}, $x$ always stands
  for a single variable.}
  \label{lang_fig2}
\end{myfig}

%%\emph{What does it look like?}

Using this syntax, we can define some common functions. |Identity|
returns its argument:
\begin{align}
  |identity| &= \lamAbs{x}{x}. \label{eq_lang2} \\
  \intertext{|Const| takes two arguments but always returns the first:}
  |const| &= \lamAbs{a}{\lamAbs{b}{a}}. \label{eq_lang4} \\
  \intertext{|Compose| takes two functions and an argument. The result of
    applying the second function to the argument is passed to the first:}
  |compose| &= \lamCompose. \label{eq_lang3} 
\end{align}
Note that function application is right-associative, meaning
\lamPApp{f}{\lamPApp{g}{x}} is the same as \lamApp{\lamApp{f}{g}}{x},
but \emph{not} the same as \lamPApp{\lamPApp{f}{g}}{x}.

\begin{myfig}[bt]
\begin{minipage}{2in}
\begin{Verbatim}
                ##  
                 #  
 ##  ## ## ###   #  
####  # ## ###   #  
#     ###  # #   #  
 ###   #   ## # ### 
\end{Verbatim}
\end{minipage}
  \caption{Evaluation rules for \lamA. These rules show 
    \emph{call-by-value}, where arguments are evaluated
    before functions.}
  \label{fig_lang2}
\end{myfig}

A \lamA term executes by rewriting the expression according to the
rules in Figure \ref{fig_lang2}. We match our term to each of the
patterns above the line. If we have a match, we rewrite according to
the pattern below the line. When no more matches can be made, we say
the term is in \emph{normal form}: we have finished executing.

The rules given implement \emph{call-by-value} evaluation order,
meaning arguments to a function are evaluated before the function
itself. Other variants include \emph{call-by-need} and
\emph{call-by-name}, where arguments are not evaluated until
needed. We do not considers those variants further, however.

\section{Compiling the \LamA}
\label{sec_lang1}

%% Define which steps in compilation we're going to worry about
Compiling even a language as simple as the \lamA involves a number of
steps, such as defining a concrete syntax, parsing source programs
into an abstract syntax tree (AST), and producing an executable
program from the AST. For our purposes, however, we just focus on the
\lamA' three fundamental operations:

\begin{itemize}
\item Naming values (\emph{variables}).
\item Apply a function to an argument (\emph{application}).
\item Create a new function (\emph{abstraction}). 
\end{itemize}

Any compiler for the \lamA must be able to produce executable programs
which implement these operations. 

%% \subsection{The Target Machine}
%% We begin by defining a \emph{target machine}, |M|, for our compiler. To
%% reduce complexity we do not target an actual computer, but one of our
%% own design. Our machine will have an infinite number of
%% \emph{registers} (i.e., storage locations) that we can refer to by
%% name. It will have an unlimited supply of memory (called the
%% \emph{heap}) in which we can allocate structured values. However, we
%% will not refer to memory locations directly. Instead, we will always
%% store references to heap values in registers. Finally, the machine
%% will execute a list of instructions (our \emph{program}), starting at
%% the beginning and proceeding in sequential order (unless otherwise
%% instructed), until reaching the end of the list. Each instruction will
%% have a definite location, but we will only refer to certain special
%% locations using named labels.

%% \subsection{M's Language: \machLam}
%% Table \ref{tbl_lang1} gives the language that our machine will
%% execute, \machLam. A benefit of defining our own machine is that we
%% can also define the language it executes -- and the language we need
%% to compile to! We cannot make it too dissimilar from a ``real''
%% machine, but at this stage it helps to keep things simple. 

%% \begin{table}[th]
%%   \centering
%%   \begin{tabular}{lp{3.5in}}
%%     \emph{Instruction} & \emph{Description} \\
%%     \cmidrule(r){1-1}\cmidrule(r){2-2}
%%     \texttt{Store \emph{R} (\emph{F}, \emph{M})} & Store the value found in register #R# to field %%
%%     #F# of the value in register #M#. \\
%%     \texttt{Load (\emph{F}, \emph{M}) \emph{R}} & Load field #F# of the value in register #M# to register #R#. \\
%%     \texttt{Set \emph{v} \emph{R}} & Sets the register #R# to name of the variable $v$. \\
%%     \texttt{Copy \emph{R} \emph{M}} & Copies the contents of register #R# to register #M#. \\
%%     #Enter# & Jump to the location indicated by the closure in
%%     register #clo#, assuming an argument in register #arg#. The next #Return# executed
%%     will return to this location, with a result in register #res#.\\
%%     #Return# & Jump to the instruction following the most recently 
%%     executed #Enter# instruction and begin executing.  \\
%%     \texttt{MkClo \emph{L} [\emph{R}, \emph{S}, \dots]} &  Create a closure pointing to the 
%%     label #L# and holding the values in registers #R#, #S#, etc. The closure will be stored in 
%%     the #res# register.
%%   \end{tabular}
%%   \caption{\machLam, the ``machine language'' executed by our machine |M|.}
%%   \label{tbl_lang1}
%%   \figend
%% \end{table}

%% Each instruction supports an some aspect of the \lamA. In brief:
%% \begin{description}
%% \item[Variables] -- #Store# and #Load# help access variables and
%%   function arguments.
%% \item[Function Application] -- #Enter# and #Return# allow us to execute a function with arguments.
%% \item[Abstraction] -- #MkClo# lets us create functions as values.
%% \end{description}
%% The following sections describe each aspect in detail.

\subsection{Variables}
\label{subsec_lang1}

%% Free variables and environment
Consider how to find a value by its name. For example, consider
the |compose| function (expression \ref{eq_lang3}):
\begin{equation}
  \lamCompose.  \label{eq_lang1}
\end{equation}
We see three variables: $f$, $g$, and $x$. We say $x$ is \emph{bound},
because it is given as an argument, and that $f$ and $g$ are
\emph{free} because, in this context, they are not arguments in a 
$\lambda$-abstraction. To evaluate this expression, though, we need
a way to find the values of these terms.  

We can describe where to find $f, g$ and $x$ in terms of memory
locations. We can say that $x$ will appear in a special location,
|arg|, because it is the argument to the function and we will always
put arguments in the same place. We can further say that another
special location, |clo|, will have two
slots. The first will contain $g$ and the second will contain
$f$. Conceptually, then, our expression can be represented as:
\begin{center}
  \begin{tabular}{c}
    \begin{math}\begin{aligned}[b]
      |arg| &= x, \\
      |clo|[0] &= g, \\
      |clo|[1] &= f 
    \end{aligned}\text{\ in}\end{math} \\
    \lamAbs{|arg|}{\lamApp{|clo|[1]}{\lamPApp{|clo|[0]}{arg}}}.
  \end{tabular}
\end{center}

\par
In general, the $|clo|$ location holds the \emph{environment} for our
expression. For any given expression, we will be able to find all the
free variables (i.e., all those except the argument) in the
environment. The compiler will be responsible for ensuring the correct
environment is available whenever a given expression is evaluated.

%% Our machine, then, must have instructions for storing and retrieving
%% values. #Store# and #Load# (from Table \ref{tbl_lang1}) serve this
%% purpose. 

\subsection{Function Application}
\label{subsec_lang2}

%% Application & closures
Associating locations with names is not enough, however. Looking again
at expression \ref{eq_lang1}, $g$ clearly represents a function to
which we pass the argument $x$. To compute the value of
$\lamPApp{g}{x}$, we must be able to execute the code representing
$g$. We already assigned a storage location for $g$ ($|clo|[0]$) -- now
we just say that the value in $|clo|[0]$ is a \emph{label} that tells
us where to find the code representing $g$. However, $g$ will need
an environment of its own, to hold any free variables for $g$. Therefore,
we pair the label indicating where to find $g$ with a list of free
variables. We call this structure a \emph{closure}.

Closures are the fundamental data structures used to compile
functional languages. They may not have the exact form described here
but they always have the same purpose: they pair a label with the free
variables used in the function represented. 

\subsection{Abstraction}
\label{subsec_lang3}
The \lamA lets us define functions which return new functions. We have
seen how to access variables in the environment and how to execute
unknown functions using closures. Now we come to the final element
needed to compile the \lamA\ -- creating closures.

Consider the following expression, where we apply the $|const|$ function (expression 
\ref{eq_lang4}) to an argument:
\begin{equation}
  \begin{split}
    |main| &= \lamApp{|const|}{s} \\
         &= \lamAppP{\lamAbs{a}{\lamAbs{b}{a}}}{s}.
  \end{split}
\end{equation}
In order to evaluate $|main|$, we need to apply the $|const|$ function
to $s$. From the previous section we know that a closure is required to
implement function application. It follows that
\lamAbs{a}{\lamAbs{b}{a}} must create a closure which will
then be used to execute the body of the $\lambda$-abstraction with the
argument $s$. In fact, the ``value'' created by a
$\lambda$-abstraction is always a closure. The closure will point to
the body of the $\lambda$-abstraction and will hold the free variables
necessary to evaluate it.

%% \subsection{Compiling from \lamA to \machLam}

%% Table \ref{tbl_lang2} gives our algorithm to compile from \lamA to
%% \machLam. We present it in in three parts, \emph{a} - \emph{c},
%% corresponding to the syntax of \lamA terms given in Figure
%% \ref{lang_fig2}. The ``fat brackets,'' \compMach{t}, represent our
%% compiler, with the term being compiled given as the argument, $t$.
%% Each term compiles to a given sequence of instructions. We also assume
%% a function $\rho$, maintained by the compiler, that knows which
%% register holds a given variable.

%% %% Compilation rules ...
%% \afterpage{\clearpage{%% Used in the languages chapter, this
%% table is placed in its own file so we can use
%% it with the afterpage command.
\begin{singlespace}
  \begin{longtable}{p{2in}p{3.5in}}
    \caption{Compilation rules from \lamA to \machLam.} \\
    \hline \\
    \endfirsthead
    \caption{Compilation rules from \lamA to \machLam \emph{(cont'd)}} \\
    \hline \\
    \endhead
    \\ \hline \multicolumn{2}{r}{\emph{Continued on next page}}
    \endfoot 
    \\ \hline
    \endlastfoot
    %% Variables
    \multicolumn{2}{c}{\emph{(a) Variable Reference}} \\ 
    \begin{minipage}[t]{2in}
      \begin{Verbatim}[commandchars=\\\{\}]
\compMach{v} = 
      \end{Verbatim}
    \end{minipage} \\

    \begin{minipage}[t]{2in}
      \begin{Verbatim}[commandchars=\\\{\}]
  Set ``v'' ``res''
  Return
      \end{Verbatim}
    \end{minipage} &  Because our \lamA does not have any real ``values'', we just
    set the #res# register to the variable name. \\ \\

    %% Application
    \multicolumn{2}{c}{\emph{(b) Function Application}} \\ 
    \begin{minipage}[t]{2in}
      \begin{Verbatim}[commandchars=\\\{\}]
\compMach{\lamApp{f}{g}} = 
      \end{Verbatim}
    \end{minipage} \\

    \begin{minipage}[t]{2in}
      \begin{Verbatim}[commandchars=\\\{\}, codes={\catcode`\_8\catcode`\$3}]
  Copy ``arg'' $r$
  Copy ``clo'' $s$
      \end{Verbatim}
    \end{minipage} & $r$ and $s$ are ``fresh'' registers. \\ \\[-.5em]

    \begin{minipage}[t]{2in}
      \begin{Verbatim}[commandchars=\\\{\}]
  Copy \compRho{g} arg
  Copy \compRho{f} clo
  Enter
      \end{Verbatim}
    \end{minipage} & $\rho$ associates variables
    to the register that they will be found in. This lookup occurs
    during compilation, not while the program executes. Here we copy
    $f$ and $g$ to the #clo# and #arg# registers, respectively. \\ \\[-.5em]

    \begin{minipage}[t]{2in}
      \begin{Verbatim}[commandchars=\\\{\}, codes={\catcode`\_8\catcode`\$3}]
  Copy $r$ ``arg''
  Copy $s$ ``clo''
      \end{Verbatim}
    \end{minipage} & Restore the previous #arg# and #clo# registers. \\ \\

    %% Abstraction
    \multicolumn{2}{c}{\emph{(c) Abstraction with an Abstraction}} \\ 
    \begin{minipage}[t]{2in}
      \begin{Verbatim}[commandchars=\\\{\}]
\compMach{\lamAbs{x}{\lamAbs{y}{t}}} = 
      \end{Verbatim}
    \end{minipage} \\

    \begin{minipage}[t]{2in}
      \begin{Verbatim}[commandchars=\\\{\}]
m : 
      \end{Verbatim}
    \end{minipage} & We mark the function body with a fresh label, #m#. \\ \\[-.5em]
    
    \begin{minipage}[t]{2in}
      \begin{Verbatim}[commandchars=\\\{\}, codes={\catcode`\_8\catcode`\$3}]
  Store r1 (``clo'', 0) 
  Store r2 (``clo'', 1) 
  \dots
  Store r$N$ (``clo'', 
            $N$)
      \end{Verbatim}
    \end{minipage} & Copy current values out of the closure. $N$
    equals the number of fields in the closure. \\ \\[-.5em]

    \begin{minipage}[t]{2in}
      \begin{Verbatim}[commandchars=\\\{\}, codes={\catcode`\_8\catcode`\$3}]
  MkClo l [r1, \dots, r$N$, 
           arg]
  Return
      \end{Verbatim}
    \end{minipage} & We call ourselves recursively and 
    retrieve a label, #l#, holding the location of the compiled body:
    \[ l = \compMach{\lamAbs{y}{t}} \].
    We then create a new closure which points to #l#. We put
    all values from the current closure into the new, and add our 
    argument, $x$, using the #arg# register. Because #MkClo# puts
    the closure created in #res#, we can immediately return. \\ \\

    %% Abstraction 2
    \multicolumn{2}{c}{\emph{(d) Abstraction with a Term}} \\* 
    \begin{minipage}[t]{2in}
      \begin{Verbatim}[commandchars=\\\{\}]
\compMach{\lamAbs{x}{t}} = 
      \end{Verbatim}
    \end{minipage} \\*

    \begin{minipage}[t]{2in}
      \begin{Verbatim}[commandchars=\\\{\}]
m : 
      \end{Verbatim}
    \end{minipage} & Again, we mark the function body with a fresh label, #m#. \\* \\*[-.5em]

    \begin{minipage}[t]{2in}
      \begin{Verbatim}[commandchars=\\\{\}, codes={\catcode`\_8\catcode`\$3}]
  Load \compRho{v_1} (0, ``clo'')
  Load \compRho{v_2} (1, ``clo'')
  \dots
  Load \compRho{v_n} ($N$, 
             ``clo'')
  Copy ``arg'' \compRho{x} 
      \end{Verbatim}
    \end{minipage} & We load all free variables $v_1, \dots, v_n$ from our
    closure into the appropriate registers, using the $\rho$ function. We also
    copy the argument $x$ from the #arg# register to the location given by
    $\rho$. \\ \\[-.5em]

    \begin{minipage}[t]{2in}
      \begin{Verbatim}[commandchars=\\\{\}]
  \compMach{t}
      \end{Verbatim}
    \end{minipage} & Now that variables are set up correctly, we compile the body
    of the abstraction and place it inline here.
  \label{tbl_lang2}
  \end{longtable}
\end{singlespace}
}\clearpage}

%% Table \ref{tbl_lang2}, part \emph{a}, shows the compilation
%% scheme for variables. Variable refrences that are not used
%% in function application can only be the body of an expression, so we
%% just copy the variable's name to the #res#
%% register and return.

%% Function application, \lamPApp{f}{g}, is shown in part
%% \emph{b}. To apply a function, we must save the current #clo#
%% and #arg# registers. The compiler creates \emph{fresh} registers,
%% guaranteed to be unused anywhere else in the program, to store #clo#
%% and #arg#. We then use $\rho$ to find the registers holding $f$ and
%% $g$. Remember that $f$ will be a closure, while $g$ will be some
%% value. We copy those values into #clo# and #arg#. The #Enter#
%% instruction will execute the code pointed to by #clo#. When that
%% function returns, we restore #clo# and #arg# from the fresh registers
%% created earlier.

%% Abstractions, such as \lamAbs{x}{t}, return a closure pointing to the
%% code implementing $t$. Therefore, our compiler needs to generate code
%% that returns a closure, which in turn points to the code generated for
%% the body of the abstraction. To accomplish this, our compiler
%% recursively calls itself on the body. We get a label back, which is the
%% location of the just compiled code. In parts \emph{c} and \emph{d}
%% the expression $l = \compMach{\lamAbs{y}{t}}$ shows this
%% recursive call, and the label that results. That label can then be used in the 
%% closure returned by the abstraction.

%% We separate compilation of abstractions into two cases, depending if
%% the body is an abstraction or not. In the first case, as shown in part
%% \emph{c}, we begin by marking the location of this code with a new label,
%% #m#. We prepare to create a new closure by copying all values out of
%% the current closure into fresh registers. We then create a closure that
%% points to the body of our abstraction, contains all the values found
%% in the current closure, and ``captures'' our argument in the new
%% closure. 

%% For example, consider compiling this expression:

%% \begin{equation}
%%   \lamAbs{x}{\lamAbs{y}{\lamApp{f}{\lamPApp{y}{x}}}}. 
%% \end{equation}

%% $f$ and $x$ must be available when the body
%% \lamPApp{f}{\lamPApp{y}{x}} executes. Therefore, the closure returned
%% by \lamAbs{x}{(\dots)} must copy all values in the existing
%% closure as well as add the argument, $x$.

%% Part \emph{d} shows the code generated when the body of an abstraction
%% is \emph{not} another abstraction. We first mark the location of the
%% start of the body with a new label, #m#.  We then find the free
%% variables in the body, calling them $v_1, \dots, v_n$. This is a
%% compile-time operation, not something the program will do when
%% executing.  We assume that value of each free variables can be found
%% in the corresponding closure slot. For example, $v_0$ will be found in
%% $clo[0]$, $v_1$ in $clo[1]$, and so on. We also copy the $arg$
%% register to the corresponding register for our argument, as determined
%% by the $\rho$ function. Now that we have placed all variables in the
%% registers expected by our function, we generate the code for our body
%% and place it inline.

\section{Conclusion}
Functional languages distinguish themselves by their ability to treat
\emph{functions} as \emph{first-class values}. The \lamA, invented
some time before the first functional language, turned out to be a
simple but effective way to model (and experiment with) the behavior
of functional languages. Therefore, understanding how to compile the
\lamA can effectively show us how to compile functional languages in
general.

This chapter gave the basic mechanisms needed to understand the \lamA:
\emph{variables}, \emph{application}, and
\emph{abstraction}. Understanding how to compile the \lamA means
understanding how to compile these three mechanisms. Variables become
\emph{locations}. Application means evaluating a function with a given
\emph{environment} for any \emph{free variables}. Abstractions create
\emph{closures} that carry two pieces of information: the location of
the compiled function body and the value of free variables to be used
when evaluating the function.

\end{document}


\documentclass[12pt]{report}
%include polycode.fmt
\usepackage[T1]{fontenc}
\usepackage{calc}
%% \usepackage{fourier}
\usepackage{palatino}
\renewcommand\ttdefault{lmtt}
\usepackage{helvet}
%% \usepackage{inconsolata}
\usepackage{comment}
\usepackage{calc}
\usepackage{xspace}
\usepackage{verbatim}
\usepackage{url}
\usepackage{fancyvrb}
\usepackage{setspace}
\usepackage{amsmath}
\usepackage{booktabs}
\usepackage[margin=\parindent, format=hang,labelfont=bf]{caption}
%% \usepackage[subrefformat=parens]{subcaption}
%% The following makes sure we get parentheses around
%% subreferences. The newest version of the subcaption
%% package has an option for this, but that's not available
%% widely.
%%
%% From http://tex.stackexchange.com/questions/25644
\usepackage[labelformat=simple]{subcaption}
\makeatletter
  \def\thesubfigure{(\alph{subfigure})}
  \providecommand\thefigsubsep{~}
  \def\p@subfigure{\@nameuse{thefigure}\thefigsubsep}
\makeatother

\usepackage{ifthen}
\usepackage{stmaryrd}
\usepackage{longtable}
\usepackage{afterpage}
\usepackage{xifthen}
\usepackage{mathtools}
\usepackage{xparse}
\usepackage[natbib=true,style=authoryear,backend=bibtex8]{biblatex}
\setlength{\bibitemsep}{\bigskipamount}
\addbibresource{thesis.bib}
\usepackage{microtype}

\usepackage{tikz}
\usetikzlibrary{arrows,automata,positioning,calc}
%% Used for CFGs.
\tikzset{
  >=stealth, 
  node distance=.5in,
  stmt/.style={rectangle,
    draw=black, thick,        
    minimum height=2em,
    %% inner sep=2pt,
    %% text centered,
    %% node distance=.5in,
  },
  entex/.style={
    minimum height=2em,
    %% inner sep=2pt,
    %% text centered,
  },
  labelfor/.style={circle, 
    draw=black, thin,
    font={\footnotesize},
    inner sep=0,
    fill=white,
    above right=-1.5mm and -1.5mm of #1,
  },
  fact/.style={overlay},
  %% Invisible node
  invis/.style={inner sep=0pt, 
    minimum height=0em}, 
  table/.style={circle, fill=white,height=2mm}
}

%% GSO margins.
\usepackage[left=1.5in, right=1in, top=1in, bottom=1in]{geometry}
\usepackage{abstract}

%% GSO requires 12 pt font for all headings
\usepackage[bf,sf,tiny,compact]{titlesec}
\titleformat{\chapter}[display]
            {}% format
            {\sffamily\bfseries\chaptertitlename\ \thechapter}
            {\baselineskip}
            {\sffamily\bfseries}
            {}

\hyphenation{data-flow mo-na-dic} 

\newboolean{lhs2tex}
\setboolean{lhs2tex}{true}

% Used by included files to know they
% are NOT standalone
\newboolean{standaloneFlag}
\setboolean{standaloneFlag}{true}

\newlength{\rulefigmargin}
\setlength{\rulefigmargin}{2\parindent}

\newcommand\figbegin{\rule{\linewidth-\rulefigmargin}{0.4pt}\\\vspace{12pt}}
\newcommand\figend{\rule{\linewidth-\rulefigmargin}{0.4pt}}

%\providecommand{\citep}[1]{(\emph{#1})\xspace}
%\renewcommand{\cite}[1]{\emph{#1}\xspace}

%% Functional languages chapter commands
\newcommand{\lamA}{\ensuremath{\lambda}-calculus\xspace}
\newcommand{\LamA}{\ensuremath{\lambda}-Calculus\xspace}
\newcommand{\lamAbs}[2]{\ensuremath{\lambda#1.\ #2}}
\newcommand{\lamApp}[2]{\ensuremath{#1\ #2}}
\newcommand{\lamPApp}[2]{\ensuremath{(#1\ #2)}}
\newcommand{\lamAPp}[2]{\ensuremath{(#1)\ #2}}
\newcommand{\lamApP}[2]{\ensuremath{#1\ (#2)}}
\newcommand{\lamAPP}[2]{\ensuremath{(#1)\ (#2)}}
\let\lamApPp=\lamApP
\let\lamAppP=\lamAPp

\newcommand{\lamId}{\lamAbs{x}{x}}
\newcommand{\lamCompose}{\lamAbs{f}{\lamAbs{g}{\lamAbs{x}{\lamApp{f}{(\lamApp{g}{x})}}}}}
\newcommand{\machLam}{\ensuremath{M_\lambda}\xspace}
\newcommand{\compMach}[1]{\ensuremath{\left\llbracket #1 \right\rrbracket}}
\newcommand{\compRho}[1]{\ensuremath{\rho(#1)}}
\newcommand{\verSub}[2]{\ensuremath{#1_{#2}}}
\newcommand{\verSup}[2]{\ensuremath{#1^{#2}}}
\newcommand{\lamC}{\ensuremath{\lambda_C}\xspace}
\newcommand{\lamPlus}{\lamAbs{m}{\lamAbs{n}{\lamAbs{s}{\lamAbs{z}{\lamApp{m}{\lamApPp{s}{\lamApp{n}{\lamApp{s}{z}}}}}}}}}
%% Substitution notation -- [#1 -> #2]
\newcommand{\lamSubst}[2]{\ensuremath{[#1 \mapsto #2]}}
%% End functional languages chapter

%% Dataflow chapter commands
\newcounter{nodeCounter}[figure]
\newcommand{\inE}{\ensuremath{\mathit{in}}\xspace}
\newcommand{\out}{\ensuremath{\mathit{out}}\xspace}
\newcommand{\In}{\ensuremath{\mathit{In}}\xspace}
\newcommand{\InBa}{\ensuremath{\mathit{In}(B)}\xspace}
\newcommand{\Out}{\ensuremath{\mathit{Out}}\xspace}
%% Out(B_x) -- fact function for an named block.
\newcommand{\OutB}[1]{\ensuremath{\mathit{Out}(B_{\ref{#1}})}\xspace}
\newcommand{\OutBa}{\ensuremath{\mathit{Out}(B)}\xspace}
%% in(B) -- fact function for an anonymous block.
\newcommand{\inBa}{\ensuremath{\mathit{in}(B)}\xspace}
%% in(X) -- fact function for an anonymous block, but using a different variable.
\newcommand{\inXa}[1]{\ensuremath{\mathit{in}(#1)}\xspace}
%% in(B,v) -- fact function for an anonymous block and some variable.
\newcommand{\inBav}[1]{\ensuremath{\mathit{in}(B, #1)}\xspace}
%% in(B_x) -- fact function for an named block.
\newcommand{\inB}[1]{\ensuremath{\mathit{in}(B_{\ref{#1}})}\xspace}
%% in(B_x,v) -- fact function for an named block and some variable.
\newcommand{\inBv}[2]{\ensuremath{\mathit{in}(B_{\ref{#1}}, #2)}\xspace}
%% out(B) -- fact function for an anonymous block.
\newcommand{\outBa}{\ensuremath{\mathit{out}(B)}\xspace}
%% out(X) -- fact function for an anonymous block, but using a different variable.
\newcommand{\outXa}[1]{\ensuremath{\mathit{out}(#1)}\xspace}
%% out(B,v) -- fact function for an anonymous block and some variable.
\newcommand{\outBav}[1]{\ensuremath{\mathit{out}(B, #1)}\xspace}
%% out(B_x) -- fact function for an named block.
\newcommand{\outB}[1]{\ensuremath{\mathit{out}(B_{\ref{#1}})}\xspace}
%% out(B_x,v) -- fact function for an named block and some variable.
\newcommand{\outBv}[2]{\ensuremath{\mathit{out}(B_{\ref{#1}}, #2)}\xspace}
\newcommand{\entryN}{\emph{E}\xspace}
\newcommand{\exitN}{\emph{X}\xspace}
\newcommand{\refNode}[1]{\ensuremath{B_{\ref{#1}}}\xspace}
\newcommand{\labelNode}[1]{\refstepcounter{nodeCounter}\label{#1}}
\newcommand{\setL}[1]{\textsc{#1}\xspace}
\newcommand{\setLC}{\setL{Const}}

%% Formats a list of facts
%% Argument should be like \facts{a/1, b/2, foobar/\bot, baz/\top}.
%% 
\newcounter{factctr}
\newtoks\varVal
\newtoks\varName
\newcommand{\facts}[1]{\begingroup%%
  %% Test if the argument given contains a forward slash (/). Expands
  %% slashTest with argument such that if a slash is NOT present the 
  %% token \noSlash will be given as argument 2 to slashTest. Otherwise
  %% there must be slash.
  \def\hasSlash##1{\expandafter\slashTest##1/\noslash\endslash}%%
  \def\slashTest##1/##2##3\endslash{\ifx\noslash##2 N\else Y\fi}%%
  \def\getArgs##1/##2{\varName={##1}%%
    \varVal={##2}}
  \ensuremath{%%
    \setcounter{factctr}{0}%%
    \foreach \var in {#1}{%%
      %% Separate list with a comma
      \ifthenelse{\value{factctr}>0}{,\allowbreak}{}%%
      %% \tracingmacros=1%%
      %% If key/val arguments, use first form. Otherwise
      %% use second.
      \ifthenelse{\equal{\hasSlash{\var}}{Y}}%%
                  {\expandafter\getArgs\var \factC{\the\varName}{\the\varVal}}%%
                  {\var}%%
      %% \tracingmacros=0%%
      \stepcounter{factctr}%%
    }}%%
\endgroup}
\newcommand{\factC}[2]{{\ensuremath{(\mathit{#1},#2)}}}
\newcommand{\doFacts}[4]{\ensuremath{#3{#1}: %%
    \left\{ %%
    \begin{minipage}[c]{#4}%%
      \facts{#2} %%
  \end{minipage}\kern -0.23em\right\}}}

\ExplSyntaxOn
\DeclareDocumentCommand \inFactsM {m m m} {\doFacts{#1}{#2}{\inB}{#3}}
\DeclareDocumentCommand \inFacts {m m O{1in}} {\doFacts{#1}{#2}{\inB}{#3}}
\DeclareDocumentCommand \outFactsM {m m m} {\doFacts{#1}{#2}{\outB}{#3}}
\DeclareDocumentCommand \outFacts {m m O{1in}} {\doFacts{#1}{#2}{\outB}{#3}}
\ExplSyntaxOff

\newcommand{\lub}{\ifthenelse{\boolean{mmode}}{\sqcap}{\raisebox{.1em}{\ensuremath{\sqcap}}}\xspace}
\newcommand{\sqlt}{\ensuremath{\sqsubset}\xspace}
\newcommand{\sqlte}{\ensuremath{\sqsubseteq}\xspace}

%% End dataflow

%% MIL Chapter
\newcommand{\compMILE}[1]{\ensuremath{\left\llbracket #1 \right\rrbracket}}
\newcommand{\compMILV}[1]{\ensuremath{\left\llbracket #1 \right\rrbracket}}
\newcommand{\compMILQ}[2]{\ensuremath{\left\llbracket #2 \right\rrbracket}}
\newcommand{\milCtx}[1]{\ensuremath{\llfloor}#1\ensuremath{\rrfloor}}
%% End MIL chapter

\newenvironment{myfig}[1][tbh]{\begin{figure}[#1]%%
\centering%%
\figbegin}{\figend%%
\end{figure}}

%% Produce a sub-caption and label it.
\newcommand{\scap}[2][1in]{\begin{minipage}{#1}%%
\subcaption{}\label{#2}\end{minipage}}

%% Produce a sub-caption with text.
\newcommand{\lscap}[3][1in]{\begin{minipage}{#1}%%
\subcaption{#3}\label{#2}\end{minipage}}

% single-argument comment. Do not put
% a space before the command when used
% or the file will have two spaces.
\newcommand{\rem}[1]{}

%% A verbatim environment with active charactesr
%% so we can use math shortcuts and macros
\DefineVerbatimEnvironment{AVerb}{Verbatim}{commandchars=\\\{\},%% 
  codes={\catcode`\_8\catcode`\$3\catcode`\^7},%%
  numberblanklines=false}

%% Turn on line numbers for Haskell code, 
%% and reset the line number counter.
\newcommand{\hsNumOn}{\numberson\numbersreset}
\newcommand{\hsNumOff}{\numbersoff}
%% Turn on line numbering in Haskell code within
%% the environment, then turn it off.
\newenvironment{withHsNum}{\numberson\numbersreset}{\numbersoff}

%% Paragraph run-in
\newcommand{\runin}[1]{\begingroup\noindent\sffamily\textbf{#1}\qquad\endgroup}

%% Chapter bibliographies
\newcommand{\standaloneBib}{%%
  \ifthenelse{\boolean{standaloneFlag}}%%
             {\begin{singlespace}
                \printbibliography
             \end{singlespace}}{}}

%% Adds an equation number on demand.
\newcommand\addtag{\refstepcounter{equation}\tag{\theequation}}

%% For typesetting set definitions like {x | x \in f(y)}
\newcommand\setdef[2]{\ensuremath{\{#1\ |\ #2\}}}

%% For typesetting function names like dom(f) or out(b).
\newcommand\mfun[1]{\ensuremath{\mathit{#1}}}

%% Marginal notes
\newcommand\margin[2]{\marginpar{\begin{singlespace}\emph{\footnotesize #2}\end{singlespace}}\relax #1}

%% Describe intent of a passage
\newcommand\intent[1]{{\leftskip = -1in\begin{singlespace}\emph{\noindent\footnotesize Intent: #1}\end{singlespace}}}

\begin{document}
\ifthenelse{\boolean{standaloneFlag}}
           {\VerbatimFootnotes
             \DefineShortVerb{\#}
             \setcounter{chapter}{0}}{}

%% Default float parameters. For case when
%% multiple chapters are included and
%% only one needs custom float settings.
\renewcommand{\textfraction}{0.2}
\renewcommand{\textfraction}{0.2}
\renewcommand{\topfraction}{0.9}


\chapter{A Monadic Intermediate Language}
\label{ref_chapter_mil}

%% What is an intermediate language? What is a ``Monadic'' one?
Most compilers do not generate executable machine code directly from a
program source file. Rather, programs get translated into one or more
\emph{intermediate forms}. The compiler may implement a pipeline of
translations, each translating the program into a more detailed (i.e.,
lower-level) representation. Frequently these intermediate forms are 
also languages, with their own syntax, semantics, type-systems,
and more. 

\section{Three-address Code}

Intermediate forms typically expose more detail about the
implementation of a program, while at the same time making some
optimization or transformation easier or even possible. 
\emph{Three-address code}, one such intermediate form, translates the
program into assembly-language like form, using registers to
hold values. Infinitely many registers can be named, making registers
more like memory locations than registers in real hardware. Each
instruction in the translated program has two operand registers and one
destination register, thus the name ``three-address.'' 

Three-address code makes all intermediate expression values explicit, 
by reducing complicated expressions to a series of assignments. 
For example, the expression:
\begin{equation}
  a = \frac{(b * c + d)}{2},
\end{equation}
would be expressed in three-address code as:
\begin{AVerb}
  s = b * c;
  t = s + d;
  a = t / 2;
\end{AVerb}
where #s# and #t# are new temporaries created by the compiler. This 
representation makes it easier for the compiler to re-order expressions,
unravel complex control-flow, and manipulate intermediate values. 

\section{Monadic Intermediate Language}

Our intermediate language, MIL, serves the same purpose as
three-address code and other intermediate forms: it exposes more
detail about the implementation of a program, while making some
optimizations simpler or even possible. In contrast to three-address
code, however, our language supports features unique to functional
languages: the ability to treat functions as first-class values, and
the representation of stateful computations in a monad.

\subsection{Monads \& Functional Programming}
As described by Wadler \citep{Wadler1990}, \emph{monads} can be used
distinguish \emph{pure} and \emph{impure} functions. A \emph{pure}
function has no side-effects: it will not print to the screen, throw
an exception, write to disk, or in any other way change the obversable
state of the machine. An \emph{impure} function may change the
machine's state.

%% Presentation drawn from http://en.wikipedia.org/wiki/Monad_%28functional_programming%29, 
%% accessed April 6 2010.
A \emph{monad} provides the abstraction that separates pure and impure
functions. Impure (or ``monadic'') functions execute ``inside'' the
monad. Values returned from a monadic function are not directly
accessible -- they are ``wrapped'' in the monad. The only way
to ``unwrap'' a monadic value is to execute the computation -- inside
the monad! 

\subsection{The Monad in MIL}

When designing MIL we wished to make all memory allocation
explicit. Besides the obvious effect of reducing free memory
available, allocation can also cause two other effects: the allocation
may fail, or a garbage-collection may occur. A monad allows us to
separate computations which (potentially) allocate memory from those
that do not.

\subsection{MIL Example: $compose$}

To give a sense of MIL, consider the definition of $compose$ given in
Figure~\ref{fig_mil1a}. Figure~\ref{fig_mil1b} shows a fragment of this 
expression in MIL. The \emph{block declaration}
on Line~\ref{fig_mil1b_block_decl} gives the name of
the block (#compose#) and arguments that will be passed in (#f#, #g#,
and #x#). Line~\ref{fig_mil1b_gofx} applies #g# to #x# and assigns
the result to #t1#. The ``enter'' operator (#@@#), represents function application.
\footnote{So called because in the expression #g @@ x#, we ``enter''
  function #g# with the argument #x#.}  We assume #g# refers to a
function (or, more precisely, a \emph{closure}). The ``bind'' operator
(#<-#) assigns the result of the operation on its right-hand side to
the location on the left. In turn, Line~\ref{fig_mil1b_fofx} applies
#f# to #t1# and assigns the result to #t2#. The last line returns
#t2#. Thus, the #compose# block returns the value of
\lamApPp{f}{\lamApp{g}{x}}, just as in our original \lamA expression.

\begin{myfig}[t]
  \begin{tabular}{cc}
    \subfloat{$compose = \lamCompose$%%
      \label{fig_mil1a}} & 
    \subfloat{\begin{minipage}{2in}%%
\begin{center}%%
\begin{minipage}{1in}%%
\begin{AVerb}[numbers=left]
compose(f,g,x) = \label{fig_mil1b_block_decl}
  t1 <- g @ x \label{fig_mil1b_gofx}
  t2 <- f @ t1 \label{fig_mil1b_fofx}
  return t2 \label{fig_mil1b_result}
\end{AVerb}
\end{minipage}%%
\end{center}%%
\end{minipage}%%
\label{fig_mil1b}} \\
    \subref{fig_mil1a} & \subref{fig_mil1b}
  \end{tabular} 
  \caption{\subref{fig_mil1a} gives a \lamA definition of the composition
    function; \subref{fig_mil1b} shows a fragment of the MIL program
    for $compose$.}
  \label{fig_mil1}
\end{myfig}

%% Closures

However, according to rules in Figure~\ref{lang_fig6},
Chapter~\ref{ref_chapter_languages} on page~\pageref{lang_fig6},
evaluating an expression which applies $compose$ actually involves the
creation of several intermediate values. Consider the expression
\begin{equation}
  main = \lamApp{\lamApp{\lamApp{compose}{a}}{b}}{c}, \label{eqn_mil4}
\end{equation}
where $a$, $b$ and $c$ are given values elsewhere. Using the
rules for call-by-value evaluation order from Figure~\ref{lang_fig6} in 
Chapter \ref{ref_chapter_languages}, we can compute the value of the expression
as follows:
\begin{align*}
  main &= \lamApp{\lamApp{\lamApp{compose}{a}}{b}}{c} \\
  &= \lamApp{\lamApp{\lamAPp{\lamCompose}{a}}{b}}{c} & \text{\emph{Definition of |compose|.}} \\
  &= \lamApp{\lamAPp{\lamAbs{g}{\lamAbs{x}{\lamApP{a}{\lamApp{g}{x}}}}}{b}}{c} & \text{\emph{E-App.}} \\
  &= \lamAPp{\lamAbs{x}{\lamApP{a}{\lamApp{b}{x}}}}{c} & \text{\emph{E-App.}} \\
  &= \lamApP{a}{\lamApp{b}{c}}. & \text{\emph{E-App.}} 
\end{align*}

We can capture each intermediate value created when evaluating this
expression by assigning each result to a new variable. 

\begin{align*}
  main &= \lamApp{\lamApp{\lamApp{compose}{a}}{b}}{c} \\
  &= \lamApp{\lamApp{\lamAPp{\lamCompose}{a}}{b}}{c} & \text{\emph{Definition of |compose|.}} \\
  t_1 &\leftarrow \lamAbs{g}{\lamAbs{x}{\lamApP{a}{\lamApp{g}{x}}}} & \text{\emph{Result of E-App.}}\\
  &= \lamApp{t_1}{\lamApp{b}{c}} \\
  t_2 &\leftarrow \lamAbs{x}{\lamApP{a}{\lamApp{b}{x}}} & \text{\emph{Result of E-App.}} \\
  &= \lamApp{t_2}{c} \\
  t_3 &\leftarrow \lamApP{a}{\lamApp{b}{c}} & \text{\emph{Result of E-App.}} \\
  &= t_3.
\end{align*}

We apply $t_1$ to $b$ to create our next intermediate value, $t_2$:
\begin{equation}
  t_2 = \lamApp{t_1}{b} = \lamAbs{x}{\lamApp{a}{\lamApp{b}{x}}}. \label{eqn_mil2}
\end{equation}
Finally, we compute our final value, $main$, by applying $t_2$ to $c$:
\begin{equation}
  main = \lamApp{t_2}{c} = \lamApp{a}{\lamApp{b}{c}}. \label{eqn_mil3}  
\end{equation}

Both $t_1$ and $t_2$ will hold \emph{closures} when evaluating
expression \eqref{eqn_mil4}. As detailed in Section \ref{subsec_lang2}, a closure
holds a pointer to a body of code and any \emph{free variables}. In this case,
$t_1$ holds $a$ and points to the code that evaulates to $t_2$. In turn, $t_2$
holds $a$ and $b$, and points to the code which evaluates to $main$. The
\lamA does not make this explicit, but our MIL does. 

\begin{myfig}[t]
  \begin{minipage}{5in}%%
\begin{center}%%
\begin{minipage}{4in}%%
\begin{AVerb}[numbers=left]
main(a,b,c) = \label{fig_mil2_main}
  t1 <- k1 @ a \label{fig_mil2_t1}
  t2 <- t1 @ b \label{fig_mil2_t2}
  t3 <- t2 @ c \label{fig_mil2_t3}
  return t3

k1 \{\} f = k2 \{f\} \label{fig_mil2_k1}
k2 \{f\} g = k3 \{f,g\} \label{fig_mil2_k2}
k3 \{f,g\} x = compose(f,g,x) \label{fig_mil2_k3}

compose(f,g,x) = \dots \emph{as in Figure \ref{fig_mil1b}} \dots 
\end{AVerb}
\end{minipage}%%
\end{center}%%
\end{minipage}%%

  \caption{The MIL program which computes $main = \lamApp{\lamApp{\lamApp{compose}{a}}{b}}{c}$. Note that $a$, $b$, and $c$ are assumed to be arguments given
    outside the program.}
  \label{fig_mil2}
\end{myfig}

Figure \ref{fig_mil2} shows the complete MIL program for $main =
\lamApp{\lamApp{\lamApp{compose}{a}}{b}}{c}$. #k1#, #k2# and #k3#
(lines \ref{fig_mil2_k1} -- \ref{fig_mil2_k3}) represent
\emph{closure-capturing} blocks. As opposed to #main#, these blocks
create new closures. In the definition #k1 {} f = k2 {f}#, the braces
on the left-hand side represent variables expected in the closure
given to this function. In this case, #k1# does not expect to find any
variables. #f# names the argument given to #k1#. The right-hand side,
#k2 {f}#, shows the creation of a new closure. The closure points to
#k2# and captures the value of #f#. In other words, evaluating #k1#
returns a closure which can be used to execute #k2#. #k2# behaves
similarly. It expects to find one value in its closure (#{f}#) and
returns a closure pointing to #k3# that copies the value #f# from the
existing closure and adds the argument, #g# (#k3 {f,g}#). #k3#,
however, does something new. Instead of returning a closure, it
executes the #compose# block (defined in Figure \ref{fig_mil1b}) with
three arguments: #f#, #g#, and #x#. This does \emph{not} return a
closure or ``enter'' a function. Instead, we jump directly to the
block. The value returned by #k3# will be the value computed by
#compose# with the arguments given.

Returning to #main# on line \ref{fig_mil2_main} in Figure
\ref{fig_mil2}, we can now see how MIL makes explicit the intemediate
closures created while evaluating
\lamApp{\lamApp{\lamApp{compose}{a}}{b}}{c}. On line
\ref{fig_mil2_t1}, we enter #k1# with the first argument, #a#. #t1#
holds the closure returned. On the next line, we enter #t1# (which
will point to #k2#) with the second argument, #b#. #t2# then holds the
closure returned. Finally, on line \ref{fig_mil2_t3}, we enter #t2#
(which will point to #k3#) with the final argument, #c#. #k3# will directly
execute #compose# with our specific arguments. #t3# holds the result returned
by #compose#. On the last line of #main# we return the value computed, #t3#.

%% Syntax of MIL
\subsection{MIL Syntax}

Figure \ref{fig_mil3} gives the syntax for MIL.  A MIL program
consists of a number of \emph{blocks}: \emph{closure} blocks (line
\ref{fig_mil3_k1}), basic blocks (line \ref{fig_mil3_b}) and top-level
blocks (line \ref{fig_mil3_t}). Though the syntax for closure blocks
seems to allow any tail, in practice they can only do one of two
things: either return a closure (\texttt{k \{\dots\}}) or jump to a
basic block (\texttt{b(\dots)}). Top-level blocks (line
\ref{fig_mil3_t}) provide an entry point for top-level functions --
they provide a closure which can be used to initially ``enter'' the
function.

\afterpage{\clearpage{\begin{myfig}
\begin{tabular}{r@{}lrr}
  \termrule block:{\ccblock k(v_1, \dots, v_n)v:\ \term body/}:Closure-Capturing Block/& \labeleq{mil_syntax_cc}\eqref{mil_syntax_cc} \\
  \termcase {\block b(v_1, \dots, v_n):\ \term body/}:Block/ &  \labeleq{mil_syntax_block}\eqref{mil_syntax_block} \\
  \termcase {\binds \lab t/\ <-\ \mkclo[k:];}:Top-level Definition/ &  \labeleq{mil_syntax_top}\eqref{mil_syntax_top} \\
  %% This row ensures the table fills the width of the page
  \multicolumn{4}{l}{\dimen0=-12pt \advance\dimen0\linewidth \hbox to \dimen0{}}\\

  \termrule body:\rlap{\begin{minipage}[t]{2in}\disableparspacing;%%
      \term stmt_1/\endgraf%%
      $\dots$\endgraf%%
      \term stmt_n/\end{minipage}}:Block Body/ &  \labeleq{mil_syntax_body}\eqref{mil_syntax_body} \\\\

  \termrule stmt:{\binds v\ <-\ \term tail/;}:Bind/ &  \labeleq{mil_syntax_stmt}\eqref{mil_syntax_stmt} \\
  \termcase {\begin{minipage}[t]{3in}\disableparspacing;%%
      \case v;\endgraf%%
      \quad \term alt_1/\endgraf%%
      \quad $\dots$\endgraf%%
      \quad \term alt_n/%%
  \end{minipage}}:Case Discrimination/ &  \labeleq{mil_syntax_case}\eqref{mil_syntax_case} \\
  \termcase \term tail/:Tail Expression/ &  \labeleq{mil_syntax_done}\eqref{mil_syntax_done} \\\\

  \termrule tail:{\return v;}:Return/ &  \labeleq{mil_syntax_return}\eqref{mil_syntax_return} \\
  \termcase \app v_1 * \ v_2/:Enter/ &  \labeleq{mil_syntax_enter}\eqref{mil_syntax_enter} \\
  \termcase \goto b(v_1, \dots, v_n):Goto Block/ &  \labeleq{mil_syntax_goto}\eqref{mil_syntax_goto} \\
  \termcase \prim p(v_1, \dots, v_n):Goto Primitive/ &  \labeleq{mil_syntax_prim}\eqref{mil_syntax_prim} \\
  \termcase \invoke v/:Execute Thunk/ &  \labeleq{mil_syntax_invoke}\eqref{mil_syntax_invoke} \\
  \termcase {\mkclo[k:v_1, \dots, v_n]}:Allocate Closure/ &  \labeleq{mil_syntax_clo}\eqref{mil_syntax_clo} \\
  \termcase {\mkthunk[m:v_1, \dots, v_n]}:Allocate Monadic Thunk/ &  \labeleq{mil_syntax_thunk}\eqref{mil_syntax_thunk}\\
  \termcase \ensurett{C\ v_1\ \dots\ v_n}:Allocate Data/ &  \labeleq{mil_syntax_cons}\eqref{mil_syntax_cons} \\\\

  \termrule alt:\alt C(v_1\ \dots\ v_n) -> \goto b(v_1, \dots, v_n);:Case Alternative/ &  \labeleq{mil_syntax_alt}\eqref{mil_syntax_alt}
  %% $alt$ := C $v_1$ \dots $v_n$ -> b(\dots) \\\\
  %% $v$ represents a variable, and only a variable. Expressions
  %% cannot be used where $v$ appears. \\ 

  %% \texttt{C $v_1$ \dots $v_n$} represents a constructor, for creating
  %% values in $tail$ expressions and inspecting values in $alt$ arms. \\
\end{tabular}
\caption{MIL syntax.}
\label{mil_fig3}
\end{myfig}
}\clearpage}

Basic blocks (line \ref{fig_mil3_body}) consist of a sequence of statements that
execute in order without any intra-block jumps or conditional
branches. Each basic block ends with a branch: either they return a
value (#done#) or take conditional branch (#case#). Conditional
branches can specify multiple destinations, though at any given time
only one will be taken.

The #case# statement (line \ref{fig_mil3_case}) specifies a list of
\emph{alternatives}, each of which matches a \emph{constructor} and
binds new variables to the values held by the constructor. #case#
examines the variable given (note, this cannot be an expression) and
selects the alternative that matches the constructor
found. Alternatives always branch immediately to some block -- they do
not allow any other statement. The result of block called becomes the
result of the #case#, which in turn becomese the result of the calling
block.

Only the binding statement (line \ref{fig_mil3_bind}) can appear multiple
times in a block. Each binding assigns the result of the \emph{tail}
on the right-hand side to a variable on the left. If a variable is
bound more than once, later bindings will ``shadow'' previous
bindings.

The #done# statement (line \ref{fig_mil3_done}) ends a block and returns
the value of tail expression specified.

\emph{Tail} expressions represent effects -- they create monadic
values. #return# (line \ref{fig_mil3_return}) takes a variable and
makes its value monadic. Notice it can only take a variable, not an
expression.  The ``enter'' operator, #@@#, expects a closure on its
left and some value on the right. It will enter the function pointed
to by the closure, with the argument given, and will evaluate to the
result of that function. #k#, the ``capture'' operator, creates a
closure from a block name and a list of variables. The name given is
not an arbitrary code pointer -- it is a location determined during
compilation. The ``goto'' expression, \texttt{b(\dots)}, jumps to the
particular block with the arguments given. Again, this is not a
computed value -- #b# represents a known location for the block. The
variables mentioned in the #goto# do not have to have the same names
as those given in the block's declaration. The constructor expression,
``C'', will create a data value with the given tag (``C'') and
variables. Primitives, which are not implemented in MIL, have the form
#p*# and are treated the same as ``goto'' expressions. They are not 
implemented in MIL, however. 

\section{Compiling \lamA to MIL}

\emph{\dots Lots of text \dots}

\section{Intermediate Languages, MIL, and Three-Address Code}

Intermediate langauges, and three-address code in particular, have at
least two purposes: making certain optimizations simpler, and exposing
more details about implementation. The intermediate language does
\emph{not} expose all details about implementation -- only those the
designers considered relevant. Three-address code accomplishes this
first goal by reducing the complexity of expresssions that need to be
analyzed. The second goal is achieved by deferring decisions about the
actual location of values to some later stage of compilation.
Finally, while not required, three-address code also easily adapts to
organizing code into basic blocks, which makes control-flow analysis
much simpler.

Our MIL shares some of the same goals as three-address code, and
accomplishes them in similar ways. Blocks do not have complex
structure -- they either return a closure, jump to another block, or
execute a series of statements followed by a return or branch. Tail
expressions ensure that all intermediate values are named, and also 
isolate monadic effects to one area of the language. Finally, the limited
number of statements ensure control-flow is straightforward.

\section{Conclusion}

This chapter presented our Monadic Intermediate Language (MIL). Our
MIL resembles three-address code in sevarl ways: infinitely many
registers can be named, nested expressions are not allowed, and
implementatino details are made explicit. The MIL's unique features
include separate representations for \emph{closure-capturing} and
basic blocks, and the use of monadic \emph{tail} expressions. We 
presented a simple scheme for compiling the \lamA given in
Chapter \ref{ref_chapter_languages} to our MIL. Later will be devoted
to optimizing those MIL programs using dataflow techniques.

%% Compiling the lambda-calculus to MIL

%% \section{Monadic Intermediate Language}

%% %% What does the language support?

%% Our monadic language takes its inspiration from Haskell's @do@
%% notation. It is a pure functional language, making allocation of data
%% structures and closures explicit via monadic syntax. Functions in MIL
%% define computations which, when run, can affect heap memory. Figure
%% \ref{figMILDef} gives the syntax of the language.

%% %% TODO: Mention that v restricts the term to variables
%% %% only.

%% MIL programs consist of a series of definitions (@defM@). Each
%% definition can be any of the following.

%% \begin{description}
%%   \item[Closure-capturing] (@k {v1, ..., vN} v = k1 {v1, ..., vN, v}@) -- This function
%%     expects to find the variables @v1, ..., vN@ in its own closure. It constructs
%%     a new closure containing the existing variables plus the newly captured variable
%%     @v@. The new closure refers to @k1@, another closure-capturing function.
%%   \item[Block-calling] (@k {v1, ..., vN} v = b(v1, ..., vN, v)@) -- This function immediately
%%     jumps to block @b@ with arguments @v1, ..., vN@ and @v@. No closure value needs to
%%     be constructed. 
%%   \item[Function block] (@b(v1, ..., vN) = bodyM@) -- This function executes the statements
%%     in the body. 
%%   \item[Top-level] (@t <- k {}@) -- This special case ensures top-level definitions in the program
%%     can be accessed like any other function. The notation indicates that @t@ holds a closure
%%     structure, referring to the definition @k@. 
%% \end{description}

%% Notice that we can distinguish syntatically between functions that
%% merely create a closure (@k { ... }@) and those that do actual work
%% (@b(...)@). The body of a @k@ functin can only allocate another
%% closure or jump to a block. A block, on the other hand, can do other
%% work, but it cannot directly return a closure. As will be described in
%% chapter \ref{ref_chapter_uncurrying} this makes it much easier to
%% recognize and elminate intermediate closures.

%% The body of each block consists of statements followed by a
%% \emph{tail}. Tails can only
%% appear as the last statement in a block or on the right-hand side of
%% the monadic arrow (``@<-@''). Tail instructions, in other words, cause 
%% effects. The three tail statements follow:

%% \begin{description}
%% \item[Return a computation] (@return v@) -- Returns the result of a computation
%%   to the caller.

%% \item[Create a closure] (@k {v1, ..., vN}@) -- Creates a closure pointing to
%%   function @k@, capturing variables @v1@ through @vN@.

%% \item[Enter a function] (@v1 @@ v2@) -- Enter the closure referred to by @v1@, with
%%   argument @v2@. In other words, function application. Note that @v1@ represents an
%%   \emph{unknown} function -- one for which we compute the address at run-time.

%% \item[Call a block] (@f(v1, ..., vN)@) -- Jump to the block labeled @f@ with the arguments
%%   given. In this case we know the function @f@ refers to and do not need to examine
%%   a closure in order to execute it.
%% \item[Create a value] (@C v1 ... vN@) -- Create a data value with tag @C@, holding
%%   the values found in variables @v1 ... vN@.
%% \end{description}

%% %% TODO: Describe alt syntax.

%% Statements in a block either bind the result of a tail statement 
%% (@v <- tailM@) or branch conditionally (@case v of ... @). Binding ``runs''
%% a computation and ``dereferences'' the result, placing
%% the value in a variable (e.g., @v@). That same variable can be bound
%% again later, but that does not affect previous uses of @v@. In essence, the old
%% name becomes hidden and its value inaccessible.

%% Though the syntax allows multiple @case@ statements in a function
%% body, only one can appear and it must be the last statement in the
%% body. The arms of the @case@ statement can only match on constructor
%% tags (@C@) and can only bind the constructor arguments to variables
%% (@v1 ... vN@). Each arm then jumps to a known block with those
%% variables as arguments. This choice makes compilation simpler.


%% %% \emph{Defines our monadic language and explains the terms in
%% %%   it. Example programs are given which illustrate closure construction
%% %%   and data allocation. The use of ``tail'' vs. statements is motivated
%% %%   and described. }

%% \emph{Need to talk about the monad we work in as well - what 
%% do bind and return mean?}

%% \section{Compiling to Our MIL}
%% \emph{A compilation scheme which uses Hoopls ``shapes'' is
%% described. This scheme will give use our initial, unoptimized
%% MIL program. An example (possibly |compose|, or |const3|) illustrates 
%% our scheme.}

\ifthenelse{\boolean{standaloneFlag}}
           {\bibliography{thesis}}{}

\end{document}


\chapter{The Hoopl Library}

\emph{Introduce the Hoopl library, describing how
it approaches dataflow analysis. Important concepts
such as shape, transfer and rewrite functions, facts and
lattices will be described. }

\documentclass[12pt]{report}
 %include polycode.fmt
\usepackage[T1]{fontenc}
\usepackage{calc}
%% \usepackage{fourier}
\usepackage{palatino}
\renewcommand\ttdefault{lmtt}
\usepackage{helvet}
%% \usepackage{inconsolata}
\usepackage{comment}
\usepackage{calc}
\usepackage{xspace}
\usepackage{verbatim}
\usepackage{url}
\usepackage{fancyvrb}
\usepackage{setspace}
\usepackage{amsmath}
\usepackage{booktabs}
\usepackage[margin=\parindent, format=hang,labelfont=bf]{caption}
%% \usepackage[subrefformat=parens]{subcaption}
%% The following makes sure we get parentheses around
%% subreferences. The newest version of the subcaption
%% package has an option for this, but that's not available
%% widely.
%%
%% From http://tex.stackexchange.com/questions/25644
\usepackage[labelformat=simple]{subcaption}
\makeatletter
  \def\thesubfigure{(\alph{subfigure})}
  \providecommand\thefigsubsep{~}
  \def\p@subfigure{\@nameuse{thefigure}\thefigsubsep}
\makeatother

\usepackage{ifthen}
\usepackage{stmaryrd}
\usepackage{longtable}
\usepackage{afterpage}
\usepackage{xifthen}
\usepackage{mathtools}
\usepackage{xparse}
\usepackage[natbib=true,style=authoryear,backend=bibtex8]{biblatex}
\setlength{\bibitemsep}{\bigskipamount}
\addbibresource{thesis.bib}
\usepackage{microtype}

\usepackage{tikz}
\usetikzlibrary{arrows,automata,positioning,calc}
%% Used for CFGs.
\tikzset{
  >=stealth, 
  node distance=.5in,
  stmt/.style={rectangle,
    draw=black, thick,        
    minimum height=2em,
    %% inner sep=2pt,
    %% text centered,
    %% node distance=.5in,
  },
  entex/.style={
    minimum height=2em,
    %% inner sep=2pt,
    %% text centered,
  },
  labelfor/.style={circle, 
    draw=black, thin,
    font={\footnotesize},
    inner sep=0,
    fill=white,
    above right=-1.5mm and -1.5mm of #1,
  },
  fact/.style={overlay},
  %% Invisible node
  invis/.style={inner sep=0pt, 
    minimum height=0em}, 
  table/.style={circle, fill=white,height=2mm}
}

%% GSO margins.
\usepackage[left=1.5in, right=1in, top=1in, bottom=1in]{geometry}
\usepackage{abstract}

%% GSO requires 12 pt font for all headings
\usepackage[bf,sf,tiny,compact]{titlesec}
\titleformat{\chapter}[display]
            {}% format
            {\sffamily\bfseries\chaptertitlename\ \thechapter}
            {\baselineskip}
            {\sffamily\bfseries}
            {}

\hyphenation{data-flow mo-na-dic} 

\newboolean{lhs2tex}
\setboolean{lhs2tex}{true}

% Used by included files to know they
% are NOT standalone
\newboolean{standaloneFlag}
\setboolean{standaloneFlag}{true}

\newlength{\rulefigmargin}
\setlength{\rulefigmargin}{2\parindent}

\newcommand\figbegin{\rule{\linewidth-\rulefigmargin}{0.4pt}\\\vspace{12pt}}
\newcommand\figend{\rule{\linewidth-\rulefigmargin}{0.4pt}}

%\providecommand{\citep}[1]{(\emph{#1})\xspace}
%\renewcommand{\cite}[1]{\emph{#1}\xspace}

%% Functional languages chapter commands
\newcommand{\lamA}{\ensuremath{\lambda}-calculus\xspace}
\newcommand{\LamA}{\ensuremath{\lambda}-Calculus\xspace}
\newcommand{\lamAbs}[2]{\ensuremath{\lambda#1.\ #2}}
\newcommand{\lamApp}[2]{\ensuremath{#1\ #2}}
\newcommand{\lamPApp}[2]{\ensuremath{(#1\ #2)}}
\newcommand{\lamAPp}[2]{\ensuremath{(#1)\ #2}}
\newcommand{\lamApP}[2]{\ensuremath{#1\ (#2)}}
\newcommand{\lamAPP}[2]{\ensuremath{(#1)\ (#2)}}
\let\lamApPp=\lamApP
\let\lamAppP=\lamAPp

\newcommand{\lamId}{\lamAbs{x}{x}}
\newcommand{\lamCompose}{\lamAbs{f}{\lamAbs{g}{\lamAbs{x}{\lamApp{f}{(\lamApp{g}{x})}}}}}
\newcommand{\machLam}{\ensuremath{M_\lambda}\xspace}
\newcommand{\compMach}[1]{\ensuremath{\left\llbracket #1 \right\rrbracket}}
\newcommand{\compRho}[1]{\ensuremath{\rho(#1)}}
\newcommand{\verSub}[2]{\ensuremath{#1_{#2}}}
\newcommand{\verSup}[2]{\ensuremath{#1^{#2}}}
\newcommand{\lamC}{\ensuremath{\lambda_C}\xspace}
\newcommand{\lamPlus}{\lamAbs{m}{\lamAbs{n}{\lamAbs{s}{\lamAbs{z}{\lamApp{m}{\lamApPp{s}{\lamApp{n}{\lamApp{s}{z}}}}}}}}}
%% Substitution notation -- [#1 -> #2]
\newcommand{\lamSubst}[2]{\ensuremath{[#1 \mapsto #2]}}
%% End functional languages chapter

%% Dataflow chapter commands
\newcounter{nodeCounter}[figure]
\newcommand{\inE}{\ensuremath{\mathit{in}}\xspace}
\newcommand{\out}{\ensuremath{\mathit{out}}\xspace}
\newcommand{\In}{\ensuremath{\mathit{In}}\xspace}
\newcommand{\InBa}{\ensuremath{\mathit{In}(B)}\xspace}
\newcommand{\Out}{\ensuremath{\mathit{Out}}\xspace}
%% Out(B_x) -- fact function for an named block.
\newcommand{\OutB}[1]{\ensuremath{\mathit{Out}(B_{\ref{#1}})}\xspace}
\newcommand{\OutBa}{\ensuremath{\mathit{Out}(B)}\xspace}
%% in(B) -- fact function for an anonymous block.
\newcommand{\inBa}{\ensuremath{\mathit{in}(B)}\xspace}
%% in(X) -- fact function for an anonymous block, but using a different variable.
\newcommand{\inXa}[1]{\ensuremath{\mathit{in}(#1)}\xspace}
%% in(B,v) -- fact function for an anonymous block and some variable.
\newcommand{\inBav}[1]{\ensuremath{\mathit{in}(B, #1)}\xspace}
%% in(B_x) -- fact function for an named block.
\newcommand{\inB}[1]{\ensuremath{\mathit{in}(B_{\ref{#1}})}\xspace}
%% in(B_x,v) -- fact function for an named block and some variable.
\newcommand{\inBv}[2]{\ensuremath{\mathit{in}(B_{\ref{#1}}, #2)}\xspace}
%% out(B) -- fact function for an anonymous block.
\newcommand{\outBa}{\ensuremath{\mathit{out}(B)}\xspace}
%% out(X) -- fact function for an anonymous block, but using a different variable.
\newcommand{\outXa}[1]{\ensuremath{\mathit{out}(#1)}\xspace}
%% out(B,v) -- fact function for an anonymous block and some variable.
\newcommand{\outBav}[1]{\ensuremath{\mathit{out}(B, #1)}\xspace}
%% out(B_x) -- fact function for an named block.
\newcommand{\outB}[1]{\ensuremath{\mathit{out}(B_{\ref{#1}})}\xspace}
%% out(B_x,v) -- fact function for an named block and some variable.
\newcommand{\outBv}[2]{\ensuremath{\mathit{out}(B_{\ref{#1}}, #2)}\xspace}
\newcommand{\entryN}{\emph{E}\xspace}
\newcommand{\exitN}{\emph{X}\xspace}
\newcommand{\refNode}[1]{\ensuremath{B_{\ref{#1}}}\xspace}
\newcommand{\labelNode}[1]{\refstepcounter{nodeCounter}\label{#1}}
\newcommand{\setL}[1]{\textsc{#1}\xspace}
\newcommand{\setLC}{\setL{Const}}

%% Formats a list of facts
%% Argument should be like \facts{a/1, b/2, foobar/\bot, baz/\top}.
%% 
\newcounter{factctr}
\newtoks\varVal
\newtoks\varName
\newcommand{\facts}[1]{\begingroup%%
  %% Test if the argument given contains a forward slash (/). Expands
  %% slashTest with argument such that if a slash is NOT present the 
  %% token \noSlash will be given as argument 2 to slashTest. Otherwise
  %% there must be slash.
  \def\hasSlash##1{\expandafter\slashTest##1/\noslash\endslash}%%
  \def\slashTest##1/##2##3\endslash{\ifx\noslash##2 N\else Y\fi}%%
  \def\getArgs##1/##2{\varName={##1}%%
    \varVal={##2}}
  \ensuremath{%%
    \setcounter{factctr}{0}%%
    \foreach \var in {#1}{%%
      %% Separate list with a comma
      \ifthenelse{\value{factctr}>0}{,\allowbreak}{}%%
      %% \tracingmacros=1%%
      %% If key/val arguments, use first form. Otherwise
      %% use second.
      \ifthenelse{\equal{\hasSlash{\var}}{Y}}%%
                  {\expandafter\getArgs\var \factC{\the\varName}{\the\varVal}}%%
                  {\var}%%
      %% \tracingmacros=0%%
      \stepcounter{factctr}%%
    }}%%
\endgroup}
\newcommand{\factC}[2]{{\ensuremath{(\mathit{#1},#2)}}}
\newcommand{\doFacts}[4]{\ensuremath{#3{#1}: %%
    \left\{ %%
    \begin{minipage}[c]{#4}%%
      \facts{#2} %%
  \end{minipage}\kern -0.23em\right\}}}

\ExplSyntaxOn
\DeclareDocumentCommand \inFactsM {m m m} {\doFacts{#1}{#2}{\inB}{#3}}
\DeclareDocumentCommand \inFacts {m m O{1in}} {\doFacts{#1}{#2}{\inB}{#3}}
\DeclareDocumentCommand \outFactsM {m m m} {\doFacts{#1}{#2}{\outB}{#3}}
\DeclareDocumentCommand \outFacts {m m O{1in}} {\doFacts{#1}{#2}{\outB}{#3}}
\ExplSyntaxOff

\newcommand{\lub}{\ifthenelse{\boolean{mmode}}{\sqcap}{\raisebox{.1em}{\ensuremath{\sqcap}}}\xspace}
\newcommand{\sqlt}{\ensuremath{\sqsubset}\xspace}
\newcommand{\sqlte}{\ensuremath{\sqsubseteq}\xspace}

%% End dataflow

%% MIL Chapter
\newcommand{\compMILE}[1]{\ensuremath{\left\llbracket #1 \right\rrbracket}}
\newcommand{\compMILV}[1]{\ensuremath{\left\llbracket #1 \right\rrbracket}}
\newcommand{\compMILQ}[2]{\ensuremath{\left\llbracket #2 \right\rrbracket}}
\newcommand{\milCtx}[1]{\ensuremath{\llfloor}#1\ensuremath{\rrfloor}}
%% End MIL chapter

\newenvironment{myfig}[1][tbh]{\begin{figure}[#1]%%
\centering%%
\figbegin}{\figend%%
\end{figure}}

%% Produce a sub-caption and label it.
\newcommand{\scap}[2][1in]{\begin{minipage}{#1}%%
\subcaption{}\label{#2}\end{minipage}}

%% Produce a sub-caption with text.
\newcommand{\lscap}[3][1in]{\begin{minipage}{#1}%%
\subcaption{#3}\label{#2}\end{minipage}}

% single-argument comment. Do not put
% a space before the command when used
% or the file will have two spaces.
\newcommand{\rem}[1]{}

%% A verbatim environment with active charactesr
%% so we can use math shortcuts and macros
\DefineVerbatimEnvironment{AVerb}{Verbatim}{commandchars=\\\{\},%% 
  codes={\catcode`\_8\catcode`\$3\catcode`\^7},%%
  numberblanklines=false}

%% Turn on line numbers for Haskell code, 
%% and reset the line number counter.
\newcommand{\hsNumOn}{\numberson\numbersreset}
\newcommand{\hsNumOff}{\numbersoff}
%% Turn on line numbering in Haskell code within
%% the environment, then turn it off.
\newenvironment{withHsNum}{\numberson\numbersreset}{\numbersoff}

%% Paragraph run-in
\newcommand{\runin}[1]{\begingroup\noindent\sffamily\textbf{#1}\qquad\endgroup}

%% Chapter bibliographies
\newcommand{\standaloneBib}{%%
  \ifthenelse{\boolean{standaloneFlag}}%%
             {\begin{singlespace}
                \printbibliography
             \end{singlespace}}{}}

%% Adds an equation number on demand.
\newcommand\addtag{\refstepcounter{equation}\tag{\theequation}}

%% For typesetting set definitions like {x | x \in f(y)}
\newcommand\setdef[2]{\ensuremath{\{#1\ |\ #2\}}}

%% For typesetting function names like dom(f) or out(b).
\newcommand\mfun[1]{\ensuremath{\mathit{#1}}}

%% Marginal notes
\newcommand\margin[2]{\marginpar{\begin{singlespace}\emph{\footnotesize #2}\end{singlespace}}\relax #1}

%% Describe intent of a passage
\newcommand\intent[1]{{\leftskip = -1in\begin{singlespace}\emph{\noindent\footnotesize Intent: #1}\end{singlespace}}}

\begin{document}
\ifthenelse{\boolean{standaloneFlag}}
           {\VerbatimFootnotes
             \DefineShortVerb{\#}
             \setcounter{chapter}{0}}{}

%% Default float parameters. For case when
%% multiple chapters are included and
%% only one needs custom float settings.
\renewcommand{\textfraction}{0.2}
\renewcommand{\textfraction}{0.2}
\renewcommand{\topfraction}{0.9}


\chapter{Uncurrying}
\label{ref_chapter_uncurrying}
%% \emph{Describes our optimization for collapsing intermediate
%% closures. Our choice of representation is analyzed to
%% show how it facilitates this optimization. We should show one
%% closure can be eliminated from a program and how the optimization
%% is applied over and over until a fixed point is reached. The format
%% for this section will vary from the other two.}

Functional languages permit definitions in two styles: \emph{curried}
and \emph{uncurried}. A curried function can be \emph{partially
  applied} --- it does not need to be given all of its arguments at
once. A function that takes the remaining arguments results from such
an application. An \emph{uncurried} function, however, must be given
all of its arguments at once. It cannot be partially applied. For
example, these Haskell fragments define @adder@ in curried style and
@divider@ in uncurried style:

\begin{code}
adder :: Int -> Int -> Int
adder a b = a + b

divider :: (Float, Float) -> Float
divider (a,b) = a / b
\end{code}

\noindent
The first definition lets us easily define specialized versions of @adder@, 
such as @add1@:

\begin{code}
add1 :: Int -> Int
add1 = adder 1
\end{code}

When applied to a single argument, @adder@ returns a function that we can re-use over and over. We cannot as easily define a simliar specialized function with @divider@.

%% Why is this a problem? Need more motivatin
The implementation of partial application, however, does come at a
cost. Each partial application requires that we construct a closure
over the arguments captured so far. That closure represents a function specialized
to the arguments given so far. In general, we don't know the address of the function
it contains when compiling -- only at run-time. Therefore, when the closure is
applied to the rest of the arguments, we cannot generate code that jumps to a
known address. Instead, we must look at the address in the closure at run-time and 
then jump. 

Because each function application @f x@ may result in another
function, the most general implementation strategy makes \emph{every}
application result in a closure. The compiler need only generate code
that inspects the closure constructed and jumps to the address
indicated. When a curried function is applied to all of its arguments
at once (e.g., @adder 1 2@), we get a chain of function calls where
most construct a closure and immediately return. It would be more
efficient to collect all arguments at once and immediately jump to the
function body. \emph{Uncurrying} is the transformation we use to turn 
fully-applied curried functions into direct calls to the function body.

%% TODO: Talk about how we can look for fully-applied forms
%% as a special case, but that is sub-optimal

%% TODO: What is an example of a fully-applied function that we cannot
%% recognize syntatically (very easily)?

\section{Example of Desired Optimization}

Recall from Section \ref{ref_foo} our definition of @foldr@:

\begin{code}
foldr :: (a -> b -> b) -> b -> [a] -> b
foldr f b (a:as) = foldr f (f a b) as
foldr f b []     = b
\end{code}

which compiles to the following blocks in our MIL:

\begin{code}

\end{code}

\section{Implementation}
\section{Reflection}
\subsection{Prior Work}

\end{document}


\documentclass[12pt]{report}
%include polycode.fmt
\usepackage[T1]{fontenc}
\usepackage{calc}
%% \usepackage{fourier}
\usepackage{palatino}
\renewcommand\ttdefault{lmtt}
\usepackage{helvet}
%% \usepackage{inconsolata}
\usepackage{comment}
\usepackage{calc}
\usepackage{xspace}
\usepackage{verbatim}
\usepackage{url}
\usepackage{fancyvrb}
\usepackage{setspace}
\usepackage{amsmath}
\usepackage{booktabs}
\usepackage[margin=\parindent, format=hang,labelfont=bf]{caption}
%% \usepackage[subrefformat=parens]{subcaption}
%% The following makes sure we get parentheses around
%% subreferences. The newest version of the subcaption
%% package has an option for this, but that's not available
%% widely.
%%
%% From http://tex.stackexchange.com/questions/25644
\usepackage[labelformat=simple]{subcaption}
\makeatletter
  \def\thesubfigure{(\alph{subfigure})}
  \providecommand\thefigsubsep{~}
  \def\p@subfigure{\@nameuse{thefigure}\thefigsubsep}
\makeatother

\usepackage{ifthen}
\usepackage{stmaryrd}
\usepackage{longtable}
\usepackage{afterpage}
\usepackage{xifthen}
\usepackage{mathtools}
\usepackage{xparse}
\usepackage[natbib=true,style=authoryear,backend=bibtex8]{biblatex}
\setlength{\bibitemsep}{\bigskipamount}
\addbibresource{thesis.bib}
\usepackage{microtype}

\usepackage{tikz}
\usetikzlibrary{arrows,automata,positioning,calc}
%% Used for CFGs.
\tikzset{
  >=stealth, 
  node distance=.5in,
  stmt/.style={rectangle,
    draw=black, thick,        
    minimum height=2em,
    %% inner sep=2pt,
    %% text centered,
    %% node distance=.5in,
  },
  entex/.style={
    minimum height=2em,
    %% inner sep=2pt,
    %% text centered,
  },
  labelfor/.style={circle, 
    draw=black, thin,
    font={\footnotesize},
    inner sep=0,
    fill=white,
    above right=-1.5mm and -1.5mm of #1,
  },
  fact/.style={overlay},
  %% Invisible node
  invis/.style={inner sep=0pt, 
    minimum height=0em}, 
  table/.style={circle, fill=white,height=2mm}
}

%% GSO margins.
\usepackage[left=1.5in, right=1in, top=1in, bottom=1in]{geometry}
\usepackage{abstract}

%% GSO requires 12 pt font for all headings
\usepackage[bf,sf,tiny,compact]{titlesec}
\titleformat{\chapter}[display]
            {}% format
            {\sffamily\bfseries\chaptertitlename\ \thechapter}
            {\baselineskip}
            {\sffamily\bfseries}
            {}

\hyphenation{data-flow mo-na-dic} 

\newboolean{lhs2tex}
\setboolean{lhs2tex}{true}

% Used by included files to know they
% are NOT standalone
\newboolean{standaloneFlag}
\setboolean{standaloneFlag}{true}

\newlength{\rulefigmargin}
\setlength{\rulefigmargin}{2\parindent}

\newcommand\figbegin{\rule{\linewidth-\rulefigmargin}{0.4pt}\\\vspace{12pt}}
\newcommand\figend{\rule{\linewidth-\rulefigmargin}{0.4pt}}

%\providecommand{\citep}[1]{(\emph{#1})\xspace}
%\renewcommand{\cite}[1]{\emph{#1}\xspace}

%% Functional languages chapter commands
\newcommand{\lamA}{\ensuremath{\lambda}-calculus\xspace}
\newcommand{\LamA}{\ensuremath{\lambda}-Calculus\xspace}
\newcommand{\lamAbs}[2]{\ensuremath{\lambda#1.\ #2}}
\newcommand{\lamApp}[2]{\ensuremath{#1\ #2}}
\newcommand{\lamPApp}[2]{\ensuremath{(#1\ #2)}}
\newcommand{\lamAPp}[2]{\ensuremath{(#1)\ #2}}
\newcommand{\lamApP}[2]{\ensuremath{#1\ (#2)}}
\newcommand{\lamAPP}[2]{\ensuremath{(#1)\ (#2)}}
\let\lamApPp=\lamApP
\let\lamAppP=\lamAPp

\newcommand{\lamId}{\lamAbs{x}{x}}
\newcommand{\lamCompose}{\lamAbs{f}{\lamAbs{g}{\lamAbs{x}{\lamApp{f}{(\lamApp{g}{x})}}}}}
\newcommand{\machLam}{\ensuremath{M_\lambda}\xspace}
\newcommand{\compMach}[1]{\ensuremath{\left\llbracket #1 \right\rrbracket}}
\newcommand{\compRho}[1]{\ensuremath{\rho(#1)}}
\newcommand{\verSub}[2]{\ensuremath{#1_{#2}}}
\newcommand{\verSup}[2]{\ensuremath{#1^{#2}}}
\newcommand{\lamC}{\ensuremath{\lambda_C}\xspace}
\newcommand{\lamPlus}{\lamAbs{m}{\lamAbs{n}{\lamAbs{s}{\lamAbs{z}{\lamApp{m}{\lamApPp{s}{\lamApp{n}{\lamApp{s}{z}}}}}}}}}
%% Substitution notation -- [#1 -> #2]
\newcommand{\lamSubst}[2]{\ensuremath{[#1 \mapsto #2]}}
%% End functional languages chapter

%% Dataflow chapter commands
\newcounter{nodeCounter}[figure]
\newcommand{\inE}{\ensuremath{\mathit{in}}\xspace}
\newcommand{\out}{\ensuremath{\mathit{out}}\xspace}
\newcommand{\In}{\ensuremath{\mathit{In}}\xspace}
\newcommand{\InBa}{\ensuremath{\mathit{In}(B)}\xspace}
\newcommand{\Out}{\ensuremath{\mathit{Out}}\xspace}
%% Out(B_x) -- fact function for an named block.
\newcommand{\OutB}[1]{\ensuremath{\mathit{Out}(B_{\ref{#1}})}\xspace}
\newcommand{\OutBa}{\ensuremath{\mathit{Out}(B)}\xspace}
%% in(B) -- fact function for an anonymous block.
\newcommand{\inBa}{\ensuremath{\mathit{in}(B)}\xspace}
%% in(X) -- fact function for an anonymous block, but using a different variable.
\newcommand{\inXa}[1]{\ensuremath{\mathit{in}(#1)}\xspace}
%% in(B,v) -- fact function for an anonymous block and some variable.
\newcommand{\inBav}[1]{\ensuremath{\mathit{in}(B, #1)}\xspace}
%% in(B_x) -- fact function for an named block.
\newcommand{\inB}[1]{\ensuremath{\mathit{in}(B_{\ref{#1}})}\xspace}
%% in(B_x,v) -- fact function for an named block and some variable.
\newcommand{\inBv}[2]{\ensuremath{\mathit{in}(B_{\ref{#1}}, #2)}\xspace}
%% out(B) -- fact function for an anonymous block.
\newcommand{\outBa}{\ensuremath{\mathit{out}(B)}\xspace}
%% out(X) -- fact function for an anonymous block, but using a different variable.
\newcommand{\outXa}[1]{\ensuremath{\mathit{out}(#1)}\xspace}
%% out(B,v) -- fact function for an anonymous block and some variable.
\newcommand{\outBav}[1]{\ensuremath{\mathit{out}(B, #1)}\xspace}
%% out(B_x) -- fact function for an named block.
\newcommand{\outB}[1]{\ensuremath{\mathit{out}(B_{\ref{#1}})}\xspace}
%% out(B_x,v) -- fact function for an named block and some variable.
\newcommand{\outBv}[2]{\ensuremath{\mathit{out}(B_{\ref{#1}}, #2)}\xspace}
\newcommand{\entryN}{\emph{E}\xspace}
\newcommand{\exitN}{\emph{X}\xspace}
\newcommand{\refNode}[1]{\ensuremath{B_{\ref{#1}}}\xspace}
\newcommand{\labelNode}[1]{\refstepcounter{nodeCounter}\label{#1}}
\newcommand{\setL}[1]{\textsc{#1}\xspace}
\newcommand{\setLC}{\setL{Const}}

%% Formats a list of facts
%% Argument should be like \facts{a/1, b/2, foobar/\bot, baz/\top}.
%% 
\newcounter{factctr}
\newtoks\varVal
\newtoks\varName
\newcommand{\facts}[1]{\begingroup%%
  %% Test if the argument given contains a forward slash (/). Expands
  %% slashTest with argument such that if a slash is NOT present the 
  %% token \noSlash will be given as argument 2 to slashTest. Otherwise
  %% there must be slash.
  \def\hasSlash##1{\expandafter\slashTest##1/\noslash\endslash}%%
  \def\slashTest##1/##2##3\endslash{\ifx\noslash##2 N\else Y\fi}%%
  \def\getArgs##1/##2{\varName={##1}%%
    \varVal={##2}}
  \ensuremath{%%
    \setcounter{factctr}{0}%%
    \foreach \var in {#1}{%%
      %% Separate list with a comma
      \ifthenelse{\value{factctr}>0}{,\allowbreak}{}%%
      %% \tracingmacros=1%%
      %% If key/val arguments, use first form. Otherwise
      %% use second.
      \ifthenelse{\equal{\hasSlash{\var}}{Y}}%%
                  {\expandafter\getArgs\var \factC{\the\varName}{\the\varVal}}%%
                  {\var}%%
      %% \tracingmacros=0%%
      \stepcounter{factctr}%%
    }}%%
\endgroup}
\newcommand{\factC}[2]{{\ensuremath{(\mathit{#1},#2)}}}
\newcommand{\doFacts}[4]{\ensuremath{#3{#1}: %%
    \left\{ %%
    \begin{minipage}[c]{#4}%%
      \facts{#2} %%
  \end{minipage}\kern -0.23em\right\}}}

\ExplSyntaxOn
\DeclareDocumentCommand \inFactsM {m m m} {\doFacts{#1}{#2}{\inB}{#3}}
\DeclareDocumentCommand \inFacts {m m O{1in}} {\doFacts{#1}{#2}{\inB}{#3}}
\DeclareDocumentCommand \outFactsM {m m m} {\doFacts{#1}{#2}{\outB}{#3}}
\DeclareDocumentCommand \outFacts {m m O{1in}} {\doFacts{#1}{#2}{\outB}{#3}}
\ExplSyntaxOff

\newcommand{\lub}{\ifthenelse{\boolean{mmode}}{\sqcap}{\raisebox{.1em}{\ensuremath{\sqcap}}}\xspace}
\newcommand{\sqlt}{\ensuremath{\sqsubset}\xspace}
\newcommand{\sqlte}{\ensuremath{\sqsubseteq}\xspace}

%% End dataflow

%% MIL Chapter
\newcommand{\compMILE}[1]{\ensuremath{\left\llbracket #1 \right\rrbracket}}
\newcommand{\compMILV}[1]{\ensuremath{\left\llbracket #1 \right\rrbracket}}
\newcommand{\compMILQ}[2]{\ensuremath{\left\llbracket #2 \right\rrbracket}}
\newcommand{\milCtx}[1]{\ensuremath{\llfloor}#1\ensuremath{\rrfloor}}
%% End MIL chapter

\newenvironment{myfig}[1][tbh]{\begin{figure}[#1]%%
\centering%%
\figbegin}{\figend%%
\end{figure}}

%% Produce a sub-caption and label it.
\newcommand{\scap}[2][1in]{\begin{minipage}{#1}%%
\subcaption{}\label{#2}\end{minipage}}

%% Produce a sub-caption with text.
\newcommand{\lscap}[3][1in]{\begin{minipage}{#1}%%
\subcaption{#3}\label{#2}\end{minipage}}

% single-argument comment. Do not put
% a space before the command when used
% or the file will have two spaces.
\newcommand{\rem}[1]{}

%% A verbatim environment with active charactesr
%% so we can use math shortcuts and macros
\DefineVerbatimEnvironment{AVerb}{Verbatim}{commandchars=\\\{\},%% 
  codes={\catcode`\_8\catcode`\$3\catcode`\^7},%%
  numberblanklines=false}

%% Turn on line numbers for Haskell code, 
%% and reset the line number counter.
\newcommand{\hsNumOn}{\numberson\numbersreset}
\newcommand{\hsNumOff}{\numbersoff}
%% Turn on line numbering in Haskell code within
%% the environment, then turn it off.
\newenvironment{withHsNum}{\numberson\numbersreset}{\numbersoff}

%% Paragraph run-in
\newcommand{\runin}[1]{\begingroup\noindent\sffamily\textbf{#1}\qquad\endgroup}

%% Chapter bibliographies
\newcommand{\standaloneBib}{%%
  \ifthenelse{\boolean{standaloneFlag}}%%
             {\begin{singlespace}
                \printbibliography
             \end{singlespace}}{}}

%% Adds an equation number on demand.
\newcommand\addtag{\refstepcounter{equation}\tag{\theequation}}

%% For typesetting set definitions like {x | x \in f(y)}
\newcommand\setdef[2]{\ensuremath{\{#1\ |\ #2\}}}

%% For typesetting function names like dom(f) or out(b).
\newcommand\mfun[1]{\ensuremath{\mathit{#1}}}

%% Marginal notes
\newcommand\margin[2]{\marginpar{\begin{singlespace}\emph{\footnotesize #2}\end{singlespace}}\relax #1}

%% Describe intent of a passage
\newcommand\intent[1]{{\leftskip = -1in\begin{singlespace}\emph{\noindent\footnotesize Intent: #1}\end{singlespace}}}

\begin{document}
\ifthenelse{\boolean{standaloneFlag}}
           {\VerbatimFootnotes
             \DefineShortVerb{\#}
             \setcounter{chapter}{0}}{}

%% Default float parameters. For case when
%% multiple chapters are included and
%% only one needs custom float settings.
\renewcommand{\textfraction}{0.2}
\renewcommand{\textfraction}{0.2}
\renewcommand{\topfraction}{0.9}


\chapter{Dead-code elimination}

Dead-code elimination seeks to remove program statements that will not
be executed or which have no visible effect. It can be applied at
multiple points during compilation, as other transformations
frequently introduce dead code. Without dead-code elimination, many
other optimizations have minimal effect, since they leave behind code
which still executes even if it is no longer needed.

For our MIL, we want to eliminate two types of dead code: bindings and
blocks. Dead bindings represent allocations we do not need to
make. Dead blocks increase the size of our program without providing
any benefit. To illustrate these two types of dead code, consider
the following program:

\begin{code}
main = do
  a <- 1 + 2
  b <- block1(a)
  c <- a + a
  return c

block1 (a) = {-"\dots \emph{elided} \dots"-}
\end{code}

In |main2|, |a| and |c| will be used, but |b| will not. The binding to
|b| is therefore dead and can be eliminated.\footnote{If |block1| had
  a side effect this would not be the case, but we assume it is does
  not for now.} With |b| gone, our program becomes:

\begin{code}
main = do
  a <- 1 + 2
  c <- a + a
  return c

block1 (a) = {-"\dots \emph{elided} \dots"-}
\end{code}

In the new program, |block1| is not called and can be eliminated. This
not only demonstrates how dead-code can be removed, it also shows that
iteratively applying the optimization can result in multiple
eliminations.

\section{Eliminating Bindings}
\label{dead_sub_elim_bindings}

A binding can be eliminated if no references to the bound variable
occur after the initial binding. Within each block, we find all the
variables which are ``live'' (i.e., used after being bound) and then
eliminate any binding which is not in the live set. 
Hoopl traverses each block backwards, applying the transfer function in turn to each statement. 
Variables used are collected in the set of live
variables, while new bindings remove variables from the set. 

After gathering facts for a statement, Hoopl then applies the rewrite
function to the next statement. If the statement binds a variable that
is not in the live set, the statement can be eliminated. Otherwise the
statement is unchanged and Hoopl applies the transfer function to it,
gathering new facts.

% Tempoarily change comment style for this example.
%if style == poly
%subst comment a = "\texttt{\emph{" a "}}"
%endif

\begin{figure}[b]
  \begin{code}
    --  1. live: [c] 
    a <- block1(c) -- -- C
    --  2. live: [c, a]
    b <- block2(a) -- -- B
    --  3. live: [c]
    a <- block3(c) -- -- A
    --  4. live: [a] 
    return a 
    --  5. live: []
  \end{code}
  \caption{A program fragment with the ``live'' variable set annotated between each fragment.}
  \label{fig_dead1}
\end{figure}

To illustrate, consider the program fragment in Figure
\ref{fig_dead1}, where the live set is annotated between each
statement. For illustration, we do not rewrite any statements.
At the end of the block (where analysis begins), no
variables are live (\#5). After |return a|, |a| is in live set
(\#4). However, the next statement, |a <- block3(c)|, removes
|a| from the live set and adds |c| (\#3). At point \#2, a new |a| (different than
the one at \#4) is added. That |a| is removed by the next statement, leaving 
only |c| in the live set (\#1). 

With rewriting, the program changes as Hoopl interleaves analysis and
rewriting. At statement #A# (the first to bind a variable),|a| is
bound and also appears in the live set (\#4). Therefore, we do not
elminate it. At #B#, |b| is bound but it does appear \emph{not} in the
live set (\#3), so we eliminate it. 

Hoopl's ability to analyze the \emph{updated} program works to our
advantage now. With line #B# eliminated, we also update the live
set. Figure \ref{fig_dead2} shows our updated program with new live
sets. 

\begin{figure}[h]
  \begin{code}
    --  1. live: [c] 
    a <- block1(c) -- -- C
    --  2. live: [c]
    a <- block3(c) -- -- A
    --  4. live: [a] 
    return a 
    --  5. live: []
  \end{code}
  \caption{Our program fragment from Figure \ref{fig_dead1} with line B removed.}
  \label{fig_dead2}
\end{figure}

Notice that the live set at \#2 no longer contains |a|. Our rewrite
function now considers line #C# with the update live sets and
eliminates it. Figure \ref{fig_dead3} shows the final program.

\begin{figure}[h]
  \begin{code}
    a <- block3(c)
    return a 
  \end{code}
  \caption{The final program with all dead bindings eliminated.}
  \label{fig_dead3}
\end{figure}

% Restore poly-style comments
%if style == poly
%subst comment a = "\mbox{\onelinecomment " a "}"
%endif

\subsection*{Safe Bindings}

As eluded to in this chapter's introduction, we can only eliminate
bindings that do not have any side-effects (except
allocation). Fortunately, all bindings with side-effects in MIL will
use the |Run| statement. As long as a binding does not takes its value
from a |Run| statement, it can be eliminated.

\subsection*{Implementation}

%let deadcode = True
%include Live.lhs

\section{Eliminating Blocks}

%% TODO - talk about direction of analysis, and more about
%% dataflow analysis.

We eliminate blocks that will never execute in order to decrease code
size and memory footprint. ``Dead'' blocks are found in much the same
way as dead variables, except we consider the whole program. Since MIL
does not have computed labels, we know that if a block is never
referenced in a |Goto| or |Closure| statement then it will never be
called. Any block which is referenced will be considered ``live''; all
other will be eliminated.

Our algorithm uses Hoopl to find all the blocks referenced within a
given block. From this set , we create a map of blocks to their
predecessors -- that is, for each block, we determine who calls that
block. This map allows us to determine which blocks have no
predecessors -- that is, they are not referernced. We eliminate those
blocks and repeat the analysis until no more blocks can be eliminated.

\subsection*{Implementation}

%include DeadBlocks.lhs

\section{Reflection}
%% \emph{What was good, what didn't work so well, and how Hoopl helped
%% or hindered the implementation}

The Hoopl library made the implementation of dead-code elimination
straightforward. Because it interleaves analysis and rewriting,
removing one useless binding can make other bindings useless, which
will in turn be removed.

%{

%% This ensures the "." in the forall statement does not
%% format as composition.

%format . = ".\ "

Unfortunately, the same cannot be said for dead block
elimination. Hoopl's interface does not allow a block to be removed
during rewrite. This is a consequence of the type our rewriting
function must have: |(FuelMonad m) => (forall e x. StatM e x -> Fact x
f -> m (Maybe (Graph StatM e x)))|.  The result type of this function
can be |Nothing| or |Just g|, where |g| is a new graph, with one or
more statements. |Nothing| means the statement is unchanged. |Just g|
does not let us express that the statement should be deleted in all
cases. When the type is |O O|, we can return |emptyGraph|. But when
the type is |C O| or |O C|, there is no constructor that lets us
delete the node. At best, we can replace statements with no-ops or
special-purpose ``markers'', but we cannot truly eliminate the block
during rewrite.

%}

Consequently, we have to implement our own iterative process to
eliminate all dead blocks (as shown in |deadBlocks| above), including
those which become dead when another block is eliminated.

\end{document}


\chapter{Monadic Optimizations}
\emph{Describes optimizations based on the monad laws: bind/return collapse and
  monadic fuzzbang (inlining)}

\section{Copy-propagation}
\emph{Collapsing ``|x <- return y; p|'' to ``|[y/x] p|''.}
\subsection{Example of Desired Optimization}
\subsection{Implementation}
\subsection{Reflection}

\section{Inlining}
\emph{Monadic inlining using the associativity law. That is:}

> y <- (z <- x; p1)
> p2

\noindent
\emph{becomes:}

> z <- x
> y <- p1
> p2

\subsection{Example of Desired Optimization}
\subsection{Implementation}
\subsection{Reflection}

\chapter{Lazy Code Motion}
\emph{Describes our implementation of LCM in terms of the four passes
  used. This section will give an overview of LCM and briefly describe
  each pass. We give a example program which will be used throughout
  the section.}

\section{Anticipated Expressions}

\section{Available Expressions}

\section{Dead-code Elimination}

\section{Reflection}

\emph{Conclusions regarding our implemenation.}

\chapter{Conclusion \& Future Work}

\emph{Where we started and where we wished we could have go to.}

\end{document}
