\documentclass[11pt]{article}
\usepackage{standalone}
%include polycode.fmt
\usepackage{sectsty}
\usepackage{palatino}
\usepackage[scaled=0.92]{helvet}
\usepackage{xspace}
\renewcommand\ttdefault{cmtt}
\allsectionsfont{\sffamily}
\usepackage{url}
\usepackage{fancyvrb}
\usepackage{setspace}
\usepackage{cmds}
\usepackage{ifthen}
\newboolean{lhs2tex}
\setboolean{lhs2tex}{true}
% Awkward Habit: Design and Optimization of A Monadic Intermediate Language
% \title{Dataflow Optimization of a Monadic Intermediate Language using Hoopl}
\title{Optimizing Monadic Programs using Hoopl}
%if False
\setboolean{lhs2tex}{false}
\author{Justin Bailey \\ \url{justinb@cs.pdx.edu}}
%else
\ifthenelse{\boolean{lhs2tex}}{\author{Justin Bailey \\ \url{justinb@@cs.pdx.edu}}}{}
%endif
\date{\today}

\begin{document}
\VerbatimFootnotes
\DefineShortVerb{\#}
\doublespacing

\maketitle

\section{Abstract}

This thesis describes a monadic intermediate language and how we used
the Hoopl library to implement optimizations for programs written in
it. We show that our monadic language makes it simpler to implement
optimizations, such as Lazy Code Motion, not normally applied to
functional languages. We also demonstrate an optimization that can
eliminate many intra- and inter-procedure closure allocations and
several optimizations based on the \emph{monad laws}.

\section{Audience}

\section{Introduction/Overview}

\section{Background}
\subsection{Dataflow Optimization}

%% A short section giving the history of dataflow optimization techniques
%% and basic concepts.

\subsubsection{Basic Blocks and Contrl--Flow Graphs}

\citet{SoAndSo}\comment{Gary Kindal?} first defined dataflow
optimization techniques. A dataflow optimization operates over a
``control-flow graph'' (CFG) of the program -- a directed graph where edges
encode branches or jumps and nodes represent statements. Programs run
by entering a node from a predecessor, executing the statements in
turn, and exiting the node to a successor. Multiple successors imply a
conditional branch, though the program can only choose one. A special
``entry'' node, with no predecssors, exists to give the program a
starting point.

The statements in each node must define a ``basic block,'' which means
there can only be one entry and one exit to the node. Each 
predeccessor starts at the same statement; execution cannot start in
the ``middle'' of the statements in the node. Each successor also
leaves from the same instruction, so only one ``branch'' can exist in
each node.

For example, consider the ``fall-through'' implied by the use of case
statements in this C language program fragment:

\begin{verbatim}
  switch(i) {
  case 1:
    printf("1");
    break;
  case 2:
    printf("2");
  case 3:
    printf("3");
  }

  printf("4");
\end{verbatim}

Figure \ref{switchCfgEg} shows a CFG for this fragment. Execution
begins at node A. Node C has two predeccessors: A and B. The edge
between Node B and C represents fall-through from the second to third
case. They cannot be combined because the node would need two distinct
entry points. Encoding a program into basic blocks usually involves
inserting similar branches. The CFG makes explicit control--flow that
exists by implication in the source program.

\begin{comment}
\begin{verbatim}
   A
  switch   ----<-
  | |  |  |      |
  | |  |  v C    ^
  | |   ->case 3 |
  | |     |      |
  | |      ->----_-- 
  | | B          |  |
  |  ->case 2 ->-   v
  |                 |
  |   D       ----<-
   ->case 1  |
     |       v
     v       |
   --------<-      
  |  E
   ->printf("3")
\end{verbatim}
\end{comment}

% What does dataflow mean?

% How do you use it? 

% Example

\subsection{Monadic Languages}

Introduce what a ``monadic'' language looks like and the advantages such a form
gives us.

\section{The Hoopl Library}

Explains the purpose of the Hoopl library and basic concepts (nodes,
graphs, rewrites and transfers)

\section{BC -- The Language of Blocks and Captures}

Defines, motivates and gives examples of the BC language.

\section{Compiling the \lamA to BC}

Demonstrates compiliation from a variant of the \lamA to BC. 

\section{Dataflow Optimization and the Hoopl Library}

Describes Hoopl and its use.

\section{Optimizing BC Programs Using Hoopl}

Catalog of optimizations implemented over BC using Hoopl.

\section{Eliminating Heap Allocation in BC using Hoopl}

Special attention to an optimization for eliminating heap allocation
in BC programs.

\section{Compiling Habit to BC to x86}

How it all came together.

\section{Conclusion \& Future Work}

\end{document}
