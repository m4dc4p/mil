\documentclass[11pt]{article}
\usepackage{standalone}
%include polycode.fmt
\usepackage{sectsty}
\usepackage{palatino}
\usepackage[scaled=0.92]{helvet}
\usepackage{xspace}
\renewcommand\ttdefault{cmtt}
\allsectionsfont{\sffamily}
\usepackage{url}
\usepackage{fancyvrb}
\usepackage{setspace}
\usepackage{cmds}
\usepackage{ifthen}
\newboolean{lhs2tex}
\setboolean{lhs2tex}{true}
% Awkward Habit: Design and Optimization of A Monadic Intermediate Language
% \title{Dataflow Optimization of a Monadic Intermediate Language using Hoopl}
\title{Optimizing Monadic Programs using Hoopl}
%if False
\setboolean{lhs2tex}{false}
\author{Justin Bailey \\ \url{justinb@cs.pdx.edu}}
%else
\ifthenelse{\boolean{lhs2tex}}{\author{Justin Bailey \\ \url{justinb@@cs.pdx.edu}}}{}
%endif
\date{\today}

\begin{document}
\VerbatimFootnotes
\DefineShortVerb{\#}
\doublespacing

\maketitle

\section{Abstract}

This thesis describes a monadic intermediate language and how we used
the Hoopl library to implement optimizations for programs written in
it. We show that our monadic language makes it simpler to implement
optimizations, such as Lazy Code Motion, not normally applied to
functional languages. We also demonstrate an optimization that can
eliminate many intra- and inter-procedure closure allocations and
several optimizations based on the \emph{monad laws}.

\section{Audience}

\section{Introduction/Overview}

\section{Background}
\subsection{Dataflow Optimization}

A short section giving the history of dataflow optimization techniques
and basic concepts.

\subsection{Monadic Languages}

Introduce what a ``monadic'' language looks like and the advantages such a form
gives us.

\section{The Hoopl Library}

Explains the purpose of the Hoopl library and basic concepts (nodes, graphs, rewrites and transfers)

\section{BC -- The Language of Blocks and Captures}

Defines, motivates and gives examples of the BC language.

\section{Compiling the \lamA to BC}

Demonstrates compiliation from a variant of the \lamA to BC. 

\section{Dataflow Optimization and the Hoopl Library}

Describes Hoopl and its use.

\section{Optimizing BC Programs Using Hoopl}

Catalog of optimizations implemented over BC using Hoopl.

\section{Eliminating Heap Allocation in BC using Hoopl}

Special attention to an optimization for eliminating heap allocation
in BC programs.

\section{Compiling Habit to BC to x86}

How it all came together.

\section{Conclusion \& Future Work}

\end{document}
