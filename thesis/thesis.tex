\documentclass[12pt]{report}
\usepackage{standalone}
%include polycode.fmt
\usepackage[T1]{fontenc}
\usepackage{calc}
\usepackage{palatino}
\usepackage{amsfonts}
\renewcommand\ttdefault{lmtt}
\usepackage{helvet}
\usepackage{xspace}
\usepackage{url}
\usepackage{fancyvrb}
\usepackage[doublespacing]{setspace}
%% below only necessary when using doublespacing -- corrects
%% the vertical space inserted when switching to singlespace
%% environment.
\def\correctspaceskip{\vskip-\baselineskip} 
\usepackage{amsmath}
\usepackage{booktabs}
\usepackage[margin=\parindent, format=hang,labelfont=bf]{caption}
%% \usepackage[subrefformat=parens]{subcaption}
%% The following makes sure we get parentheses around
%% subreferences. The newest version of the subcaption
%% package has an option for this, but that's not available
%% widely.
%%
%% From http://tex.stackexchange.com/questions/25644
\usepackage[labelformat=simple]{subcaption}
\makeatletter
  \def\thesubfigure{(\alph{subfigure})}
  \providecommand\thefigsubsep{~}
  \def\p@subfigure{\@nameuse{thefigure}\thefigsubsep}
\makeatother

\usepackage{ifthen}
\usepackage{stmaryrd}
\usepackage{longtable}
\usepackage{afterpage}
\usepackage{xifthen}
\usepackage{mathtools}
\usepackage[natbib=true,style=authoryear,backend=bibtex8]{biblatex}
\setlength{\bibitemsep}{\bigskipamount}
\addbibresource{thesis.bib}
\usepackage{microtype}

%% GSO margins.
\usepackage[left=1.5in, right=1in, top=1in, bottom=1in]{geometry}
\usepackage{abstract}

%% GSO requires 12 pt font for all headings
\usepackage[bf,sf,tiny,compact]{titlesec}
\titleformat{\chapter}[display]
            {}% format
            {\sffamily\bfseries\chaptertitlename\ \thechapter}
            {\baselineskip}
            {\sffamily\bfseries}
            {}

\hyphenation{data-flow mo-na-dic} 

%% Should unindent all haskell code set in a dispay (versus inline)
\makeatletter
  \@ifundefined{hscodestyle}
               {}
               {\renewcommand{\hscodestyle}{\advance\leftskip -\mathindent}}
\makeatother

% Used by included files to know they
% are NOT standalone
\newboolean{standaloneFlag}
\setboolean{standaloneFlag}{true}

\newlength{\rulefigmargin}
\setlength{\rulefigmargin}{2\parindent}

\newcommand\figbegin{\rule{\linewidth}{0.4pt}\\\vspace{12pt}}
\newcommand\figend{\rule{\linewidth}{0.4pt}}

%% Sets
\newcommand{\setL}[1]{\textsc{#1}\xspace}
\newcommand{\setLC}{\setL{Const}}

%% Lub, subset operators.
\protected\def\lub{\ifmmode\sqcap\else\raisebox{.1em}{\ensuremath{\sqcap}}\fi\xspace}
\newcommand{\sqlt}{\ensuremath{\sqsubset}\xspace}
\newcommand{\sqlte}{\ensuremath{\sqsubseteq}\xspace}

%% Subscripting with typewriter
\def\subtt#1{\ifmmode_{\ensurett{#1}}%%
  \else$_{\ensurett{#1}}$%%
  \fi}
%% Superscripting with typerwriter
\def\suptt#1{\ifmmode^{\ensurett{#1}}%%
  \else$^{\ensurett{#1}}$%%
  \fi}
%% Functional languages chapter commands
\newcommand{\lamA}{\ensuremath{\lambda}-calculus\xspace}
\newcommand{\LamA}{\ensuremath{\lambda}-Calculus\xspace}
\newcommand{\lamAbs}[2]{\ensuremath{\lambda#1.\ #2}}
\newcommand{\lamApp}[2]{\ensuremath{#1\ #2}}
\newcommand{\lamPApp}[2]{\ensuremath{(#1\ #2)}}
\newcommand{\lamAPp}[2]{\ensuremath{(#1)\ #2}}
\newcommand{\lamApP}[2]{\ensuremath{#1\ (#2)}}
\newcommand{\lamAPP}[2]{\ensuremath{(#1)\ (#2)}}
\let\lamApPp=\lamApP
\let\lamAppP=\lamAPp
%% LC definition
\newtoks\toksA
\protected\def\lcname#1/{\ensuremath{\mathit{#1}}}
\protected\def\lcdef#1(#2)=#3;{\def\arg{#2}%%
  \def\lcargs##1,##2/{\def\arg{##2}%%
    \ifx\empty\arg%%
    \lcname ##1/%%
    \else\lcname ##1/\ \lcargs ##2/%%
    \fi}%%
  \ifx\empty\arg\toksA={\ }%%
  \else\toksA={\ \lcargs #2,/\ }%%
  \fi%%
  \ensuremath{\lcname#1/\the\toksA =\ #3}}
%% Arbitary number of applied arguments, separated
%% by asterisks (*).
\protected\def\lcapp#1/{\def\lcappB##1*##2/{\def\arg{##2}%
    \ensuremath{\ifx\arg\empty%%
      \lcname ##1/%%
      \else%%
      \lcname##1/\ \lcappB##2/%%
      \fi}}%%
  %% Adding a star here makes
  %% sure our applicaitn always ends with an asterisks, ensuring
  %% #2 will be \empty at some point.
  \lcappB#1*/}
\protected\def\lcabs#1.{\ensuremath{\lambda#1.\ }}

\newcommand{\lamId}{\lamAbs{x}{x}}
\newcommand{\lamCompose}{\lamAbs{f}{\lamAbs{g}{\lamAbs{x}{\lamApp{f}{(\lamApp{g}{x})}}}}}
\newcommand{\machLam}{\ensuremath{M_\lambda}\xspace}
\newcommand{\compMach}[1]{\ensuremath{\left\llbracket #1 \right\rrbracket}}
\newcommand{\compRho}[1]{\ensuremath{\rho(#1)}}
\newcommand{\verSub}[2]{\ensuremath{#1_{#2}}}
\newcommand{\verSup}[2]{\ensuremath{#1^{#2}}}
\newcommand{\lamC}{\ensuremath{\lambda_C}\xspace}
\newcommand{\lamPlus}{\lamAbs{m}{\lamAbs{n}{\lamAbs{s}{\lamAbs{z}{\lamApp{m}{\lamApPp{s}{\lamApp{n}{\lamApp{s}{z}}}}}}}}}
%% Substitution notation -- [#1 -> #2]
\newcommand{\lamSubst}[2]{\ensuremath{[#1 \mapsto #2]}}
%% End functional languages chapter


%% MIL Chapter
\newcommand{\compMILE}[1]{\ensuremath{\left\llbracket #1 \right\rrbracket}}
\newcommand{\compMILV}[1]{\ensuremath{\left\llbracket #1 \right\rrbracket}}
\newcommand{\compMILQ}[2]{\ensuremath{\left\llbracket #2 \right\rrbracket}}
\newcommand{\milCtx}[1]{\ensuremath{\llfloor}#1\ensuremath{\rrfloor}}

%% This dimension makes sure the same amount of space
%% follows | and := in syntax rules like:
%%
%% term := var       (Variable)
%%      |  var var    (Application)
%%      |  \x. var    (Abstraction)
%%
\newdimen\termalign
\setbox0=\hbox{$:=$}
\termalign=\wd0 
\protected\def\term#1/{\ensuremath{\mathit{#1}}}
\protected\def\syntaxrule#1/{\hfil\text{\emph{#1}}}
\protected\long\def\termrule#1:#2:#3/{\term #1/ &\hbox{$:=$}\ensuremath{\ #2} & \syntaxrule #3/}
\protected\def\termcase#1:#2/{&\hbox to \termalign{$|$\hss}\ensuremath{\ #1} & \syntaxrule #2/}


%% End MIL chapter

%% Dataflow Chapter
% Domain function
\def\dom(#1){\ensuremath{\mfun{dom}(#1)}\xspace}
% Set of all integers.
\def\ZZ{\ensuremath{\mathbb{Z}}}
%%

%% Uncurrrying Chapter 
%% A space equal to a \thinspace, but we
%% can break a line at it.
\newskip\thinskipamt \thinskipamt=.16667em 
\protected\def\thinskip{\hskip \thinskipamt\relax}
\protected\def\thinnerskip{\hskip .5\thinskipamt\relax}
%% Capture a space token. Use a ``control-symbol'' (\. instead of \mksp)
%% to keep the trailing space from getting gobbled.
{\def\.{\global\let\sp= } \. }
%% Define \asp, which will capture the macro definition attached to space,
%% if one exists. Otherwise, \spa is relax after this.
{\catcode`\ =\active\gdef\asp{\ifx \relax\let\spa\relax\else\let\spa= \fi}}
\newtoks\foo
%% Removes spaces, implicit, active and explicit.
\protected\def\removespaces{\asp\afterassignment\removesp\let\next= }
\def\removesp{\foo={\next}\ifcat\noexpand\next\sp\foo={\removespace}%%
 \else\ifx\next\spa\foo={\removespaces}\fi%%
 \fi\the\foo}
%% MIL reserved word
\protected\def\milres#1/{\text{\ttfamily\bfseries #1}}
\protected\def\lab#1/{\textbf{\ensurett{\removespaces #1}}}
%% Constructs a closure: l { v1, ..., vN }
\protected\long\def\mkclo[#1:#2]{\lab #1/\ensuremath{\,\{\ensurett{#2}\}}\xspace}
%% Tuple version of closurs: {l: v1, ..., vN}.
\protected\long\def\clo[#1:#2]{\def\argA{#1}\def\argB{#2}\ensuremath{\{%%
      \ifx\argA\empty%%
      \else\lab #1/%%
        \ifx\argB\empty%%
        \else\ensurett{:\thinskip}%%
        \fi%%
      \fi\ensurett{#2}\}}\xspace}
%% Construct a thunk
\newbox\bracklbox \newbox\brackrbox
\setbox0=\hbox{$\{$} \setbox\bracklbox=\hbox to \wd0{\hfil[\kern0.25mm}
\setbox0=\hbox{$\}$} \setbox\brackrbox=\hbox to \wd0{\kern0.25mm]\hfil}
\protected\def\mkthunk[#1:#2]{\lab #1/%%
  \ensuremath{\,%%
    \mathopen{\copy\bracklbox}%%
    \ensurett{#2}%%
    \mathclose{\copy\brackrbox}\xspace}}
%% Binding statement: v <- {...}
\protected\def\binds#1<-#2;{\ensurett{\removespaces #1\texttt{<-}#2}\xspace}
%% In order to use \binds in verbatim environment, have to define
%% delimiters while they are active. The below defines \vbinds which
%% must be used in AVerb environments.  Notice the active space as
%% well - that is necessary so the space after \vbinds (and before the
%% first argument) in the verbatim environment gets eaten.
\begingroup\catcode`\!=\active \lccode`\!=`\< \lccode`\~=`\- 
  \catcode`\ =\active\lowercase{\endgroup\def\vbinds#1!~#2;}{\binds#1<-#2;}
%% Return statement: return ... ;
\protected\def\return#1;{\milres return/\ensurett{\ \removespaces #1}}
%% A closure capturing block. k {v1, ..., vN} x: ...
\protected\def\ccblock#1(#2)#3:{\lab#1/\ensuremath{\thinspace\{\ensurett{#2}\}}\ \ensurett{#3\hbox{:}}}
%% A normal block
\protected\def\block#1(#2):{\lab #1/\ensuremath{\thinspace(\ensurett{#2})}\ensurett{:}}
%% A goto expression
\protected\def\goto#1(#2){\lab #1/\thinspace\ensuremath{(\ensurett{#2})}}
%% An enter expression
\protected\def\app#1*#2/{\ensurett{\removespaces #1\ifmmode\ \fi{\text{\tt @}}\ifmmode\ \fi#2}}
\protected\def\bind{\texttt{<-}\xspace}
%% Primitive expression
\protected\def\prim#1(#2){\lab #1/\suptt*\ensuremath{(\ensurett{#2})}}
%% Program variable
\protected\def\var#1/{\ensurett{\removespaces #1}\xspace}
%% Case statement
\protected\def\case#1;{\milres case\ \ensuremath{\ensurett{\removespaces #1}}\ of/}
%% Case alternative
\protected\def\alt#1(#2)#3->#4;{\ensuremath{\ensurett{#1\ \ignorespaces#2\ \texttt{->}\ \ignorespaces #4}}}
%% Invoke
\protected\def\invoke#1/{\milres invoke/\ensurett{\ \removespaces #1}}
\def\rhs{right--hand side\xspace}
\def\lhs{left--hand side\xspace}
\def\enter{\texttt{@}\xspace}
\def\cc{closure--capturing\xspace}
\def\Cc{Closure--capturing\xspace}
%%

\newenvironment{myfig}[1][tbh]{\begin{figure}[#1]%%
\begin{singlespace}\centering%%
\figbegin}{\figend\end{singlespace}%%
\end{figure}}

%% Produce a sub-caption and label it.
\newcommand{\scap}[2][1in]{\begin{minipage}{#1}%%
\subcaption{}\label{#2}\end{minipage}}

%% Produce a sub-caption with text.
\newcommand{\lscap}[3][\hsize]{\begin{minipage}{#1}%%
\subcaption{#3}\label{#2}\end{minipage}}

% single-argument comment. Do not put
% a space before the command when used
% or the file will have two spaces.
\newcommand{\rem}[1]{}

%% A verbatim environment with active charactesr
%% so we can use math shortcuts and macros
\DefineVerbatimEnvironment{AVerb}{Verbatim}{commandchars=\\\{\},%% 
  codes={\catcode`\_8\catcode`\$3\catcode`\^7},%%
  numberblanklines=false}

\DefineVerbatimEnvironment{Verb}{Verbatim}{commandchars=\\\[\],%% 
  numberblanklines=false}

%% Turn on line numbers for Haskell code, 
%% and reset the line number counter.
\newcommand{\hsNumOn}{\numberson\numbersreset}
\newcommand{\hsNumOff}{\numbersoff}
%% Turn on line numbering in Haskell code within
%% the environment, then turn it off. The optional
%% argument specifies a prefix that \hslabel can
%% use to generate line number references. If no prefix
%% is givne, \hslabel will have no effect.
\newtoks\prefixtoks
\def\mkhslabel#1{\prefixtoks={#1}\let\prefix=a}
\def\hslabel#1{\ifx\prefix\relax%%
  \else\label{\the\prefixtoks_#1}%%
  \fi}
\def\unhslabel{\let\prefix=\relax}
\newenvironment{withHsNum}{\numberson\numbersreset}{\numbersoff}
\newenvironment{withHsLabeled}[1]{\numberson\numbersreset\mkhslabel{#1}}{\unhslabel\numbersoff}

%% Paragraph run-in
\newcommand{\runin}[1]{\begingroup\noindent\sffamily\textbf{#1}\qquad\endgroup}

%% Chapter bibliographies
\newcommand{\standaloneBib}{%%
  \ifthenelse{\boolean{standaloneFlag}}%%
             {\begin{singlespace}
                 \printbibliography
             \end{singlespace}}{}}

%% Adds an equation number on demand.
\newcommand\addtag{\refstepcounter{equation}\tag{\theequation}}

%% For typesetting set definitions like {x | x \in f(y)}
\newcommand\setdef[2]{\ensuremath{\{#1\ |\ #2\}}}

%% For typesetting function names like dom(f) or out(b).
\newcommand\mfun[1]{\ensuremath{\mathit{#1}}}

%% Marginal notes
\newcommand\margin[2]{\marginpar{\begin{singlespace}\emph{\footnotesize #2}\end{singlespace}}\relax #1}

%% Describe intent of a passage
\newcommand\intent[1]{{\begin{singlespace}\noindent\leftskip=-1in\emph{\footnotesize Intent: #1}\end{singlespace}}\nopagebreak[1]}

%% In aligned/alignedat/gathered environments, you don't et
%% automatice equation numbers. This command makes sure to
%% label them properly.
\newcommand\labeleq[1]{\refstepcounter{equation}\label{#1}}

%% Creates a hanging paragraph, where the first line is not
%% indented but all other lines are.
\def\itempar#1{\noindent\hangindent=\parindent\hangafter=1 #1\quad}

%% Disable overfull messages with ridiculous hfuzz value
\def\disableoverfull{\hfuzz=10in}

%% Set parfillskip so stretching does NOT occur at the end of
%% a paragraph (i.e., list of elements). Disable indent at beginning
%% of paragraph. Also turn off underfull hbox warnings.
%%
%% Intended to be used in a \vbox that forms part of a table or graphic,
%% which we want to be line-broken but not exactly like a normal paragraph.
\long\def\disableparspacing#1;{\def\arg{#1}\hbadness=100000\parindent=0pt\parfillskip=0pt\leftskip=0pt\rightskip=0pt%%
  \ifx\arg\empty\else\hsize=#1\relax\fi}
%% This stuff makes !+<text>+! write <text> in typewriter font.  

%% We make ! and + active characters early, then manipulate their
%% meaning to produce the right effect. Initially, + produces +. When
%% !  appears w/o a + following, it produces ``!''. When ``+''
%% follows, we start writing in teleteype (\ttfamily). The definition
%% of ``!'' changes to produce a bang. ``+'' changes such that it
%% looks for trailing ``!''. When no ``!'' appears, ``+'' produces ``+''. 
%% If a ``!'' appears, we shift out of \ttfamily (by ending the group) and
%% reset the meaning of ``!'' and ``+'' so we can start again.
\makeatletter
\let\mdplus=+\let\mdbang=!      %% Preserve meaning of + and ! so we can put them into document.
%% Turn off mark down for everyone
\outer\def\nomd{\catcode`!=12\catcode`+=12}
%% Turn mark down on for everyone
\outer\def\domd{\catcode`!=\active\catcode`+=\active %%
  \initialmd}
%% Use only with a group IMMEDIETALY following. Turns off
%% markdown for the group-to-come, without actually tokenizing the
%% group. If no group follows, this has no effect.
\protected\def\pausemd{\def\dopause{\catcode`!=12\catcode`+=12}%%
  \def\pausemdB{\ifx\next\bgroup%%
    %% A ``partial'' application of expandwith is used
    %% so we don't double up the group argument (which is what
    %% happens if we expand \next). This has the effect of 
    %% inserting \expandafter\dowith in front of the upcoming {. 
    %% If no brace is coming, \withmdC will have no effect.
    \def\pausemdC{\expandafter\dopause}
  \else
    \let\pausedmC=\relax
  \fi\pausemdC}
  %% \futurelet has to end the macro so it grabs the next token
  %% from the input file. Otherwise, it grabs it *from* this
  %% definition.
  \futurelet\next\pausemdB} %%
%% Turns markdown on for the group-to-come, without actually
%% tokenizing the group. Only has an effect when
%% used in front of a group, otherwise its a no-op.
\protected\def\withmd{\def\dowith{\catcode`!=\active\catcode`+=\active\initialmd}%%
  \def\withmdB{\ifx\next\bgroup %%
    %% A ``partial'' application of expandafter is used
    %% so we don't double up the group argument (which is what
    %% happens if we expand \next). This has the effect of 
    %% inserting \expandafter\dowith in front of the upcoming {. 
    %% If no brace is coming, \withmdC will have no effect.
      \def\withmdC{\expandafter\dowith} %%
    \else %%
      \let\withmdC=\relax %%
    \fi\withmdC}%%
  %% \futurelet has to end the macro so it grabs the next token
  %% from the input file. Otherwise, it grabs it *from* this
  %% definition.
  \futurelet\next\withmdB} %%
%% Make ! and + active in the following group so they have the right
%% catcode in the definitions to follow.
\catcode`!=\active\catcode`+=\active %%
%% Initial definitions associated with ! and +.
\def\initialmd{\protected\def!{\startTTA} %%
  \protected\def+{\stopTTA}} %%
%% Step 1 of startTT. Inital meaning of !; capture next token in \next, go to next step.
\def\startTTA{\futurelet\next\startTTB} %%
%% Step 2 of startTT. Compare captured token to + and go to step 3 if true. Otherwise
%% output a ! (since that started our macro), the argument captured and stop
%% processing.
\long\def\startTTB{\ifx\next+\expandafter\startTTC\expandafter\@gobble\else\mdbang\fi} %%
%% Step 3 of startTT. Shift into teletype mode and change definition of 
%% + and ! so we can stop processing.
\def\startTTC{\begingroup\ifmmode %%
  \let \math@bgroup \relax %%
  \def \math@egroup {\let \math@bgroup \@@math@bgroup %%
    \let \math@egroup \@@math@egroup} %%
  \mathtt\relax %%
  \else  %%
  \ttfamily\fi} %%
%% Step 1, 2  and 3 of stopTT follow the same pattern as startTT.
\def\stopTTA{\futurelet\next\stopTTB} %%
\long\def\stopTTB{\ifx\next!\expandafter\stopTTC\expandafter\@gobble\else\mdplus\fi} %%
\def\stopTTC{\endgroup}%%
\catcode`!=12\catcode`+=12
\makeatother

\domd

%% Place an input file on the next page
\def\onnextpage#1{\afterpage{\clearpage\input{#1}\clearpage}}

\setboolean{standaloneFlag}{false}

\begin{document}

\VerbatimFootnotes
\DefineShortVerb{\#}
\doublespacing
             
\input{frontmatter.f.tex}

\chapter{Introduction}

Compilers for imperative languages implement many optimizations using
\emph{dataflow analysis}. This method treats the program as a graph,
where edges represent execution paths and nodes represent statements
in the program. Dataflow analysis computes facts about each node and
then transforms the graph into an equivalent, yet faster (or smaller,
or more efficient, etc.) program. Optimizations which use dataflow
analysis include constant propagation, dead-code elimination,
common-subexpression elimination and many others.

Dataflow analysis on imperative programs arises naturally due to the
explicit flow-of-control from statement to statement. For pure
functional languages, with flow-of-control determined by evaluation
order, the fit seems more awkward. However, call-by-value, pure,
\emph{monadic} functional programs embody the best of both
styles: expression-based evaluation \emph{and} explicit
control-flow. 

Our work, then, defines a monadic language and optimizes programs in
it using dataflow analysis. We implement a number of optimizations
common to imperative and functional languages, including constant
propagation and dead-code elimination. We implement an optimization
which eliminates intermediate closure construction, showing that this
technique can be used for optimizations specific to functional
languages. Finally, we implement \emph{lazy code motion}, which to our
knowledge has not been applied to programs in a monadic language
before.

We use the Hoopl library\rem{reference} to implement our
optimizations. Besides showing that it is possible (and even
desirable) to use dataflow analysis in this context, our work also
serves as a case-study for advanced uses of Hoopl.

\chapter{Background}

%% A short section giving the history of dataflow optimization techniques
%% and basic concepts.

% Describe dataflow analysis in general terms and defines key
% concepts: basic blocks, control flow, facts, and
% rewrites. Bind/Return elimination is used as an an example.

\section{Dataflow Optimization}

\citet{SoAndSo} first described dataflow
analysis. It works by transforming a \emph{control-flow graph} (CFG)
representation of the program. This section introduces the concepts
necessary to understand how dataflow optimization works and describes
how the technique is usually implemented.

\subsection{Basic Blocks and Control-Flow Graphs}

A dataflow optimization operates over a ``control-flow graph'' (CFG)
of the program -- a directed graph where edges encode branches or
jumps and nodes represent statements. Programs run by entering a node
from a predecessor, executing the statements in turn, and exiting the
node to a successor. Multiple successors imply a conditional branch,
though the program can only choose one. A special ``entry'' node, with
no predecssors, exists to give the program a starting point.

The statements in each node must define a ``basic block,'' which means
there can only be one entry and one exit to the node. Each 
predeccessor starts at the same statement; execution cannot start in
the ``middle'' of the statements in the node. Each successor also
leaves from the same instruction, so only one ``branch'' can exist in
each node.

For example, consider the ``fall-through'' implied by the use of #case#
statements in this C-language program fragment:

\begin{verbatim}
  switch(i) {
  case 1:
    printf("1");
    break;
  case 2:
    printf("2");
  case 3:
    printf("3");
  }
\end{verbatim}

\begin{figure}[h]
\begin{verbatim}
   A
  switch   ----<-
  | |  |  |      |
  | |  |  v C    ^
  | |   ->case 3 |
  | |     |      |
  | |      ->----_-- 
  | | B          |  |
  |  ->case 2 ->-   v
  |                 |
  |   D       ----<-
   ->case 1  |
     |       v
     v       |
   --+-----<-      
  |  
   -> ...
\end{verbatim}
\caption{CFG illustrating \emph{fall-through} allowed by the
  C-language \texttt{switch} statement.}
\label{switchCfgEg}
\end{figure}

Figure \ref{switchCfgEg} shows a CFG for this fragment. Execution
begins at node A. Node C has two predeccessors: A and B. The edge
between Node B and C represents fall-through from the second to third
case. They cannot be combined because the node would need two distinct
entry points. Encoding a program into basic blocks usually involves
inserting similar branches. The CFG makes explicit control--flow that
exists by implication in the source program.

\subsection{Direction, Facts and Rewrites}

\subsection{Example: Bind/Return Collapse}

Dataflow optimizations transform the CFG representation of a program,
with the goal of making a faster (or smaller, or more efficient, etc.)
program. Dataflow computes a set of ``entry'' assumptions and ``exit''
facts for each node in the graph. Facts for one node become
assumptions for the nodes' successors (thus the term
``dataflow''). The algorithm iteratves over the entire graph until a
fixed point is reached -- that is, facts and assumptions no longer
change. The computed facts can then be used to transform the graph.

\emph{Constant propagation example -- or something more functional?}

\emph{Introduce forward and backwards dataflow.}

% What does dataflow mean?

% How do you use it? 

% Example

\chapter{The Hoopl Library}

\emph{Introduce the Hoopl library, describing how
it approaches dataflow analysis. Important concepts
such as shape, transfer and rewrite functions, facts and
lattices will be described. }

\chapter{Languages}

\emph{Defines the languages used in the thesis and a simple
  compilation scheme from a \lamA variant to our monadic language. }

\section{Source Language}

\emph{Defines a \lamA variant with some monadic effects, enough to
  illustrate interesting programs.}

\section{Monadic Intermediate Language}

\emph{Defines our monadic language and explains the terms in
  it. Example programs are given which illustrate closure construction
  and data allocation. The use of ``tail'' vs. statements is motivated
  and described. }

\emph{Need to talk about the monad we work in as well - what 
do bind and return mean?}

\section{Compiling to Our MIL}
\emph{A compilation scheme which uses Hoopls ``shapes'' is
described. This scheme will give use our initial, unoptimized
MIL program. An example (possibly |compose|, or |const3|) illustrates 
our scheme.}

\chapter{``Traditional'' Optimizations}

\emph{The bulk of our work. Each optimization implemented is
  described. Each subsection should follow a common recipe: describe
  the optimization, give an example program and show how we want it
  changed, show salient points about the optimization (with code
  snippets and references to the full library), and reflect on the
  implementation.  Lazy Code Motion may get its own section.}

\emph{One question: should optimizations be categorized by direction? We could
describe all forward optimizations, then all backward ones. That may help
set up the concepts necessary to describe LCM.}

\section{Constant-Propagation}
\emph{This portion gives an overview of the optimization, without
code or (much) notation.}

\subsection{Example of Desired Optimization}
\emph{A program is given and we show what we'd like it to be
transformed to.}

\subsection{Implementation}
\emph{``Interesting'' pieces of the implementation are described.}

\subsection{Reflection}
\emph{What was good, what didn't work so well, and how Hoopl helped
or hindered the implementation}

\section{Dead-code elimination}
\subsection{Example of Desired Optimization}
\subsection{Implementation}
\subsection{Reflection}

\chapter{Monadic Optimizations}
\emph{Describes optimizations based on the monad laws: bind/return collapse and
  monadic fuzzbang (inlining)}

\section{Copy-propagation}
\emph{Collapsing ``|x <- return y; p|'' to ``|[y/x] p|''.}
\subsection{Example of Desired Optimization}
\subsection{Implementation}
\subsection{Reflection}

\section{Inlining}
\emph{``FuzzBang'' --  monadic inlining. That is:}

> y <- (z <- x; p1)
> p2

\noindent
\emph{becomes:}

> z <- x
> y <- p1
> p2

\subsection{Example of Desired Optimization}
\subsection{Implementation}
\subsection{Reflection}

\chapter{Intermediate Closure Elimimation}
\emph{Describes our optimization for collapsing intermediate
closures. Our choice of representation is analyzed to
show how it facilitates this optimization. We should show one
closure can be eliminated from a program and how the optimization
is applied over and over until a fixed point is reached. The format
for this section will vary from the other two.}

\chapter{Implementing Lazy Code Motion}
\emph{Describes our implementation of LCM in terms of the four passes
  used. This section will give an overview of LCM and briefly describe
  each pass. We give a example program which will be used throughout
  the section.}

\section{Anticipated Expressions}

\section{Available Expressions}

\section{Dead-code Elimination}

\section{Reflection}

\emph{Conclusions regarding our implemenation.}

\chapter{Conclusion \& Future Work}

\emph{Where we started and where we wished we could have go to.}

\end{document}
