\documentclass[12pt]{report}
\usepackage{standalone}
%include polycode.fmt
\usepackage[T1]{fontenc}
\usepackage{calc}
\usepackage{palatino}
\usepackage{amsfonts}
\renewcommand\ttdefault{lmtt}
\usepackage{helvet}
\usepackage{xspace}
\usepackage{url}
\usepackage{fancyvrb}
\usepackage[doublespacing]{setspace}
%% below only necessary when using doublespacing -- corrects
%% the vertical space inserted when switching to singlespace
%% environment.
\def\correctspaceskip{\vskip-\baselineskip} 
\usepackage{amsmath}
\usepackage{booktabs}
\usepackage[margin=\parindent, format=hang,labelfont=bf]{caption}
%% \usepackage[subrefformat=parens]{subcaption}
%% The following makes sure we get parentheses around
%% subreferences. The newest version of the subcaption
%% package has an option for this, but that's not available
%% widely.
%%
%% From http://tex.stackexchange.com/questions/25644
\usepackage[labelformat=simple]{subcaption}
\makeatletter
  \def\thesubfigure{(\alph{subfigure})}
  \providecommand\thefigsubsep{~}
  \def\p@subfigure{\@nameuse{thefigure}\thefigsubsep}
\makeatother

\usepackage{ifthen}
\usepackage{stmaryrd}
\usepackage{longtable}
\usepackage{afterpage}
\usepackage{xifthen}
\usepackage{mathtools}
\usepackage[natbib=true,style=authoryear,backend=bibtex8]{biblatex}
\setlength{\bibitemsep}{\bigskipamount}
\addbibresource{thesis.bib}
\usepackage{microtype}

%% GSO margins.
\usepackage[left=1.5in, right=1in, top=1in, bottom=1in]{geometry}
\usepackage{abstract}

%% GSO requires 12 pt font for all headings
\usepackage[bf,sf,tiny,compact]{titlesec}
\titleformat{\chapter}[display]
            {}% format
            {\sffamily\bfseries\chaptertitlename\ \thechapter}
            {\baselineskip}
            {\sffamily\bfseries}
            {}

\hyphenation{data-flow mo-na-dic} 

%% Should unindent all haskell code set in a dispay (versus inline)
\makeatletter
  \@ifundefined{hscodestyle}
               {}
               {\renewcommand{\hscodestyle}{\advance\leftskip -\mathindent}}
\makeatother

% Used by included files to know they
% are NOT standalone
\newboolean{standaloneFlag}
\setboolean{standaloneFlag}{true}

\newlength{\rulefigmargin}
\setlength{\rulefigmargin}{2\parindent}

\newcommand\figbegin{\rule{\linewidth}{0.4pt}\\\vspace{12pt}}
\newcommand\figend{\rule{\linewidth}{0.4pt}}

%% Sets
\newcommand{\setL}[1]{\textsc{#1}\xspace}
\newcommand{\setLC}{\setL{Const}}

%% Lub, subset operators.
\protected\def\lub{\ifmmode\sqcap\else\raisebox{.1em}{\ensuremath{\sqcap}}\fi\xspace}
\newcommand{\sqlt}{\ensuremath{\sqsubset}\xspace}
\newcommand{\sqlte}{\ensuremath{\sqsubseteq}\xspace}

%% Subscripting with typewriter
\def\subtt#1{\ifmmode_{\ensurett{#1}}%%
  \else$_{\ensurett{#1}}$%%
  \fi}
%% Superscripting with typerwriter
\def\suptt#1{\ifmmode^{\ensurett{#1}}%%
  \else$^{\ensurett{#1}}$%%
  \fi}
%% Functional languages chapter commands
\newcommand{\lamA}{\ensuremath{\lambda}-calculus\xspace}
\newcommand{\LamA}{\ensuremath{\lambda}-Calculus\xspace}
\newcommand{\lamAbs}[2]{\ensuremath{\lambda#1.\ #2}}
\newcommand{\lamApp}[2]{\ensuremath{#1\ #2}}
\newcommand{\lamPApp}[2]{\ensuremath{(#1\ #2)}}
\newcommand{\lamAPp}[2]{\ensuremath{(#1)\ #2}}
\newcommand{\lamApP}[2]{\ensuremath{#1\ (#2)}}
\newcommand{\lamAPP}[2]{\ensuremath{(#1)\ (#2)}}
\let\lamApPp=\lamApP
\let\lamAppP=\lamAPp
%% LC definition
\newtoks\toksA
\protected\def\lcname#1/{\ensuremath{\mathit{#1}}}
\protected\def\lcdef#1(#2)=#3;{\def\arg{#2}%%
  \def\lcargs##1,##2/{\def\arg{##2}%%
    \ifx\empty\arg%%
    \lcname ##1/%%
    \else\lcname ##1/\ \lcargs ##2/%%
    \fi}%%
  \ifx\empty\arg\toksA={\ }%%
  \else\toksA={\ \lcargs #2,/\ }%%
  \fi%%
  \ensuremath{\lcname#1/\the\toksA =\ #3}}
%% Arbitary number of applied arguments, separated
%% by asterisks (*).
\protected\def\lcapp#1/{\def\lcappB##1*##2/{\def\arg{##2}%
    \ensuremath{\ifx\arg\empty%%
      \lcname ##1/%%
      \else%%
      \lcname##1/\ \lcappB##2/%%
      \fi}}%%
  %% Adding a star here makes
  %% sure our applicaitn always ends with an asterisks, ensuring
  %% #2 will be \empty at some point.
  \lcappB#1*/}
\protected\def\lcabs#1.{\ensuremath{\lambda#1.\ }}

\newcommand{\lamId}{\lamAbs{x}{x}}
\newcommand{\lamCompose}{\lamAbs{f}{\lamAbs{g}{\lamAbs{x}{\lamApp{f}{(\lamApp{g}{x})}}}}}
\newcommand{\machLam}{\ensuremath{M_\lambda}\xspace}
\newcommand{\compMach}[1]{\ensuremath{\left\llbracket #1 \right\rrbracket}}
\newcommand{\compRho}[1]{\ensuremath{\rho(#1)}}
\newcommand{\verSub}[2]{\ensuremath{#1_{#2}}}
\newcommand{\verSup}[2]{\ensuremath{#1^{#2}}}
\newcommand{\lamC}{\ensuremath{\lambda_C}\xspace}
\newcommand{\lamPlus}{\lamAbs{m}{\lamAbs{n}{\lamAbs{s}{\lamAbs{z}{\lamApp{m}{\lamApPp{s}{\lamApp{n}{\lamApp{s}{z}}}}}}}}}
%% Substitution notation -- [#1 -> #2]
\newcommand{\lamSubst}[2]{\ensuremath{[#1 \mapsto #2]}}
%% End functional languages chapter


%% MIL Chapter
\newcommand{\compMILE}[1]{\ensuremath{\left\llbracket #1 \right\rrbracket}}
\newcommand{\compMILV}[1]{\ensuremath{\left\llbracket #1 \right\rrbracket}}
\newcommand{\compMILQ}[2]{\ensuremath{\left\llbracket #2 \right\rrbracket}}
\newcommand{\milCtx}[1]{\ensuremath{\llfloor}#1\ensuremath{\rrfloor}}

%% This dimension makes sure the same amount of space
%% follows | and := in syntax rules like:
%%
%% term := var       (Variable)
%%      |  var var    (Application)
%%      |  \x. var    (Abstraction)
%%
\newdimen\termalign
\setbox0=\hbox{$:=$}
\termalign=\wd0 
\protected\def\term#1/{\ensuremath{\mathit{#1}}}
\protected\def\syntaxrule#1/{\hfil\text{\emph{#1}}}
\protected\long\def\termrule#1:#2:#3/{\term #1/ &\hbox{$:=$}\ensuremath{\ #2} & \syntaxrule #3/}
\protected\def\termcase#1:#2/{&\hbox to \termalign{$|$\hss}\ensuremath{\ #1} & \syntaxrule #2/}


%% End MIL chapter

%% Dataflow Chapter
% Domain function
\def\dom(#1){\ensuremath{\mfun{dom}(#1)}\xspace}
% Set of all integers.
\def\ZZ{\ensuremath{\mathbb{Z}}}
%%

%% Uncurrrying Chapter 
%% A space equal to a \thinspace, but we
%% can break a line at it.
\newskip\thinskipamt \thinskipamt=.16667em 
\protected\def\thinskip{\hskip \thinskipamt\relax}
\protected\def\thinnerskip{\hskip .5\thinskipamt\relax}
%% Capture a space token. Use a ``control-symbol'' (\. instead of \mksp)
%% to keep the trailing space from getting gobbled.
{\def\.{\global\let\sp= } \. }
%% Define \asp, which will capture the macro definition attached to space,
%% if one exists. Otherwise, \spa is relax after this.
{\catcode`\ =\active\gdef\asp{\ifx \relax\let\spa\relax\else\let\spa= \fi}}
\newtoks\foo
%% Removes spaces, implicit, active and explicit.
\protected\def\removespaces{\asp\afterassignment\removesp\let\next= }
\def\removesp{\foo={\next}\ifcat\noexpand\next\sp\foo={\removespace}%%
 \else\ifx\next\spa\foo={\removespaces}\fi%%
 \fi\the\foo}
%% MIL reserved word
\protected\def\milres#1/{\text{\ttfamily\bfseries #1}}
\protected\def\lab#1/{\textbf{\ensurett{\removespaces #1}}}
%% Constructs a closure: l { v1, ..., vN }
\protected\long\def\mkclo[#1:#2]{\lab #1/\ensuremath{\,\{\ensurett{#2}\}}\xspace}
%% Tuple version of closurs: {l: v1, ..., vN}.
\protected\long\def\clo[#1:#2]{\def\argA{#1}\def\argB{#2}\ensuremath{\{%%
      \ifx\argA\empty%%
      \else\lab #1/%%
        \ifx\argB\empty%%
        \else\ensurett{:\thinskip}%%
        \fi%%
      \fi\ensurett{#2}\}}\xspace}
%% Construct a thunk
\newbox\bracklbox \newbox\brackrbox
\setbox0=\hbox{$\{$} \setbox\bracklbox=\hbox to \wd0{\hfil[\kern0.25mm}
\setbox0=\hbox{$\}$} \setbox\brackrbox=\hbox to \wd0{\kern0.25mm]\hfil}
\protected\def\mkthunk[#1:#2]{\lab #1/%%
  \ensuremath{\,%%
    \mathopen{\copy\bracklbox}%%
    \ensurett{#2}%%
    \mathclose{\copy\brackrbox}\xspace}}
%% Binding statement: v <- {...}
\protected\def\binds#1<-#2;{\ensurett{\removespaces #1\texttt{<-}#2}\xspace}
%% In order to use \binds in verbatim environment, have to define
%% delimiters while they are active. The below defines \vbinds which
%% must be used in AVerb environments.  Notice the active space as
%% well - that is necessary so the space after \vbinds (and before the
%% first argument) in the verbatim environment gets eaten.
\begingroup\catcode`\!=\active \lccode`\!=`\< \lccode`\~=`\- 
  \catcode`\ =\active\lowercase{\endgroup\def\vbinds#1!~#2;}{\binds#1<-#2;}
%% Return statement: return ... ;
\protected\def\return#1;{\milres return/\ensurett{\ \removespaces #1}}
%% A closure capturing block. k {v1, ..., vN} x: ...
\protected\def\ccblock#1(#2)#3:{\lab#1/\ensuremath{\thinspace\{\ensurett{#2}\}}\ \ensurett{#3\hbox{:}}}
%% A normal block
\protected\def\block#1(#2):{\lab #1/\ensuremath{\thinspace(\ensurett{#2})}\ensurett{:}}
%% A goto expression
\protected\def\goto#1(#2){\lab #1/\thinspace\ensuremath{(\ensurett{#2})}}
%% An enter expression
\protected\def\app#1*#2/{\ensurett{\removespaces #1\ifmmode\ \fi{\text{\tt @}}\ifmmode\ \fi#2}}
\protected\def\bind{\texttt{<-}\xspace}
%% Primitive expression
\protected\def\prim#1(#2){\lab #1/\suptt*\ensuremath{(\ensurett{#2})}}
%% Program variable
\protected\def\var#1/{\ensurett{\removespaces #1}\xspace}
%% Case statement
\protected\def\case#1;{\milres case\ \ensuremath{\ensurett{\removespaces #1}}\ of/}
%% Case alternative
\protected\def\alt#1(#2)#3->#4;{\ensuremath{\ensurett{#1\ \ignorespaces#2\ \texttt{->}\ \ignorespaces #4}}}
%% Invoke
\protected\def\invoke#1/{\milres invoke/\ensurett{\ \removespaces #1}}
\def\rhs{right--hand side\xspace}
\def\lhs{left--hand side\xspace}
\def\enter{\texttt{@}\xspace}
\def\cc{closure--capturing\xspace}
\def\Cc{Closure--capturing\xspace}
%%

\newenvironment{myfig}[1][tbh]{\begin{figure}[#1]%%
\begin{singlespace}\centering%%
\figbegin}{\figend\end{singlespace}%%
\end{figure}}

%% Produce a sub-caption and label it.
\newcommand{\scap}[2][1in]{\begin{minipage}{#1}%%
\subcaption{}\label{#2}\end{minipage}}

%% Produce a sub-caption with text.
\newcommand{\lscap}[3][\hsize]{\begin{minipage}{#1}%%
\subcaption{#3}\label{#2}\end{minipage}}

% single-argument comment. Do not put
% a space before the command when used
% or the file will have two spaces.
\newcommand{\rem}[1]{}

%% A verbatim environment with active charactesr
%% so we can use math shortcuts and macros
\DefineVerbatimEnvironment{AVerb}{Verbatim}{commandchars=\\\{\},%% 
  codes={\catcode`\_8\catcode`\$3\catcode`\^7},%%
  numberblanklines=false}

\DefineVerbatimEnvironment{Verb}{Verbatim}{commandchars=\\\[\],%% 
  numberblanklines=false}

%% Turn on line numbers for Haskell code, 
%% and reset the line number counter.
\newcommand{\hsNumOn}{\numberson\numbersreset}
\newcommand{\hsNumOff}{\numbersoff}
%% Turn on line numbering in Haskell code within
%% the environment, then turn it off. The optional
%% argument specifies a prefix that \hslabel can
%% use to generate line number references. If no prefix
%% is givne, \hslabel will have no effect.
\newtoks\prefixtoks
\def\mkhslabel#1{\prefixtoks={#1}\let\prefix=a}
\def\hslabel#1{\ifx\prefix\relax%%
  \else\label{\the\prefixtoks_#1}%%
  \fi}
\def\unhslabel{\let\prefix=\relax}
\newenvironment{withHsNum}{\numberson\numbersreset}{\numbersoff}
\newenvironment{withHsLabeled}[1]{\numberson\numbersreset\mkhslabel{#1}}{\unhslabel\numbersoff}

%% Paragraph run-in
\newcommand{\runin}[1]{\begingroup\noindent\sffamily\textbf{#1}\qquad\endgroup}

%% Chapter bibliographies
\newcommand{\standaloneBib}{%%
  \ifthenelse{\boolean{standaloneFlag}}%%
             {\begin{singlespace}
                 \printbibliography
             \end{singlespace}}{}}

%% Adds an equation number on demand.
\newcommand\addtag{\refstepcounter{equation}\tag{\theequation}}

%% For typesetting set definitions like {x | x \in f(y)}
\newcommand\setdef[2]{\ensuremath{\{#1\ |\ #2\}}}

%% For typesetting function names like dom(f) or out(b).
\newcommand\mfun[1]{\ensuremath{\mathit{#1}}}

%% Marginal notes
\newcommand\margin[2]{\marginpar{\begin{singlespace}\emph{\footnotesize #2}\end{singlespace}}\relax #1}

%% Describe intent of a passage
\newcommand\intent[1]{{\begin{singlespace}\noindent\leftskip=-1in\emph{\footnotesize Intent: #1}\end{singlespace}}\nopagebreak[1]}

%% In aligned/alignedat/gathered environments, you don't et
%% automatice equation numbers. This command makes sure to
%% label them properly.
\newcommand\labeleq[1]{\refstepcounter{equation}\label{#1}}

%% Creates a hanging paragraph, where the first line is not
%% indented but all other lines are.
\def\itempar#1{\noindent\hangindent=\parindent\hangafter=1 #1\quad}

%% Disable overfull messages with ridiculous hfuzz value
\def\disableoverfull{\hfuzz=10in}

%% Set parfillskip so stretching does NOT occur at the end of
%% a paragraph (i.e., list of elements). Disable indent at beginning
%% of paragraph. Also turn off underfull hbox warnings.
%%
%% Intended to be used in a \vbox that forms part of a table or graphic,
%% which we want to be line-broken but not exactly like a normal paragraph.
\long\def\disableparspacing#1;{\def\arg{#1}\hbadness=100000\parindent=0pt\parfillskip=0pt\leftskip=0pt\rightskip=0pt%%
  \ifx\arg\empty\else\hsize=#1\relax\fi}
%% This stuff makes !+<text>+! write <text> in typewriter font.  

%% We make ! and + active characters early, then manipulate their
%% meaning to produce the right effect. Initially, + produces +. When
%% !  appears w/o a + following, it produces ``!''. When ``+''
%% follows, we start writing in teleteype (\ttfamily). The definition
%% of ``!'' changes to produce a bang. ``+'' changes such that it
%% looks for trailing ``!''. When no ``!'' appears, ``+'' produces ``+''. 
%% If a ``!'' appears, we shift out of \ttfamily (by ending the group) and
%% reset the meaning of ``!'' and ``+'' so we can start again.
\makeatletter
\let\mdplus=+\let\mdbang=!      %% Preserve meaning of + and ! so we can put them into document.
%% Turn off mark down for everyone
\outer\def\nomd{\catcode`!=12\catcode`+=12}
%% Turn mark down on for everyone
\outer\def\domd{\catcode`!=\active\catcode`+=\active %%
  \initialmd}
%% Use only with a group IMMEDIETALY following. Turns off
%% markdown for the group-to-come, without actually tokenizing the
%% group. If no group follows, this has no effect.
\protected\def\pausemd{\def\dopause{\catcode`!=12\catcode`+=12}%%
  \def\pausemdB{\ifx\next\bgroup%%
    %% A ``partial'' application of expandwith is used
    %% so we don't double up the group argument (which is what
    %% happens if we expand \next). This has the effect of 
    %% inserting \expandafter\dowith in front of the upcoming {. 
    %% If no brace is coming, \withmdC will have no effect.
    \def\pausemdC{\expandafter\dopause}
  \else
    \let\pausedmC=\relax
  \fi\pausemdC}
  %% \futurelet has to end the macro so it grabs the next token
  %% from the input file. Otherwise, it grabs it *from* this
  %% definition.
  \futurelet\next\pausemdB} %%
%% Turns markdown on for the group-to-come, without actually
%% tokenizing the group. Only has an effect when
%% used in front of a group, otherwise its a no-op.
\protected\def\withmd{\def\dowith{\catcode`!=\active\catcode`+=\active\initialmd}%%
  \def\withmdB{\ifx\next\bgroup %%
    %% A ``partial'' application of expandafter is used
    %% so we don't double up the group argument (which is what
    %% happens if we expand \next). This has the effect of 
    %% inserting \expandafter\dowith in front of the upcoming {. 
    %% If no brace is coming, \withmdC will have no effect.
      \def\withmdC{\expandafter\dowith} %%
    \else %%
      \let\withmdC=\relax %%
    \fi\withmdC}%%
  %% \futurelet has to end the macro so it grabs the next token
  %% from the input file. Otherwise, it grabs it *from* this
  %% definition.
  \futurelet\next\withmdB} %%
%% Make ! and + active in the following group so they have the right
%% catcode in the definitions to follow.
\catcode`!=\active\catcode`+=\active %%
%% Initial definitions associated with ! and +.
\def\initialmd{\protected\def!{\startTTA} %%
  \protected\def+{\stopTTA}} %%
%% Step 1 of startTT. Inital meaning of !; capture next token in \next, go to next step.
\def\startTTA{\futurelet\next\startTTB} %%
%% Step 2 of startTT. Compare captured token to + and go to step 3 if true. Otherwise
%% output a ! (since that started our macro), the argument captured and stop
%% processing.
\long\def\startTTB{\ifx\next+\expandafter\startTTC\expandafter\@gobble\else\mdbang\fi} %%
%% Step 3 of startTT. Shift into teletype mode and change definition of 
%% + and ! so we can stop processing.
\def\startTTC{\begingroup\ifmmode %%
  \let \math@bgroup \relax %%
  \def \math@egroup {\let \math@bgroup \@@math@bgroup %%
    \let \math@egroup \@@math@egroup} %%
  \mathtt\relax %%
  \else  %%
  \ttfamily\fi} %%
%% Step 1, 2  and 3 of stopTT follow the same pattern as startTT.
\def\stopTTA{\futurelet\next\stopTTB} %%
\long\def\stopTTB{\ifx\next!\expandafter\stopTTC\expandafter\@gobble\else\mdplus\fi} %%
\def\stopTTC{\endgroup}%%
\catcode`!=12\catcode`+=12
\makeatother

\domd

%% Place an input file on the next page
\def\onnextpage#1{\afterpage{\clearpage\input{#1}\clearpage}}


% Used by included files to know they
% are NOT standalone
\setboolean{standaloneFlag}{false}

\begin{document}

\VerbatimFootnotes
\DefineShortVerb{\#}
\doublespacing
             
\documentclass[12pt]{article}
%include polycode.fmt
\usepackage[T1]{fontenc}
\usepackage{calc}
\usepackage{palatino}
\usepackage{amsfonts}
\renewcommand\ttdefault{lmtt}
\usepackage{helvet}
\usepackage{xspace}
\usepackage{url}
\usepackage{fancyvrb}
\usepackage[doublespacing]{setspace}
%% below only necessary when using doublespacing -- corrects
%% the vertical space inserted when switching to singlespace
%% environment.
\def\correctspaceskip{\vskip-\baselineskip} 
\usepackage{amsmath}
\usepackage{booktabs}
\usepackage[margin=\parindent, format=hang,labelfont=bf]{caption}
%% \usepackage[subrefformat=parens]{subcaption}
%% The following makes sure we get parentheses around
%% subreferences. The newest version of the subcaption
%% package has an option for this, but that's not available
%% widely.
%%
%% From http://tex.stackexchange.com/questions/25644
\usepackage[labelformat=simple]{subcaption}
\makeatletter
  \def\thesubfigure{(\alph{subfigure})}
  \providecommand\thefigsubsep{~}
  \def\p@subfigure{\@nameuse{thefigure}\thefigsubsep}
\makeatother

\usepackage{ifthen}
\usepackage{stmaryrd}
\usepackage{longtable}
\usepackage{afterpage}
\usepackage{xifthen}
\usepackage{mathtools}
\usepackage[natbib=true,style=authoryear,backend=bibtex8]{biblatex}
\setlength{\bibitemsep}{\bigskipamount}
\addbibresource{thesis.bib}
\usepackage{microtype}

%% GSO margins.
\usepackage[left=1.5in, right=1in, top=1in, bottom=1in]{geometry}
\usepackage{abstract}

%% GSO requires 12 pt font for all headings
\usepackage[bf,sf,tiny,compact]{titlesec}
\titleformat{\chapter}[display]
            {}% format
            {\sffamily\bfseries\chaptertitlename\ \thechapter}
            {\baselineskip}
            {\sffamily\bfseries}
            {}

\hyphenation{data-flow mo-na-dic} 

%% Should unindent all haskell code set in a dispay (versus inline)
\makeatletter
  \@ifundefined{hscodestyle}
               {}
               {\renewcommand{\hscodestyle}{\advance\leftskip -\mathindent}}
\makeatother

% Used by included files to know they
% are NOT standalone
\newboolean{standaloneFlag}
\setboolean{standaloneFlag}{true}

\newlength{\rulefigmargin}
\setlength{\rulefigmargin}{2\parindent}

\newcommand\figbegin{\rule{\linewidth}{0.4pt}\\\vspace{12pt}}
\newcommand\figend{\rule{\linewidth}{0.4pt}}

%% Sets
\newcommand{\setL}[1]{\textsc{#1}\xspace}
\newcommand{\setLC}{\setL{Const}}

%% Lub, subset operators.
\protected\def\lub{\ifmmode\sqcap\else\raisebox{.1em}{\ensuremath{\sqcap}}\fi\xspace}
\newcommand{\sqlt}{\ensuremath{\sqsubset}\xspace}
\newcommand{\sqlte}{\ensuremath{\sqsubseteq}\xspace}

%% Subscripting with typewriter
\def\subtt#1{\ifmmode_{\ensurett{#1}}%%
  \else$_{\ensurett{#1}}$%%
  \fi}
%% Superscripting with typerwriter
\def\suptt#1{\ifmmode^{\ensurett{#1}}%%
  \else$^{\ensurett{#1}}$%%
  \fi}
%% Functional languages chapter commands
\newcommand{\lamA}{\ensuremath{\lambda}-calculus\xspace}
\newcommand{\LamA}{\ensuremath{\lambda}-Calculus\xspace}
\newcommand{\lamAbs}[2]{\ensuremath{\lambda#1.\ #2}}
\newcommand{\lamApp}[2]{\ensuremath{#1\ #2}}
\newcommand{\lamPApp}[2]{\ensuremath{(#1\ #2)}}
\newcommand{\lamAPp}[2]{\ensuremath{(#1)\ #2}}
\newcommand{\lamApP}[2]{\ensuremath{#1\ (#2)}}
\newcommand{\lamAPP}[2]{\ensuremath{(#1)\ (#2)}}
\let\lamApPp=\lamApP
\let\lamAppP=\lamAPp
%% LC definition
\newtoks\toksA
\protected\def\lcname#1/{\ensuremath{\mathit{#1}}}
\protected\def\lcdef#1(#2)=#3;{\def\arg{#2}%%
  \def\lcargs##1,##2/{\def\arg{##2}%%
    \ifx\empty\arg%%
    \lcname ##1/%%
    \else\lcname ##1/\ \lcargs ##2/%%
    \fi}%%
  \ifx\empty\arg\toksA={\ }%%
  \else\toksA={\ \lcargs #2,/\ }%%
  \fi%%
  \ensuremath{\lcname#1/\the\toksA =\ #3}}
%% Arbitary number of applied arguments, separated
%% by asterisks (*).
\protected\def\lcapp#1/{\def\lcappB##1*##2/{\def\arg{##2}%
    \ensuremath{\ifx\arg\empty%%
      \lcname ##1/%%
      \else%%
      \lcname##1/\ \lcappB##2/%%
      \fi}}%%
  %% Adding a star here makes
  %% sure our applicaitn always ends with an asterisks, ensuring
  %% #2 will be \empty at some point.
  \lcappB#1*/}
\protected\def\lcabs#1.{\ensuremath{\lambda#1.\ }}

\newcommand{\lamId}{\lamAbs{x}{x}}
\newcommand{\lamCompose}{\lamAbs{f}{\lamAbs{g}{\lamAbs{x}{\lamApp{f}{(\lamApp{g}{x})}}}}}
\newcommand{\machLam}{\ensuremath{M_\lambda}\xspace}
\newcommand{\compMach}[1]{\ensuremath{\left\llbracket #1 \right\rrbracket}}
\newcommand{\compRho}[1]{\ensuremath{\rho(#1)}}
\newcommand{\verSub}[2]{\ensuremath{#1_{#2}}}
\newcommand{\verSup}[2]{\ensuremath{#1^{#2}}}
\newcommand{\lamC}{\ensuremath{\lambda_C}\xspace}
\newcommand{\lamPlus}{\lamAbs{m}{\lamAbs{n}{\lamAbs{s}{\lamAbs{z}{\lamApp{m}{\lamApPp{s}{\lamApp{n}{\lamApp{s}{z}}}}}}}}}
%% Substitution notation -- [#1 -> #2]
\newcommand{\lamSubst}[2]{\ensuremath{[#1 \mapsto #2]}}
%% End functional languages chapter


%% MIL Chapter
\newcommand{\compMILE}[1]{\ensuremath{\left\llbracket #1 \right\rrbracket}}
\newcommand{\compMILV}[1]{\ensuremath{\left\llbracket #1 \right\rrbracket}}
\newcommand{\compMILQ}[2]{\ensuremath{\left\llbracket #2 \right\rrbracket}}
\newcommand{\milCtx}[1]{\ensuremath{\llfloor}#1\ensuremath{\rrfloor}}

%% This dimension makes sure the same amount of space
%% follows | and := in syntax rules like:
%%
%% term := var       (Variable)
%%      |  var var    (Application)
%%      |  \x. var    (Abstraction)
%%
\newdimen\termalign
\setbox0=\hbox{$:=$}
\termalign=\wd0 
\protected\def\term#1/{\ensuremath{\mathit{#1}}}
\protected\def\syntaxrule#1/{\hfil\text{\emph{#1}}}
\protected\long\def\termrule#1:#2:#3/{\term #1/ &\hbox{$:=$}\ensuremath{\ #2} & \syntaxrule #3/}
\protected\def\termcase#1:#2/{&\hbox to \termalign{$|$\hss}\ensuremath{\ #1} & \syntaxrule #2/}


%% End MIL chapter

%% Dataflow Chapter
% Domain function
\def\dom(#1){\ensuremath{\mfun{dom}(#1)}\xspace}
% Set of all integers.
\def\ZZ{\ensuremath{\mathbb{Z}}}
%%

%% Uncurrrying Chapter 
%% A space equal to a \thinspace, but we
%% can break a line at it.
\newskip\thinskipamt \thinskipamt=.16667em 
\protected\def\thinskip{\hskip \thinskipamt\relax}
\protected\def\thinnerskip{\hskip .5\thinskipamt\relax}
%% Capture a space token. Use a ``control-symbol'' (\. instead of \mksp)
%% to keep the trailing space from getting gobbled.
{\def\.{\global\let\sp= } \. }
%% Define \asp, which will capture the macro definition attached to space,
%% if one exists. Otherwise, \spa is relax after this.
{\catcode`\ =\active\gdef\asp{\ifx \relax\let\spa\relax\else\let\spa= \fi}}
\newtoks\foo
%% Removes spaces, implicit, active and explicit.
\protected\def\removespaces{\asp\afterassignment\removesp\let\next= }
\def\removesp{\foo={\next}\ifcat\noexpand\next\sp\foo={\removespace}%%
 \else\ifx\next\spa\foo={\removespaces}\fi%%
 \fi\the\foo}
%% MIL reserved word
\protected\def\milres#1/{\text{\ttfamily\bfseries #1}}
\protected\def\lab#1/{\textbf{\ensurett{\removespaces #1}}}
%% Constructs a closure: l { v1, ..., vN }
\protected\long\def\mkclo[#1:#2]{\lab #1/\ensuremath{\,\{\ensurett{#2}\}}\xspace}
%% Tuple version of closurs: {l: v1, ..., vN}.
\protected\long\def\clo[#1:#2]{\def\argA{#1}\def\argB{#2}\ensuremath{\{%%
      \ifx\argA\empty%%
      \else\lab #1/%%
        \ifx\argB\empty%%
        \else\ensurett{:\thinskip}%%
        \fi%%
      \fi\ensurett{#2}\}}\xspace}
%% Construct a thunk
\newbox\bracklbox \newbox\brackrbox
\setbox0=\hbox{$\{$} \setbox\bracklbox=\hbox to \wd0{\hfil[\kern0.25mm}
\setbox0=\hbox{$\}$} \setbox\brackrbox=\hbox to \wd0{\kern0.25mm]\hfil}
\protected\def\mkthunk[#1:#2]{\lab #1/%%
  \ensuremath{\,%%
    \mathopen{\copy\bracklbox}%%
    \ensurett{#2}%%
    \mathclose{\copy\brackrbox}\xspace}}
%% Binding statement: v <- {...}
\protected\def\binds#1<-#2;{\ensurett{\removespaces #1\texttt{<-}#2}\xspace}
%% In order to use \binds in verbatim environment, have to define
%% delimiters while they are active. The below defines \vbinds which
%% must be used in AVerb environments.  Notice the active space as
%% well - that is necessary so the space after \vbinds (and before the
%% first argument) in the verbatim environment gets eaten.
\begingroup\catcode`\!=\active \lccode`\!=`\< \lccode`\~=`\- 
  \catcode`\ =\active\lowercase{\endgroup\def\vbinds#1!~#2;}{\binds#1<-#2;}
%% Return statement: return ... ;
\protected\def\return#1;{\milres return/\ensurett{\ \removespaces #1}}
%% A closure capturing block. k {v1, ..., vN} x: ...
\protected\def\ccblock#1(#2)#3:{\lab#1/\ensuremath{\thinspace\{\ensurett{#2}\}}\ \ensurett{#3\hbox{:}}}
%% A normal block
\protected\def\block#1(#2):{\lab #1/\ensuremath{\thinspace(\ensurett{#2})}\ensurett{:}}
%% A goto expression
\protected\def\goto#1(#2){\lab #1/\thinspace\ensuremath{(\ensurett{#2})}}
%% An enter expression
\protected\def\app#1*#2/{\ensurett{\removespaces #1\ifmmode\ \fi{\text{\tt @}}\ifmmode\ \fi#2}}
\protected\def\bind{\texttt{<-}\xspace}
%% Primitive expression
\protected\def\prim#1(#2){\lab #1/\suptt*\ensuremath{(\ensurett{#2})}}
%% Program variable
\protected\def\var#1/{\ensurett{\removespaces #1}\xspace}
%% Case statement
\protected\def\case#1;{\milres case\ \ensuremath{\ensurett{\removespaces #1}}\ of/}
%% Case alternative
\protected\def\alt#1(#2)#3->#4;{\ensuremath{\ensurett{#1\ \ignorespaces#2\ \texttt{->}\ \ignorespaces #4}}}
%% Invoke
\protected\def\invoke#1/{\milres invoke/\ensurett{\ \removespaces #1}}
\def\rhs{right--hand side\xspace}
\def\lhs{left--hand side\xspace}
\def\enter{\texttt{@}\xspace}
\def\cc{closure--capturing\xspace}
\def\Cc{Closure--capturing\xspace}
%%

\newenvironment{myfig}[1][tbh]{\begin{figure}[#1]%%
\begin{singlespace}\centering%%
\figbegin}{\figend\end{singlespace}%%
\end{figure}}

%% Produce a sub-caption and label it.
\newcommand{\scap}[2][1in]{\begin{minipage}{#1}%%
\subcaption{}\label{#2}\end{minipage}}

%% Produce a sub-caption with text.
\newcommand{\lscap}[3][\hsize]{\begin{minipage}{#1}%%
\subcaption{#3}\label{#2}\end{minipage}}

% single-argument comment. Do not put
% a space before the command when used
% or the file will have two spaces.
\newcommand{\rem}[1]{}

%% A verbatim environment with active charactesr
%% so we can use math shortcuts and macros
\DefineVerbatimEnvironment{AVerb}{Verbatim}{commandchars=\\\{\},%% 
  codes={\catcode`\_8\catcode`\$3\catcode`\^7},%%
  numberblanklines=false}

\DefineVerbatimEnvironment{Verb}{Verbatim}{commandchars=\\\[\],%% 
  numberblanklines=false}

%% Turn on line numbers for Haskell code, 
%% and reset the line number counter.
\newcommand{\hsNumOn}{\numberson\numbersreset}
\newcommand{\hsNumOff}{\numbersoff}
%% Turn on line numbering in Haskell code within
%% the environment, then turn it off. The optional
%% argument specifies a prefix that \hslabel can
%% use to generate line number references. If no prefix
%% is givne, \hslabel will have no effect.
\newtoks\prefixtoks
\def\mkhslabel#1{\prefixtoks={#1}\let\prefix=a}
\def\hslabel#1{\ifx\prefix\relax%%
  \else\label{\the\prefixtoks_#1}%%
  \fi}
\def\unhslabel{\let\prefix=\relax}
\newenvironment{withHsNum}{\numberson\numbersreset}{\numbersoff}
\newenvironment{withHsLabeled}[1]{\numberson\numbersreset\mkhslabel{#1}}{\unhslabel\numbersoff}

%% Paragraph run-in
\newcommand{\runin}[1]{\begingroup\noindent\sffamily\textbf{#1}\qquad\endgroup}

%% Chapter bibliographies
\newcommand{\standaloneBib}{%%
  \ifthenelse{\boolean{standaloneFlag}}%%
             {\begin{singlespace}
                 \printbibliography
             \end{singlespace}}{}}

%% Adds an equation number on demand.
\newcommand\addtag{\refstepcounter{equation}\tag{\theequation}}

%% For typesetting set definitions like {x | x \in f(y)}
\newcommand\setdef[2]{\ensuremath{\{#1\ |\ #2\}}}

%% For typesetting function names like dom(f) or out(b).
\newcommand\mfun[1]{\ensuremath{\mathit{#1}}}

%% Marginal notes
\newcommand\margin[2]{\marginpar{\begin{singlespace}\emph{\footnotesize #2}\end{singlespace}}\relax #1}

%% Describe intent of a passage
\newcommand\intent[1]{{\begin{singlespace}\noindent\leftskip=-1in\emph{\footnotesize Intent: #1}\end{singlespace}}\nopagebreak[1]}

%% In aligned/alignedat/gathered environments, you don't et
%% automatice equation numbers. This command makes sure to
%% label them properly.
\newcommand\labeleq[1]{\refstepcounter{equation}\label{#1}}

%% Creates a hanging paragraph, where the first line is not
%% indented but all other lines are.
\def\itempar#1{\noindent\hangindent=\parindent\hangafter=1 #1\quad}

%% Disable overfull messages with ridiculous hfuzz value
\def\disableoverfull{\hfuzz=10in}

%% Set parfillskip so stretching does NOT occur at the end of
%% a paragraph (i.e., list of elements). Disable indent at beginning
%% of paragraph. Also turn off underfull hbox warnings.
%%
%% Intended to be used in a \vbox that forms part of a table or graphic,
%% which we want to be line-broken but not exactly like a normal paragraph.
\long\def\disableparspacing#1;{\def\arg{#1}\hbadness=100000\parindent=0pt\parfillskip=0pt\leftskip=0pt\rightskip=0pt%%
  \ifx\arg\empty\else\hsize=#1\relax\fi}
%% This stuff makes !+<text>+! write <text> in typewriter font.  

%% We make ! and + active characters early, then manipulate their
%% meaning to produce the right effect. Initially, + produces +. When
%% !  appears w/o a + following, it produces ``!''. When ``+''
%% follows, we start writing in teleteype (\ttfamily). The definition
%% of ``!'' changes to produce a bang. ``+'' changes such that it
%% looks for trailing ``!''. When no ``!'' appears, ``+'' produces ``+''. 
%% If a ``!'' appears, we shift out of \ttfamily (by ending the group) and
%% reset the meaning of ``!'' and ``+'' so we can start again.
\makeatletter
\let\mdplus=+\let\mdbang=!      %% Preserve meaning of + and ! so we can put them into document.
%% Turn off mark down for everyone
\outer\def\nomd{\catcode`!=12\catcode`+=12}
%% Turn mark down on for everyone
\outer\def\domd{\catcode`!=\active\catcode`+=\active %%
  \initialmd}
%% Use only with a group IMMEDIETALY following. Turns off
%% markdown for the group-to-come, without actually tokenizing the
%% group. If no group follows, this has no effect.
\protected\def\pausemd{\def\dopause{\catcode`!=12\catcode`+=12}%%
  \def\pausemdB{\ifx\next\bgroup%%
    %% A ``partial'' application of expandwith is used
    %% so we don't double up the group argument (which is what
    %% happens if we expand \next). This has the effect of 
    %% inserting \expandafter\dowith in front of the upcoming {. 
    %% If no brace is coming, \withmdC will have no effect.
    \def\pausemdC{\expandafter\dopause}
  \else
    \let\pausedmC=\relax
  \fi\pausemdC}
  %% \futurelet has to end the macro so it grabs the next token
  %% from the input file. Otherwise, it grabs it *from* this
  %% definition.
  \futurelet\next\pausemdB} %%
%% Turns markdown on for the group-to-come, without actually
%% tokenizing the group. Only has an effect when
%% used in front of a group, otherwise its a no-op.
\protected\def\withmd{\def\dowith{\catcode`!=\active\catcode`+=\active\initialmd}%%
  \def\withmdB{\ifx\next\bgroup %%
    %% A ``partial'' application of expandafter is used
    %% so we don't double up the group argument (which is what
    %% happens if we expand \next). This has the effect of 
    %% inserting \expandafter\dowith in front of the upcoming {. 
    %% If no brace is coming, \withmdC will have no effect.
      \def\withmdC{\expandafter\dowith} %%
    \else %%
      \let\withmdC=\relax %%
    \fi\withmdC}%%
  %% \futurelet has to end the macro so it grabs the next token
  %% from the input file. Otherwise, it grabs it *from* this
  %% definition.
  \futurelet\next\withmdB} %%
%% Make ! and + active in the following group so they have the right
%% catcode in the definitions to follow.
\catcode`!=\active\catcode`+=\active %%
%% Initial definitions associated with ! and +.
\def\initialmd{\protected\def!{\startTTA} %%
  \protected\def+{\stopTTA}} %%
%% Step 1 of startTT. Inital meaning of !; capture next token in \next, go to next step.
\def\startTTA{\futurelet\next\startTTB} %%
%% Step 2 of startTT. Compare captured token to + and go to step 3 if true. Otherwise
%% output a ! (since that started our macro), the argument captured and stop
%% processing.
\long\def\startTTB{\ifx\next+\expandafter\startTTC\expandafter\@gobble\else\mdbang\fi} %%
%% Step 3 of startTT. Shift into teletype mode and change definition of 
%% + and ! so we can stop processing.
\def\startTTC{\begingroup\ifmmode %%
  \let \math@bgroup \relax %%
  \def \math@egroup {\let \math@bgroup \@@math@bgroup %%
    \let \math@egroup \@@math@egroup} %%
  \mathtt\relax %%
  \else  %%
  \ttfamily\fi} %%
%% Step 1, 2  and 3 of stopTT follow the same pattern as startTT.
\def\stopTTA{\futurelet\next\stopTTB} %%
\long\def\stopTTB{\ifx\next!\expandafter\stopTTC\expandafter\@gobble\else\mdplus\fi} %%
\def\stopTTC{\endgroup}%%
\catcode`!=12\catcode`+=12
\makeatother

\domd

%% Place an input file on the next page
\def\onnextpage#1{\afterpage{\clearpage\input{#1}\clearpage}}

\begin{document}
\ifthenelse{\boolean{standaloneFlag}}
           {\VerbatimFootnotes
             \DefineShortVerb{\#}
             \doublespacing
             \setcounter{chapter}{0}}{}

%% Default float parameters. For case when
%% multiple chapters are included and
%% only one needs custom float settings.
\renewcommand{\textfraction}{0.2}
\renewcommand{\topfraction}{0.9}


%if False
% lhs2tex ignores this section
\newcommand{\authorEmail}{\url{justinb@cs.pdx.edu}}
%else
% LaTeX ignores this section, unless pre-processed with lhs2Text
\ifthenelse{\boolean{lhs2tex}}%
           {\newcommand{\authorEmail}{\url{justinb@@cs.pdx.edu}}}%
           {}
%endif

\date{}
\author{Justin Bailey \\ \authorEmail}
\title{Using Dataflow Optimization Techniques with a Monadic Intermediate Language}

\maketitle 
\ifthenelse{\standaloneFlag}
           {\thispagestyle{empty}}
           {}

\renewcommand{\abstractnamefont}{\normalfont\small\sffamily\bfseries}
\begin{abstract}
  Dataflow analysis of programs represented as control-flow graphs
  (CFGs) of basic blocks underlies many optimizations implemented by
  imperative language compilers. Functional language compilers, in
  contrast, traditionally optimize by rewriting programs according to
  algebraic laws. We show that, by compiling to a \emph{monadic
    intermediate language}, we can treat our functional programs as
  CFGs of basic blocks. Doing so enables us to re-use the rich body of
  dataflow techniques developed for imperative language compilers. We
  first implement dead code elimination, an optimization common to
  both functional and imperative compilers. We then demonstrate
  \emph{uncurrying}, an optimization specific to functional language
  compilers. Next, we use the \emph{monad laws} to derive inlining and
  copy-propagation optimizations. Finally, we implement \emph{Lazy
    Code Motion} (LCM), one of the most complicated dataflow-based
  optimizations. To our knowledge this is the first implementation of
  LCM over a monadic intermediate language. Throughout, we use the
  \emph{Hoopl} library to implement our optimizations, making our work
  a case-study for the library as well.
\end{abstract}
\end{document}


\documentclass[12pt]{report}
%include polycode.fmt
\usepackage[T1]{fontenc}
\usepackage{calc}
\usepackage{palatino}
\usepackage{amsfonts}
\renewcommand\ttdefault{lmtt}
\usepackage{helvet}
\usepackage{xspace}
\usepackage{url}
\usepackage{fancyvrb}
\usepackage[doublespacing]{setspace}
%% below only necessary when using doublespacing -- corrects
%% the vertical space inserted when switching to singlespace
%% environment.
\def\correctspaceskip{\vskip-\baselineskip} 
\usepackage{amsmath}
\usepackage{booktabs}
\usepackage[margin=\parindent, format=hang,labelfont=bf]{caption}
%% \usepackage[subrefformat=parens]{subcaption}
%% The following makes sure we get parentheses around
%% subreferences. The newest version of the subcaption
%% package has an option for this, but that's not available
%% widely.
%%
%% From http://tex.stackexchange.com/questions/25644
\usepackage[labelformat=simple]{subcaption}
\makeatletter
  \def\thesubfigure{(\alph{subfigure})}
  \providecommand\thefigsubsep{~}
  \def\p@subfigure{\@nameuse{thefigure}\thefigsubsep}
\makeatother

\usepackage{ifthen}
\usepackage{stmaryrd}
\usepackage{longtable}
\usepackage{afterpage}
\usepackage{xifthen}
\usepackage{mathtools}
\usepackage[natbib=true,style=authoryear,backend=bibtex8]{biblatex}
\setlength{\bibitemsep}{\bigskipamount}
\addbibresource{thesis.bib}
\usepackage{microtype}

%% GSO margins.
\usepackage[left=1.5in, right=1in, top=1in, bottom=1in]{geometry}
\usepackage{abstract}

%% GSO requires 12 pt font for all headings
\usepackage[bf,sf,tiny,compact]{titlesec}
\titleformat{\chapter}[display]
            {}% format
            {\sffamily\bfseries\chaptertitlename\ \thechapter}
            {\baselineskip}
            {\sffamily\bfseries}
            {}

\hyphenation{data-flow mo-na-dic} 

%% Should unindent all haskell code set in a dispay (versus inline)
\makeatletter
  \@ifundefined{hscodestyle}
               {}
               {\renewcommand{\hscodestyle}{\advance\leftskip -\mathindent}}
\makeatother

% Used by included files to know they
% are NOT standalone
\newboolean{standaloneFlag}
\setboolean{standaloneFlag}{true}

\newlength{\rulefigmargin}
\setlength{\rulefigmargin}{2\parindent}

\newcommand\figbegin{\rule{\linewidth}{0.4pt}\\\vspace{12pt}}
\newcommand\figend{\rule{\linewidth}{0.4pt}}

%% Sets
\newcommand{\setL}[1]{\textsc{#1}\xspace}
\newcommand{\setLC}{\setL{Const}}

%% Lub, subset operators.
\protected\def\lub{\ifmmode\sqcap\else\raisebox{.1em}{\ensuremath{\sqcap}}\fi\xspace}
\newcommand{\sqlt}{\ensuremath{\sqsubset}\xspace}
\newcommand{\sqlte}{\ensuremath{\sqsubseteq}\xspace}

%% Subscripting with typewriter
\def\subtt#1{\ifmmode_{\ensurett{#1}}%%
  \else$_{\ensurett{#1}}$%%
  \fi}
%% Superscripting with typerwriter
\def\suptt#1{\ifmmode^{\ensurett{#1}}%%
  \else$^{\ensurett{#1}}$%%
  \fi}
%% Functional languages chapter commands
\newcommand{\lamA}{\ensuremath{\lambda}-calculus\xspace}
\newcommand{\LamA}{\ensuremath{\lambda}-Calculus\xspace}
\newcommand{\lamAbs}[2]{\ensuremath{\lambda#1.\ #2}}
\newcommand{\lamApp}[2]{\ensuremath{#1\ #2}}
\newcommand{\lamPApp}[2]{\ensuremath{(#1\ #2)}}
\newcommand{\lamAPp}[2]{\ensuremath{(#1)\ #2}}
\newcommand{\lamApP}[2]{\ensuremath{#1\ (#2)}}
\newcommand{\lamAPP}[2]{\ensuremath{(#1)\ (#2)}}
\let\lamApPp=\lamApP
\let\lamAppP=\lamAPp
%% LC definition
\newtoks\toksA
\protected\def\lcname#1/{\ensuremath{\mathit{#1}}}
\protected\def\lcdef#1(#2)=#3;{\def\arg{#2}%%
  \def\lcargs##1,##2/{\def\arg{##2}%%
    \ifx\empty\arg%%
    \lcname ##1/%%
    \else\lcname ##1/\ \lcargs ##2/%%
    \fi}%%
  \ifx\empty\arg\toksA={\ }%%
  \else\toksA={\ \lcargs #2,/\ }%%
  \fi%%
  \ensuremath{\lcname#1/\the\toksA =\ #3}}
%% Arbitary number of applied arguments, separated
%% by asterisks (*).
\protected\def\lcapp#1/{\def\lcappB##1*##2/{\def\arg{##2}%
    \ensuremath{\ifx\arg\empty%%
      \lcname ##1/%%
      \else%%
      \lcname##1/\ \lcappB##2/%%
      \fi}}%%
  %% Adding a star here makes
  %% sure our applicaitn always ends with an asterisks, ensuring
  %% #2 will be \empty at some point.
  \lcappB#1*/}
\protected\def\lcabs#1.{\ensuremath{\lambda#1.\ }}

\newcommand{\lamId}{\lamAbs{x}{x}}
\newcommand{\lamCompose}{\lamAbs{f}{\lamAbs{g}{\lamAbs{x}{\lamApp{f}{(\lamApp{g}{x})}}}}}
\newcommand{\machLam}{\ensuremath{M_\lambda}\xspace}
\newcommand{\compMach}[1]{\ensuremath{\left\llbracket #1 \right\rrbracket}}
\newcommand{\compRho}[1]{\ensuremath{\rho(#1)}}
\newcommand{\verSub}[2]{\ensuremath{#1_{#2}}}
\newcommand{\verSup}[2]{\ensuremath{#1^{#2}}}
\newcommand{\lamC}{\ensuremath{\lambda_C}\xspace}
\newcommand{\lamPlus}{\lamAbs{m}{\lamAbs{n}{\lamAbs{s}{\lamAbs{z}{\lamApp{m}{\lamApPp{s}{\lamApp{n}{\lamApp{s}{z}}}}}}}}}
%% Substitution notation -- [#1 -> #2]
\newcommand{\lamSubst}[2]{\ensuremath{[#1 \mapsto #2]}}
%% End functional languages chapter


%% MIL Chapter
\newcommand{\compMILE}[1]{\ensuremath{\left\llbracket #1 \right\rrbracket}}
\newcommand{\compMILV}[1]{\ensuremath{\left\llbracket #1 \right\rrbracket}}
\newcommand{\compMILQ}[2]{\ensuremath{\left\llbracket #2 \right\rrbracket}}
\newcommand{\milCtx}[1]{\ensuremath{\llfloor}#1\ensuremath{\rrfloor}}

%% This dimension makes sure the same amount of space
%% follows | and := in syntax rules like:
%%
%% term := var       (Variable)
%%      |  var var    (Application)
%%      |  \x. var    (Abstraction)
%%
\newdimen\termalign
\setbox0=\hbox{$:=$}
\termalign=\wd0 
\protected\def\term#1/{\ensuremath{\mathit{#1}}}
\protected\def\syntaxrule#1/{\hfil\text{\emph{#1}}}
\protected\long\def\termrule#1:#2:#3/{\term #1/ &\hbox{$:=$}\ensuremath{\ #2} & \syntaxrule #3/}
\protected\def\termcase#1:#2/{&\hbox to \termalign{$|$\hss}\ensuremath{\ #1} & \syntaxrule #2/}


%% End MIL chapter

%% Dataflow Chapter
% Domain function
\def\dom(#1){\ensuremath{\mfun{dom}(#1)}\xspace}
% Set of all integers.
\def\ZZ{\ensuremath{\mathbb{Z}}}
%%

%% Uncurrrying Chapter 
%% A space equal to a \thinspace, but we
%% can break a line at it.
\newskip\thinskipamt \thinskipamt=.16667em 
\protected\def\thinskip{\hskip \thinskipamt\relax}
\protected\def\thinnerskip{\hskip .5\thinskipamt\relax}
%% Capture a space token. Use a ``control-symbol'' (\. instead of \mksp)
%% to keep the trailing space from getting gobbled.
{\def\.{\global\let\sp= } \. }
%% Define \asp, which will capture the macro definition attached to space,
%% if one exists. Otherwise, \spa is relax after this.
{\catcode`\ =\active\gdef\asp{\ifx \relax\let\spa\relax\else\let\spa= \fi}}
\newtoks\foo
%% Removes spaces, implicit, active and explicit.
\protected\def\removespaces{\asp\afterassignment\removesp\let\next= }
\def\removesp{\foo={\next}\ifcat\noexpand\next\sp\foo={\removespace}%%
 \else\ifx\next\spa\foo={\removespaces}\fi%%
 \fi\the\foo}
%% MIL reserved word
\protected\def\milres#1/{\text{\ttfamily\bfseries #1}}
\protected\def\lab#1/{\textbf{\ensurett{\removespaces #1}}}
%% Constructs a closure: l { v1, ..., vN }
\protected\long\def\mkclo[#1:#2]{\lab #1/\ensuremath{\,\{\ensurett{#2}\}}\xspace}
%% Tuple version of closurs: {l: v1, ..., vN}.
\protected\long\def\clo[#1:#2]{\def\argA{#1}\def\argB{#2}\ensuremath{\{%%
      \ifx\argA\empty%%
      \else\lab #1/%%
        \ifx\argB\empty%%
        \else\ensurett{:\thinskip}%%
        \fi%%
      \fi\ensurett{#2}\}}\xspace}
%% Construct a thunk
\newbox\bracklbox \newbox\brackrbox
\setbox0=\hbox{$\{$} \setbox\bracklbox=\hbox to \wd0{\hfil[\kern0.25mm}
\setbox0=\hbox{$\}$} \setbox\brackrbox=\hbox to \wd0{\kern0.25mm]\hfil}
\protected\def\mkthunk[#1:#2]{\lab #1/%%
  \ensuremath{\,%%
    \mathopen{\copy\bracklbox}%%
    \ensurett{#2}%%
    \mathclose{\copy\brackrbox}\xspace}}
%% Binding statement: v <- {...}
\protected\def\binds#1<-#2;{\ensurett{\removespaces #1\texttt{<-}#2}\xspace}
%% In order to use \binds in verbatim environment, have to define
%% delimiters while they are active. The below defines \vbinds which
%% must be used in AVerb environments.  Notice the active space as
%% well - that is necessary so the space after \vbinds (and before the
%% first argument) in the verbatim environment gets eaten.
\begingroup\catcode`\!=\active \lccode`\!=`\< \lccode`\~=`\- 
  \catcode`\ =\active\lowercase{\endgroup\def\vbinds#1!~#2;}{\binds#1<-#2;}
%% Return statement: return ... ;
\protected\def\return#1;{\milres return/\ensurett{\ \removespaces #1}}
%% A closure capturing block. k {v1, ..., vN} x: ...
\protected\def\ccblock#1(#2)#3:{\lab#1/\ensuremath{\thinspace\{\ensurett{#2}\}}\ \ensurett{#3\hbox{:}}}
%% A normal block
\protected\def\block#1(#2):{\lab #1/\ensuremath{\thinspace(\ensurett{#2})}\ensurett{:}}
%% A goto expression
\protected\def\goto#1(#2){\lab #1/\thinspace\ensuremath{(\ensurett{#2})}}
%% An enter expression
\protected\def\app#1*#2/{\ensurett{\removespaces #1\ifmmode\ \fi{\text{\tt @}}\ifmmode\ \fi#2}}
\protected\def\bind{\texttt{<-}\xspace}
%% Primitive expression
\protected\def\prim#1(#2){\lab #1/\suptt*\ensuremath{(\ensurett{#2})}}
%% Program variable
\protected\def\var#1/{\ensurett{\removespaces #1}\xspace}
%% Case statement
\protected\def\case#1;{\milres case\ \ensuremath{\ensurett{\removespaces #1}}\ of/}
%% Case alternative
\protected\def\alt#1(#2)#3->#4;{\ensuremath{\ensurett{#1\ \ignorespaces#2\ \texttt{->}\ \ignorespaces #4}}}
%% Invoke
\protected\def\invoke#1/{\milres invoke/\ensurett{\ \removespaces #1}}
\def\rhs{right--hand side\xspace}
\def\lhs{left--hand side\xspace}
\def\enter{\texttt{@}\xspace}
\def\cc{closure--capturing\xspace}
\def\Cc{Closure--capturing\xspace}
%%

\newenvironment{myfig}[1][tbh]{\begin{figure}[#1]%%
\begin{singlespace}\centering%%
\figbegin}{\figend\end{singlespace}%%
\end{figure}}

%% Produce a sub-caption and label it.
\newcommand{\scap}[2][1in]{\begin{minipage}{#1}%%
\subcaption{}\label{#2}\end{minipage}}

%% Produce a sub-caption with text.
\newcommand{\lscap}[3][\hsize]{\begin{minipage}{#1}%%
\subcaption{#3}\label{#2}\end{minipage}}

% single-argument comment. Do not put
% a space before the command when used
% or the file will have two spaces.
\newcommand{\rem}[1]{}

%% A verbatim environment with active charactesr
%% so we can use math shortcuts and macros
\DefineVerbatimEnvironment{AVerb}{Verbatim}{commandchars=\\\{\},%% 
  codes={\catcode`\_8\catcode`\$3\catcode`\^7},%%
  numberblanklines=false}

\DefineVerbatimEnvironment{Verb}{Verbatim}{commandchars=\\\[\],%% 
  numberblanklines=false}

%% Turn on line numbers for Haskell code, 
%% and reset the line number counter.
\newcommand{\hsNumOn}{\numberson\numbersreset}
\newcommand{\hsNumOff}{\numbersoff}
%% Turn on line numbering in Haskell code within
%% the environment, then turn it off. The optional
%% argument specifies a prefix that \hslabel can
%% use to generate line number references. If no prefix
%% is givne, \hslabel will have no effect.
\newtoks\prefixtoks
\def\mkhslabel#1{\prefixtoks={#1}\let\prefix=a}
\def\hslabel#1{\ifx\prefix\relax%%
  \else\label{\the\prefixtoks_#1}%%
  \fi}
\def\unhslabel{\let\prefix=\relax}
\newenvironment{withHsNum}{\numberson\numbersreset}{\numbersoff}
\newenvironment{withHsLabeled}[1]{\numberson\numbersreset\mkhslabel{#1}}{\unhslabel\numbersoff}

%% Paragraph run-in
\newcommand{\runin}[1]{\begingroup\noindent\sffamily\textbf{#1}\qquad\endgroup}

%% Chapter bibliographies
\newcommand{\standaloneBib}{%%
  \ifthenelse{\boolean{standaloneFlag}}%%
             {\begin{singlespace}
                 \printbibliography
             \end{singlespace}}{}}

%% Adds an equation number on demand.
\newcommand\addtag{\refstepcounter{equation}\tag{\theequation}}

%% For typesetting set definitions like {x | x \in f(y)}
\newcommand\setdef[2]{\ensuremath{\{#1\ |\ #2\}}}

%% For typesetting function names like dom(f) or out(b).
\newcommand\mfun[1]{\ensuremath{\mathit{#1}}}

%% Marginal notes
\newcommand\margin[2]{\marginpar{\begin{singlespace}\emph{\footnotesize #2}\end{singlespace}}\relax #1}

%% Describe intent of a passage
\newcommand\intent[1]{{\begin{singlespace}\noindent\leftskip=-1in\emph{\footnotesize Intent: #1}\end{singlespace}}\nopagebreak[1]}

%% In aligned/alignedat/gathered environments, you don't et
%% automatice equation numbers. This command makes sure to
%% label them properly.
\newcommand\labeleq[1]{\refstepcounter{equation}\label{#1}}

%% Creates a hanging paragraph, where the first line is not
%% indented but all other lines are.
\def\itempar#1{\noindent\hangindent=\parindent\hangafter=1 #1\quad}

%% Disable overfull messages with ridiculous hfuzz value
\def\disableoverfull{\hfuzz=10in}

%% Set parfillskip so stretching does NOT occur at the end of
%% a paragraph (i.e., list of elements). Disable indent at beginning
%% of paragraph. Also turn off underfull hbox warnings.
%%
%% Intended to be used in a \vbox that forms part of a table or graphic,
%% which we want to be line-broken but not exactly like a normal paragraph.
\long\def\disableparspacing#1;{\def\arg{#1}\hbadness=100000\parindent=0pt\parfillskip=0pt\leftskip=0pt\rightskip=0pt%%
  \ifx\arg\empty\else\hsize=#1\relax\fi}
%% This stuff makes !+<text>+! write <text> in typewriter font.  

%% We make ! and + active characters early, then manipulate their
%% meaning to produce the right effect. Initially, + produces +. When
%% !  appears w/o a + following, it produces ``!''. When ``+''
%% follows, we start writing in teleteype (\ttfamily). The definition
%% of ``!'' changes to produce a bang. ``+'' changes such that it
%% looks for trailing ``!''. When no ``!'' appears, ``+'' produces ``+''. 
%% If a ``!'' appears, we shift out of \ttfamily (by ending the group) and
%% reset the meaning of ``!'' and ``+'' so we can start again.
\makeatletter
\let\mdplus=+\let\mdbang=!      %% Preserve meaning of + and ! so we can put them into document.
%% Turn off mark down for everyone
\outer\def\nomd{\catcode`!=12\catcode`+=12}
%% Turn mark down on for everyone
\outer\def\domd{\catcode`!=\active\catcode`+=\active %%
  \initialmd}
%% Use only with a group IMMEDIETALY following. Turns off
%% markdown for the group-to-come, without actually tokenizing the
%% group. If no group follows, this has no effect.
\protected\def\pausemd{\def\dopause{\catcode`!=12\catcode`+=12}%%
  \def\pausemdB{\ifx\next\bgroup%%
    %% A ``partial'' application of expandwith is used
    %% so we don't double up the group argument (which is what
    %% happens if we expand \next). This has the effect of 
    %% inserting \expandafter\dowith in front of the upcoming {. 
    %% If no brace is coming, \withmdC will have no effect.
    \def\pausemdC{\expandafter\dopause}
  \else
    \let\pausedmC=\relax
  \fi\pausemdC}
  %% \futurelet has to end the macro so it grabs the next token
  %% from the input file. Otherwise, it grabs it *from* this
  %% definition.
  \futurelet\next\pausemdB} %%
%% Turns markdown on for the group-to-come, without actually
%% tokenizing the group. Only has an effect when
%% used in front of a group, otherwise its a no-op.
\protected\def\withmd{\def\dowith{\catcode`!=\active\catcode`+=\active\initialmd}%%
  \def\withmdB{\ifx\next\bgroup %%
    %% A ``partial'' application of expandafter is used
    %% so we don't double up the group argument (which is what
    %% happens if we expand \next). This has the effect of 
    %% inserting \expandafter\dowith in front of the upcoming {. 
    %% If no brace is coming, \withmdC will have no effect.
      \def\withmdC{\expandafter\dowith} %%
    \else %%
      \let\withmdC=\relax %%
    \fi\withmdC}%%
  %% \futurelet has to end the macro so it grabs the next token
  %% from the input file. Otherwise, it grabs it *from* this
  %% definition.
  \futurelet\next\withmdB} %%
%% Make ! and + active in the following group so they have the right
%% catcode in the definitions to follow.
\catcode`!=\active\catcode`+=\active %%
%% Initial definitions associated with ! and +.
\def\initialmd{\protected\def!{\startTTA} %%
  \protected\def+{\stopTTA}} %%
%% Step 1 of startTT. Inital meaning of !; capture next token in \next, go to next step.
\def\startTTA{\futurelet\next\startTTB} %%
%% Step 2 of startTT. Compare captured token to + and go to step 3 if true. Otherwise
%% output a ! (since that started our macro), the argument captured and stop
%% processing.
\long\def\startTTB{\ifx\next+\expandafter\startTTC\expandafter\@gobble\else\mdbang\fi} %%
%% Step 3 of startTT. Shift into teletype mode and change definition of 
%% + and ! so we can stop processing.
\def\startTTC{\begingroup\ifmmode %%
  \let \math@bgroup \relax %%
  \def \math@egroup {\let \math@bgroup \@@math@bgroup %%
    \let \math@egroup \@@math@egroup} %%
  \mathtt\relax %%
  \else  %%
  \ttfamily\fi} %%
%% Step 1, 2  and 3 of stopTT follow the same pattern as startTT.
\def\stopTTA{\futurelet\next\stopTTB} %%
\long\def\stopTTB{\ifx\next!\expandafter\stopTTC\expandafter\@gobble\else\mdplus\fi} %%
\def\stopTTC{\endgroup}%%
\catcode`!=12\catcode`+=12
\makeatother

\domd

%% Place an input file on the next page
\def\onnextpage#1{\afterpage{\clearpage\input{#1}\clearpage}}

\begin{document}
\ifthenelse{\boolean{standaloneFlag}}
           {\VerbatimFootnotes
             \DefineShortVerb{\#}
             \doublespacing
             \setcounter{chapter}{0}}{}

%% Default float parameters. For case when
%% multiple chapters are included and
%% only one needs custom float settings.
\renewcommand{\textfraction}{0.2}
\renewcommand{\topfraction}{0.9}


\chapter{Introduction}

Compilers for imperative languages implement many optimizations using
\emph{dataflow analysis}. This method treats the program as a graph,
where edges represent execution paths and nodes represent statements
in the program. Dataflow analysis computes facts about each node and
then transforms the graph into an equivalent, yet faster (or smaller,
or more efficient, etc.) program. Optimizations which use dataflow
analysis include constant propagation, dead-code elimination,
common-subexpression elimination and many others.

Dataflow analysis on imperative programs arises naturally due to the
explicit flow-of-control from statement to statement. For pure
functional languages, with flow-of-control determined by evaluation
order, the fit seems more awkward. However, call-by-value, pure,
\emph{monadic} functional programs embody the best of both
styles: expression-based evaluation \emph{and} explicit
control-flow. 

Our work, then, defines a monadic language and optimizes programs in
it using dataflow analysis. We implement a number of optimizations
common to imperative and functional languages, including constant
propagation and dead-code elimination. We implement an optimization
which eliminates intermediate closure construction, showing that this
technique can be used for optimizations specific to functional
languages. Finally, we implement \emph{lazy code motion}, which to our
knowledge has not been applied to programs in a monadic language
before.

We use the Hoopl library\rem{reference} to implement our
optimizations. Besides showing that it is possible (and even
desirable) to use dataflow analysis in this context, our work also
serves as a case-study for advanced uses of Hoopl.

\end{document}


\documentclass[12pt]{report}
%include polycode.fmt
\usepackage[T1]{fontenc}
\usepackage{calc}
\usepackage{palatino}
\usepackage{amsfonts}
\renewcommand\ttdefault{lmtt}
\usepackage{helvet}
\usepackage{xspace}
\usepackage{url}
\usepackage{fancyvrb}
\usepackage[doublespacing]{setspace}
%% below only necessary when using doublespacing -- corrects
%% the vertical space inserted when switching to singlespace
%% environment.
\def\correctspaceskip{\vskip-\baselineskip} 
\usepackage{amsmath}
\usepackage{booktabs}
\usepackage[margin=\parindent, format=hang,labelfont=bf]{caption}
%% \usepackage[subrefformat=parens]{subcaption}
%% The following makes sure we get parentheses around
%% subreferences. The newest version of the subcaption
%% package has an option for this, but that's not available
%% widely.
%%
%% From http://tex.stackexchange.com/questions/25644
\usepackage[labelformat=simple]{subcaption}
\makeatletter
  \def\thesubfigure{(\alph{subfigure})}
  \providecommand\thefigsubsep{~}
  \def\p@subfigure{\@nameuse{thefigure}\thefigsubsep}
\makeatother

\usepackage{ifthen}
\usepackage{stmaryrd}
\usepackage{longtable}
\usepackage{afterpage}
\usepackage{xifthen}
\usepackage{mathtools}
\usepackage[natbib=true,style=authoryear,backend=bibtex8]{biblatex}
\setlength{\bibitemsep}{\bigskipamount}
\addbibresource{thesis.bib}
\usepackage{microtype}

%% GSO margins.
\usepackage[left=1.5in, right=1in, top=1in, bottom=1in]{geometry}
\usepackage{abstract}

%% GSO requires 12 pt font for all headings
\usepackage[bf,sf,tiny,compact]{titlesec}
\titleformat{\chapter}[display]
            {}% format
            {\sffamily\bfseries\chaptertitlename\ \thechapter}
            {\baselineskip}
            {\sffamily\bfseries}
            {}

\hyphenation{data-flow mo-na-dic} 

%% Should unindent all haskell code set in a dispay (versus inline)
\makeatletter
  \@ifundefined{hscodestyle}
               {}
               {\renewcommand{\hscodestyle}{\advance\leftskip -\mathindent}}
\makeatother

% Used by included files to know they
% are NOT standalone
\newboolean{standaloneFlag}
\setboolean{standaloneFlag}{true}

\newlength{\rulefigmargin}
\setlength{\rulefigmargin}{2\parindent}

\newcommand\figbegin{\rule{\linewidth}{0.4pt}\\\vspace{12pt}}
\newcommand\figend{\rule{\linewidth}{0.4pt}}

%% Sets
\newcommand{\setL}[1]{\textsc{#1}\xspace}
\newcommand{\setLC}{\setL{Const}}

%% Lub, subset operators.
\protected\def\lub{\ifmmode\sqcap\else\raisebox{.1em}{\ensuremath{\sqcap}}\fi\xspace}
\newcommand{\sqlt}{\ensuremath{\sqsubset}\xspace}
\newcommand{\sqlte}{\ensuremath{\sqsubseteq}\xspace}

%% Subscripting with typewriter
\def\subtt#1{\ifmmode_{\ensurett{#1}}%%
  \else$_{\ensurett{#1}}$%%
  \fi}
%% Superscripting with typerwriter
\def\suptt#1{\ifmmode^{\ensurett{#1}}%%
  \else$^{\ensurett{#1}}$%%
  \fi}
%% Functional languages chapter commands
\newcommand{\lamA}{\ensuremath{\lambda}-calculus\xspace}
\newcommand{\LamA}{\ensuremath{\lambda}-Calculus\xspace}
\newcommand{\lamAbs}[2]{\ensuremath{\lambda#1.\ #2}}
\newcommand{\lamApp}[2]{\ensuremath{#1\ #2}}
\newcommand{\lamPApp}[2]{\ensuremath{(#1\ #2)}}
\newcommand{\lamAPp}[2]{\ensuremath{(#1)\ #2}}
\newcommand{\lamApP}[2]{\ensuremath{#1\ (#2)}}
\newcommand{\lamAPP}[2]{\ensuremath{(#1)\ (#2)}}
\let\lamApPp=\lamApP
\let\lamAppP=\lamAPp
%% LC definition
\newtoks\toksA
\protected\def\lcname#1/{\ensuremath{\mathit{#1}}}
\protected\def\lcdef#1(#2)=#3;{\def\arg{#2}%%
  \def\lcargs##1,##2/{\def\arg{##2}%%
    \ifx\empty\arg%%
    \lcname ##1/%%
    \else\lcname ##1/\ \lcargs ##2/%%
    \fi}%%
  \ifx\empty\arg\toksA={\ }%%
  \else\toksA={\ \lcargs #2,/\ }%%
  \fi%%
  \ensuremath{\lcname#1/\the\toksA =\ #3}}
%% Arbitary number of applied arguments, separated
%% by asterisks (*).
\protected\def\lcapp#1/{\def\lcappB##1*##2/{\def\arg{##2}%
    \ensuremath{\ifx\arg\empty%%
      \lcname ##1/%%
      \else%%
      \lcname##1/\ \lcappB##2/%%
      \fi}}%%
  %% Adding a star here makes
  %% sure our applicaitn always ends with an asterisks, ensuring
  %% #2 will be \empty at some point.
  \lcappB#1*/}
\protected\def\lcabs#1.{\ensuremath{\lambda#1.\ }}

\newcommand{\lamId}{\lamAbs{x}{x}}
\newcommand{\lamCompose}{\lamAbs{f}{\lamAbs{g}{\lamAbs{x}{\lamApp{f}{(\lamApp{g}{x})}}}}}
\newcommand{\machLam}{\ensuremath{M_\lambda}\xspace}
\newcommand{\compMach}[1]{\ensuremath{\left\llbracket #1 \right\rrbracket}}
\newcommand{\compRho}[1]{\ensuremath{\rho(#1)}}
\newcommand{\verSub}[2]{\ensuremath{#1_{#2}}}
\newcommand{\verSup}[2]{\ensuremath{#1^{#2}}}
\newcommand{\lamC}{\ensuremath{\lambda_C}\xspace}
\newcommand{\lamPlus}{\lamAbs{m}{\lamAbs{n}{\lamAbs{s}{\lamAbs{z}{\lamApp{m}{\lamApPp{s}{\lamApp{n}{\lamApp{s}{z}}}}}}}}}
%% Substitution notation -- [#1 -> #2]
\newcommand{\lamSubst}[2]{\ensuremath{[#1 \mapsto #2]}}
%% End functional languages chapter


%% MIL Chapter
\newcommand{\compMILE}[1]{\ensuremath{\left\llbracket #1 \right\rrbracket}}
\newcommand{\compMILV}[1]{\ensuremath{\left\llbracket #1 \right\rrbracket}}
\newcommand{\compMILQ}[2]{\ensuremath{\left\llbracket #2 \right\rrbracket}}
\newcommand{\milCtx}[1]{\ensuremath{\llfloor}#1\ensuremath{\rrfloor}}

%% This dimension makes sure the same amount of space
%% follows | and := in syntax rules like:
%%
%% term := var       (Variable)
%%      |  var var    (Application)
%%      |  \x. var    (Abstraction)
%%
\newdimen\termalign
\setbox0=\hbox{$:=$}
\termalign=\wd0 
\protected\def\term#1/{\ensuremath{\mathit{#1}}}
\protected\def\syntaxrule#1/{\hfil\text{\emph{#1}}}
\protected\long\def\termrule#1:#2:#3/{\term #1/ &\hbox{$:=$}\ensuremath{\ #2} & \syntaxrule #3/}
\protected\def\termcase#1:#2/{&\hbox to \termalign{$|$\hss}\ensuremath{\ #1} & \syntaxrule #2/}


%% End MIL chapter

%% Dataflow Chapter
% Domain function
\def\dom(#1){\ensuremath{\mfun{dom}(#1)}\xspace}
% Set of all integers.
\def\ZZ{\ensuremath{\mathbb{Z}}}
%%

%% Uncurrrying Chapter 
%% A space equal to a \thinspace, but we
%% can break a line at it.
\newskip\thinskipamt \thinskipamt=.16667em 
\protected\def\thinskip{\hskip \thinskipamt\relax}
\protected\def\thinnerskip{\hskip .5\thinskipamt\relax}
%% Capture a space token. Use a ``control-symbol'' (\. instead of \mksp)
%% to keep the trailing space from getting gobbled.
{\def\.{\global\let\sp= } \. }
%% Define \asp, which will capture the macro definition attached to space,
%% if one exists. Otherwise, \spa is relax after this.
{\catcode`\ =\active\gdef\asp{\ifx \relax\let\spa\relax\else\let\spa= \fi}}
\newtoks\foo
%% Removes spaces, implicit, active and explicit.
\protected\def\removespaces{\asp\afterassignment\removesp\let\next= }
\def\removesp{\foo={\next}\ifcat\noexpand\next\sp\foo={\removespace}%%
 \else\ifx\next\spa\foo={\removespaces}\fi%%
 \fi\the\foo}
%% MIL reserved word
\protected\def\milres#1/{\text{\ttfamily\bfseries #1}}
\protected\def\lab#1/{\textbf{\ensurett{\removespaces #1}}}
%% Constructs a closure: l { v1, ..., vN }
\protected\long\def\mkclo[#1:#2]{\lab #1/\ensuremath{\,\{\ensurett{#2}\}}\xspace}
%% Tuple version of closurs: {l: v1, ..., vN}.
\protected\long\def\clo[#1:#2]{\def\argA{#1}\def\argB{#2}\ensuremath{\{%%
      \ifx\argA\empty%%
      \else\lab #1/%%
        \ifx\argB\empty%%
        \else\ensurett{:\thinskip}%%
        \fi%%
      \fi\ensurett{#2}\}}\xspace}
%% Construct a thunk
\newbox\bracklbox \newbox\brackrbox
\setbox0=\hbox{$\{$} \setbox\bracklbox=\hbox to \wd0{\hfil[\kern0.25mm}
\setbox0=\hbox{$\}$} \setbox\brackrbox=\hbox to \wd0{\kern0.25mm]\hfil}
\protected\def\mkthunk[#1:#2]{\lab #1/%%
  \ensuremath{\,%%
    \mathopen{\copy\bracklbox}%%
    \ensurett{#2}%%
    \mathclose{\copy\brackrbox}\xspace}}
%% Binding statement: v <- {...}
\protected\def\binds#1<-#2;{\ensurett{\removespaces #1\texttt{<-}#2}\xspace}
%% In order to use \binds in verbatim environment, have to define
%% delimiters while they are active. The below defines \vbinds which
%% must be used in AVerb environments.  Notice the active space as
%% well - that is necessary so the space after \vbinds (and before the
%% first argument) in the verbatim environment gets eaten.
\begingroup\catcode`\!=\active \lccode`\!=`\< \lccode`\~=`\- 
  \catcode`\ =\active\lowercase{\endgroup\def\vbinds#1!~#2;}{\binds#1<-#2;}
%% Return statement: return ... ;
\protected\def\return#1;{\milres return/\ensurett{\ \removespaces #1}}
%% A closure capturing block. k {v1, ..., vN} x: ...
\protected\def\ccblock#1(#2)#3:{\lab#1/\ensuremath{\thinspace\{\ensurett{#2}\}}\ \ensurett{#3\hbox{:}}}
%% A normal block
\protected\def\block#1(#2):{\lab #1/\ensuremath{\thinspace(\ensurett{#2})}\ensurett{:}}
%% A goto expression
\protected\def\goto#1(#2){\lab #1/\thinspace\ensuremath{(\ensurett{#2})}}
%% An enter expression
\protected\def\app#1*#2/{\ensurett{\removespaces #1\ifmmode\ \fi{\text{\tt @}}\ifmmode\ \fi#2}}
\protected\def\bind{\texttt{<-}\xspace}
%% Primitive expression
\protected\def\prim#1(#2){\lab #1/\suptt*\ensuremath{(\ensurett{#2})}}
%% Program variable
\protected\def\var#1/{\ensurett{\removespaces #1}\xspace}
%% Case statement
\protected\def\case#1;{\milres case\ \ensuremath{\ensurett{\removespaces #1}}\ of/}
%% Case alternative
\protected\def\alt#1(#2)#3->#4;{\ensuremath{\ensurett{#1\ \ignorespaces#2\ \texttt{->}\ \ignorespaces #4}}}
%% Invoke
\protected\def\invoke#1/{\milres invoke/\ensurett{\ \removespaces #1}}
\def\rhs{right--hand side\xspace}
\def\lhs{left--hand side\xspace}
\def\enter{\texttt{@}\xspace}
\def\cc{closure--capturing\xspace}
\def\Cc{Closure--capturing\xspace}
%%

\newenvironment{myfig}[1][tbh]{\begin{figure}[#1]%%
\begin{singlespace}\centering%%
\figbegin}{\figend\end{singlespace}%%
\end{figure}}

%% Produce a sub-caption and label it.
\newcommand{\scap}[2][1in]{\begin{minipage}{#1}%%
\subcaption{}\label{#2}\end{minipage}}

%% Produce a sub-caption with text.
\newcommand{\lscap}[3][\hsize]{\begin{minipage}{#1}%%
\subcaption{#3}\label{#2}\end{minipage}}

% single-argument comment. Do not put
% a space before the command when used
% or the file will have two spaces.
\newcommand{\rem}[1]{}

%% A verbatim environment with active charactesr
%% so we can use math shortcuts and macros
\DefineVerbatimEnvironment{AVerb}{Verbatim}{commandchars=\\\{\},%% 
  codes={\catcode`\_8\catcode`\$3\catcode`\^7},%%
  numberblanklines=false}

\DefineVerbatimEnvironment{Verb}{Verbatim}{commandchars=\\\[\],%% 
  numberblanklines=false}

%% Turn on line numbers for Haskell code, 
%% and reset the line number counter.
\newcommand{\hsNumOn}{\numberson\numbersreset}
\newcommand{\hsNumOff}{\numbersoff}
%% Turn on line numbering in Haskell code within
%% the environment, then turn it off. The optional
%% argument specifies a prefix that \hslabel can
%% use to generate line number references. If no prefix
%% is givne, \hslabel will have no effect.
\newtoks\prefixtoks
\def\mkhslabel#1{\prefixtoks={#1}\let\prefix=a}
\def\hslabel#1{\ifx\prefix\relax%%
  \else\label{\the\prefixtoks_#1}%%
  \fi}
\def\unhslabel{\let\prefix=\relax}
\newenvironment{withHsNum}{\numberson\numbersreset}{\numbersoff}
\newenvironment{withHsLabeled}[1]{\numberson\numbersreset\mkhslabel{#1}}{\unhslabel\numbersoff}

%% Paragraph run-in
\newcommand{\runin}[1]{\begingroup\noindent\sffamily\textbf{#1}\qquad\endgroup}

%% Chapter bibliographies
\newcommand{\standaloneBib}{%%
  \ifthenelse{\boolean{standaloneFlag}}%%
             {\begin{singlespace}
                 \printbibliography
             \end{singlespace}}{}}

%% Adds an equation number on demand.
\newcommand\addtag{\refstepcounter{equation}\tag{\theequation}}

%% For typesetting set definitions like {x | x \in f(y)}
\newcommand\setdef[2]{\ensuremath{\{#1\ |\ #2\}}}

%% For typesetting function names like dom(f) or out(b).
\newcommand\mfun[1]{\ensuremath{\mathit{#1}}}

%% Marginal notes
\newcommand\margin[2]{\marginpar{\begin{singlespace}\emph{\footnotesize #2}\end{singlespace}}\relax #1}

%% Describe intent of a passage
\newcommand\intent[1]{{\begin{singlespace}\noindent\leftskip=-1in\emph{\footnotesize Intent: #1}\end{singlespace}}\nopagebreak[1]}

%% In aligned/alignedat/gathered environments, you don't et
%% automatice equation numbers. This command makes sure to
%% label them properly.
\newcommand\labeleq[1]{\refstepcounter{equation}\label{#1}}

%% Creates a hanging paragraph, where the first line is not
%% indented but all other lines are.
\def\itempar#1{\noindent\hangindent=\parindent\hangafter=1 #1\quad}

%% Disable overfull messages with ridiculous hfuzz value
\def\disableoverfull{\hfuzz=10in}

%% Set parfillskip so stretching does NOT occur at the end of
%% a paragraph (i.e., list of elements). Disable indent at beginning
%% of paragraph. Also turn off underfull hbox warnings.
%%
%% Intended to be used in a \vbox that forms part of a table or graphic,
%% which we want to be line-broken but not exactly like a normal paragraph.
\long\def\disableparspacing#1;{\def\arg{#1}\hbadness=100000\parindent=0pt\parfillskip=0pt\leftskip=0pt\rightskip=0pt%%
  \ifx\arg\empty\else\hsize=#1\relax\fi}
%% This stuff makes !+<text>+! write <text> in typewriter font.  

%% We make ! and + active characters early, then manipulate their
%% meaning to produce the right effect. Initially, + produces +. When
%% !  appears w/o a + following, it produces ``!''. When ``+''
%% follows, we start writing in teleteype (\ttfamily). The definition
%% of ``!'' changes to produce a bang. ``+'' changes such that it
%% looks for trailing ``!''. When no ``!'' appears, ``+'' produces ``+''. 
%% If a ``!'' appears, we shift out of \ttfamily (by ending the group) and
%% reset the meaning of ``!'' and ``+'' so we can start again.
\makeatletter
\let\mdplus=+\let\mdbang=!      %% Preserve meaning of + and ! so we can put them into document.
%% Turn off mark down for everyone
\outer\def\nomd{\catcode`!=12\catcode`+=12}
%% Turn mark down on for everyone
\outer\def\domd{\catcode`!=\active\catcode`+=\active %%
  \initialmd}
%% Use only with a group IMMEDIETALY following. Turns off
%% markdown for the group-to-come, without actually tokenizing the
%% group. If no group follows, this has no effect.
\protected\def\pausemd{\def\dopause{\catcode`!=12\catcode`+=12}%%
  \def\pausemdB{\ifx\next\bgroup%%
    %% A ``partial'' application of expandwith is used
    %% so we don't double up the group argument (which is what
    %% happens if we expand \next). This has the effect of 
    %% inserting \expandafter\dowith in front of the upcoming {. 
    %% If no brace is coming, \withmdC will have no effect.
    \def\pausemdC{\expandafter\dopause}
  \else
    \let\pausedmC=\relax
  \fi\pausemdC}
  %% \futurelet has to end the macro so it grabs the next token
  %% from the input file. Otherwise, it grabs it *from* this
  %% definition.
  \futurelet\next\pausemdB} %%
%% Turns markdown on for the group-to-come, without actually
%% tokenizing the group. Only has an effect when
%% used in front of a group, otherwise its a no-op.
\protected\def\withmd{\def\dowith{\catcode`!=\active\catcode`+=\active\initialmd}%%
  \def\withmdB{\ifx\next\bgroup %%
    %% A ``partial'' application of expandafter is used
    %% so we don't double up the group argument (which is what
    %% happens if we expand \next). This has the effect of 
    %% inserting \expandafter\dowith in front of the upcoming {. 
    %% If no brace is coming, \withmdC will have no effect.
      \def\withmdC{\expandafter\dowith} %%
    \else %%
      \let\withmdC=\relax %%
    \fi\withmdC}%%
  %% \futurelet has to end the macro so it grabs the next token
  %% from the input file. Otherwise, it grabs it *from* this
  %% definition.
  \futurelet\next\withmdB} %%
%% Make ! and + active in the following group so they have the right
%% catcode in the definitions to follow.
\catcode`!=\active\catcode`+=\active %%
%% Initial definitions associated with ! and +.
\def\initialmd{\protected\def!{\startTTA} %%
  \protected\def+{\stopTTA}} %%
%% Step 1 of startTT. Inital meaning of !; capture next token in \next, go to next step.
\def\startTTA{\futurelet\next\startTTB} %%
%% Step 2 of startTT. Compare captured token to + and go to step 3 if true. Otherwise
%% output a ! (since that started our macro), the argument captured and stop
%% processing.
\long\def\startTTB{\ifx\next+\expandafter\startTTC\expandafter\@gobble\else\mdbang\fi} %%
%% Step 3 of startTT. Shift into teletype mode and change definition of 
%% + and ! so we can stop processing.
\def\startTTC{\begingroup\ifmmode %%
  \let \math@bgroup \relax %%
  \def \math@egroup {\let \math@bgroup \@@math@bgroup %%
    \let \math@egroup \@@math@egroup} %%
  \mathtt\relax %%
  \else  %%
  \ttfamily\fi} %%
%% Step 1, 2  and 3 of stopTT follow the same pattern as startTT.
\def\stopTTA{\futurelet\next\stopTTB} %%
\long\def\stopTTB{\ifx\next!\expandafter\stopTTC\expandafter\@gobble\else\mdplus\fi} %%
\def\stopTTC{\endgroup}%%
\catcode`!=12\catcode`+=12
\makeatother

\domd

%% Place an input file on the next page
\def\onnextpage#1{\afterpage{\clearpage\input{#1}\clearpage}}

\begin{document}
\ifthenelse{\boolean{standaloneFlag}}
           {\VerbatimFootnotes
             \DefineShortVerb{\#}
             \doublespacing
             \setcounter{chapter}{0}}{}

%% Default float parameters. For case when
%% multiple chapters are included and
%% only one needs custom float settings.
\renewcommand{\textfraction}{0.2}
\renewcommand{\topfraction}{0.9}


%% Float parameters
\renewcommand{\textfraction}{0.1}
\renewcommand{\topfraction}{0.9}

\chapter{Dataflow Optimization}
\label{ref_chapter_background}

%% A short section giving the history of dataflow optimization techniques
%% and basic concepts.

% Describe dataflow analysis in general terms and defines key
% concepts: basic blocks, control flow, facts, and
% rewrites. Bind/Return elimination is used as an an example.

The term ``program optimization'' generally refers to the idea of
transforming a program with some undesirable property to one with the
same semantics (i.e., result ) but without the unwanted property.  For
example, an optimized program may run faster, use less memory, consume
less power, or by whatever measure perform ``better'' than the
unoptimized program. Optimizations can be performed by the programmer
as part of writing a given program or they can be applied
automatically by analyzing the program and transforming it according
to an algorithm.

``Dataflow analysis'' (or ``dataflow optimization''), first introduced
by Gary Kildall \citep{Kildall1973}, refers to an algorithm for
applying an ``optimizing function'' to a given program. In itself it
does not give a specific optimization; rather, it gives a technique
for applying many different optimizations. In today's terms, dataflow
analysis treats a program as a ``control-flow graph'' (CFG), applies a
``transfer'' function to compute ``facts'' about the execution of the
program, and then applies a ``rewriting'' function to transform the
program based on those facts. Dataflow analysis is now considered
standard technique and can be found in most compilere textbooks.

\section{Dataflow Concepts}

The CFG for a program shows all possible execution ``paths'' that may
occur when the program runs. Nodes represent statements, while
directed edges show the order that statements may be executed.
Program execution enters the graph from one or more entry points and
leaves the graph through one or exit points.

Figure \ref{fig_back1} gives the CFG for a C-language program
fragment.  ``#E#'' represents the entry point for the program. The
conditional statement #if(a > b)# will be executed first. Execution
can then move to either of the two following nodes, depending on the
values of #a# and #b#. Finally, execution leaves the graph through one of
the exit nodes, indicated by ``#X#'' in the diagram.

\begin{figure}[th]
\centering
\figbegin

\begin{tabular}{cc}
\begin{minipage}[t]{1in}
\begin{Verbatim}[numbers=left]
if(a > b)
  c = a; 
else     
  c = b; 
\end{Verbatim}
\end{minipage} \vline & 
\begin{minipage}[t]{1in}
  \begin{Verbatim}[gobble=2]
      E
      ||
      V
    ---------        ------
    ||if(a > b)||-->||c = a||-->X
    ---------        ------
      ||
      V
     ------
    ||c = b||
     ------
      ||
      X
  \end{Verbatim}
\end{minipage} \\
 (a) & (b) \rule{0pt}{24pt}
\end{tabular}
\caption{A C-language program fragment (\emph{a}) and its associated
  control-flow graph (\emph{b}). Notice that all possible paths are
  shown in the CFG. Each specific execution of the program will
  depend on the values of \verb=a= and \verb=b=.}
\label{fig_back1}
\figend
\end{figure}

Dataflow analysis applies a ``transfer function'' to each node in the
CFG to compute ``facts'' about the program's state before and after
the execution of each node. The facts computed and the transfer
function used depend on the specific optimization, but dataflow
analysis always applies them in the same way. Each node has two sets
of facts -- ``in'' and ``out.'' The transfer function uses the ``in''
facts to compute ``out'' facts for a node. ``Out'' facts on a node
become ``in'' facts on the node analyzed next.  If the CFG for a
program contains loops, then ``in'' facts for a node may change based
on later ``out'' facts. The transfer function will be applied
repeatedly until the facts stop changing -- they reach a ``fixed
point.''

If the CFG for a program contains nodes with multiple predecessors,
``in'' facts must be combined in some way. The ``meet'' operator for
each optimization combines facts. Usually, the operator takes the
intersection or union of all the facts. Intersection implies the
analysis computes something is true for \emph{all} incoming paths. Union
implies computing something for \emph{any} incoming path.

Because we represent the CFG for a program as a directed graph, we can
choose which direction to traverse the CFG -- forwards or backwards.
When traversing forward, we usually compute facts about program
execution past a certain point (e.g., does a variable's value
change?); a backwards analysis computes facts up to a certain point
(e.g., what variables will be referenced following a given
statement?). Where a forwards analysis begins at the entry point(s)
for the CFG, a backwards analysis begins at the exit points.

Direction, the meet operator, facts, and the transfer function
together define the optimizing function applied by dataflow analysis
for a particular optimization. The result of the analysis is then used
to alter, or ``rewrite,'' the CFG. The meaning of the new program will
not be different than the old, but it will behave differently: execute
faster, use less memory, or whatever characteristic the optimization
should improve.

\section{Example: Dead-Code Elimination}

Consider Figure \ref{fig_back2}, again showing a C-language fragment.
After assignment on line \ref{fig_back2_dead_line}, #b# is not
referenced. Removing the #b# will not affect the program and,
if nothing else, will reduce the size of the program. It may even make
it run faster or use less memory. We call this optimization
\emph{dead-code elimination}.

\begin{figure}[ht]\centering
\figbegin
\begin{minipage}{1in}
  \begin{Verbatim}[numbers=left,commandchars=\\\{\}]
    a = 1;
    b = a + 1;\label{fig_back2_dead_line}
    return a + 1;
  \end{Verbatim}
\end{minipage}
\caption{A C-language fragment illustrating \emph{dead code}. After
assignment on line \ref{fig_back2_dead_line}, \verb=b= is not used
and can be considered ``dead.''}
\label{fig_back2}
\figend
\end{figure}

Of course, people do not normally write programs with such obviously
useless statements, but other compiler optimizations can produce (or
leave behind) many such statements. \emph{Uncurrying}, described in
chapter \ref{chap_uncurrying}, in fact depends on dead-code elmination.

To eliminate the assignment like that on line
\ref{fig_back2_dead_line}, we really need to determine which variables
are referenced after assignment. Such variables are ``live''; if a
variable is \emph{not} live, then it is dead. We use this ``liveness''
analysis to determine if a particular assignment is dead.

To determine if a variable is live, we need to know if it is
referenced after assignment.  Such variables make up the \emph{the
  live set} which we can compute between each statement. To compute
the live set, we can choose to traverse the CFG for the program forwards or
backwards.  In the forwards case, we must track each assignment and
determine, when we exit the fragment, if the variable was used
afterwards. In general we would need to track every assignment until
our traversal finished. However, if we traverse backwards, we only
need to note any reference to a variable. When we see an assignment to
a variable \emph{not} in our live set, we know it will not be
referenced afterwards. Therefore we compute ``liveness'' using a
backwards travesal over the CFG.

\begin{figure}[th]\centering
\figbegin
\begin{minipage}{2in}
\begin{Verbatim}[commandchars=\\\{\}]
       E
       ||      
       v
     -----
    ||a = 1||    \emph{live:}  \ensuremath{\emptyset}
     -----
       ||      
       V
   ---------
  ||b = a + 1||  \emph{live:} \{a\}  
   ---------
       ||      
       V
  ------------
 ||return a + 1|| \emph{live:} \{a\}
  ------------
       ||      
       X          \emph{live:}  \ensuremath{\emptyset}
\end{Verbatim}
\end{minipage}
\caption{The CFG for our example program, annotated with the live
set for each node.}
\label{fig_back3}
\figend
\end{figure}

Figure \ref{fig_back3} shows the CFG for this example, with annotations
between each statement showing the live set. Though
execution follows the arrows in the CFG, our analysis proceeds
backwards. For example, the input to node 2 is the live set computed
for node 3 (``$\{a\}$'' in this case).

Our transfer function computes the live set based on \emph{uses} and
\emph{definitions} in a statement. Any reference (or use) of a
variable goes into the live set. Any assignment (or definition) of a
variable removes it from the live set. We can then define our transfer
function, $live$, for a statement as:

\begin{align}
  & live(s) = (in(s) \cup use(s)) - def(s), \label{eqn_back1} \\
\intertext{where}
  & s     & \text{Statement considered.} \notag\\
  & use(s) &  \text{Set of variables used in $s$}. \notag\\
  & def(s) & \text{Variable assigned to in $s$ (a singleton set)}. \notag\\
  & in(s) & \text{Live variables computed for $s$' successor}. \notag
\end{align}

Table \ref{tbl_back1} shows the $use$ and $def$ sets for each
statement. The live set computed, $live$, becomes the input, $in$, for
the statement's predecessor. We include the exit node (``#X#'') in the
table to show the initial value of $in$ for the last statement --
$\emptyset$, the empty set. Our analysis then works backwards through the
program. If our program (and its CFG) contained any loops, we would
need to run this algorithm multiple times, until the live set for each
statement reached a fixed point.

\begin{table}
  \centering
  \begin{tabular}{lcccc}
    $s$ & $use(s)$ & $def(s)$ & $in(s)$ &  $live(s)$ \\
    \cmidrule(r){1-1}\cmidrule(r){2-2}\cmidrule(r){3-3}\cmidrule(r){4-4}\cmidrule(r){5-5}
    #X# & & & & $\emptyset$ \\
    #return a + 1# & $\{a\}$ & $\emptyset$ & $\emptyset$ & $\{a\}$ \\
    #b = a + 1# & $\{a\}$ & $\{b\}$ & $\{a\}$ & $\{a\}$ \\
    #a = 1# & $\emptyset$ & $\{a\}$ & $\{a\}$ & $\emptyset$ \\
    \bottomrule
  \end{tabular}
  \caption{The $use$, $def$ and $live$ sets computed using equation \ref{eqn_back1} for our example program.}
  \label{tbl_back1}
\end{table}

With the live set computed for each statement, our analysis can now
determine which statements to eliminate. Only nodes 1 and 2 in Figure
\ref{fig_back3} perform an assignment. The live set for node 1 (``#a = 1#'')
contains #a#, so we do not eliminate it. In node 2 (``#b = a + 1#''),
the live set does \emph{not} contain #b#. Therefore, we can eliminate
node 2, giving us a new program without any dead code:

\begin{Verbatim}
a = 1;
return a + 1;
\end{Verbatim}

\section{Conclusion}

This chapter gave an overview of \emph{dataflow optimization}, a
technique we used extensively in our work. The dataflow
\emph{algorithm} gives a general technique for applying an
\emph{optimizing function} to the \emph{control flow graph} (CFG)
representing a give program. The optimizing function computes
\emph{facts} about each node in the graph, using a \emph{transfer}
function to turn input facts into output facts. The CFG can be
traversed forwards or backwards (depending on the particular
optimization), and it may need to be traversed many times until the
computed facts reach a \emph{fixed point}.  Each optimization defines
a specific \emph{meet operator} that combines facts for nodes with
multiple inputs. Finally, the facts computed are used to
\emph{rewrite} the CFG, transforming the program so it still has the
same meaning, but behaves better, according to the optimization used.


%% \subsection{Basic Blocks and Control-Flow Graphs}

%% A dataflow optimization operates over a ``control-flow graph'' (CFG)
%% of the program -- a directed graph where edges encode branches or
%% jumps and nodes represent statements. Programs run by entering a node
%% from a predecessor, executing the statements in turn, and exiting the
%% node to a successor. Multiple successors imply a conditional branch,
%% though the program can only choose one. A special ``entry'' node, with
%% no predecssors, exists to give the program a starting point.

%% The statements in each node must define a ``basic block,'' which means
%% there can only be one entry and one exit to the node. Each
%% predeccessor starts at the same statement; execution cannot start in
%% the ``middle'' of the statements in the node. Each successor also
%% leaves from the same instruction, so only one ``branch'' can exist in
%% each node.

%% For example, consider the ``fall-through'' implied by the use of #case#
%% statements in this C-language program fragment:

%% \begin{verbatim}
%%   switch(i) {
%%   case 1:
%%     printf("1");
%%     break;
%%   case 2:
%%     printf("2");
%%   case 3:
%%     printf("3");
%%   }
%% \end{verbatim}

%% \begin{figure}[h]
%% \begin{verbatim}
%%    A
%%   switch   ----<-
%%   | |  |  |      |
%%   | |  |  v C    ^
%%   | |   ->case 3 |
%%   | |     |      |
%%   | |      ->----_--
%%   | | B          |  |
%%   |  ->case 2 ->-   v
%%   |                 |
%%   |   D       ----<-
%%    ->case 1  |
%%      |       v
%%      v       |
%%    --+-----<-
%%   |
%%    -> ...
%% \end{verbatim}
%% \caption{CFG illustrating \emph{fall-through} allowed by the
%%   C-language \texttt{switch} statement.}
%% \label{switchCfgEg}
%% \end{figure}

%% Figure \ref{switchCfgEg} shows a CFG for this fragment. Execution
%% begins at node A. Node C has two predeccessors: A and B. The edge
%% between Node B and C represents fall-through from the second to third
%% case. They cannot be combined because the node would need two distinct
%% entry points. Encoding a program into basic blocks usually involves
%% inserting similar branches. The CFG makes explicit control--flow that
%% exists by implication in the source program.

%% \subsection{Direction, Facts and Rewrites}

%% \subsection{Example: Bind/Return Collapse}

%% Dataflow optimizations transform the CFG representation of a program,
%% with the goal of making a faster (or smaller, or more efficient, etc.)
%% program. Dataflow computes a set of ``entry'' assumptions and ``exit''
%% facts for each node in the graph. Facts for one node become
%% assumptions for the nodes' successors (thus the term
%% ``dataflow''). The algorithm iteratves over the entire graph until a
%% fixed point is reached -- that is, facts and assumptions no longer
%% change. The computed facts can then be used to transform the graph.

%% \emph{Constant propagation example -- or something more functional?}

%% \emph{Introduce forward and backwards dataflow.}

% What does dataflow mean?

% How do you use it?

% Example

\end{document}


\documentclass[12pt]{report}
%include polycode.fmt
\usepackage[T1]{fontenc}
\usepackage{calc}
\usepackage{palatino}
\usepackage{amsfonts}
\renewcommand\ttdefault{lmtt}
\usepackage{helvet}
\usepackage{xspace}
\usepackage{url}
\usepackage{fancyvrb}
\usepackage[doublespacing]{setspace}
%% below only necessary when using doublespacing -- corrects
%% the vertical space inserted when switching to singlespace
%% environment.
\def\correctspaceskip{\vskip-\baselineskip} 
\usepackage{amsmath}
\usepackage{booktabs}
\usepackage[margin=\parindent, format=hang,labelfont=bf]{caption}
%% \usepackage[subrefformat=parens]{subcaption}
%% The following makes sure we get parentheses around
%% subreferences. The newest version of the subcaption
%% package has an option for this, but that's not available
%% widely.
%%
%% From http://tex.stackexchange.com/questions/25644
\usepackage[labelformat=simple]{subcaption}
\makeatletter
  \def\thesubfigure{(\alph{subfigure})}
  \providecommand\thefigsubsep{~}
  \def\p@subfigure{\@nameuse{thefigure}\thefigsubsep}
\makeatother

\usepackage{ifthen}
\usepackage{stmaryrd}
\usepackage{longtable}
\usepackage{afterpage}
\usepackage{xifthen}
\usepackage{mathtools}
\usepackage[natbib=true,style=authoryear,backend=bibtex8]{biblatex}
\setlength{\bibitemsep}{\bigskipamount}
\addbibresource{thesis.bib}
\usepackage{microtype}

%% GSO margins.
\usepackage[left=1.5in, right=1in, top=1in, bottom=1in]{geometry}
\usepackage{abstract}

%% GSO requires 12 pt font for all headings
\usepackage[bf,sf,tiny,compact]{titlesec}
\titleformat{\chapter}[display]
            {}% format
            {\sffamily\bfseries\chaptertitlename\ \thechapter}
            {\baselineskip}
            {\sffamily\bfseries}
            {}

\hyphenation{data-flow mo-na-dic} 

%% Should unindent all haskell code set in a dispay (versus inline)
\makeatletter
  \@ifundefined{hscodestyle}
               {}
               {\renewcommand{\hscodestyle}{\advance\leftskip -\mathindent}}
\makeatother

% Used by included files to know they
% are NOT standalone
\newboolean{standaloneFlag}
\setboolean{standaloneFlag}{true}

\newlength{\rulefigmargin}
\setlength{\rulefigmargin}{2\parindent}

\newcommand\figbegin{\rule{\linewidth}{0.4pt}\\\vspace{12pt}}
\newcommand\figend{\rule{\linewidth}{0.4pt}}

%% Sets
\newcommand{\setL}[1]{\textsc{#1}\xspace}
\newcommand{\setLC}{\setL{Const}}

%% Lub, subset operators.
\protected\def\lub{\ifmmode\sqcap\else\raisebox{.1em}{\ensuremath{\sqcap}}\fi\xspace}
\newcommand{\sqlt}{\ensuremath{\sqsubset}\xspace}
\newcommand{\sqlte}{\ensuremath{\sqsubseteq}\xspace}

%% Subscripting with typewriter
\def\subtt#1{\ifmmode_{\ensurett{#1}}%%
  \else$_{\ensurett{#1}}$%%
  \fi}
%% Superscripting with typerwriter
\def\suptt#1{\ifmmode^{\ensurett{#1}}%%
  \else$^{\ensurett{#1}}$%%
  \fi}
%% Functional languages chapter commands
\newcommand{\lamA}{\ensuremath{\lambda}-calculus\xspace}
\newcommand{\LamA}{\ensuremath{\lambda}-Calculus\xspace}
\newcommand{\lamAbs}[2]{\ensuremath{\lambda#1.\ #2}}
\newcommand{\lamApp}[2]{\ensuremath{#1\ #2}}
\newcommand{\lamPApp}[2]{\ensuremath{(#1\ #2)}}
\newcommand{\lamAPp}[2]{\ensuremath{(#1)\ #2}}
\newcommand{\lamApP}[2]{\ensuremath{#1\ (#2)}}
\newcommand{\lamAPP}[2]{\ensuremath{(#1)\ (#2)}}
\let\lamApPp=\lamApP
\let\lamAppP=\lamAPp
%% LC definition
\newtoks\toksA
\protected\def\lcname#1/{\ensuremath{\mathit{#1}}}
\protected\def\lcdef#1(#2)=#3;{\def\arg{#2}%%
  \def\lcargs##1,##2/{\def\arg{##2}%%
    \ifx\empty\arg%%
    \lcname ##1/%%
    \else\lcname ##1/\ \lcargs ##2/%%
    \fi}%%
  \ifx\empty\arg\toksA={\ }%%
  \else\toksA={\ \lcargs #2,/\ }%%
  \fi%%
  \ensuremath{\lcname#1/\the\toksA =\ #3}}
%% Arbitary number of applied arguments, separated
%% by asterisks (*).
\protected\def\lcapp#1/{\def\lcappB##1*##2/{\def\arg{##2}%
    \ensuremath{\ifx\arg\empty%%
      \lcname ##1/%%
      \else%%
      \lcname##1/\ \lcappB##2/%%
      \fi}}%%
  %% Adding a star here makes
  %% sure our applicaitn always ends with an asterisks, ensuring
  %% #2 will be \empty at some point.
  \lcappB#1*/}
\protected\def\lcabs#1.{\ensuremath{\lambda#1.\ }}

\newcommand{\lamId}{\lamAbs{x}{x}}
\newcommand{\lamCompose}{\lamAbs{f}{\lamAbs{g}{\lamAbs{x}{\lamApp{f}{(\lamApp{g}{x})}}}}}
\newcommand{\machLam}{\ensuremath{M_\lambda}\xspace}
\newcommand{\compMach}[1]{\ensuremath{\left\llbracket #1 \right\rrbracket}}
\newcommand{\compRho}[1]{\ensuremath{\rho(#1)}}
\newcommand{\verSub}[2]{\ensuremath{#1_{#2}}}
\newcommand{\verSup}[2]{\ensuremath{#1^{#2}}}
\newcommand{\lamC}{\ensuremath{\lambda_C}\xspace}
\newcommand{\lamPlus}{\lamAbs{m}{\lamAbs{n}{\lamAbs{s}{\lamAbs{z}{\lamApp{m}{\lamApPp{s}{\lamApp{n}{\lamApp{s}{z}}}}}}}}}
%% Substitution notation -- [#1 -> #2]
\newcommand{\lamSubst}[2]{\ensuremath{[#1 \mapsto #2]}}
%% End functional languages chapter


%% MIL Chapter
\newcommand{\compMILE}[1]{\ensuremath{\left\llbracket #1 \right\rrbracket}}
\newcommand{\compMILV}[1]{\ensuremath{\left\llbracket #1 \right\rrbracket}}
\newcommand{\compMILQ}[2]{\ensuremath{\left\llbracket #2 \right\rrbracket}}
\newcommand{\milCtx}[1]{\ensuremath{\llfloor}#1\ensuremath{\rrfloor}}

%% This dimension makes sure the same amount of space
%% follows | and := in syntax rules like:
%%
%% term := var       (Variable)
%%      |  var var    (Application)
%%      |  \x. var    (Abstraction)
%%
\newdimen\termalign
\setbox0=\hbox{$:=$}
\termalign=\wd0 
\protected\def\term#1/{\ensuremath{\mathit{#1}}}
\protected\def\syntaxrule#1/{\hfil\text{\emph{#1}}}
\protected\long\def\termrule#1:#2:#3/{\term #1/ &\hbox{$:=$}\ensuremath{\ #2} & \syntaxrule #3/}
\protected\def\termcase#1:#2/{&\hbox to \termalign{$|$\hss}\ensuremath{\ #1} & \syntaxrule #2/}


%% End MIL chapter

%% Dataflow Chapter
% Domain function
\def\dom(#1){\ensuremath{\mfun{dom}(#1)}\xspace}
% Set of all integers.
\def\ZZ{\ensuremath{\mathbb{Z}}}
%%

%% Uncurrrying Chapter 
%% A space equal to a \thinspace, but we
%% can break a line at it.
\newskip\thinskipamt \thinskipamt=.16667em 
\protected\def\thinskip{\hskip \thinskipamt\relax}
\protected\def\thinnerskip{\hskip .5\thinskipamt\relax}
%% Capture a space token. Use a ``control-symbol'' (\. instead of \mksp)
%% to keep the trailing space from getting gobbled.
{\def\.{\global\let\sp= } \. }
%% Define \asp, which will capture the macro definition attached to space,
%% if one exists. Otherwise, \spa is relax after this.
{\catcode`\ =\active\gdef\asp{\ifx \relax\let\spa\relax\else\let\spa= \fi}}
\newtoks\foo
%% Removes spaces, implicit, active and explicit.
\protected\def\removespaces{\asp\afterassignment\removesp\let\next= }
\def\removesp{\foo={\next}\ifcat\noexpand\next\sp\foo={\removespace}%%
 \else\ifx\next\spa\foo={\removespaces}\fi%%
 \fi\the\foo}
%% MIL reserved word
\protected\def\milres#1/{\text{\ttfamily\bfseries #1}}
\protected\def\lab#1/{\textbf{\ensurett{\removespaces #1}}}
%% Constructs a closure: l { v1, ..., vN }
\protected\long\def\mkclo[#1:#2]{\lab #1/\ensuremath{\,\{\ensurett{#2}\}}\xspace}
%% Tuple version of closurs: {l: v1, ..., vN}.
\protected\long\def\clo[#1:#2]{\def\argA{#1}\def\argB{#2}\ensuremath{\{%%
      \ifx\argA\empty%%
      \else\lab #1/%%
        \ifx\argB\empty%%
        \else\ensurett{:\thinskip}%%
        \fi%%
      \fi\ensurett{#2}\}}\xspace}
%% Construct a thunk
\newbox\bracklbox \newbox\brackrbox
\setbox0=\hbox{$\{$} \setbox\bracklbox=\hbox to \wd0{\hfil[\kern0.25mm}
\setbox0=\hbox{$\}$} \setbox\brackrbox=\hbox to \wd0{\kern0.25mm]\hfil}
\protected\def\mkthunk[#1:#2]{\lab #1/%%
  \ensuremath{\,%%
    \mathopen{\copy\bracklbox}%%
    \ensurett{#2}%%
    \mathclose{\copy\brackrbox}\xspace}}
%% Binding statement: v <- {...}
\protected\def\binds#1<-#2;{\ensurett{\removespaces #1\texttt{<-}#2}\xspace}
%% In order to use \binds in verbatim environment, have to define
%% delimiters while they are active. The below defines \vbinds which
%% must be used in AVerb environments.  Notice the active space as
%% well - that is necessary so the space after \vbinds (and before the
%% first argument) in the verbatim environment gets eaten.
\begingroup\catcode`\!=\active \lccode`\!=`\< \lccode`\~=`\- 
  \catcode`\ =\active\lowercase{\endgroup\def\vbinds#1!~#2;}{\binds#1<-#2;}
%% Return statement: return ... ;
\protected\def\return#1;{\milres return/\ensurett{\ \removespaces #1}}
%% A closure capturing block. k {v1, ..., vN} x: ...
\protected\def\ccblock#1(#2)#3:{\lab#1/\ensuremath{\thinspace\{\ensurett{#2}\}}\ \ensurett{#3\hbox{:}}}
%% A normal block
\protected\def\block#1(#2):{\lab #1/\ensuremath{\thinspace(\ensurett{#2})}\ensurett{:}}
%% A goto expression
\protected\def\goto#1(#2){\lab #1/\thinspace\ensuremath{(\ensurett{#2})}}
%% An enter expression
\protected\def\app#1*#2/{\ensurett{\removespaces #1\ifmmode\ \fi{\text{\tt @}}\ifmmode\ \fi#2}}
\protected\def\bind{\texttt{<-}\xspace}
%% Primitive expression
\protected\def\prim#1(#2){\lab #1/\suptt*\ensuremath{(\ensurett{#2})}}
%% Program variable
\protected\def\var#1/{\ensurett{\removespaces #1}\xspace}
%% Case statement
\protected\def\case#1;{\milres case\ \ensuremath{\ensurett{\removespaces #1}}\ of/}
%% Case alternative
\protected\def\alt#1(#2)#3->#4;{\ensuremath{\ensurett{#1\ \ignorespaces#2\ \texttt{->}\ \ignorespaces #4}}}
%% Invoke
\protected\def\invoke#1/{\milres invoke/\ensurett{\ \removespaces #1}}
\def\rhs{right--hand side\xspace}
\def\lhs{left--hand side\xspace}
\def\enter{\texttt{@}\xspace}
\def\cc{closure--capturing\xspace}
\def\Cc{Closure--capturing\xspace}
%%

\newenvironment{myfig}[1][tbh]{\begin{figure}[#1]%%
\begin{singlespace}\centering%%
\figbegin}{\figend\end{singlespace}%%
\end{figure}}

%% Produce a sub-caption and label it.
\newcommand{\scap}[2][1in]{\begin{minipage}{#1}%%
\subcaption{}\label{#2}\end{minipage}}

%% Produce a sub-caption with text.
\newcommand{\lscap}[3][\hsize]{\begin{minipage}{#1}%%
\subcaption{#3}\label{#2}\end{minipage}}

% single-argument comment. Do not put
% a space before the command when used
% or the file will have two spaces.
\newcommand{\rem}[1]{}

%% A verbatim environment with active charactesr
%% so we can use math shortcuts and macros
\DefineVerbatimEnvironment{AVerb}{Verbatim}{commandchars=\\\{\},%% 
  codes={\catcode`\_8\catcode`\$3\catcode`\^7},%%
  numberblanklines=false}

\DefineVerbatimEnvironment{Verb}{Verbatim}{commandchars=\\\[\],%% 
  numberblanklines=false}

%% Turn on line numbers for Haskell code, 
%% and reset the line number counter.
\newcommand{\hsNumOn}{\numberson\numbersreset}
\newcommand{\hsNumOff}{\numbersoff}
%% Turn on line numbering in Haskell code within
%% the environment, then turn it off. The optional
%% argument specifies a prefix that \hslabel can
%% use to generate line number references. If no prefix
%% is givne, \hslabel will have no effect.
\newtoks\prefixtoks
\def\mkhslabel#1{\prefixtoks={#1}\let\prefix=a}
\def\hslabel#1{\ifx\prefix\relax%%
  \else\label{\the\prefixtoks_#1}%%
  \fi}
\def\unhslabel{\let\prefix=\relax}
\newenvironment{withHsNum}{\numberson\numbersreset}{\numbersoff}
\newenvironment{withHsLabeled}[1]{\numberson\numbersreset\mkhslabel{#1}}{\unhslabel\numbersoff}

%% Paragraph run-in
\newcommand{\runin}[1]{\begingroup\noindent\sffamily\textbf{#1}\qquad\endgroup}

%% Chapter bibliographies
\newcommand{\standaloneBib}{%%
  \ifthenelse{\boolean{standaloneFlag}}%%
             {\begin{singlespace}
                 \printbibliography
             \end{singlespace}}{}}

%% Adds an equation number on demand.
\newcommand\addtag{\refstepcounter{equation}\tag{\theequation}}

%% For typesetting set definitions like {x | x \in f(y)}
\newcommand\setdef[2]{\ensuremath{\{#1\ |\ #2\}}}

%% For typesetting function names like dom(f) or out(b).
\newcommand\mfun[1]{\ensuremath{\mathit{#1}}}

%% Marginal notes
\newcommand\margin[2]{\marginpar{\begin{singlespace}\emph{\footnotesize #2}\end{singlespace}}\relax #1}

%% Describe intent of a passage
\newcommand\intent[1]{{\begin{singlespace}\noindent\leftskip=-1in\emph{\footnotesize Intent: #1}\end{singlespace}}\nopagebreak[1]}

%% In aligned/alignedat/gathered environments, you don't et
%% automatice equation numbers. This command makes sure to
%% label them properly.
\newcommand\labeleq[1]{\refstepcounter{equation}\label{#1}}

%% Creates a hanging paragraph, where the first line is not
%% indented but all other lines are.
\def\itempar#1{\noindent\hangindent=\parindent\hangafter=1 #1\quad}

%% Disable overfull messages with ridiculous hfuzz value
\def\disableoverfull{\hfuzz=10in}

%% Set parfillskip so stretching does NOT occur at the end of
%% a paragraph (i.e., list of elements). Disable indent at beginning
%% of paragraph. Also turn off underfull hbox warnings.
%%
%% Intended to be used in a \vbox that forms part of a table or graphic,
%% which we want to be line-broken but not exactly like a normal paragraph.
\long\def\disableparspacing#1;{\def\arg{#1}\hbadness=100000\parindent=0pt\parfillskip=0pt\leftskip=0pt\rightskip=0pt%%
  \ifx\arg\empty\else\hsize=#1\relax\fi}
%% This stuff makes !+<text>+! write <text> in typewriter font.  

%% We make ! and + active characters early, then manipulate their
%% meaning to produce the right effect. Initially, + produces +. When
%% !  appears w/o a + following, it produces ``!''. When ``+''
%% follows, we start writing in teleteype (\ttfamily). The definition
%% of ``!'' changes to produce a bang. ``+'' changes such that it
%% looks for trailing ``!''. When no ``!'' appears, ``+'' produces ``+''. 
%% If a ``!'' appears, we shift out of \ttfamily (by ending the group) and
%% reset the meaning of ``!'' and ``+'' so we can start again.
\makeatletter
\let\mdplus=+\let\mdbang=!      %% Preserve meaning of + and ! so we can put them into document.
%% Turn off mark down for everyone
\outer\def\nomd{\catcode`!=12\catcode`+=12}
%% Turn mark down on for everyone
\outer\def\domd{\catcode`!=\active\catcode`+=\active %%
  \initialmd}
%% Use only with a group IMMEDIETALY following. Turns off
%% markdown for the group-to-come, without actually tokenizing the
%% group. If no group follows, this has no effect.
\protected\def\pausemd{\def\dopause{\catcode`!=12\catcode`+=12}%%
  \def\pausemdB{\ifx\next\bgroup%%
    %% A ``partial'' application of expandwith is used
    %% so we don't double up the group argument (which is what
    %% happens if we expand \next). This has the effect of 
    %% inserting \expandafter\dowith in front of the upcoming {. 
    %% If no brace is coming, \withmdC will have no effect.
    \def\pausemdC{\expandafter\dopause}
  \else
    \let\pausedmC=\relax
  \fi\pausemdC}
  %% \futurelet has to end the macro so it grabs the next token
  %% from the input file. Otherwise, it grabs it *from* this
  %% definition.
  \futurelet\next\pausemdB} %%
%% Turns markdown on for the group-to-come, without actually
%% tokenizing the group. Only has an effect when
%% used in front of a group, otherwise its a no-op.
\protected\def\withmd{\def\dowith{\catcode`!=\active\catcode`+=\active\initialmd}%%
  \def\withmdB{\ifx\next\bgroup %%
    %% A ``partial'' application of expandafter is used
    %% so we don't double up the group argument (which is what
    %% happens if we expand \next). This has the effect of 
    %% inserting \expandafter\dowith in front of the upcoming {. 
    %% If no brace is coming, \withmdC will have no effect.
      \def\withmdC{\expandafter\dowith} %%
    \else %%
      \let\withmdC=\relax %%
    \fi\withmdC}%%
  %% \futurelet has to end the macro so it grabs the next token
  %% from the input file. Otherwise, it grabs it *from* this
  %% definition.
  \futurelet\next\withmdB} %%
%% Make ! and + active in the following group so they have the right
%% catcode in the definitions to follow.
\catcode`!=\active\catcode`+=\active %%
%% Initial definitions associated with ! and +.
\def\initialmd{\protected\def!{\startTTA} %%
  \protected\def+{\stopTTA}} %%
%% Step 1 of startTT. Inital meaning of !; capture next token in \next, go to next step.
\def\startTTA{\futurelet\next\startTTB} %%
%% Step 2 of startTT. Compare captured token to + and go to step 3 if true. Otherwise
%% output a ! (since that started our macro), the argument captured and stop
%% processing.
\long\def\startTTB{\ifx\next+\expandafter\startTTC\expandafter\@gobble\else\mdbang\fi} %%
%% Step 3 of startTT. Shift into teletype mode and change definition of 
%% + and ! so we can stop processing.
\def\startTTC{\begingroup\ifmmode %%
  \let \math@bgroup \relax %%
  \def \math@egroup {\let \math@bgroup \@@math@bgroup %%
    \let \math@egroup \@@math@egroup} %%
  \mathtt\relax %%
  \else  %%
  \ttfamily\fi} %%
%% Step 1, 2  and 3 of stopTT follow the same pattern as startTT.
\def\stopTTA{\futurelet\next\stopTTB} %%
\long\def\stopTTB{\ifx\next!\expandafter\stopTTC\expandafter\@gobble\else\mdplus\fi} %%
\def\stopTTC{\endgroup}%%
\catcode`!=12\catcode`+=12
\makeatother

\domd

%% Place an input file on the next page
\def\onnextpage#1{\afterpage{\clearpage\input{#1}\clearpage}}

\begin{document}
\ifthenelse{\boolean{standaloneFlag}}
           {\VerbatimFootnotes
             \DefineShortVerb{\#}
             \doublespacing
             \setcounter{chapter}{0}}{}

%% Default float parameters. For case when
%% multiple chapters are included and
%% only one needs custom float settings.
\renewcommand{\textfraction}{0.2}
\renewcommand{\topfraction}{0.9}


\chapter{Functional Languages}
\label{ref_chapter_languages}

%% \emph{Brief definitino of ``functional languages.'' Introduce our
%%   lambda calculus variant. Shows a compilation scheme from the variant
%%   to Caffeine's RTL as an example}

\section{The \LamA}
%%\emph{Why is it important}

In mathematics, a ``function'' takes one or more arguments and
produces some value. ``Functional programming languages'' are based
around the same idea -- the definition and evaluation of
functions. They all share share the ability to manipulate functions
during execution; that is, they are not limited to executing only the
function definitions specified by the programmer -- they can define
new functions as part of a computation.

%%\emph{What is the \lamA}

Alonzo Church defined his \lamA (``lambda calculus'') in 19XX
\citep{ChurchXX} to study systems of recursive equations. Being
Turing-complete, it can be used to model the behavior of any
computational system. However, it is particularly useful for modeling
functional programming languages. %% why?

Figure \ref{fig_lang1} gives a syntax for the ``pure''
\lamA. ``Abstraction'' defines a new function, while ``application''
gives a particular argument to a function. ``Variables'' are defined
when mentioned. 

\begin{myfig}[ht]
\begin{minipage}{3in}
\begin{Verbatim}
##                 ##      ##       
 #                  #       #       
 #  ###  ########   ###   ###  ###  
 #  ###   #  #  #   #  # #  #  ###  
 #  # #   #  #  #   #  # #  #  # #  
### ## # #########  ###   #### ## # 
\end{Verbatim}
\end{minipage}
  \caption{The \lamA' syntax.}
  \label{fig_lang1}
\end{myfig}

%%\emph{What does it look like?}

Using this syntax, we can define some common functions. \emph{Identity} 
returns its argument:
\begin{align}
  & \lamAbs{x}{x}. \\
\intertext{\emph{Compose} takes two functions and an argument. The result of
applying the second function to the argument is passed to the first:}
  & \lamCompose. \\
\intertext{\emph{Const} takes two arguments but always returns the first:}
  & \lamAbs{f}{\lamAbs{a}{\lamAbs{b}{a}}}.
\end{align}

\begin{myfig}[bt]
\begin{minipage}{2in}
\begin{Verbatim}
                ##  
                 #  
 ##  ## ## ###   #  
####  # ## ###   #  
#     ###  # #   #  
 ###   #   ## # ### 
\end{Verbatim}
\end{minipage}
  \caption{Evaluation rules for \lamA. These rules show 
    \emph{call-by-value}, where arguments are evaluated
    before functions.}
  \label{fig_lang2}
\end{myfig}

A \lamA term executes by rewriting the expression according to the
rules in Figure \ref{fig_lang2}. We match our term to each of the
patterns above the line. If we have a match, we rewrite according to
the pattern below the line. When no more matches can be made, we say
the term is in \emph{normal form}: we have finished executing.

The rules given implement \emph{call-by-value} evaluation order,
meaning arguments to a function are evaluated before the function
itself. Other variants include \emph{call-by-need} and
\emph{call-by-name}, where arguments are not evaluated until
needed. We do not considers those variants further, however.

\section{Compiling the \LamA}
\label{sec_lang1}

%% Define which steps in compilation we're going to worry about
Compiling even a language as simple as the \lamA involves a number of
steps, such as defining a concrete syntax, parsing source programs
into an \emph{abstract syntax tree} (AST), and producing an executable
program from the AST. For our purposes, however, we just focus on the
\lamA' three fundamental operations:

\begin{itemize}
\item Find a value by name (\emph{variables}).
\item Apply a function to an argument (\emph{application}).
\item Create a new function (\emph{abstraction}). 
\end{itemize}

Any compiler for the \lamA must be able to produce executable programs
which implement these operations. 

\subsection{The Target Machine}
We begin by defining a \emph{target machine}, |M|, for our compiler. To
reduce complexity we do not target an actual computer, but one of our
own design. Our machine will have an infinite number of
\emph{registers} (i.e., storage locations) that we can refer to by
name. It will have an unlimited supply of memory (called the
\emph{heap}) in which we can allocate structured values. However, we
will not refer to memory locations directly. Instead, we will always
store references to heap values in registers. Finally, the machine
will execute a list of instructions (our \emph{program}), starting at
the beginning and proceeding in sequential order (unless otherwise
instructed), until reaching the end of the list. Each instruction will
have a definite location, but we will only refer to certain special
locations using named labels.

\subsection{M's Language: \machLam}
Table \ref{tbl_lang1} gives the language that our machine will
execute, \machLam. A benefit of defining our own machine is that we
can also define the language it executes -- and the language we need
to compile to! We cannot make it too dissimilar from a ``real''
machine, but at this stage it helps to keep things simple. 

\begin{table}[th]
  \centering
  \begin{tabular}{lp{3.5in}}
    \emph{Instruction} & \emph{Description} \\
    \cmidrule(r){1-1}\cmidrule(r){2-2}
    \texttt{Store \emph{R} (\emph{F}, \emph{M})} & Store the value found in register #R# to field %%
    #F# of the value in register #M#. \\
    \texttt{Load (\emph{F}, \emph{M}) \emph{R}} & Load field #F# of the value in register #M# to register #R#. \\
    \texttt{Set \emph{v} \emph{R}} & Sets the register #R# to name of the variable $v$. \\
    \texttt{Copy \emph{R} \emph{M}} & Copies the contents of register #R# to register #M#. \\
    #Enter# & Jump to the location indicated by the closure in
    register #clo#, assuming an argument in register #arg#. The next #Return# executed
    will return to this location, with a result in register #res#.\\
    #Return# & Jump to the instruction following the most recently 
    executed #Enter# instruction and begin executing.  \\
    \texttt{MkClo \emph{L} [\emph{R}, \emph{S}, \dots]} &  Create a closure pointing to the 
    label #L# and holding the values in registers #R#, #S#, etc. The closure will be stored in 
    the #res# register.
  \end{tabular}
  \caption{\machLam, the ``machine language'' executed by our machine |M|.}
  \label{tbl_lang1}
  \figend
\end{table}

Each instruction supports an some aspect of the \lamA. In brief:
\begin{description}
\item[Variables] -- #Store# and #Load# help access variables and
  function arguments.
\item[Function Application] -- #Enter# and #Return# allow us to execute a function with arguments.
\item[Abstraction] -- #MkClo# lets us create functions as values.
\end{description}
The following sections describe each aspect in detail.

\subsection{Variables}

%% Free variables and environment
Consider how to find a value by its name. For example, in the
following program fragment
\begin{equation}
  \lamAbs{x}{\lamApp{f}{\lamPApp{g}{x}}}.
  \label{eq_lang1}
\end{equation}
we see three variables: $f$, $g$, and $x$. We say $x$ is \emph{bound},
because it is given as an argument, and that $f$ and $g$ are
\emph{free} because, in this context, they are not bound by a
$\lambda$-abstraction. To evaluate this expression, though, we need
a way to find the values of these terms.  

We can describe where to find $f$, $g$ and $x$ in terms of memory
locations. We can say that $x$ will appear in a special location,
$arg$, because it is the argument to the function and we will always
put arguments in the same place. We can further say that another
special location, $clo$, will have two
slots. The first will contain $g$ and the second will contain
$f$. Conceptually, then, our expression can be represented as:
\begin{center}
  \begin{tabular}{c}
    \begin{math}\begin{aligned}[b]
      arg &= x, \\
      clo[0] &= g, \\
      clo[1] &= f 
    \end{aligned}\text{\ in}\end{math} \\
    \lamAbs{arg}{\lamApp{clo[1]}{\lamPApp{clo[0]}{arg}}}.
  \end{tabular}
\end{center}

\par
In general, the $clo$ location holds the \emph{environment} for our
expression. For any given expression, we will be able to find all the
free variables (i.e., all those except the argument) in the
environment. The compiler will be responsible for ensuring the correct
environment is available whenever a given expression is evaluated.

Our machine, then, must have instructions for storing and retrieving
values. #Store# and #Load# (from Table \ref{tbl_lang1}) serve this
purpose. 

\subsection{Function Application}

%% Application & closures
Associating location with names is not enough, however. Looking again
at Equation \ref{eq_lang1}, it is clear $g$ represents a function, to
which we are passing the argument $x$. To compute the value of
$\lamPApp{g}{x}$, we must be able to execute the code representing
$g$. Because we have a storage location for $g$ already, we can
say that the value in $clo[0]$ holds a \emph{label} that tells us
where to find the code representing $g$. 

We now have two different types of values: an environment which tells
the currently evaluating function where to find its free variables;
and labels which tells us where to find the code representing a
function. However, $g$ may refer to any number of free variables, so
it makes sense to pair the label in $clo[0]$ with $g$'s
environment. We call this data structure a \emph{closure}. Closures
are the fundamental data structures used to compile functional
languages. They may not have the exact form described here but they
always have the same purpose: they pair a label with the free
variables used in the function represented.

#Enter# and #Return# (give in Table \ref{tbl_lang1}) implement
function application. #Enter# expects to find a closure in register
#clo# and will cause the machine to start executing the code at the
label given in the closure. A #Return# instruction will cause the
machine to jump back to the instruction following the most recently
executed #Enter# instruction. The machine will maintain a stack of
return locations so that #Enter# instructions can be nested.

We also use the #Copy# instruction to save and restore registers before
and after an #Enter# instruction. 

\subsection{Abstraction}
The \lamA lets us define functions which return new functions. We have
seen how to access variables in the environment and how to execute
unknown functions using closures. Now we come to the final element
needed to compile our \lamA to \machLam -- how to create
closures.

Consider the following expression, where we apply the |identity|
function to an argument:
\[\lamApp{(\lamAbs{x}{x})}{s}.\]
From the previous section, we know #Enter# and #Return# are used to
implement the application of \lamAbs{x}{x} to $s$. It follows that
\lamAbs{x}{x} must create a closure which #Enter# will then use to
execute the body of the $\lambda$-abstraction. In fact, Each
$\lambda$ in our source program returns a closure and that closure
points to code that implements the body of the $\lambda$ term.

The #MkClo# (from Table \ref{tbl_lang1}) instruction creates closures
for us. It takes a label, pointing to the code which will be executed
by the #Enter# instruction, and a list of registers, holding the free
variables found in the body of the expression.

\subsection{Compiling from \lamA to \machLam}

Table \ref{tbl_lang2} gives our algorithm to compile from \lamA to
\machLam. We present it in in four parts, \emph{a} - \emph{d},
corresponding to the syntax of \lamA terms given in Figure
\ref{fig_lang1}. The ``fat brackets,'' \compMach{t}, represent our
compiler, with the term being compiled given as the argument, $t$.
Each term compiles to a given sequence of instructions. We also assume
a function $\rho$, maintained by the compiler, that knows which
register holds a given variable.

%% Compilation rules ...
\afterpage{\clearpage{%% Used in the languages chapter, this
%% table is placed in its own file so we can use
%% it with the afterpage command.
\begin{singlespace}
  \begin{longtable}{p{2in}l@{\vline\hspace{.1in}}p{3.5in}}
    \caption{Compilation rules from \lamA to \machLam.} \\
    \hline \\
    \endfirsthead
    \caption{Compilation rules from \lamA to \machLam \emph{(cont'd)}} \\
    \hline \\
    \endhead
    \\ \hline \multicolumn{3}{r}{\emph{Continued on next page}}
    \endfoot 
    \\ \hline
    \endlastfoot
    %% Variables
    \multicolumn{3}{c}{\emph{(a) Variable Reference}} \\ \\[-.5em]
    \begin{minipage}[t]{2in}
      \begin{AVerb}
\compMach{v} = 
  Copy \compRho{v} ``res''
      \end{AVerb}
    \end{minipage} & & We use the $\rho$ function to find the register containing the variable $v$. As the compiler
    maintains $\rho$, this lookup is performed during compilation, not while the program executes.\\ \\

    %% Application
    \multicolumn{3}{c}{\emph{(b) Function Application}} \\ \\[-.5em]
    \begin{minipage}[t]{2in}
      \begin{AVerb}
\compMach{\lamApp{f}{g}} = 
      \end{AVerb}
    \end{minipage} \\

    \begin{minipage}[t]{2in}
      \begin{AVerb}
  Copy ``arg'' $r$
  Copy ``clo'' $s$
      \end{AVerb}
    \end{minipage} &  & $r$ and $s$ are ``fresh'' registers. \\ \\[-.5em]

    \begin{minipage}[t]{2in}
      \begin{AVerb}
  \compMach{g}
  Copy ``res'' $t$
  \compMach{f}
  Copy ``res'' $u$
      \end{AVerb}
    \end{minipage} &  & Because $g$ and $f$ may be terms themselves, we compile their code inline 
    first. We copy the result of evaluating each from #res# to $t$ and
    $u$, respectively (both fresh registers). We do not use #arg# and
    #clo# because they may get overwritten while evaluating $g$ and
    $f$. \\ \\[-.5em]

    \begin{minipage}[t]{2in}
      \begin{AVerb}
  Copy $t$ ``arg''
  Copy $u$ ``clo''
  Enter
      \end{AVerb}
    \end{minipage} &  & We copy the result of $g$ and $f$ to #arg# and #clo#, 
    respectively. After #Enter# executes, #res# will hold the result
    of \lamPApp{f}{g}. However, we do not mention #res# here becouse
    our case \emph{(a)} handles the results of functions.\\ \\[-.5em]

    \begin{minipage}[t]{2in}
      \begin{AVerb}
  Copy $r$ ``arg''
  Copy $s$ ``clo''
      \end{AVerb}
    \end{minipage} &  & Restore the previous #arg# and #clo# registers. \\ \\

    %% Abstraction
    \multicolumn{3}{c}{\emph{(c) Abstraction}} \\ \\[-.5em]
    \begin{minipage}[t]{2in}
      \begin{AVerb}
\compMach{\lamAbs{x}{t}} = 
l : 
  \compMach{t}
  Return 
      \end{AVerb}
    \end{minipage} &  & We mark the location of $t$'s compiled form with #l#, a fresh
    label. We place the compiled code for $t$ inline. Because a function always
    produces a result, we follow the code with #Return#. \\ \\[-.5em]

    \begin{minipage}[t]{2in}
      \begin{AVerb}
m : 
      \end{AVerb}
    \end{minipage} &  & Again, we mark this portion of the program 
    with a fresh label, #m#. \\* \\*[-.5em]
    
    \begin{minipage}[t]{2in}
      \begin{AVerb}
  MkClo l [\compRho{v_1}, \dots, 
           \compRho{v_N}, 
           ``arg'']
  Return
      \end{AVerb}
    \end{minipage} &  & We build a closure holding the \emph{free variables} found in
    $t$. $v_1, \dots, v_N$ represent the free variables, and we use
    $\rho$ to find the location of those variables. We also add our
    argument to the end of the closure. Finally, the closure points to
    #l#, the label marking the location of the compiled body,
    $t$. Because #MkClo# puts the closure created in #res#, we can
    immediately return. 

  \label{tbl_lang2}
  \end{longtable}
\end{singlespace}
}\clearpage}

Table \ref{tbl_lang2}, part \emph{a}, shows the compilation
scheme for variables. Variable refrences that are not used
in function application can only be the body of an expression, so we
just copy the variable's name to the #res#
register and return.

Function application, \lamPApp{f}{g}, is shown in part
\emph{b}. To apply a function, we must save the current #clo#
and #arg# registers. The compiler creates \emph{fresh} registers,
guaranteed to be unused anywhere else in the program, to store #clo#
and #arg#. We then use $\rho$ to find the registers holding $f$ and
$g$. Remember that $f$ will be a closure, while $g$ will be some
value. We copy those values into #clo# and #arg#. The #Enter#
instruction will execute the code pointed to by #clo#. When that
function returns, we restore #clo# and #arg# from the fresh registers
created earlier.

Abstractions, such as \lamAbs{x}{t}, return a closure pointing to the
code implementing $t$. Therefore, our compiler needs to generate code
that returns a closure, which in turn points to the code generated for
the body of the abstraction. To accomplish this, our compiler
recursively calls itself on the body. We get a label back, which is the
location of the just compiled code. In parts \emph{c} and \emph{d}
the expression $l = \compMach{\lamAbs{y}{t}}$ shows this
recursive call, and the label that results. That label can then be used in the 
closure returned by the abstraction.

We separate compilation of abstractions into two cases, depending if
the body is an abstraction or not. In the first case, as shown in part
\emph{c}, we begin by marking the location of this code with a new label,
#m#. We prepare to create a new closure by copying all values out of
the current closure into fresh registers. We then create a closure that
points to the body of our abstraction, contains all the values found
in the current closure, and ``captures'' our argument in the new
closure. 

For example, consider compiling this expression:

\begin{equation}
  \lamAbs{x}{\lamAbs{y}{\lamApp{f}{\lamPApp{y}{x}}}}. 
\end{equation}

$f$ and $x$ must be available when the body
\lamPApp{f}{\lamPApp{y}{x}} executes. Therefore, the closure returned
by \lamAbs{x}{(\dots)} must copy all values in the existing
closure as well as add the argument, $x$.

Part \emph{d} shows the code generated when the body of an abstraction
is \emph{not} another abstraction. We first mark the location of the
start of the body with a new label, #m#.  We then find the free
variables in the body, calling them $v_1, \dots, v_n$. This is a
compile-time operation, not something the program will do when
executing.  We assume that value of each free variables can be found
in the corresponding closure slot. For example, $v_0$ will be found in
$clo[0]$, $v_1$ in $clo[1]$, and so on. We also copy the $arg$
register to the corresponding register for our argument, as determined
by the $\rho$ function. Now that we have placed all variables in the
registers expected by our function, we generate the code for our body
and place it inline.

\end{document}


\documentclass[12pt]{report}
%include polycode.fmt
\usepackage[T1]{fontenc}
\usepackage{calc}
\usepackage{palatino}
\usepackage{amsfonts}
\renewcommand\ttdefault{lmtt}
\usepackage{helvet}
\usepackage{xspace}
\usepackage{url}
\usepackage{fancyvrb}
\usepackage[doublespacing]{setspace}
%% below only necessary when using doublespacing -- corrects
%% the vertical space inserted when switching to singlespace
%% environment.
\def\correctspaceskip{\vskip-\baselineskip} 
\usepackage{amsmath}
\usepackage{booktabs}
\usepackage[margin=\parindent, format=hang,labelfont=bf]{caption}
%% \usepackage[subrefformat=parens]{subcaption}
%% The following makes sure we get parentheses around
%% subreferences. The newest version of the subcaption
%% package has an option for this, but that's not available
%% widely.
%%
%% From http://tex.stackexchange.com/questions/25644
\usepackage[labelformat=simple]{subcaption}
\makeatletter
  \def\thesubfigure{(\alph{subfigure})}
  \providecommand\thefigsubsep{~}
  \def\p@subfigure{\@nameuse{thefigure}\thefigsubsep}
\makeatother

\usepackage{ifthen}
\usepackage{stmaryrd}
\usepackage{longtable}
\usepackage{afterpage}
\usepackage{xifthen}
\usepackage{mathtools}
\usepackage[natbib=true,style=authoryear,backend=bibtex8]{biblatex}
\setlength{\bibitemsep}{\bigskipamount}
\addbibresource{thesis.bib}
\usepackage{microtype}

%% GSO margins.
\usepackage[left=1.5in, right=1in, top=1in, bottom=1in]{geometry}
\usepackage{abstract}

%% GSO requires 12 pt font for all headings
\usepackage[bf,sf,tiny,compact]{titlesec}
\titleformat{\chapter}[display]
            {}% format
            {\sffamily\bfseries\chaptertitlename\ \thechapter}
            {\baselineskip}
            {\sffamily\bfseries}
            {}

\hyphenation{data-flow mo-na-dic} 

%% Should unindent all haskell code set in a dispay (versus inline)
\makeatletter
  \@ifundefined{hscodestyle}
               {}
               {\renewcommand{\hscodestyle}{\advance\leftskip -\mathindent}}
\makeatother

% Used by included files to know they
% are NOT standalone
\newboolean{standaloneFlag}
\setboolean{standaloneFlag}{true}

\newlength{\rulefigmargin}
\setlength{\rulefigmargin}{2\parindent}

\newcommand\figbegin{\rule{\linewidth}{0.4pt}\\\vspace{12pt}}
\newcommand\figend{\rule{\linewidth}{0.4pt}}

%% Sets
\newcommand{\setL}[1]{\textsc{#1}\xspace}
\newcommand{\setLC}{\setL{Const}}

%% Lub, subset operators.
\protected\def\lub{\ifmmode\sqcap\else\raisebox{.1em}{\ensuremath{\sqcap}}\fi\xspace}
\newcommand{\sqlt}{\ensuremath{\sqsubset}\xspace}
\newcommand{\sqlte}{\ensuremath{\sqsubseteq}\xspace}

%% Subscripting with typewriter
\def\subtt#1{\ifmmode_{\ensurett{#1}}%%
  \else$_{\ensurett{#1}}$%%
  \fi}
%% Superscripting with typerwriter
\def\suptt#1{\ifmmode^{\ensurett{#1}}%%
  \else$^{\ensurett{#1}}$%%
  \fi}
%% Functional languages chapter commands
\newcommand{\lamA}{\ensuremath{\lambda}-calculus\xspace}
\newcommand{\LamA}{\ensuremath{\lambda}-Calculus\xspace}
\newcommand{\lamAbs}[2]{\ensuremath{\lambda#1.\ #2}}
\newcommand{\lamApp}[2]{\ensuremath{#1\ #2}}
\newcommand{\lamPApp}[2]{\ensuremath{(#1\ #2)}}
\newcommand{\lamAPp}[2]{\ensuremath{(#1)\ #2}}
\newcommand{\lamApP}[2]{\ensuremath{#1\ (#2)}}
\newcommand{\lamAPP}[2]{\ensuremath{(#1)\ (#2)}}
\let\lamApPp=\lamApP
\let\lamAppP=\lamAPp
%% LC definition
\newtoks\toksA
\protected\def\lcname#1/{\ensuremath{\mathit{#1}}}
\protected\def\lcdef#1(#2)=#3;{\def\arg{#2}%%
  \def\lcargs##1,##2/{\def\arg{##2}%%
    \ifx\empty\arg%%
    \lcname ##1/%%
    \else\lcname ##1/\ \lcargs ##2/%%
    \fi}%%
  \ifx\empty\arg\toksA={\ }%%
  \else\toksA={\ \lcargs #2,/\ }%%
  \fi%%
  \ensuremath{\lcname#1/\the\toksA =\ #3}}
%% Arbitary number of applied arguments, separated
%% by asterisks (*).
\protected\def\lcapp#1/{\def\lcappB##1*##2/{\def\arg{##2}%
    \ensuremath{\ifx\arg\empty%%
      \lcname ##1/%%
      \else%%
      \lcname##1/\ \lcappB##2/%%
      \fi}}%%
  %% Adding a star here makes
  %% sure our applicaitn always ends with an asterisks, ensuring
  %% #2 will be \empty at some point.
  \lcappB#1*/}
\protected\def\lcabs#1.{\ensuremath{\lambda#1.\ }}

\newcommand{\lamId}{\lamAbs{x}{x}}
\newcommand{\lamCompose}{\lamAbs{f}{\lamAbs{g}{\lamAbs{x}{\lamApp{f}{(\lamApp{g}{x})}}}}}
\newcommand{\machLam}{\ensuremath{M_\lambda}\xspace}
\newcommand{\compMach}[1]{\ensuremath{\left\llbracket #1 \right\rrbracket}}
\newcommand{\compRho}[1]{\ensuremath{\rho(#1)}}
\newcommand{\verSub}[2]{\ensuremath{#1_{#2}}}
\newcommand{\verSup}[2]{\ensuremath{#1^{#2}}}
\newcommand{\lamC}{\ensuremath{\lambda_C}\xspace}
\newcommand{\lamPlus}{\lamAbs{m}{\lamAbs{n}{\lamAbs{s}{\lamAbs{z}{\lamApp{m}{\lamApPp{s}{\lamApp{n}{\lamApp{s}{z}}}}}}}}}
%% Substitution notation -- [#1 -> #2]
\newcommand{\lamSubst}[2]{\ensuremath{[#1 \mapsto #2]}}
%% End functional languages chapter


%% MIL Chapter
\newcommand{\compMILE}[1]{\ensuremath{\left\llbracket #1 \right\rrbracket}}
\newcommand{\compMILV}[1]{\ensuremath{\left\llbracket #1 \right\rrbracket}}
\newcommand{\compMILQ}[2]{\ensuremath{\left\llbracket #2 \right\rrbracket}}
\newcommand{\milCtx}[1]{\ensuremath{\llfloor}#1\ensuremath{\rrfloor}}

%% This dimension makes sure the same amount of space
%% follows | and := in syntax rules like:
%%
%% term := var       (Variable)
%%      |  var var    (Application)
%%      |  \x. var    (Abstraction)
%%
\newdimen\termalign
\setbox0=\hbox{$:=$}
\termalign=\wd0 
\protected\def\term#1/{\ensuremath{\mathit{#1}}}
\protected\def\syntaxrule#1/{\hfil\text{\emph{#1}}}
\protected\long\def\termrule#1:#2:#3/{\term #1/ &\hbox{$:=$}\ensuremath{\ #2} & \syntaxrule #3/}
\protected\def\termcase#1:#2/{&\hbox to \termalign{$|$\hss}\ensuremath{\ #1} & \syntaxrule #2/}


%% End MIL chapter

%% Dataflow Chapter
% Domain function
\def\dom(#1){\ensuremath{\mfun{dom}(#1)}\xspace}
% Set of all integers.
\def\ZZ{\ensuremath{\mathbb{Z}}}
%%

%% Uncurrrying Chapter 
%% A space equal to a \thinspace, but we
%% can break a line at it.
\newskip\thinskipamt \thinskipamt=.16667em 
\protected\def\thinskip{\hskip \thinskipamt\relax}
\protected\def\thinnerskip{\hskip .5\thinskipamt\relax}
%% Capture a space token. Use a ``control-symbol'' (\. instead of \mksp)
%% to keep the trailing space from getting gobbled.
{\def\.{\global\let\sp= } \. }
%% Define \asp, which will capture the macro definition attached to space,
%% if one exists. Otherwise, \spa is relax after this.
{\catcode`\ =\active\gdef\asp{\ifx \relax\let\spa\relax\else\let\spa= \fi}}
\newtoks\foo
%% Removes spaces, implicit, active and explicit.
\protected\def\removespaces{\asp\afterassignment\removesp\let\next= }
\def\removesp{\foo={\next}\ifcat\noexpand\next\sp\foo={\removespace}%%
 \else\ifx\next\spa\foo={\removespaces}\fi%%
 \fi\the\foo}
%% MIL reserved word
\protected\def\milres#1/{\text{\ttfamily\bfseries #1}}
\protected\def\lab#1/{\textbf{\ensurett{\removespaces #1}}}
%% Constructs a closure: l { v1, ..., vN }
\protected\long\def\mkclo[#1:#2]{\lab #1/\ensuremath{\,\{\ensurett{#2}\}}\xspace}
%% Tuple version of closurs: {l: v1, ..., vN}.
\protected\long\def\clo[#1:#2]{\def\argA{#1}\def\argB{#2}\ensuremath{\{%%
      \ifx\argA\empty%%
      \else\lab #1/%%
        \ifx\argB\empty%%
        \else\ensurett{:\thinskip}%%
        \fi%%
      \fi\ensurett{#2}\}}\xspace}
%% Construct a thunk
\newbox\bracklbox \newbox\brackrbox
\setbox0=\hbox{$\{$} \setbox\bracklbox=\hbox to \wd0{\hfil[\kern0.25mm}
\setbox0=\hbox{$\}$} \setbox\brackrbox=\hbox to \wd0{\kern0.25mm]\hfil}
\protected\def\mkthunk[#1:#2]{\lab #1/%%
  \ensuremath{\,%%
    \mathopen{\copy\bracklbox}%%
    \ensurett{#2}%%
    \mathclose{\copy\brackrbox}\xspace}}
%% Binding statement: v <- {...}
\protected\def\binds#1<-#2;{\ensurett{\removespaces #1\texttt{<-}#2}\xspace}
%% In order to use \binds in verbatim environment, have to define
%% delimiters while they are active. The below defines \vbinds which
%% must be used in AVerb environments.  Notice the active space as
%% well - that is necessary so the space after \vbinds (and before the
%% first argument) in the verbatim environment gets eaten.
\begingroup\catcode`\!=\active \lccode`\!=`\< \lccode`\~=`\- 
  \catcode`\ =\active\lowercase{\endgroup\def\vbinds#1!~#2;}{\binds#1<-#2;}
%% Return statement: return ... ;
\protected\def\return#1;{\milres return/\ensurett{\ \removespaces #1}}
%% A closure capturing block. k {v1, ..., vN} x: ...
\protected\def\ccblock#1(#2)#3:{\lab#1/\ensuremath{\thinspace\{\ensurett{#2}\}}\ \ensurett{#3\hbox{:}}}
%% A normal block
\protected\def\block#1(#2):{\lab #1/\ensuremath{\thinspace(\ensurett{#2})}\ensurett{:}}
%% A goto expression
\protected\def\goto#1(#2){\lab #1/\thinspace\ensuremath{(\ensurett{#2})}}
%% An enter expression
\protected\def\app#1*#2/{\ensurett{\removespaces #1\ifmmode\ \fi{\text{\tt @}}\ifmmode\ \fi#2}}
\protected\def\bind{\texttt{<-}\xspace}
%% Primitive expression
\protected\def\prim#1(#2){\lab #1/\suptt*\ensuremath{(\ensurett{#2})}}
%% Program variable
\protected\def\var#1/{\ensurett{\removespaces #1}\xspace}
%% Case statement
\protected\def\case#1;{\milres case\ \ensuremath{\ensurett{\removespaces #1}}\ of/}
%% Case alternative
\protected\def\alt#1(#2)#3->#4;{\ensuremath{\ensurett{#1\ \ignorespaces#2\ \texttt{->}\ \ignorespaces #4}}}
%% Invoke
\protected\def\invoke#1/{\milres invoke/\ensurett{\ \removespaces #1}}
\def\rhs{right--hand side\xspace}
\def\lhs{left--hand side\xspace}
\def\enter{\texttt{@}\xspace}
\def\cc{closure--capturing\xspace}
\def\Cc{Closure--capturing\xspace}
%%

\newenvironment{myfig}[1][tbh]{\begin{figure}[#1]%%
\begin{singlespace}\centering%%
\figbegin}{\figend\end{singlespace}%%
\end{figure}}

%% Produce a sub-caption and label it.
\newcommand{\scap}[2][1in]{\begin{minipage}{#1}%%
\subcaption{}\label{#2}\end{minipage}}

%% Produce a sub-caption with text.
\newcommand{\lscap}[3][\hsize]{\begin{minipage}{#1}%%
\subcaption{#3}\label{#2}\end{minipage}}

% single-argument comment. Do not put
% a space before the command when used
% or the file will have two spaces.
\newcommand{\rem}[1]{}

%% A verbatim environment with active charactesr
%% so we can use math shortcuts and macros
\DefineVerbatimEnvironment{AVerb}{Verbatim}{commandchars=\\\{\},%% 
  codes={\catcode`\_8\catcode`\$3\catcode`\^7},%%
  numberblanklines=false}

\DefineVerbatimEnvironment{Verb}{Verbatim}{commandchars=\\\[\],%% 
  numberblanklines=false}

%% Turn on line numbers for Haskell code, 
%% and reset the line number counter.
\newcommand{\hsNumOn}{\numberson\numbersreset}
\newcommand{\hsNumOff}{\numbersoff}
%% Turn on line numbering in Haskell code within
%% the environment, then turn it off. The optional
%% argument specifies a prefix that \hslabel can
%% use to generate line number references. If no prefix
%% is givne, \hslabel will have no effect.
\newtoks\prefixtoks
\def\mkhslabel#1{\prefixtoks={#1}\let\prefix=a}
\def\hslabel#1{\ifx\prefix\relax%%
  \else\label{\the\prefixtoks_#1}%%
  \fi}
\def\unhslabel{\let\prefix=\relax}
\newenvironment{withHsNum}{\numberson\numbersreset}{\numbersoff}
\newenvironment{withHsLabeled}[1]{\numberson\numbersreset\mkhslabel{#1}}{\unhslabel\numbersoff}

%% Paragraph run-in
\newcommand{\runin}[1]{\begingroup\noindent\sffamily\textbf{#1}\qquad\endgroup}

%% Chapter bibliographies
\newcommand{\standaloneBib}{%%
  \ifthenelse{\boolean{standaloneFlag}}%%
             {\begin{singlespace}
                 \printbibliography
             \end{singlespace}}{}}

%% Adds an equation number on demand.
\newcommand\addtag{\refstepcounter{equation}\tag{\theequation}}

%% For typesetting set definitions like {x | x \in f(y)}
\newcommand\setdef[2]{\ensuremath{\{#1\ |\ #2\}}}

%% For typesetting function names like dom(f) or out(b).
\newcommand\mfun[1]{\ensuremath{\mathit{#1}}}

%% Marginal notes
\newcommand\margin[2]{\marginpar{\begin{singlespace}\emph{\footnotesize #2}\end{singlespace}}\relax #1}

%% Describe intent of a passage
\newcommand\intent[1]{{\begin{singlespace}\noindent\leftskip=-1in\emph{\footnotesize Intent: #1}\end{singlespace}}\nopagebreak[1]}

%% In aligned/alignedat/gathered environments, you don't et
%% automatice equation numbers. This command makes sure to
%% label them properly.
\newcommand\labeleq[1]{\refstepcounter{equation}\label{#1}}

%% Creates a hanging paragraph, where the first line is not
%% indented but all other lines are.
\def\itempar#1{\noindent\hangindent=\parindent\hangafter=1 #1\quad}

%% Disable overfull messages with ridiculous hfuzz value
\def\disableoverfull{\hfuzz=10in}

%% Set parfillskip so stretching does NOT occur at the end of
%% a paragraph (i.e., list of elements). Disable indent at beginning
%% of paragraph. Also turn off underfull hbox warnings.
%%
%% Intended to be used in a \vbox that forms part of a table or graphic,
%% which we want to be line-broken but not exactly like a normal paragraph.
\long\def\disableparspacing#1;{\def\arg{#1}\hbadness=100000\parindent=0pt\parfillskip=0pt\leftskip=0pt\rightskip=0pt%%
  \ifx\arg\empty\else\hsize=#1\relax\fi}
%% This stuff makes !+<text>+! write <text> in typewriter font.  

%% We make ! and + active characters early, then manipulate their
%% meaning to produce the right effect. Initially, + produces +. When
%% !  appears w/o a + following, it produces ``!''. When ``+''
%% follows, we start writing in teleteype (\ttfamily). The definition
%% of ``!'' changes to produce a bang. ``+'' changes such that it
%% looks for trailing ``!''. When no ``!'' appears, ``+'' produces ``+''. 
%% If a ``!'' appears, we shift out of \ttfamily (by ending the group) and
%% reset the meaning of ``!'' and ``+'' so we can start again.
\makeatletter
\let\mdplus=+\let\mdbang=!      %% Preserve meaning of + and ! so we can put them into document.
%% Turn off mark down for everyone
\outer\def\nomd{\catcode`!=12\catcode`+=12}
%% Turn mark down on for everyone
\outer\def\domd{\catcode`!=\active\catcode`+=\active %%
  \initialmd}
%% Use only with a group IMMEDIETALY following. Turns off
%% markdown for the group-to-come, without actually tokenizing the
%% group. If no group follows, this has no effect.
\protected\def\pausemd{\def\dopause{\catcode`!=12\catcode`+=12}%%
  \def\pausemdB{\ifx\next\bgroup%%
    %% A ``partial'' application of expandwith is used
    %% so we don't double up the group argument (which is what
    %% happens if we expand \next). This has the effect of 
    %% inserting \expandafter\dowith in front of the upcoming {. 
    %% If no brace is coming, \withmdC will have no effect.
    \def\pausemdC{\expandafter\dopause}
  \else
    \let\pausedmC=\relax
  \fi\pausemdC}
  %% \futurelet has to end the macro so it grabs the next token
  %% from the input file. Otherwise, it grabs it *from* this
  %% definition.
  \futurelet\next\pausemdB} %%
%% Turns markdown on for the group-to-come, without actually
%% tokenizing the group. Only has an effect when
%% used in front of a group, otherwise its a no-op.
\protected\def\withmd{\def\dowith{\catcode`!=\active\catcode`+=\active\initialmd}%%
  \def\withmdB{\ifx\next\bgroup %%
    %% A ``partial'' application of expandafter is used
    %% so we don't double up the group argument (which is what
    %% happens if we expand \next). This has the effect of 
    %% inserting \expandafter\dowith in front of the upcoming {. 
    %% If no brace is coming, \withmdC will have no effect.
      \def\withmdC{\expandafter\dowith} %%
    \else %%
      \let\withmdC=\relax %%
    \fi\withmdC}%%
  %% \futurelet has to end the macro so it grabs the next token
  %% from the input file. Otherwise, it grabs it *from* this
  %% definition.
  \futurelet\next\withmdB} %%
%% Make ! and + active in the following group so they have the right
%% catcode in the definitions to follow.
\catcode`!=\active\catcode`+=\active %%
%% Initial definitions associated with ! and +.
\def\initialmd{\protected\def!{\startTTA} %%
  \protected\def+{\stopTTA}} %%
%% Step 1 of startTT. Inital meaning of !; capture next token in \next, go to next step.
\def\startTTA{\futurelet\next\startTTB} %%
%% Step 2 of startTT. Compare captured token to + and go to step 3 if true. Otherwise
%% output a ! (since that started our macro), the argument captured and stop
%% processing.
\long\def\startTTB{\ifx\next+\expandafter\startTTC\expandafter\@gobble\else\mdbang\fi} %%
%% Step 3 of startTT. Shift into teletype mode and change definition of 
%% + and ! so we can stop processing.
\def\startTTC{\begingroup\ifmmode %%
  \let \math@bgroup \relax %%
  \def \math@egroup {\let \math@bgroup \@@math@bgroup %%
    \let \math@egroup \@@math@egroup} %%
  \mathtt\relax %%
  \else  %%
  \ttfamily\fi} %%
%% Step 1, 2  and 3 of stopTT follow the same pattern as startTT.
\def\stopTTA{\futurelet\next\stopTTB} %%
\long\def\stopTTB{\ifx\next!\expandafter\stopTTC\expandafter\@gobble\else\mdplus\fi} %%
\def\stopTTC{\endgroup}%%
\catcode`!=12\catcode`+=12
\makeatother

\domd

%% Place an input file on the next page
\def\onnextpage#1{\afterpage{\clearpage\input{#1}\clearpage}}

\begin{document}
\ifthenelse{\boolean{standaloneFlag}}
           {\VerbatimFootnotes
             \DefineShortVerb{\#}
             \doublespacing
             \setcounter{chapter}{0}}{}

%% Default float parameters. For case when
%% multiple chapters are included and
%% only one needs custom float settings.
\renewcommand{\textfraction}{0.2}
\renewcommand{\topfraction}{0.9}


\chapter{A Monadic Intermediate Language}
\label{ref_chapter_mil}

%% What is an intermediate language? What is a ``Monadic'' one?
Most compilers do not generate executable machine code directly from a
program source file. Rather, programs get translated into one or more
\emph{intermediate forms}. The compiler may implement a pipeline of
translations, each translating the program into a more detailed (i.e.,
lower-level) representation. Frequently these intermediate forms are 
also languages, with their own syntax, semantics, type-systems,
and more. 

Intermediate forms typically expose more detail about the
implementation of a program, while at the same time making some
optimization or transformation easier or even possible. 
\emph{Three-address code}, one such intermediate form, translates the
program into assembly-language like form, using \emph{registers} to
hold values. Infinitely many registers can be named, making registers
more like memory locations than registers in real hardware. Each
instruction in the translated program has two operand and one
destination register, thus the name ``three-address.'' 

For example,
the add instruction might be written as:
\begin{AVerb}
  add \emph{r}, \emph{s}, \emph{d}
\end{AVerb}
where $r$ and $s$ are source registers and $d$ the destination. When
executed, the instruction would place the sum of $r$ and $s$ in $d$.



%% Motivation for MIL

%% Syntax of MIL

%% Compiling the lambda-calculus to MIL

%% \section{Monadic Intermediate Language}

%% %% What does the language support?

%% Our monadic language takes its inspiration from Haskell's @do@
%% notation. It is a pure functional language, making allocation of data
%% structures and closures explicit via monadic syntax. Functions in MIL
%% define computations which, when run, can affect heap memory. Figure
%% \ref{figMILDef} gives the syntax of the language.

%% %% TODO: Mention that v restricts the term to variables
%% %% only.

%% \begin{figure}[h]
%% \begin{code}
%%   defM := k {v1, ..., vN} v = k1 {v1, ..., vN, v} 
%%     | k {v1, ..., vN} v = b(v1, ..., vN, v)
%%     | b(v1, ..., vN) = bodyM
%%     | t <- k {}

%%   bodyM := do 
%%     stmtM1 
%%     ... 
%%     stmtMN 
%%     tailM

%%   stmtM := v <- tailM
%%     | case v of [alt1, ..., altN]


%%   tailM := return v
%%     | v1 @ v2
%%     | k {v1, ..., vN}
%%     | f(v1, ..., vN)
%%     | C v1 ... vN

%%   alt := C v1 ... vN -> b(v1, ..., vM) -- m <= n
%% \end{code}
%% \caption{Concrete syntax for our monadic intermediate language.}
%% \label{figMILDef}
%% \end{figure}

%% MIL programs consist of a series of definitions (@defM@). Each
%% definition can be any of the following.

%% \begin{description}
%%   \item[Closure-capturing] (@k {v1, ..., vN} v = k1 {v1, ..., vN, v}@) -- This function
%%     expects to find the variables @v1, ..., vN@ in its own closure. It constructs
%%     a new closure containing the existing variables plus the newly captured variable
%%     @v@. The new closure refers to @k1@, another closure-capturing function.
%%   \item[Block-calling] (@k {v1, ..., vN} v = b(v1, ..., vN, v)@) -- This function immediately
%%     jumps to block @b@ with arguments @v1, ..., vN@ and @v@. No closure value needs to
%%     be constructed. 
%%   \item[Function block] (@b(v1, ..., vN) = bodyM@) -- This function executes the statements
%%     in the body. 
%%   \item[Top-level] (@t <- k {}@) -- This special case ensures top-level definitions in the program
%%     can be accessed like any other function. The notation indicates that @t@ holds a closure
%%     structure, referring to the definition @k@. 
%% \end{description}

%% Notice that we can distinguish syntatically between functions that
%% merely create a closure (@k { ... }@) and those that do actual work
%% (@b(...)@). The body of a @k@ functin can only allocate another
%% closure or jump to a block. A block, on the other hand, can do other
%% work, but it cannot directly return a closure. As will be described in
%% chapter \ref{ref_chapter_uncurrying} this makes it much easier to
%% recognize and elminate intermediate closures.

%% The body of each block consists of statements followed by a
%% \emph{tail}. Tails can only
%% appear as the last statement in a block or on the right-hand side of
%% the monadic arrow (``@<-@''). Tail instructions, in other words, cause 
%% effects. The three tail statements follow:

%% \begin{description}
%% \item[Return a computation] (@return v@) -- Returns the result of a computation
%%   to the caller.

%% \item[Create a closure] (@k {v1, ..., vN}@) -- Creates a closure pointing to
%%   function @k@, capturing variables @v1@ through @vN@.

%% \item[Enter a function] (@v1 @@ v2@) -- Enter the closure referred to by @v1@, with
%%   argument @v2@. In other words, function application. Note that @v1@ represents an
%%   \emph{unknown} function -- one for which we compute the address at run-time.

%% \item[Call a block] (@f(v1, ..., vN)@) -- Jump to the block labeled @f@ with the arguments
%%   given. In this case we know the function @f@ refers to and do not need to examine
%%   a closure in order to execute it.
%% \item[Create a value] (@C v1 ... vN@) -- Create a data value with tag @C@, holding
%%   the values found in variables @v1 ... vN@.
%% \end{description}

%% %% TODO: Describe alt syntax.

%% Statements in a block either bind the result of a tail statement 
%% (@v <- tailM@) or branch conditionally (@case v of ... @). Binding ``runs''
%% a computation and ``dereferences'' the result, placing
%% the value in a variable (e.g., @v@). That same variable can be bound
%% again later, but that does not affect previous uses of @v@. In essence, the old
%% name becomes hidden and its value inaccessible.

%% Though the syntax allows multiple @case@ statements in a function
%% body, only one can appear and it must be the last statement in the
%% body. The arms of the @case@ statement can only match on constructor
%% tags (@C@) and can only bind the constructor arguments to variables
%% (@v1 ... vN@). Each arm then jumps to a known block with those
%% variables as arguments. This choice makes compilation simpler.


%% %% \emph{Defines our monadic language and explains the terms in
%% %%   it. Example programs are given which illustrate closure construction
%% %%   and data allocation. The use of ``tail'' vs. statements is motivated
%% %%   and described. }

%% \emph{Need to talk about the monad we work in as well - what 
%% do bind and return mean?}

%% \section{Compiling to Our MIL}
%% \emph{A compilation scheme which uses Hoopls ``shapes'' is
%% described. This scheme will give use our initial, unoptimized
%% MIL program. An example (possibly |compose|, or |const3|) illustrates 
%% our scheme.}


\end{document}


\chapter{The Hoopl Library}

\emph{Introduce the Hoopl library, describing how
it approaches dataflow analysis. Important concepts
such as shape, transfer and rewrite functions, facts and
lattices will be described. }

\documentclass[12pt]{report}
 %include polycode.fmt
\usepackage[T1]{fontenc}
\usepackage{calc}
\usepackage{palatino}
\usepackage{amsfonts}
\renewcommand\ttdefault{lmtt}
\usepackage{helvet}
\usepackage{xspace}
\usepackage{url}
\usepackage{fancyvrb}
\usepackage[doublespacing]{setspace}
%% below only necessary when using doublespacing -- corrects
%% the vertical space inserted when switching to singlespace
%% environment.
\def\correctspaceskip{\vskip-\baselineskip} 
\usepackage{amsmath}
\usepackage{booktabs}
\usepackage[margin=\parindent, format=hang,labelfont=bf]{caption}
%% \usepackage[subrefformat=parens]{subcaption}
%% The following makes sure we get parentheses around
%% subreferences. The newest version of the subcaption
%% package has an option for this, but that's not available
%% widely.
%%
%% From http://tex.stackexchange.com/questions/25644
\usepackage[labelformat=simple]{subcaption}
\makeatletter
  \def\thesubfigure{(\alph{subfigure})}
  \providecommand\thefigsubsep{~}
  \def\p@subfigure{\@nameuse{thefigure}\thefigsubsep}
\makeatother

\usepackage{ifthen}
\usepackage{stmaryrd}
\usepackage{longtable}
\usepackage{afterpage}
\usepackage{xifthen}
\usepackage{mathtools}
\usepackage[natbib=true,style=authoryear,backend=bibtex8]{biblatex}
\setlength{\bibitemsep}{\bigskipamount}
\addbibresource{thesis.bib}
\usepackage{microtype}

%% GSO margins.
\usepackage[left=1.5in, right=1in, top=1in, bottom=1in]{geometry}
\usepackage{abstract}

%% GSO requires 12 pt font for all headings
\usepackage[bf,sf,tiny,compact]{titlesec}
\titleformat{\chapter}[display]
            {}% format
            {\sffamily\bfseries\chaptertitlename\ \thechapter}
            {\baselineskip}
            {\sffamily\bfseries}
            {}

\hyphenation{data-flow mo-na-dic} 

%% Should unindent all haskell code set in a dispay (versus inline)
\makeatletter
  \@ifundefined{hscodestyle}
               {}
               {\renewcommand{\hscodestyle}{\advance\leftskip -\mathindent}}
\makeatother

% Used by included files to know they
% are NOT standalone
\newboolean{standaloneFlag}
\setboolean{standaloneFlag}{true}

\newlength{\rulefigmargin}
\setlength{\rulefigmargin}{2\parindent}

\newcommand\figbegin{\rule{\linewidth}{0.4pt}\\\vspace{12pt}}
\newcommand\figend{\rule{\linewidth}{0.4pt}}

%% Sets
\newcommand{\setL}[1]{\textsc{#1}\xspace}
\newcommand{\setLC}{\setL{Const}}

%% Lub, subset operators.
\protected\def\lub{\ifmmode\sqcap\else\raisebox{.1em}{\ensuremath{\sqcap}}\fi\xspace}
\newcommand{\sqlt}{\ensuremath{\sqsubset}\xspace}
\newcommand{\sqlte}{\ensuremath{\sqsubseteq}\xspace}

%% Subscripting with typewriter
\def\subtt#1{\ifmmode_{\ensurett{#1}}%%
  \else$_{\ensurett{#1}}$%%
  \fi}
%% Superscripting with typerwriter
\def\suptt#1{\ifmmode^{\ensurett{#1}}%%
  \else$^{\ensurett{#1}}$%%
  \fi}
%% Functional languages chapter commands
\newcommand{\lamA}{\ensuremath{\lambda}-calculus\xspace}
\newcommand{\LamA}{\ensuremath{\lambda}-Calculus\xspace}
\newcommand{\lamAbs}[2]{\ensuremath{\lambda#1.\ #2}}
\newcommand{\lamApp}[2]{\ensuremath{#1\ #2}}
\newcommand{\lamPApp}[2]{\ensuremath{(#1\ #2)}}
\newcommand{\lamAPp}[2]{\ensuremath{(#1)\ #2}}
\newcommand{\lamApP}[2]{\ensuremath{#1\ (#2)}}
\newcommand{\lamAPP}[2]{\ensuremath{(#1)\ (#2)}}
\let\lamApPp=\lamApP
\let\lamAppP=\lamAPp
%% LC definition
\newtoks\toksA
\protected\def\lcname#1/{\ensuremath{\mathit{#1}}}
\protected\def\lcdef#1(#2)=#3;{\def\arg{#2}%%
  \def\lcargs##1,##2/{\def\arg{##2}%%
    \ifx\empty\arg%%
    \lcname ##1/%%
    \else\lcname ##1/\ \lcargs ##2/%%
    \fi}%%
  \ifx\empty\arg\toksA={\ }%%
  \else\toksA={\ \lcargs #2,/\ }%%
  \fi%%
  \ensuremath{\lcname#1/\the\toksA =\ #3}}
%% Arbitary number of applied arguments, separated
%% by asterisks (*).
\protected\def\lcapp#1/{\def\lcappB##1*##2/{\def\arg{##2}%
    \ensuremath{\ifx\arg\empty%%
      \lcname ##1/%%
      \else%%
      \lcname##1/\ \lcappB##2/%%
      \fi}}%%
  %% Adding a star here makes
  %% sure our applicaitn always ends with an asterisks, ensuring
  %% #2 will be \empty at some point.
  \lcappB#1*/}
\protected\def\lcabs#1.{\ensuremath{\lambda#1.\ }}

\newcommand{\lamId}{\lamAbs{x}{x}}
\newcommand{\lamCompose}{\lamAbs{f}{\lamAbs{g}{\lamAbs{x}{\lamApp{f}{(\lamApp{g}{x})}}}}}
\newcommand{\machLam}{\ensuremath{M_\lambda}\xspace}
\newcommand{\compMach}[1]{\ensuremath{\left\llbracket #1 \right\rrbracket}}
\newcommand{\compRho}[1]{\ensuremath{\rho(#1)}}
\newcommand{\verSub}[2]{\ensuremath{#1_{#2}}}
\newcommand{\verSup}[2]{\ensuremath{#1^{#2}}}
\newcommand{\lamC}{\ensuremath{\lambda_C}\xspace}
\newcommand{\lamPlus}{\lamAbs{m}{\lamAbs{n}{\lamAbs{s}{\lamAbs{z}{\lamApp{m}{\lamApPp{s}{\lamApp{n}{\lamApp{s}{z}}}}}}}}}
%% Substitution notation -- [#1 -> #2]
\newcommand{\lamSubst}[2]{\ensuremath{[#1 \mapsto #2]}}
%% End functional languages chapter


%% MIL Chapter
\newcommand{\compMILE}[1]{\ensuremath{\left\llbracket #1 \right\rrbracket}}
\newcommand{\compMILV}[1]{\ensuremath{\left\llbracket #1 \right\rrbracket}}
\newcommand{\compMILQ}[2]{\ensuremath{\left\llbracket #2 \right\rrbracket}}
\newcommand{\milCtx}[1]{\ensuremath{\llfloor}#1\ensuremath{\rrfloor}}

%% This dimension makes sure the same amount of space
%% follows | and := in syntax rules like:
%%
%% term := var       (Variable)
%%      |  var var    (Application)
%%      |  \x. var    (Abstraction)
%%
\newdimen\termalign
\setbox0=\hbox{$:=$}
\termalign=\wd0 
\protected\def\term#1/{\ensuremath{\mathit{#1}}}
\protected\def\syntaxrule#1/{\hfil\text{\emph{#1}}}
\protected\long\def\termrule#1:#2:#3/{\term #1/ &\hbox{$:=$}\ensuremath{\ #2} & \syntaxrule #3/}
\protected\def\termcase#1:#2/{&\hbox to \termalign{$|$\hss}\ensuremath{\ #1} & \syntaxrule #2/}


%% End MIL chapter

%% Dataflow Chapter
% Domain function
\def\dom(#1){\ensuremath{\mfun{dom}(#1)}\xspace}
% Set of all integers.
\def\ZZ{\ensuremath{\mathbb{Z}}}
%%

%% Uncurrrying Chapter 
%% A space equal to a \thinspace, but we
%% can break a line at it.
\newskip\thinskipamt \thinskipamt=.16667em 
\protected\def\thinskip{\hskip \thinskipamt\relax}
\protected\def\thinnerskip{\hskip .5\thinskipamt\relax}
%% Capture a space token. Use a ``control-symbol'' (\. instead of \mksp)
%% to keep the trailing space from getting gobbled.
{\def\.{\global\let\sp= } \. }
%% Define \asp, which will capture the macro definition attached to space,
%% if one exists. Otherwise, \spa is relax after this.
{\catcode`\ =\active\gdef\asp{\ifx \relax\let\spa\relax\else\let\spa= \fi}}
\newtoks\foo
%% Removes spaces, implicit, active and explicit.
\protected\def\removespaces{\asp\afterassignment\removesp\let\next= }
\def\removesp{\foo={\next}\ifcat\noexpand\next\sp\foo={\removespace}%%
 \else\ifx\next\spa\foo={\removespaces}\fi%%
 \fi\the\foo}
%% MIL reserved word
\protected\def\milres#1/{\text{\ttfamily\bfseries #1}}
\protected\def\lab#1/{\textbf{\ensurett{\removespaces #1}}}
%% Constructs a closure: l { v1, ..., vN }
\protected\long\def\mkclo[#1:#2]{\lab #1/\ensuremath{\,\{\ensurett{#2}\}}\xspace}
%% Tuple version of closurs: {l: v1, ..., vN}.
\protected\long\def\clo[#1:#2]{\def\argA{#1}\def\argB{#2}\ensuremath{\{%%
      \ifx\argA\empty%%
      \else\lab #1/%%
        \ifx\argB\empty%%
        \else\ensurett{:\thinskip}%%
        \fi%%
      \fi\ensurett{#2}\}}\xspace}
%% Construct a thunk
\newbox\bracklbox \newbox\brackrbox
\setbox0=\hbox{$\{$} \setbox\bracklbox=\hbox to \wd0{\hfil[\kern0.25mm}
\setbox0=\hbox{$\}$} \setbox\brackrbox=\hbox to \wd0{\kern0.25mm]\hfil}
\protected\def\mkthunk[#1:#2]{\lab #1/%%
  \ensuremath{\,%%
    \mathopen{\copy\bracklbox}%%
    \ensurett{#2}%%
    \mathclose{\copy\brackrbox}\xspace}}
%% Binding statement: v <- {...}
\protected\def\binds#1<-#2;{\ensurett{\removespaces #1\texttt{<-}#2}\xspace}
%% In order to use \binds in verbatim environment, have to define
%% delimiters while they are active. The below defines \vbinds which
%% must be used in AVerb environments.  Notice the active space as
%% well - that is necessary so the space after \vbinds (and before the
%% first argument) in the verbatim environment gets eaten.
\begingroup\catcode`\!=\active \lccode`\!=`\< \lccode`\~=`\- 
  \catcode`\ =\active\lowercase{\endgroup\def\vbinds#1!~#2;}{\binds#1<-#2;}
%% Return statement: return ... ;
\protected\def\return#1;{\milres return/\ensurett{\ \removespaces #1}}
%% A closure capturing block. k {v1, ..., vN} x: ...
\protected\def\ccblock#1(#2)#3:{\lab#1/\ensuremath{\thinspace\{\ensurett{#2}\}}\ \ensurett{#3\hbox{:}}}
%% A normal block
\protected\def\block#1(#2):{\lab #1/\ensuremath{\thinspace(\ensurett{#2})}\ensurett{:}}
%% A goto expression
\protected\def\goto#1(#2){\lab #1/\thinspace\ensuremath{(\ensurett{#2})}}
%% An enter expression
\protected\def\app#1*#2/{\ensurett{\removespaces #1\ifmmode\ \fi{\text{\tt @}}\ifmmode\ \fi#2}}
\protected\def\bind{\texttt{<-}\xspace}
%% Primitive expression
\protected\def\prim#1(#2){\lab #1/\suptt*\ensuremath{(\ensurett{#2})}}
%% Program variable
\protected\def\var#1/{\ensurett{\removespaces #1}\xspace}
%% Case statement
\protected\def\case#1;{\milres case\ \ensuremath{\ensurett{\removespaces #1}}\ of/}
%% Case alternative
\protected\def\alt#1(#2)#3->#4;{\ensuremath{\ensurett{#1\ \ignorespaces#2\ \texttt{->}\ \ignorespaces #4}}}
%% Invoke
\protected\def\invoke#1/{\milres invoke/\ensurett{\ \removespaces #1}}
\def\rhs{right--hand side\xspace}
\def\lhs{left--hand side\xspace}
\def\enter{\texttt{@}\xspace}
\def\cc{closure--capturing\xspace}
\def\Cc{Closure--capturing\xspace}
%%

\newenvironment{myfig}[1][tbh]{\begin{figure}[#1]%%
\begin{singlespace}\centering%%
\figbegin}{\figend\end{singlespace}%%
\end{figure}}

%% Produce a sub-caption and label it.
\newcommand{\scap}[2][1in]{\begin{minipage}{#1}%%
\subcaption{}\label{#2}\end{minipage}}

%% Produce a sub-caption with text.
\newcommand{\lscap}[3][\hsize]{\begin{minipage}{#1}%%
\subcaption{#3}\label{#2}\end{minipage}}

% single-argument comment. Do not put
% a space before the command when used
% or the file will have two spaces.
\newcommand{\rem}[1]{}

%% A verbatim environment with active charactesr
%% so we can use math shortcuts and macros
\DefineVerbatimEnvironment{AVerb}{Verbatim}{commandchars=\\\{\},%% 
  codes={\catcode`\_8\catcode`\$3\catcode`\^7},%%
  numberblanklines=false}

\DefineVerbatimEnvironment{Verb}{Verbatim}{commandchars=\\\[\],%% 
  numberblanklines=false}

%% Turn on line numbers for Haskell code, 
%% and reset the line number counter.
\newcommand{\hsNumOn}{\numberson\numbersreset}
\newcommand{\hsNumOff}{\numbersoff}
%% Turn on line numbering in Haskell code within
%% the environment, then turn it off. The optional
%% argument specifies a prefix that \hslabel can
%% use to generate line number references. If no prefix
%% is givne, \hslabel will have no effect.
\newtoks\prefixtoks
\def\mkhslabel#1{\prefixtoks={#1}\let\prefix=a}
\def\hslabel#1{\ifx\prefix\relax%%
  \else\label{\the\prefixtoks_#1}%%
  \fi}
\def\unhslabel{\let\prefix=\relax}
\newenvironment{withHsNum}{\numberson\numbersreset}{\numbersoff}
\newenvironment{withHsLabeled}[1]{\numberson\numbersreset\mkhslabel{#1}}{\unhslabel\numbersoff}

%% Paragraph run-in
\newcommand{\runin}[1]{\begingroup\noindent\sffamily\textbf{#1}\qquad\endgroup}

%% Chapter bibliographies
\newcommand{\standaloneBib}{%%
  \ifthenelse{\boolean{standaloneFlag}}%%
             {\begin{singlespace}
                 \printbibliography
             \end{singlespace}}{}}

%% Adds an equation number on demand.
\newcommand\addtag{\refstepcounter{equation}\tag{\theequation}}

%% For typesetting set definitions like {x | x \in f(y)}
\newcommand\setdef[2]{\ensuremath{\{#1\ |\ #2\}}}

%% For typesetting function names like dom(f) or out(b).
\newcommand\mfun[1]{\ensuremath{\mathit{#1}}}

%% Marginal notes
\newcommand\margin[2]{\marginpar{\begin{singlespace}\emph{\footnotesize #2}\end{singlespace}}\relax #1}

%% Describe intent of a passage
\newcommand\intent[1]{{\begin{singlespace}\noindent\leftskip=-1in\emph{\footnotesize Intent: #1}\end{singlespace}}\nopagebreak[1]}

%% In aligned/alignedat/gathered environments, you don't et
%% automatice equation numbers. This command makes sure to
%% label them properly.
\newcommand\labeleq[1]{\refstepcounter{equation}\label{#1}}

%% Creates a hanging paragraph, where the first line is not
%% indented but all other lines are.
\def\itempar#1{\noindent\hangindent=\parindent\hangafter=1 #1\quad}

%% Disable overfull messages with ridiculous hfuzz value
\def\disableoverfull{\hfuzz=10in}

%% Set parfillskip so stretching does NOT occur at the end of
%% a paragraph (i.e., list of elements). Disable indent at beginning
%% of paragraph. Also turn off underfull hbox warnings.
%%
%% Intended to be used in a \vbox that forms part of a table or graphic,
%% which we want to be line-broken but not exactly like a normal paragraph.
\long\def\disableparspacing#1;{\def\arg{#1}\hbadness=100000\parindent=0pt\parfillskip=0pt\leftskip=0pt\rightskip=0pt%%
  \ifx\arg\empty\else\hsize=#1\relax\fi}
%% This stuff makes !+<text>+! write <text> in typewriter font.  

%% We make ! and + active characters early, then manipulate their
%% meaning to produce the right effect. Initially, + produces +. When
%% !  appears w/o a + following, it produces ``!''. When ``+''
%% follows, we start writing in teleteype (\ttfamily). The definition
%% of ``!'' changes to produce a bang. ``+'' changes such that it
%% looks for trailing ``!''. When no ``!'' appears, ``+'' produces ``+''. 
%% If a ``!'' appears, we shift out of \ttfamily (by ending the group) and
%% reset the meaning of ``!'' and ``+'' so we can start again.
\makeatletter
\let\mdplus=+\let\mdbang=!      %% Preserve meaning of + and ! so we can put them into document.
%% Turn off mark down for everyone
\outer\def\nomd{\catcode`!=12\catcode`+=12}
%% Turn mark down on for everyone
\outer\def\domd{\catcode`!=\active\catcode`+=\active %%
  \initialmd}
%% Use only with a group IMMEDIETALY following. Turns off
%% markdown for the group-to-come, without actually tokenizing the
%% group. If no group follows, this has no effect.
\protected\def\pausemd{\def\dopause{\catcode`!=12\catcode`+=12}%%
  \def\pausemdB{\ifx\next\bgroup%%
    %% A ``partial'' application of expandwith is used
    %% so we don't double up the group argument (which is what
    %% happens if we expand \next). This has the effect of 
    %% inserting \expandafter\dowith in front of the upcoming {. 
    %% If no brace is coming, \withmdC will have no effect.
    \def\pausemdC{\expandafter\dopause}
  \else
    \let\pausedmC=\relax
  \fi\pausemdC}
  %% \futurelet has to end the macro so it grabs the next token
  %% from the input file. Otherwise, it grabs it *from* this
  %% definition.
  \futurelet\next\pausemdB} %%
%% Turns markdown on for the group-to-come, without actually
%% tokenizing the group. Only has an effect when
%% used in front of a group, otherwise its a no-op.
\protected\def\withmd{\def\dowith{\catcode`!=\active\catcode`+=\active\initialmd}%%
  \def\withmdB{\ifx\next\bgroup %%
    %% A ``partial'' application of expandafter is used
    %% so we don't double up the group argument (which is what
    %% happens if we expand \next). This has the effect of 
    %% inserting \expandafter\dowith in front of the upcoming {. 
    %% If no brace is coming, \withmdC will have no effect.
      \def\withmdC{\expandafter\dowith} %%
    \else %%
      \let\withmdC=\relax %%
    \fi\withmdC}%%
  %% \futurelet has to end the macro so it grabs the next token
  %% from the input file. Otherwise, it grabs it *from* this
  %% definition.
  \futurelet\next\withmdB} %%
%% Make ! and + active in the following group so they have the right
%% catcode in the definitions to follow.
\catcode`!=\active\catcode`+=\active %%
%% Initial definitions associated with ! and +.
\def\initialmd{\protected\def!{\startTTA} %%
  \protected\def+{\stopTTA}} %%
%% Step 1 of startTT. Inital meaning of !; capture next token in \next, go to next step.
\def\startTTA{\futurelet\next\startTTB} %%
%% Step 2 of startTT. Compare captured token to + and go to step 3 if true. Otherwise
%% output a ! (since that started our macro), the argument captured and stop
%% processing.
\long\def\startTTB{\ifx\next+\expandafter\startTTC\expandafter\@gobble\else\mdbang\fi} %%
%% Step 3 of startTT. Shift into teletype mode and change definition of 
%% + and ! so we can stop processing.
\def\startTTC{\begingroup\ifmmode %%
  \let \math@bgroup \relax %%
  \def \math@egroup {\let \math@bgroup \@@math@bgroup %%
    \let \math@egroup \@@math@egroup} %%
  \mathtt\relax %%
  \else  %%
  \ttfamily\fi} %%
%% Step 1, 2  and 3 of stopTT follow the same pattern as startTT.
\def\stopTTA{\futurelet\next\stopTTB} %%
\long\def\stopTTB{\ifx\next!\expandafter\stopTTC\expandafter\@gobble\else\mdplus\fi} %%
\def\stopTTC{\endgroup}%%
\catcode`!=12\catcode`+=12
\makeatother

\domd

%% Place an input file on the next page
\def\onnextpage#1{\afterpage{\clearpage\input{#1}\clearpage}}

\begin{document}
\ifthenelse{\boolean{standaloneFlag}}
           {\VerbatimFootnotes
             \DefineShortVerb{\#}
             \doublespacing
             \setcounter{chapter}{0}}{}

%% Default float parameters. For case when
%% multiple chapters are included and
%% only one needs custom float settings.
\renewcommand{\textfraction}{0.2}
\renewcommand{\topfraction}{0.9}


\chapter{Uncurrying}
\label{ref_chapter_uncurrying}
%% \emph{Describes our optimization for collapsing intermediate
%% closures. Our choice of representation is analyzed to
%% show how it facilitates this optimization. We should show one
%% closure can be eliminated from a program and how the optimization
%% is applied over and over until a fixed point is reached. The format
%% for this section will vary from the other two.}

Functional languages permit definitions in two styles: \emph{curried}
and \emph{uncurried}. A curried function can be \emph{partially
  applied} --- it does not need to be given all of its arguments at
once. A function that takes the remaining arguments results from such
an application. An \emph{uncurried} function, however, must be given
all of its arguments at once. It cannot be partially applied. For
example, these Haskell fragments define @adder@ in curried style and
@divider@ in uncurried style:

\begin{code}
adder :: Int -> Int -> Int
adder a b = a + b

divider :: (Float, Float) -> Float
divider (a,b) = a / b
\end{code}

\noindent
The first definition lets us easily define specialized versions of @adder@, 
such as @add1@:

\begin{code}
add1 :: Int -> Int
add1 = adder 1
\end{code}

When applied to a single argument, @adder@ returns a function that we can re-use over and over. We cannot as easily define a simliar specialized function with @divider@.

%% Why is this a problem? Need more motivatin
The implementation of partial application, however, does come at a
cost. Each partial application requires that we construct a closure
over the arguments captured so far. That closure represents a function specialized
to the arguments given so far. In general, we don't know the address of the function
it contains when compiling -- only at run-time. Therefore, when the closure is
applied to the rest of the arguments, we cannot generate code that jumps to a
known address. Instead, we must look at the address in the closure at run-time and 
then jump. 

Because each function application @f x@ may result in another
function, the most general implementation strategy makes \emph{every}
application result in a closure. The compiler need only generate code
that inspects the closure constructed and jumps to the address
indicated. When a curried function is applied to all of its arguments
at once (e.g., @adder 1 2@), we get a chain of function calls where
most construct a closure and immediately return. It would be more
efficient to collect all arguments at once and immediately jump to the
function body. \emph{Uncurrying} is the transformation we use to turn 
fully-applied curried functions into direct calls to the function body.

%% TODO: Talk about how we can look for fully-applied forms
%% as a special case, but that is sub-optimal

%% TODO: What is an example of a fully-applied function that we cannot
%% recognize syntatically (very easily)?

\section{Example of Desired Optimization}

Recall from Section \ref{ref_foo} our definition of @foldr@:

\begin{code}
foldr :: (a -> b -> b) -> b -> [a] -> b
foldr f b (a:as) = foldr f (f a b) as
foldr f b []     = b
\end{code}

which compiles to the following blocks in our MIL:

\begin{code}

\end{code}

\section{Implementation}
\section{Reflection}
\subsection{Prior Work}

\end{document}


\documentclass[12pt]{report}
 %include polycode.fmt
\usepackage[T1]{fontenc}
\usepackage{calc}
\usepackage{palatino}
\usepackage{amsfonts}
\renewcommand\ttdefault{lmtt}
\usepackage{helvet}
\usepackage{xspace}
\usepackage{url}
\usepackage{fancyvrb}
\usepackage[doublespacing]{setspace}
%% below only necessary when using doublespacing -- corrects
%% the vertical space inserted when switching to singlespace
%% environment.
\def\correctspaceskip{\vskip-\baselineskip} 
\usepackage{amsmath}
\usepackage{booktabs}
\usepackage[margin=\parindent, format=hang,labelfont=bf]{caption}
%% \usepackage[subrefformat=parens]{subcaption}
%% The following makes sure we get parentheses around
%% subreferences. The newest version of the subcaption
%% package has an option for this, but that's not available
%% widely.
%%
%% From http://tex.stackexchange.com/questions/25644
\usepackage[labelformat=simple]{subcaption}
\makeatletter
  \def\thesubfigure{(\alph{subfigure})}
  \providecommand\thefigsubsep{~}
  \def\p@subfigure{\@nameuse{thefigure}\thefigsubsep}
\makeatother

\usepackage{ifthen}
\usepackage{stmaryrd}
\usepackage{longtable}
\usepackage{afterpage}
\usepackage{xifthen}
\usepackage{mathtools}
\usepackage[natbib=true,style=authoryear,backend=bibtex8]{biblatex}
\setlength{\bibitemsep}{\bigskipamount}
\addbibresource{thesis.bib}
\usepackage{microtype}

%% GSO margins.
\usepackage[left=1.5in, right=1in, top=1in, bottom=1in]{geometry}
\usepackage{abstract}

%% GSO requires 12 pt font for all headings
\usepackage[bf,sf,tiny,compact]{titlesec}
\titleformat{\chapter}[display]
            {}% format
            {\sffamily\bfseries\chaptertitlename\ \thechapter}
            {\baselineskip}
            {\sffamily\bfseries}
            {}

\hyphenation{data-flow mo-na-dic} 

%% Should unindent all haskell code set in a dispay (versus inline)
\makeatletter
  \@ifundefined{hscodestyle}
               {}
               {\renewcommand{\hscodestyle}{\advance\leftskip -\mathindent}}
\makeatother

% Used by included files to know they
% are NOT standalone
\newboolean{standaloneFlag}
\setboolean{standaloneFlag}{true}

\newlength{\rulefigmargin}
\setlength{\rulefigmargin}{2\parindent}

\newcommand\figbegin{\rule{\linewidth}{0.4pt}\\\vspace{12pt}}
\newcommand\figend{\rule{\linewidth}{0.4pt}}

%% Sets
\newcommand{\setL}[1]{\textsc{#1}\xspace}
\newcommand{\setLC}{\setL{Const}}

%% Lub, subset operators.
\protected\def\lub{\ifmmode\sqcap\else\raisebox{.1em}{\ensuremath{\sqcap}}\fi\xspace}
\newcommand{\sqlt}{\ensuremath{\sqsubset}\xspace}
\newcommand{\sqlte}{\ensuremath{\sqsubseteq}\xspace}

%% Subscripting with typewriter
\def\subtt#1{\ifmmode_{\ensurett{#1}}%%
  \else$_{\ensurett{#1}}$%%
  \fi}
%% Superscripting with typerwriter
\def\suptt#1{\ifmmode^{\ensurett{#1}}%%
  \else$^{\ensurett{#1}}$%%
  \fi}
%% Functional languages chapter commands
\newcommand{\lamA}{\ensuremath{\lambda}-calculus\xspace}
\newcommand{\LamA}{\ensuremath{\lambda}-Calculus\xspace}
\newcommand{\lamAbs}[2]{\ensuremath{\lambda#1.\ #2}}
\newcommand{\lamApp}[2]{\ensuremath{#1\ #2}}
\newcommand{\lamPApp}[2]{\ensuremath{(#1\ #2)}}
\newcommand{\lamAPp}[2]{\ensuremath{(#1)\ #2}}
\newcommand{\lamApP}[2]{\ensuremath{#1\ (#2)}}
\newcommand{\lamAPP}[2]{\ensuremath{(#1)\ (#2)}}
\let\lamApPp=\lamApP
\let\lamAppP=\lamAPp
%% LC definition
\newtoks\toksA
\protected\def\lcname#1/{\ensuremath{\mathit{#1}}}
\protected\def\lcdef#1(#2)=#3;{\def\arg{#2}%%
  \def\lcargs##1,##2/{\def\arg{##2}%%
    \ifx\empty\arg%%
    \lcname ##1/%%
    \else\lcname ##1/\ \lcargs ##2/%%
    \fi}%%
  \ifx\empty\arg\toksA={\ }%%
  \else\toksA={\ \lcargs #2,/\ }%%
  \fi%%
  \ensuremath{\lcname#1/\the\toksA =\ #3}}
%% Arbitary number of applied arguments, separated
%% by asterisks (*).
\protected\def\lcapp#1/{\def\lcappB##1*##2/{\def\arg{##2}%
    \ensuremath{\ifx\arg\empty%%
      \lcname ##1/%%
      \else%%
      \lcname##1/\ \lcappB##2/%%
      \fi}}%%
  %% Adding a star here makes
  %% sure our applicaitn always ends with an asterisks, ensuring
  %% #2 will be \empty at some point.
  \lcappB#1*/}
\protected\def\lcabs#1.{\ensuremath{\lambda#1.\ }}

\newcommand{\lamId}{\lamAbs{x}{x}}
\newcommand{\lamCompose}{\lamAbs{f}{\lamAbs{g}{\lamAbs{x}{\lamApp{f}{(\lamApp{g}{x})}}}}}
\newcommand{\machLam}{\ensuremath{M_\lambda}\xspace}
\newcommand{\compMach}[1]{\ensuremath{\left\llbracket #1 \right\rrbracket}}
\newcommand{\compRho}[1]{\ensuremath{\rho(#1)}}
\newcommand{\verSub}[2]{\ensuremath{#1_{#2}}}
\newcommand{\verSup}[2]{\ensuremath{#1^{#2}}}
\newcommand{\lamC}{\ensuremath{\lambda_C}\xspace}
\newcommand{\lamPlus}{\lamAbs{m}{\lamAbs{n}{\lamAbs{s}{\lamAbs{z}{\lamApp{m}{\lamApPp{s}{\lamApp{n}{\lamApp{s}{z}}}}}}}}}
%% Substitution notation -- [#1 -> #2]
\newcommand{\lamSubst}[2]{\ensuremath{[#1 \mapsto #2]}}
%% End functional languages chapter


%% MIL Chapter
\newcommand{\compMILE}[1]{\ensuremath{\left\llbracket #1 \right\rrbracket}}
\newcommand{\compMILV}[1]{\ensuremath{\left\llbracket #1 \right\rrbracket}}
\newcommand{\compMILQ}[2]{\ensuremath{\left\llbracket #2 \right\rrbracket}}
\newcommand{\milCtx}[1]{\ensuremath{\llfloor}#1\ensuremath{\rrfloor}}

%% This dimension makes sure the same amount of space
%% follows | and := in syntax rules like:
%%
%% term := var       (Variable)
%%      |  var var    (Application)
%%      |  \x. var    (Abstraction)
%%
\newdimen\termalign
\setbox0=\hbox{$:=$}
\termalign=\wd0 
\protected\def\term#1/{\ensuremath{\mathit{#1}}}
\protected\def\syntaxrule#1/{\hfil\text{\emph{#1}}}
\protected\long\def\termrule#1:#2:#3/{\term #1/ &\hbox{$:=$}\ensuremath{\ #2} & \syntaxrule #3/}
\protected\def\termcase#1:#2/{&\hbox to \termalign{$|$\hss}\ensuremath{\ #1} & \syntaxrule #2/}


%% End MIL chapter

%% Dataflow Chapter
% Domain function
\def\dom(#1){\ensuremath{\mfun{dom}(#1)}\xspace}
% Set of all integers.
\def\ZZ{\ensuremath{\mathbb{Z}}}
%%

%% Uncurrrying Chapter 
%% A space equal to a \thinspace, but we
%% can break a line at it.
\newskip\thinskipamt \thinskipamt=.16667em 
\protected\def\thinskip{\hskip \thinskipamt\relax}
\protected\def\thinnerskip{\hskip .5\thinskipamt\relax}
%% Capture a space token. Use a ``control-symbol'' (\. instead of \mksp)
%% to keep the trailing space from getting gobbled.
{\def\.{\global\let\sp= } \. }
%% Define \asp, which will capture the macro definition attached to space,
%% if one exists. Otherwise, \spa is relax after this.
{\catcode`\ =\active\gdef\asp{\ifx \relax\let\spa\relax\else\let\spa= \fi}}
\newtoks\foo
%% Removes spaces, implicit, active and explicit.
\protected\def\removespaces{\asp\afterassignment\removesp\let\next= }
\def\removesp{\foo={\next}\ifcat\noexpand\next\sp\foo={\removespace}%%
 \else\ifx\next\spa\foo={\removespaces}\fi%%
 \fi\the\foo}
%% MIL reserved word
\protected\def\milres#1/{\text{\ttfamily\bfseries #1}}
\protected\def\lab#1/{\textbf{\ensurett{\removespaces #1}}}
%% Constructs a closure: l { v1, ..., vN }
\protected\long\def\mkclo[#1:#2]{\lab #1/\ensuremath{\,\{\ensurett{#2}\}}\xspace}
%% Tuple version of closurs: {l: v1, ..., vN}.
\protected\long\def\clo[#1:#2]{\def\argA{#1}\def\argB{#2}\ensuremath{\{%%
      \ifx\argA\empty%%
      \else\lab #1/%%
        \ifx\argB\empty%%
        \else\ensurett{:\thinskip}%%
        \fi%%
      \fi\ensurett{#2}\}}\xspace}
%% Construct a thunk
\newbox\bracklbox \newbox\brackrbox
\setbox0=\hbox{$\{$} \setbox\bracklbox=\hbox to \wd0{\hfil[\kern0.25mm}
\setbox0=\hbox{$\}$} \setbox\brackrbox=\hbox to \wd0{\kern0.25mm]\hfil}
\protected\def\mkthunk[#1:#2]{\lab #1/%%
  \ensuremath{\,%%
    \mathopen{\copy\bracklbox}%%
    \ensurett{#2}%%
    \mathclose{\copy\brackrbox}\xspace}}
%% Binding statement: v <- {...}
\protected\def\binds#1<-#2;{\ensurett{\removespaces #1\texttt{<-}#2}\xspace}
%% In order to use \binds in verbatim environment, have to define
%% delimiters while they are active. The below defines \vbinds which
%% must be used in AVerb environments.  Notice the active space as
%% well - that is necessary so the space after \vbinds (and before the
%% first argument) in the verbatim environment gets eaten.
\begingroup\catcode`\!=\active \lccode`\!=`\< \lccode`\~=`\- 
  \catcode`\ =\active\lowercase{\endgroup\def\vbinds#1!~#2;}{\binds#1<-#2;}
%% Return statement: return ... ;
\protected\def\return#1;{\milres return/\ensurett{\ \removespaces #1}}
%% A closure capturing block. k {v1, ..., vN} x: ...
\protected\def\ccblock#1(#2)#3:{\lab#1/\ensuremath{\thinspace\{\ensurett{#2}\}}\ \ensurett{#3\hbox{:}}}
%% A normal block
\protected\def\block#1(#2):{\lab #1/\ensuremath{\thinspace(\ensurett{#2})}\ensurett{:}}
%% A goto expression
\protected\def\goto#1(#2){\lab #1/\thinspace\ensuremath{(\ensurett{#2})}}
%% An enter expression
\protected\def\app#1*#2/{\ensurett{\removespaces #1\ifmmode\ \fi{\text{\tt @}}\ifmmode\ \fi#2}}
\protected\def\bind{\texttt{<-}\xspace}
%% Primitive expression
\protected\def\prim#1(#2){\lab #1/\suptt*\ensuremath{(\ensurett{#2})}}
%% Program variable
\protected\def\var#1/{\ensurett{\removespaces #1}\xspace}
%% Case statement
\protected\def\case#1;{\milres case\ \ensuremath{\ensurett{\removespaces #1}}\ of/}
%% Case alternative
\protected\def\alt#1(#2)#3->#4;{\ensuremath{\ensurett{#1\ \ignorespaces#2\ \texttt{->}\ \ignorespaces #4}}}
%% Invoke
\protected\def\invoke#1/{\milres invoke/\ensurett{\ \removespaces #1}}
\def\rhs{right--hand side\xspace}
\def\lhs{left--hand side\xspace}
\def\enter{\texttt{@}\xspace}
\def\cc{closure--capturing\xspace}
\def\Cc{Closure--capturing\xspace}
%%

\newenvironment{myfig}[1][tbh]{\begin{figure}[#1]%%
\begin{singlespace}\centering%%
\figbegin}{\figend\end{singlespace}%%
\end{figure}}

%% Produce a sub-caption and label it.
\newcommand{\scap}[2][1in]{\begin{minipage}{#1}%%
\subcaption{}\label{#2}\end{minipage}}

%% Produce a sub-caption with text.
\newcommand{\lscap}[3][\hsize]{\begin{minipage}{#1}%%
\subcaption{#3}\label{#2}\end{minipage}}

% single-argument comment. Do not put
% a space before the command when used
% or the file will have two spaces.
\newcommand{\rem}[1]{}

%% A verbatim environment with active charactesr
%% so we can use math shortcuts and macros
\DefineVerbatimEnvironment{AVerb}{Verbatim}{commandchars=\\\{\},%% 
  codes={\catcode`\_8\catcode`\$3\catcode`\^7},%%
  numberblanklines=false}

\DefineVerbatimEnvironment{Verb}{Verbatim}{commandchars=\\\[\],%% 
  numberblanklines=false}

%% Turn on line numbers for Haskell code, 
%% and reset the line number counter.
\newcommand{\hsNumOn}{\numberson\numbersreset}
\newcommand{\hsNumOff}{\numbersoff}
%% Turn on line numbering in Haskell code within
%% the environment, then turn it off. The optional
%% argument specifies a prefix that \hslabel can
%% use to generate line number references. If no prefix
%% is givne, \hslabel will have no effect.
\newtoks\prefixtoks
\def\mkhslabel#1{\prefixtoks={#1}\let\prefix=a}
\def\hslabel#1{\ifx\prefix\relax%%
  \else\label{\the\prefixtoks_#1}%%
  \fi}
\def\unhslabel{\let\prefix=\relax}
\newenvironment{withHsNum}{\numberson\numbersreset}{\numbersoff}
\newenvironment{withHsLabeled}[1]{\numberson\numbersreset\mkhslabel{#1}}{\unhslabel\numbersoff}

%% Paragraph run-in
\newcommand{\runin}[1]{\begingroup\noindent\sffamily\textbf{#1}\qquad\endgroup}

%% Chapter bibliographies
\newcommand{\standaloneBib}{%%
  \ifthenelse{\boolean{standaloneFlag}}%%
             {\begin{singlespace}
                 \printbibliography
             \end{singlespace}}{}}

%% Adds an equation number on demand.
\newcommand\addtag{\refstepcounter{equation}\tag{\theequation}}

%% For typesetting set definitions like {x | x \in f(y)}
\newcommand\setdef[2]{\ensuremath{\{#1\ |\ #2\}}}

%% For typesetting function names like dom(f) or out(b).
\newcommand\mfun[1]{\ensuremath{\mathit{#1}}}

%% Marginal notes
\newcommand\margin[2]{\marginpar{\begin{singlespace}\emph{\footnotesize #2}\end{singlespace}}\relax #1}

%% Describe intent of a passage
\newcommand\intent[1]{{\begin{singlespace}\noindent\leftskip=-1in\emph{\footnotesize Intent: #1}\end{singlespace}}\nopagebreak[1]}

%% In aligned/alignedat/gathered environments, you don't et
%% automatice equation numbers. This command makes sure to
%% label them properly.
\newcommand\labeleq[1]{\refstepcounter{equation}\label{#1}}

%% Creates a hanging paragraph, where the first line is not
%% indented but all other lines are.
\def\itempar#1{\noindent\hangindent=\parindent\hangafter=1 #1\quad}

%% Disable overfull messages with ridiculous hfuzz value
\def\disableoverfull{\hfuzz=10in}

%% Set parfillskip so stretching does NOT occur at the end of
%% a paragraph (i.e., list of elements). Disable indent at beginning
%% of paragraph. Also turn off underfull hbox warnings.
%%
%% Intended to be used in a \vbox that forms part of a table or graphic,
%% which we want to be line-broken but not exactly like a normal paragraph.
\long\def\disableparspacing#1;{\def\arg{#1}\hbadness=100000\parindent=0pt\parfillskip=0pt\leftskip=0pt\rightskip=0pt%%
  \ifx\arg\empty\else\hsize=#1\relax\fi}
%% This stuff makes !+<text>+! write <text> in typewriter font.  

%% We make ! and + active characters early, then manipulate their
%% meaning to produce the right effect. Initially, + produces +. When
%% !  appears w/o a + following, it produces ``!''. When ``+''
%% follows, we start writing in teleteype (\ttfamily). The definition
%% of ``!'' changes to produce a bang. ``+'' changes such that it
%% looks for trailing ``!''. When no ``!'' appears, ``+'' produces ``+''. 
%% If a ``!'' appears, we shift out of \ttfamily (by ending the group) and
%% reset the meaning of ``!'' and ``+'' so we can start again.
\makeatletter
\let\mdplus=+\let\mdbang=!      %% Preserve meaning of + and ! so we can put them into document.
%% Turn off mark down for everyone
\outer\def\nomd{\catcode`!=12\catcode`+=12}
%% Turn mark down on for everyone
\outer\def\domd{\catcode`!=\active\catcode`+=\active %%
  \initialmd}
%% Use only with a group IMMEDIETALY following. Turns off
%% markdown for the group-to-come, without actually tokenizing the
%% group. If no group follows, this has no effect.
\protected\def\pausemd{\def\dopause{\catcode`!=12\catcode`+=12}%%
  \def\pausemdB{\ifx\next\bgroup%%
    %% A ``partial'' application of expandwith is used
    %% so we don't double up the group argument (which is what
    %% happens if we expand \next). This has the effect of 
    %% inserting \expandafter\dowith in front of the upcoming {. 
    %% If no brace is coming, \withmdC will have no effect.
    \def\pausemdC{\expandafter\dopause}
  \else
    \let\pausedmC=\relax
  \fi\pausemdC}
  %% \futurelet has to end the macro so it grabs the next token
  %% from the input file. Otherwise, it grabs it *from* this
  %% definition.
  \futurelet\next\pausemdB} %%
%% Turns markdown on for the group-to-come, without actually
%% tokenizing the group. Only has an effect when
%% used in front of a group, otherwise its a no-op.
\protected\def\withmd{\def\dowith{\catcode`!=\active\catcode`+=\active\initialmd}%%
  \def\withmdB{\ifx\next\bgroup %%
    %% A ``partial'' application of expandafter is used
    %% so we don't double up the group argument (which is what
    %% happens if we expand \next). This has the effect of 
    %% inserting \expandafter\dowith in front of the upcoming {. 
    %% If no brace is coming, \withmdC will have no effect.
      \def\withmdC{\expandafter\dowith} %%
    \else %%
      \let\withmdC=\relax %%
    \fi\withmdC}%%
  %% \futurelet has to end the macro so it grabs the next token
  %% from the input file. Otherwise, it grabs it *from* this
  %% definition.
  \futurelet\next\withmdB} %%
%% Make ! and + active in the following group so they have the right
%% catcode in the definitions to follow.
\catcode`!=\active\catcode`+=\active %%
%% Initial definitions associated with ! and +.
\def\initialmd{\protected\def!{\startTTA} %%
  \protected\def+{\stopTTA}} %%
%% Step 1 of startTT. Inital meaning of !; capture next token in \next, go to next step.
\def\startTTA{\futurelet\next\startTTB} %%
%% Step 2 of startTT. Compare captured token to + and go to step 3 if true. Otherwise
%% output a ! (since that started our macro), the argument captured and stop
%% processing.
\long\def\startTTB{\ifx\next+\expandafter\startTTC\expandafter\@gobble\else\mdbang\fi} %%
%% Step 3 of startTT. Shift into teletype mode and change definition of 
%% + and ! so we can stop processing.
\def\startTTC{\begingroup\ifmmode %%
  \let \math@bgroup \relax %%
  \def \math@egroup {\let \math@bgroup \@@math@bgroup %%
    \let \math@egroup \@@math@egroup} %%
  \mathtt\relax %%
  \else  %%
  \ttfamily\fi} %%
%% Step 1, 2  and 3 of stopTT follow the same pattern as startTT.
\def\stopTTA{\futurelet\next\stopTTB} %%
\long\def\stopTTB{\ifx\next!\expandafter\stopTTC\expandafter\@gobble\else\mdplus\fi} %%
\def\stopTTC{\endgroup}%%
\catcode`!=12\catcode`+=12
\makeatother

\domd

%% Place an input file on the next page
\def\onnextpage#1{\afterpage{\clearpage\input{#1}\clearpage}}

\begin{document}
\ifthenelse{\boolean{standaloneFlag}}
           {\VerbatimFootnotes
             \DefineShortVerb{\#}
             \doublespacing
             \setcounter{chapter}{0}}{}

%% Default float parameters. For case when
%% multiple chapters are included and
%% only one needs custom float settings.
\renewcommand{\textfraction}{0.2}
\renewcommand{\topfraction}{0.9}


\chapter{Dead-code elimination}

Dead-code elimination seeks to remove program statements that will not
be executed or which have no visible effect. It can be applied at multiple
points during compilation, as other transformations frequently introduce 
dead code. Without dead-code elimination, many other optimizations have
minimal effect, since they leave behind code which still executes even if
it is no longer needed.

\emph{Uncurrying} (descrbed in Chapter \ref{ref_chapter_uncurrying}), especially
when repeatedly applied, can leave behind many unusued bindings. Consider
this program, which executes @const3@ to return 3:

\begin{code}
const3 a b c = c
main = const3 1 2 3
\end{code}

@main@ compiles to this monadic program:

\begin{code}
-- Not strictly accurate, main
-- really is main <- mainK {}
main = do
 t1 <- const3 @ 1
 t2 <- t1 @ 2
 t3 <- t2 @ 3
 return t3

const3 <- k1 {}
k1 {} a = k2 {a} 
k2 {a} b = k3 {a, b}
k3 {a, b} c = b(a,b,c)
b(a,b,c) = return c
\end{code}

Uncurrying will rewrite @t1 @@ 2@ as @k2 {1} @@ 2@:

\begin{code}
main = do
 t1 <- const3 @ 1
 t2 <- k2 {a} @ 2 -- TODO: Does syntax allow a closure value to be entered here?
 t3 <- t2 @ 3
 return t3
\end{code}

This changes makes @t1@ useless --- it causes an allocation but otherwise 
has no effect. Our dead-code elimination optimization will remove the binding.

%% \emph{The bulk of our work. Each optimization implemented is
%%   described. Each subsection should follow a common recipe: describe
%%   the optimization, give an example program and show how we want it
%%   changed, show salient points about the optimization (with code
%%   snippets and references to the full library), and reflect on the
%%   implementation.  Lazy Code Motion may get its own section.}

%% \emph{One question: should optimizations be categorized by direction? We could
%% describe all forward optimizations, then all backward ones. That may help
%% set up the concepts necessary to describe LCM.}

%% \emph{This portion gives an overview of the optimization, without
%% code or (much) notation. We motivate the optimization by 
%% showing an example.}

\section{Implementation}
We seek to remove two types of dead code: unused bindings (and,
therefore, allocation) and unused blocks. Each begins with a
``liveness'' analysis. Bindings (or blocks) that are referenced are ``live''
and therefore cannot be eliminated. Everything else is ``dead.'' 

\subsection{Eliminating Bindings}

A binding can be eliminated if no references to the bound variable
occur except the initial binding. We must travese each block backwards, noting when
a variable is used. We do not need to worry about \emph{how} the variable is used, so
we can just maintain a set of variables. When we see a binding, we eliminate the bound
variable from the set, since variables can be boudn more than once (and they cannot be used before being bound). 

%% TODO - leave out woTops?  

Figure \ref{ref_fig_liveTransfer} shows the algorithm for determining
liveness.  Top-level definitions are excluded from the set by the
@woTops@ function and are not considered here. 
At each statement, we add referenced or remove variables from
the live set @f@. \emph{Tail} (@Done@) and \emph{case} (@CaseM@)
statements add all referenced variables. Additionally, case statements
must inspect all alternatives and add the union of those
sets. @tailVars@ collects the variables used in a tail expression and
its definition is not shown here. \emph{Binding} (@Bind@) statements
remove the bound variable (@v@) and add any variables in the tail
(@t@).

\begin{figure}[h]
\begin{code}
live (Bind v t) f = Set.delete v f  `Set.union` tailVars mapEmpty t
live (CaseM v alts) f = Set.insert v (Set.unions (map (setAlt f) alts))
live (Done t) f = tailVars f t
\end{code}
\caption{Our transfer function for determing live variables within a block.}
\label{ref_fig_liveTransfer}
\end{figure}

With the facts collected by @live@, we can then eliminate useless
bindngs by looking up the bound variable in the set and removing the
binding if the variable is not there. The @deadRewrite@ functin,
shown in Figure \ref{ref_fig_deadRewrite}, gives our implementation
of this process. Recall that Hoopl interleaves analysis and
rewriting. %% TODO: Better mention this!  
Therefore, at each statement
we know what variables will be live. When a bind is seen, we check if
the bound variable (@v@) is absent from the live set (@f@). We also
take a conservative view and make sure the tail portion (@t@) is an
allocation and not an application ($@@$) or block call. We then eliminate
the binding when both are true. Otherwise, we leave the program unchanged.

\begin{figure}[h]
\begin{code}
rewrite (Bind v t) f 
  | safeTail t && not (v `Set.member` f) = return (Just emptyGraph)
rewrite _ _ = return Nothing
\end{code}
\caption{Our rewrite function that elminates useless bindings.}
\label{ref_fig_deadRewrite}
\end{figure}

%% \emph{``Interesting'' pieces of the implementation are described.}

\section{Reflection}
\emph{What was good, what didn't work so well, and how Hoopl helped
or hindered the implementation}

\end{document}


\chapter{Monadic Optimizations}
\emph{Describes optimizations based on the monad laws: bind/return collapse and
  monadic fuzzbang (inlining)}

\section{Copy-propagation}
\emph{Collapsing ``|x <- return y; p|'' to ``|[y/x] p|''.}
\subsection{Example of Desired Optimization}
\subsection{Implementation}
\subsection{Reflection}

\section{Inlining}
\emph{Monadic inlining using the associativity law. That is:}

> y <- (z <- x; p1)
> p2

\noindent
\emph{becomes:}

> z <- x
> y <- p1
> p2

\subsection{Example of Desired Optimization}
\subsection{Implementation}
\subsection{Reflection}

\chapter{Lazy Code Motion}
\emph{Describes our implementation of LCM in terms of the four passes
  used. This section will give an overview of LCM and briefly describe
  each pass. We give a example program which will be used throughout
  the section.}

\section{Anticipated Expressions}

\section{Available Expressions}

\section{Dead-code Elimination}

\section{Reflection}

\emph{Conclusions regarding our implemenation.}

\chapter{Conclusion \& Future Work}

\emph{Where we started and where we wished we could have go to.}

\end{document}
