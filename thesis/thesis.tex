\documentclass[11pt]{article}
\usepackage{standalone}
%include polycode.fmt
\usepackage{sectsty}
\usepackage{palatino}
\usepackage[scaled=0.92]{helvet}
\usepackage{xspace}
\renewcommand\ttdefault{cmtt}
\allsectionsfont{\sffamily}
\usepackage{url}
\usepackage{fancyvrb}
\usepackage{setspace}
\usepackage{cmds}
\usepackage{ifthen}
\newboolean{lhs2tex}
\setboolean{lhs2tex}{true}
% Awkward Habit: Design and Optimization of A Monadic Intermediate Language
\title{Design and optimization of a monadic intermediate language}
%if False
\setboolean{lhs2tex}{false}
\author{Justin Bailey \\ \url{justinb@cs.pdx.edu}}
%else
\ifthenelse{\boolean{lhs2tex}}{\author{Justin Bailey \\ \url{justinb@@cs.pdx.edu}}}{}
%endif
\date{\today}

\begin{document}
\VerbatimFootnotes
\DefineShortVerb{\#}
\doublespacing

\maketitle

\section{Abstract}

This thesis describes two related efforts in Caffeine, a compiler for
the Habit programming language. The first, a novel monadic
intermediate language named BC, makes allocation of heap objects such
as closures and data values explicit. The second uses the Hoopl
library to implement dataflow optimizations for BC based on the
``monad laws.''

\section{Audience}

\section{Introduction/Overview}

\section{BC -- The Language of Blocks and Captures}

Defines, motivates and gives examples of the BC language.

\section{Compiling the \lamA to BC}

Demonstrates compiliation from a variant of the \lamA to BC. 

\section{Dataflow Optimization and the Hoopl Library}

Describes Hoopl and its use.

\section{Optimizing BC Programs Using Hoopl}

Catalog of optimizations implemented over BC using Hoopl.

\section{Eliminating Heap Allocation in BC using Hoopl}

Special attention to an optimization for eliminating heap allocation
in BC programs.

\section{Compiling Habit to BC to x86}

How it all came together.

\section{Conclusion \& Future Work}

\end{document}
