%% Used in the languages chapter, this
%% table is placed in its own file so we can use
%% it with the afterpage command.
\begin{singlespace}
  \begin{longtable}{p{2in}l@{\vline\hspace{.1in}}p{3.5in}}
    \caption{Compilation rules from \lamA to \machLam.} \\
    \hline \\
    \endfirsthead
    \caption{Compilation rules from \lamA to \machLam \emph{(cont'd)}} \\
    \hline \\
    \endhead
    \\ \hline \multicolumn{3}{r}{\emph{Continued on next page}}
    \endfoot 
    \\ \hline
    \endlastfoot
    %% Variables
    \multicolumn{3}{c}{\emph{(a) Variable Reference}} \\ \\[-.5em]
    \begin{minipage}[t]{2in}
      \begin{AVerb}
\compMach{v} = 
  Copy \compRho{v} ``res''
      \end{AVerb}
    \end{minipage} & & We use the $\rho$ function to find the register containing the variable $v$. As the compiler
    maintains $\rho$, this lookup is performed during compilation, not while the program executes.\\ \\

    %% Application
    \multicolumn{3}{c}{\emph{(b) Function Application}} \\ \\[-.5em]
    \begin{minipage}[t]{2in}
      \begin{AVerb}
\compMach{\lamApp{f}{g}} = 
      \end{AVerb}
    \end{minipage} \\

    \begin{minipage}[t]{2in}
      \begin{AVerb}
  Copy ``arg'' $r$
  Copy ``clo'' $s$
      \end{AVerb}
    \end{minipage} &  & $r$ and $s$ are ``fresh'' registers. \\ \\[-.5em]

    \begin{minipage}[t]{2in}
      \begin{AVerb}
  \compMach{g}
  Copy ``res'' $t$
  \compMach{f}
  Copy ``res'' $u$
      \end{AVerb}
    \end{minipage} &  & Because $g$ and $f$ may be terms themselves, we compile their code inline 
    first. We copy the result of evaluating each from #res# to $t$ and
    $u$, respectively (both fresh registers). We do not use #arg# and
    #clo# because they may get overwritten while evaluating $g$ and
    $f$. \\ \\[-.5em]

    \begin{minipage}[t]{2in}
      \begin{AVerb}
  Copy $t$ ``arg''
  Copy $u$ ``clo''
  Enter
      \end{AVerb}
    \end{minipage} &  & We copy the result of $g$ and $f$ to #arg# and #clo#, 
    respectively. After #Enter# executes, #res# will hold the result
    of \lamPApp{f}{g}. However, we do not mention #res# here becouse
    our case \emph{(a)} handles the results of functions.\\ \\[-.5em]

    \begin{minipage}[t]{2in}
      \begin{AVerb}
  Copy $r$ ``arg''
  Copy $s$ ``clo''
      \end{AVerb}
    \end{minipage} &  & Restore the previous #arg# and #clo# registers. \\ \\

    %% Abstraction
    \multicolumn{3}{c}{\emph{(c) Abstraction}} \\ \\[-.5em]
    \begin{minipage}[t]{2in}
      \begin{AVerb}
\compMach{\lamAbs{x}{t}} = 
l : 
  \compMach{t}
  Return 
      \end{AVerb}
    \end{minipage} &  & We mark the location of $t$'s compiled form with #l#, a fresh
    label. We place the compiled code for $t$ inline. Because a function always
    produces a result, we follow the code with #Return#. \\ \\[-.5em]

    \begin{minipage}[t]{2in}
      \begin{AVerb}
m : 
      \end{AVerb}
    \end{minipage} &  & Again, we mark this portion of the program 
    with a fresh label, #m#. \\* \\*[-.5em]
    
    \begin{minipage}[t]{2in}
      \begin{AVerb}
  MkClo l [\compRho{v_1}, \dots, 
           \compRho{v_N}, 
           ``arg'']
  Return
      \end{AVerb}
    \end{minipage} &  & We build a closure holding the \emph{free variables} found in
    $t$. $v_1, \dots, v_N$ represent the free variables, and we use
    $\rho$ to find the location of those variables. We also add our
    argument to the end of the closure. Finally, the closure points to
    #l#, the label marking the location of the compiled body,
    $t$. Because #MkClo# puts the closure created in #res#, we can
    immediately return. 

  \label{tbl_lang2}
  \end{longtable}
\end{singlespace}
