%% Used in the languages chapter, this
%% table is placed in its own file so we can use
%% it with the afterpage command.
\begin{singlespace}
  \begin{longtable}{p{2in}p{3.5in}}
    \caption{Compilation rules from \lamA to \machLam.} \\
    \hline \\
    \endfirsthead
    \caption{Compilation rules from \lamA to \machLam \emph{(cont'd)}} \\
    \hline \\
    \endhead
    \\ \hline \multicolumn{2}{r}{\emph{Continued on next page}}
    \endfoot 
    \\ \hline
    \endlastfoot
    %% Variables
    \multicolumn{2}{c}{\emph{(a) Variable Reference}} \\ 
    \begin{minipage}[t]{2in}
      \begin{Verbatim}[commandchars=\\\{\}]
\compMach{v} = 
      \end{Verbatim}
    \end{minipage} \\

    \begin{minipage}[t]{2in}
      \begin{Verbatim}[commandchars=\\\{\}]
  Set ``v'' ``res''
  Return
      \end{Verbatim}
    \end{minipage} &  Because our \lamA does not have any real ``values'', we just
    set the #res# register to the variable name. \\ \\

    %% Application
    \multicolumn{2}{c}{\emph{(b) Function Application}} \\ 
    \begin{minipage}[t]{2in}
      \begin{Verbatim}[commandchars=\\\{\}]
\compMach{\lamApp{f}{g}} = 
      \end{Verbatim}
    \end{minipage} \\

    \begin{minipage}[t]{2in}
      \begin{Verbatim}[commandchars=\\\{\}, codes={\catcode`\_8\catcode`\$3}]
  Copy ``arg'' $r$
  Copy ``clo'' $s$
      \end{Verbatim}
    \end{minipage} & $r$ and $s$ are ``fresh'' registers. \\ \\[-.5em]

    \begin{minipage}[t]{2in}
      \begin{Verbatim}[commandchars=\\\{\}]
  Copy \compRho{g} arg
  Copy \compRho{f} clo
  Enter
      \end{Verbatim}
    \end{minipage} & $\rho$ associates variables
    to the register that they will be found in. This lookup occurs
    during compilation, not while the program executes. Here we copy
    $f$ and $g$ to the #clo# and #arg# registers, respectively. \\ \\[-.5em]

    \begin{minipage}[t]{2in}
      \begin{Verbatim}[commandchars=\\\{\}, codes={\catcode`\_8\catcode`\$3}]
  Copy $r$ ``arg''
  Copy $s$ ``clo''
      \end{Verbatim}
    \end{minipage} & Restore the previous #arg# and #clo# registers. \\ \\

    %% Abstraction
    \multicolumn{2}{c}{\emph{(c) Abstraction with an Abstraction}} \\ 
    \begin{minipage}[t]{2in}
      \begin{Verbatim}[commandchars=\\\{\}]
\compMach{\lamAbs{x}{\lamAbs{y}{t}}} = 
      \end{Verbatim}
    \end{minipage} \\

    \begin{minipage}[t]{2in}
      \begin{Verbatim}[commandchars=\\\{\}]
m : 
      \end{Verbatim}
    \end{minipage} & We mark the function body with a fresh label, #m#. \\ \\[-.5em]
    
    \begin{minipage}[t]{2in}
      \begin{Verbatim}[commandchars=\\\{\}, codes={\catcode`\_8\catcode`\$3}]
  Store r1 (``clo'', 0) 
  Store r2 (``clo'', 1) 
  \dots
  Store r$N$ (``clo'', 
            $N$)
      \end{Verbatim}
    \end{minipage} & Copy current values out of the closure. $N$
    equals the number of fields in the closure. \\ \\[-.5em]

    \begin{minipage}[t]{2in}
      \begin{Verbatim}[commandchars=\\\{\}, codes={\catcode`\_8\catcode`\$3}]
  MkClo l [r1, \dots, r$N$, 
           arg]
  Return
      \end{Verbatim}
    \end{minipage} & We call ourselves recursively and 
    retrieve a label, #l#, holding the location of the compiled body:
    \[ l = \compMach{\lamAbs{y}{t}} \].
    We then create a new closure which points to #l#. We put
    all values from the current closure into the new, and add our 
    argument, $x$, using the #arg# register. Because #MkClo# puts
    the closure created in #res#, we can immediately return. \\ \\

    %% Abstraction 2
    \multicolumn{2}{c}{\emph{(d) Abstraction with a Term}} \\* 
    \begin{minipage}[t]{2in}
      \begin{Verbatim}[commandchars=\\\{\}]
\compMach{\lamAbs{x}{t}} = 
      \end{Verbatim}
    \end{minipage} \\*

    \begin{minipage}[t]{2in}
      \begin{Verbatim}[commandchars=\\\{\}]
m : 
      \end{Verbatim}
    \end{minipage} & Again, we mark the function body with a fresh label, #m#. \\* \\*[-.5em]

    \begin{minipage}[t]{2in}
      \begin{Verbatim}[commandchars=\\\{\}, codes={\catcode`\_8\catcode`\$3}]
  Load \compRho{v_1} (0, ``clo'')
  Load \compRho{v_2} (1, ``clo'')
  \dots
  Load \compRho{v_n} ($N$, 
             ``clo'')
  Copy ``arg'' \compRho{x} 
      \end{Verbatim}
    \end{minipage} & We load all free variables $v_1, \dots, v_n$ from our
    closure into the appropriate registers, using the $\rho$ function. We also
    copy the argument $x$ from the #arg# register to the location given by
    $\rho$. \\ \\[-.5em]

    \begin{minipage}[t]{2in}
      \begin{Verbatim}[commandchars=\\\{\}]
  \compMach{t}
      \end{Verbatim}
    \end{minipage} & Now that variables are set up correctly, we compile the body
    of the abstraction and place it inline here.
  \label{tbl_lang2}
  \end{longtable}
\end{singlespace}
