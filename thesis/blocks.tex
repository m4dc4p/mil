\section{Basic Blocks}
\label{sec_back3}

%% Begin by placing the specific concept in the overall context of
%% dataflow. Give a small example highlighting the concept. Point
%% out fine points or subtleties that occur when generalizing the concept. End
%% by summarizing how the concept fits into dataflow (in a bit larger
%% sense than the first summary).

\begin{myfig}[th]
\begin{tabular}{ccc}
\begin{minipage}[b]{1.5in}
  \subfloat{\begin{minipage}[t]{1.5in}
\begin{AVerb}[numbers=left]
int a = 1; \label{lst_back3_start}
int b = 2; 
int c = 3; 
int d = 4; \label{lst_back3_end}

if(a + d > b + c)
  \dots
else
  \dots
\end{AVerb}
\end{minipage}
\label{fig_back4_a}}
\end{minipage} & %%
\begin{minipage}[b]{1.5in}
  \subfloat{\begin{tikzpicture}[>=stealth, node distance=.5in]
  \withmd{\node[invis] (entry) {};

  \node[stmt, below of=entry] (assigna) {!+a = 1+!\labelNode{lst_back4_assigna}};
  \node[labelfor=assigna] () {\refNode{lst_back4_assigna}};

  \node[stmt, below of=assigna] (assignb) {!+b = 2+!\labelNode{lst_back4_assignb}};
  \node[labelfor=assignb] () {\refNode{lst_back4_assignb}};

  \node[stmt, below of=assignb] (assignc) {!+c = 3+!\labelNode{lst_back4_assignc}};
  \node[labelfor=assignc] () {\refNode{lst_back4_assignc}};

  \node[stmt, below of=assignc] (assignd) {!+d = 4+!\labelNode{lst_back4_assignd}};
  \node[labelfor=assignd] () {\refNode{lst_back4_assignd}};

  \node[invis, below of=assignd] (exit) {};

  \draw [->>] (entry) to (assigna);
  \draw [->] (assigna) to (assignb);
  \draw [->] (assignb) to (assignc);
  \draw [->] (assignc) to (assignd);
  \draw [->>] (assignd) to (exit);}

\end{tikzpicture}
\label{fig_back4_b}}
\end{minipage}& %%
\begin{minipage}[b]{1.5in}
  \subfloat{\begin{minipage}[t]{2in}
\begin{AVerb}
         E (1)
         |
         V
       -----(2)
      |a = 1|
      |b = 2|
      |c = 3|
      |d = 4|
       -----
         |
         V
 -----------------(3)
|if(a + d > b + c)|--|
 -----------------   V
         |          ---(4)
         V         |\dots|
     ---(5)         ---
    |\dots|--> X <---|
     ---
\end{AVerb}
\end{minipage}





\label{fig_back4_c}}
\end{minipage} \\
\subref{fig_back4_a} & \subref{fig_back4_b} & \subref{fig_back4_c} \\
\end{tabular}
\caption{\subref{fig_back4_a}: A C-language fragment to illustrate
  \emph{basic blocks}.  \subref{fig_back4_b}: The CFG for
  \subref{fig_back4_a} without basic blocks. \subref{fig_back4_c}: The
  CFG for \subref{fig_back4_c} using basic blocks.}
\label{fig_back4}
\end{myfig}

%% Basic blocks
Consider the C-language fragment and CFGs in Figure~\ref{fig_back4}. 
Part~\subref{fig_back4_b} shows the CFG for part~\subref{fig_back4_a}: a long, straight sequence of nodes, one after
another. Part~\subref{fig_back4_C}, however, collapses all those nodes
into~\refNode{lst_back5_assign}. Our graph of statements becomes a
graph of \emph{blocks}.

Part~\subref{fig_back4_c} represents the assignment statements on
lines~\ref{lst_back3_start} -- \ref{lst_back3_end} as a \emph{basic
  block}: a sequence of statements with one entry, one exit, and no
branches in-between. Execution cannot start in the ``middle'' of the
block, nor can it branch anywhere but at the end of the block. In fact,
Part~\ref{fig_back4_b} also shows four basic blocks -- they just happen
to consist of one statement each.

The representation given in Part~\subref{fig_back4_c} has a number of
advantages. It tends to reduce both the number of nodes and the number
of edges in the graph. The dataflow algorithm maintains two sets of
\emph{facts} for every node -- reducing the number of nodes obviously
reduces the number of facts stored. The algorithm also iteratively
propagates facts along edges -- so reducing the number of edges
reduces the amount of work we need to do. When rewriting, blocks allow
us to move larger amounts of the program at once. It also can be shown
(see \citep{AhoXX}) that we do not lose any information by collapsing
statements into blocks. For efficiency and brevity, we will work with
basic blocks rather than statements from here forward.
