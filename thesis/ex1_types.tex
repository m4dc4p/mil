\documentclass[11pt]{article}
%include polycode.fmt
\usepackage{palatino}
\usepackage[scaled=0.92]{helvet}
\usepackage{fullpage}
\usepackage{url}
\usepackage{fancyvrb}
\usepackage{setspace}
\renewcommand\ttdefault{cmtt}
\begin{document}
\doublespacing
\fvset{fontfamily=cmtt}

%if False
\CustomVerbatimEnvironment{code}{Verbatim}{}
\DefineShortVerb{\|%|
}
%endif

\VerbatimFootnotes
\DefineShortVerb{\#}

\section*{Scope}

This analysis only considers simple values and a few different instructions. 

\section*{Values}

Currently, we consider two types of values: characters (32 bit values)
and strings. Characters are 32 bit values which are either NULL (0) or
some other value. A string is a pointer to a contiguous sequence of
characters, terminated by a NULL-valued character.

These values cannot be unrestricted, however. For example, we do not
want to allow a |String| to point past its terminator or de-reference a
|NULL| pointer. Haskell notation gives us convenient notation for expressing 
allowed values:

< String = NULL | EOS | Ptr Char
< Char = NULL | 1 | 2 | ... 

In other words, a |String| is a NULL pointer, points to a NULL-character (``end-of-string'' -- EOS), or points to a valid character. A |Char| is either NULL (0) or itself.

At the machine level many of these values have the same representation. We hope to use types to distinguish each value.

\section*{Incrementing Pointers}

Consider this instruction, where |r1| is a register:

\begin{verbatim}
  addl    $4, r1
\end{verbatim}

If |r1| holds a character, the instruction is not very interesting. If
|r1| is a string, though, then this instruction will move the pointer
to the next character in the string. Incrementing the pointer is only
safe if two conditions holds: the pointer is not NULL and it does not
already point to the end of the string.

If we only talk about the type held in |r1|, we can't guarantee it is
safe at all. |r1| could be NULL or point to the end of the
string. Incrementing it in those cases is not safe! Types alone are
not enough -- we need to mention the value held as well. To that end,
we write the type of |r1| as a |case| expression. The rule expresses that
if |r1| matches one of the arms, then r1's type is determined by that
arm of the case:

< r1 = case r1 of 
<   Ptr Char -> String 
<   Char -> Char

This says that if we know |r1| is a pointer to a character (``|Ptr
Char|''), we can stay it is a |String| afterwards\footnote{We cannot
  say it is a Ptr or EOS because we haven't tested the value pointed
  to yet.}. If |r1| is a character, it is still a character
afterwards.

What is more interesting is the cases that are NOT allowed. Imagine these cases:

< r1 = case r1 of 
<   ...
<   String -> String
<   NULL -> ...
<   EOS -> ...

Each would open the door to arbitrary manipulation of the string pointer.

\section*{Comparing Values \& Conditional Branches}

Consider this instruction:

\begin{verbatim}
  cmpl    $0, r1
\end{verbatim}

|cmpl| will set the Zero Flag (ZF) if r1 equals 0. Otherwise, ZF will
be 1. If r1 is a character, this doesn't tell us much. If it is a
|String|, we know a little more. If |r1| is 0, we know it holds
NULL. Otherwise, it points to a character or the end of a string. Using
the case notation from above, we add these two conditions to the test
for |r1|:

< r1 = case r1 of
<   String -> ZF = 0 -> NULL
<   String -> ZF /= 0 -> {EOS | Ptr Char}

The second branch indicates that we don't know what |r1| points to,
but we know it is not NULL.

Now consider what happens when we branch after a comparison:

\begin{verbatim}
test:   
  cmpl    $0, r1
  jnz     loop
...
loop:   
\end{verbatim}

If control passed to |loop|, we know that ZF was not 0 and therefore
|r1| is is a |Ptr Char| rather than |EOS|. If control falls through the
branch, then we know |r1| is |EOS| and therefore we have reached
the end of the string. 

Conditional branches, therefore, allow us to determine which ``branch'' should
be taken on a match. 

\end{document}
