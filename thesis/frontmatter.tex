\documentclass[12pt]{article}
\include{preamble.f.tex}
\begin{document}
\ifthenelse{\boolean{standaloneFlag}}
           {\doublespacing}{}

%if False
% lhs2tex ignores this section
\newcommand{\authorEmail}{\url{justinb@cs.pdx.edu}}
%else
% LaTeX ignores this section, unless pre-processed with lhs2Text
\ifthenelse{\boolean{lhs2tex}}%
           {\newcommand{\authorEmail}{\url{justinb@@cs.pdx.edu}}}%
           {}
%endif

\date{}
\author{Justin Bailey \\ \authorEmail}
\title{Using Dataflow Optimization Techniques with a Monadic Intermediate Language}

\maketitle 

\renewcommand{\abstractnamefont}{\normalfont\small\sffamily\bfseries}
\begin{abstract}
  Dataflow analysis of programs represented as control-flow graphs
  (CFGs) of basic blocks underlies many optimizations implemented by
  imperative language compilers. Functional language compilers, in
  contrast, traditionally optimize by rewriting programs according to
  algebraic laws. We show that, by compiling to a \emph{monadic
    intermediate language}, we can treat our functional programs as
  CFGs of basic blocks. Doing so enables us to re-use the rich body of
  dataflow techniques developed for imperative language compilers. We
  first implement dead code elimination, an optimization common to
  both functional and imperative compilers. We then demonstrate an
  optimization specific to functional language compilers: elimination
  of intermediate closures. Next, we show how two optimizations,
  inlining and copy-propagation, derive from the \emph{monad laws} in
  our representation. Finally, we implement \emph{Lazy Code Motion}
  (LCM), one of the most complicated dataflow-based optimizations. To
  our knowledge this is the first implementation of LCM over a monadic
  intermediate language. Throughout, we use the \emph{Hoopl} library
  to implement our optimizations, making our work a case-study for the
  library as well.
\end{abstract}
\end{document}
