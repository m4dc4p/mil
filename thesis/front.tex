%&preamble
%% Looking at LaTeX sources reveals \usepackage and
%% \RequirePackage are made equivalent when \documentclass
%% executes, so we can test if \documentclass has
%% been executed by comparing those two control sequences.
\def\ifnodocclass{\def\loaded{\iffalse}%
  \def\notloaded{\iftrue}%%
  \let\which=\loaded%%
  \ifx\usepackage\undefined\let\which=\notloaded%%
  \else%%
    \ifx\usepackage\RequirePackage%%
    \else\let\which=\notloaded%%
    \fi%%
  \fi\which}
\dodocclass
\begin{document}
\ifthenelse{\boolean{standaloneFlag}}
           {\VerbatimFootnotes
             \DefineShortVerb{\#}
             \setcounter{chapter}{0}}{}

%% Default float parameters. For case when
%% multiple chapters are included and
%% only one needs custom float settings.
\renewcommand{\textfraction}{0.2}
\renewcommand{\textfraction}{0.2}
\renewcommand{\topfraction}{0.9}

\thispagestyle{empty}
\date{\today}
\author{by \\ Justin Bailey \\[24pt] 
               {\normalsize Thesis Committee:} \\
               Mark P. Jones, Chair\\
               James Hook\\
               Andrew Tolmach \\[48pt] 
               Portland State University}
\title{Using Dataflow Optimization Techniques with a Monadic Intermediate Language}
\par\begin{centering}{\Large
Using Dataflow Optimization Techniques with
a Monadic Intermediate Language} \par

by \\
Justin George Bailey\\
\texttt{jgbailey@codeslower.com}\par\vfill

\begin{singlespace}
A thesis submitted to Portland State University \\ 
in partial fulfillment of the requirements \\
for the degree of\\
Master of Science in Computer Science\vfill

Thesis Committee:\\
Dr. Mark P. Jones, Chair\\
Dr. James Hook\\
Dr. Andrew Tolmach\par\vfill

\copyright2012
\end{singlespace}
\end{centering}
\newpage

\pagenumbering{roman}\pagestyle{plain}
\addcontentsline{toc}{section}{Abstract}
\section*{Abstract}
{%%
%% don't allow paragraph to end without a little space
\parfillskip 3em plus 1fil%
%% protusion on the first line looks odd, disable it for 
%% this paragraph.
%\microtypesetup{protrusion=false}%
Our work applies the \emph{dataflow algorithm} to an area outside its
traditional scope: functional languages. Our approach relies on
a \emph{monadic intermediate language} that provides low-level,
imperative features like computed jumps and explicit allocations,
while at the same time supporting high-level, functional-language
features like case discrimination and partial application. We
prototyped our work in Haskell using the \hoopl library and this
dissertation shows numerous examples demonstrating its use. We prove
the efficacy of our approach by giving a novel description of
the \emph{uncurrying} optimization in terms of the dataflow algorithm, as
well as a complete implementation of the optimization using \hoopl.\par}
\newpage

\addcontentsline{toc}{section}{Acknowledgments}
\section*{Acknowledgments}
I wish to give my heartfelt thanks to Erin, my very patient and very understanding
wife. She provided invaluable support throughout a long, long
project. I also want to express my deep gratitude to my advisor, Dr. Mark P. Jones, who
agreed to answer the question ``So, how do functional language compilers
work?'' His ideas, enthusiasm, and willingness to mentor me made this
project both challenging and highly rewarding. 

I wish to thank my managers at \textsc{adp} who gave me the time and
freedom to pursue this and other projects: Mark Rankin, Tom Douglass,
and Kathy Oullette. Without their support, I would simply have not
been able to afford the time.

I want to give special thanks to my thesis committee, Dr. James Hook
and Dr. Andrew Tolmach, for their time and willingness to review and
comment on my work. Finally, my thanks to the other faculty and staff in the
Computer Science department at Portland State University. I feel
blessed that a place full of such smart, interesting and friendly
people was just a little ways away from my home and work.

\newpage

\singlespacing
\tableofcontents
\newpage
\addcontentsline{toc}{section}{\listfigurename}
\listoffigures
\newpage
\end{document}
