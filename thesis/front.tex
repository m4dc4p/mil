%&preamble
%% Looking at LaTeX sources reveals \usepackage and
%% \RequirePackage are made equivalent when \documentclass
%% executes, so we can test if \documentclass has
%% been executed by comparing those two control sequences.
\def\ifnodocclass{\def\loaded{\iffalse}%
  \def\notloaded{\iftrue}%%
  \let\which=\loaded%%
  \ifx\usepackage\undefined\let\which=\notloaded%%
  \else%%
    \ifx\usepackage\RequirePackage%%
    \else\let\which=\notloaded%%
    \fi%%
  \fi\which}
\dodocclass
\begin{document}
\ifthenelse{\boolean{standaloneFlag}}
           {\VerbatimFootnotes
             \DefineShortVerb{\#}
             \setcounter{chapter}{0}}{}

%% Default float parameters. For case when
%% multiple chapters are included and
%% only one needs custom float settings.
\renewcommand{\textfraction}{0.2}
\renewcommand{\textfraction}{0.2}
\renewcommand{\topfraction}{0.9}


\date{}
\author{Justin Bailey \\ \url{justinb@cs.pdx.edu}}
\title{Using Dataflow Optimization Techniques with a Monadic Intermediate Language}
\maketitle 

\pagenumbering{roman}\pagestyle{plain}
\addcontentsline{toc}{section}{Abstract}
\section*{Abstract}
The \emph{dataflow algorithm}, which analyzes programs using
their \emph{control-flow graph}, underlies many optimizations
implemented in modern compilers for imperative languages. Functional
language compilers, in contrast, traditionally optimize by rewriting
programs according to algebraic laws. By compiling functional programs
to a \emph{monadic intermediate language} (\mil), we can describe
traditional functional language optimizations using \emph{dataflow
equations}, the parameters that drive dataflow algorithm. Our work
gives an overview of dataflow analysis and describes how to
use \hoopl, a Haskell library implementing the dataflow algorithm. We introduce
\mil, an intermediate language designed for monadic programming, 
implementation of high-level functional language features, and
dataflow analysis. This work gives a novel description of
the \emph{uncurrying} optimization in terms of dataflow equations. We
also discuss how \mil programs can be optimized using the \emph{monad
laws} (independent of the dataflow algorithm), and a way in which
dataflow analysis could eliminate unnecessary allocations in the
presence of conditional expressions. 
\newpage

\addcontentsline{toc}{section}{Acknowledgments}
\section*{Acknowledgments}
Thanks very much!
\newpage

\singlespacing
\tableofcontents
\newpage
\addcontentsline{toc}{section}{List of Figures}
\listoffigures
\newpage
\end{document}
