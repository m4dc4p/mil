%&preamble
%% Looking at LaTeX sources reveals \usepackage and
%% \RequirePackage are made equivalent when \documentclass
%% executes, so we can test if \documentclass has
%% been executed by comparing those two control sequences.
\def\ifnodocclass{\def\loaded{\iffalse}%
  \def\notloaded{\iftrue}%%
  \let\which=\loaded%%
  \ifx\usepackage\undefined\let\which=\notloaded%%
  \else%%
    \ifx\usepackage\RequirePackage%%
    \else\let\which=\notloaded%%
    \fi%%
  \fi\which}
\dodocclass
\begin{document}
\ifthenelse{\boolean{standaloneFlag}}
           {\VerbatimFootnotes
             \DefineShortVerb{\#}
             \setcounter{chapter}{0}}{}

%% Default float parameters. For case when
%% multiple chapters are included and
%% only one needs custom float settings.
\renewcommand{\textfraction}{0.2}
\renewcommand{\textfraction}{0.2}
\renewcommand{\topfraction}{0.9}


\date{}
\author{Justin Bailey \\ \url{justinb@cs.pdx.edu}}
\title{Using Dataflow Optimization Techniques with a Monadic Intermediate Language}
\maketitle 

\pagenumbering{roman}\pagestyle{plain}
\addcontentsline{toc}{section}{\textbf{Acknowledgements}}
\section*{Acknowledgments}
Thanks very much!
\newpage

\begin{center}
  {\sffamily\bfseries Abstract}
  \addcontentsline{toc}{section}{\textbf{Abstract}}
\end{center}
\bigskip
\noindent
Dataflow analysis of programs represented as control-flow graphs
(CFGs) of basic blocks underlies many optimizations implemented by
imperative language compilers. Functional language compilers, in
contrast, traditionally optimize by rewriting programs according to
algebraic laws. We show that, by compiling to a \emph{monadic
  intermediate language}, we can treat our functional programs as CFGs
of basic blocks. Doing so enables us to re-use the rich body of
dataflow techniques developed for imperative language compilers. We
first implement dead code elimination, an optimization common to both
functional and imperative compilers. We then demonstrate
\emph{uncurrying}, an optimization specific to functional language
compilers. Next, we use the \emph{monad laws} to derive inlining and
copy-propagation optimizations.  Throughout, we use the \emph{Hoopl}
library to implement our optimizations, making our work a case-study
for the library as well.
\newpage

\singlespacing
\tableofcontents
\newpage
\addcontentsline{toc}{section}{\textbf{List of Figures}}
\listoffigures
\newpage
\end{document}
