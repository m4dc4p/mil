%&preamble
%% Looking at LaTeX sources reveals \usepackage and
%% \RequirePackage are made equivalent when \documentclass
%% executes, so we can test if \documentclass has
%% been executed by comparing those two control sequences.
\def\ifnodocclass{\def\loaded{\iffalse}%
  \def\notloaded{\iftrue}%%
  \let\which=\loaded%%
  \ifx\usepackage\undefined\let\which=\notloaded%%
  \else%%
    \ifx\usepackage\RequirePackage%%
    \else\let\which=\notloaded%%
    \fi%%
  \fi\which}
\dodocclass
\begin{document}
\ifthenelse{\boolean{standaloneFlag}}
           {\VerbatimFootnotes
             \DefineShortVerb{\#}
             \setcounter{chapter}{0}}{}

%% Default float parameters. For case when
%% multiple chapters are included and
%% only one needs custom float settings.
\renewcommand{\textfraction}{0.2}
\renewcommand{\textfraction}{0.2}
\renewcommand{\topfraction}{0.9}


\date{\today \\Draft}
\author{Justin Bailey \\ \url{justinb@cs.pdx.edu}}
\title{Using Dataflow Optimization Techniques with a Monadic Intermediate Language}
\maketitle 

\pagenumbering{roman}\pagestyle{plain}
\addcontentsline{toc}{section}{Abstract}
\section*{Abstract}
{%%
%% don't allow paragraph to end without a little space
\parfillskip 3em plus 1fil%
%% protusion on the first line looks odd, disable it for 
%% this paragraph.
%\microtypesetup{protrusion=false}%
Our work applies the \emph{dataflow algorithm} to an area outside the
algorithm's traditional scope: functional languages. Our approach
relies on a \emph{monadic intermediate language} that allows us to
represent both low-level, imperative features such as computed jumps
and explicit allocation, as well as high-level, functional-language
features like pattern-matching and partial application. We prototyped
our work in Haskell, relying extensively on the \hoopl library, and
this dissertation demonstrates many of the library's features. We
demonstrate the efficacy of our approach by giving a novel description
of the \emph{uncurrying} optimization in terms of the dataflow
algorithm. We believe this document shows that dataflow analysis can
be applied to functional languages in practical and interesting
ways.\par}
\newpage

\addcontentsline{toc}{section}{Acknowledgments}
\section*{Acknowledgments}
Thanks very much!
\newpage

\singlespacing
\tableofcontents
\newpage
\addcontentsline{toc}{section}{Figures}
\listoffigures
\newpage
\end{document}
