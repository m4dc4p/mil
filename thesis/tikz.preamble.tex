%% TikZ and LaTeX3-specific macros.
\usepackage{xparse}
\ExplSyntaxOn
\DeclareDocumentCommand \inFactsM {m m m} {\doFacts{#1}{#2}{\inB}{#3}}
\DeclareDocumentCommand \inFacts {m m O{1in}} {\doFacts{#1}{#2}{\inB}{#3}}
\DeclareDocumentCommand \outFactsM {m m m} {\doFacts{#1}{#2}{\outB}{#3}}
\DeclareDocumentCommand \outFacts {m m O{1in}} {\doFacts{#1}{#2}{\outB}{#3}}
\ExplSyntaxOff

\usepackage{tikz}
\usetikzlibrary{arrows,automata,positioning,calc}
%% Used for CFGs.
\tikzset{
  >=stealth, 
  node distance=.5in,
  stmt/.style={rectangle,
    draw=black, thick,        
    minimum height=2em,
    %% inner sep=2pt,
    %% text centered,
    %% node distance=.5in,
  },
  entex/.style={
    minimum height=2em,
    %% inner sep=2pt,
    %% text centered,
  },
  labelfor/.style={circle, 
    draw=black, thin,
    font={\footnotesize},
    inner sep=0,
    fill=white,
    above right=-1.5mm and -1.5mm of #1,
  },
  fact/.style={overlay},
  %% Invisible node
  invis/.style={inner sep=0pt, 
    minimum height=0em}, 
  table/.style={circle, fill=white,height=2mm}
}

%% Dataflow chapter commands
\newcounter{nodeCounter}[figure]
\newcommand{\inE}{\ensuremath{\mathit{in}}\xspace}
\newcommand{\out}{\ensuremath{\mathit{out}}\xspace}
\newcommand{\In}{\ensuremath{\mathit{In}}\xspace}
\newcommand{\InBa}{\ensuremath{\mathit{In}(B)}\xspace}
\newcommand{\Out}{\ensuremath{\mathit{Out}}\xspace}
%% Out(B_x) -- fact function for an named block.
\newcommand{\OutB}[1]{\ensuremath{\mathit{Out}(B_{\ref{#1}})}\xspace}
\newcommand{\OutBa}{\ensuremath{\mathit{Out}(B)}\xspace}
%% in(B) -- fact function for an anonymous block.
\newcommand{\inBa}{\ensuremath{\mathit{in}(B)}\xspace}
%% in(X) -- fact function for an anonymous block, but using a different variable.
\newcommand{\inXa}[1]{\ensuremath{\mathit{in}(#1)}\xspace}
%% in(B,v) -- fact function for an anonymous block and some variable.
\newcommand{\inBav}[1]{\ensuremath{\mathit{in}(B, #1)}\xspace}
%% in(B_x) -- fact function for an named block.
\newcommand{\inB}[1]{\ensuremath{\mathit{in}(B_{\ref{#1}})}\xspace}
%% in(B_x,v) -- fact function for an named block and some variable.
\newcommand{\inBv}[2]{\ensuremath{\mathit{in}(B_{\ref{#1}}, #2)}\xspace}
%% out(B) -- fact function for an anonymous block.
\newcommand{\outBa}{\ensuremath{\mathit{out}(B)}\xspace}
%% out(Bn) -- fact function for an named block.
\newcommand{\outBn}[1]{\ensuremath{\mathit{out}(B_{#1})}\xspace}
%% out(X) -- fact function for an anonymous block, but using a different variable.
\newcommand{\outXa}[1]{\ensuremath{\mathit{out}(#1)}\xspace}
%% out(B,v) -- fact function for an anonymous block and some variable.
\newcommand{\outBav}[1]{\ensuremath{\mathit{out}(B, #1)}\xspace}
%% out(B_x) -- fact function for an named block.
\newcommand{\outB}[1]{\ensuremath{\mathit{out}(B_{\ref{#1}})}\xspace}
%% out(B_x,v) -- fact function for an named block and some variable.
\newcommand{\outBv}[2]{\ensuremath{\mathit{out}(B_{\ref{#1}}, #2)}\xspace}
\newcommand{\entryN}{\emph{E}\xspace}
\newcommand{\exitN}{\emph{X}\xspace}
\newcommand{\refNode}[1]{\ensuremath{B_{\ref{#1}}}\xspace}
\newcommand{\labelNode}[1]{\refstepcounter{nodeCounter}\label{#1}}

%% Formats a list of facts
%% Argument should be like \facts{a/1, b/2, foobar/\bot, baz/\top}.
%% 
\newcounter{factctr}
\newcounter{breakctr}
\newtoks\varVal
\newtoks\varName
\newcommand{\facts}[1]{%%
  %% Test if the argument given contains a forward slash (/). Expands
  %% slashTest with argument such that if a slash is NOT present the 
  %% token \noSlash will be given as argument 2 to slashTest. Otherwise
  %% there must be slash.
  \def\hasSlash##1{\expandafter\slashTest##1/\noslash\endslash}%%
  \def\slashTest##1/##2##3\endslash{\ifx\noslash##2 N\else Y\fi}%%
  \def\getArgs##1/##2{\varName={##1}%%
    \varVal={##2}}
  \ensuremath{%%
    \setcounter{factctr}{0}%%
    \foreach \var in {#1}{%%
      \setcounter{breakctr}{\value{factctr}/2}
      %% Separate list with a comma
      \ifthenelse{\value{factctr}>0}{\text{,}\allowbreak}{}%%
      %% \tracingmacros=1%%
      %% If key/val arguments, use first form. Otherwise
      %% use second.
      \ifthenelse{\equal{\hasSlash{\var}}{Y}}%%
                  {\expandafter\getArgs\var \factC{\the\varName}{\the\varVal}}%%
                  {\var}%%
      %% \tracingmacros=0%%
      \stepcounter{factctr}%%
    }}}
\newcommand{\factC}[2]{{\ensuremath{(\mathit{#1},#2)}}}
\newcommand{\doFacts}[4]{\ensuremath{#3{#1}: %%
    \left\{\vcenter{\hsize=#4\facts{#2}}\right\}}}

%% End dataflow

