\begin{myfig}
  \begin{math}
    \begin{array}{rlr}

      \multicolumn{3}{c}{\emph{Facts}} \\

      \setLC &= \{\bot, \top\} \cup \ZZ.\\
      \setL{Var} &= \text{Set of all variables.} \\
      \setL{Fact} &= \setL{Var} \times \setLC. \\\\

      \multicolumn{3}{c}{\emph{Meet}} \\

      F_1 \wedge\ F_2 &= \begin{array}{rl}
        \{(a, x \lub y)\ | & a \in \dom(F_1), a \in \dom(F_2)\}\ \cup \\
        \{(a, y)\ | & a \in \dom(F_1), a \not\in \dom(F_2)\ \text{or} \\
                     & a \not\in \dom(F_1), a \in \dom(F_2)\},
      \end{array} \labeleq{eqn_back13} \labeleq{eqn_back12} & \eqref{eqn_back12} \\
      & \text{where\ } F_1, F_2 \in \setL{Fact}, \lub\ \text{as in Table~\ref{tbl_back4}.} \\\\

      \multicolumn{3}{c}{\emph{Transfer}} \\
      t (F, a\ \text{\tt =}\ C) &= \{(a, x \lub C), \text{when\ } (a, x) \in F\ \text{or} \\
                     & \phantom{= \{}(a, C), \text{when\ } a \not\in \dom(F)\}\ \cup \\
                     & \phantom{=} F\ \backslash\ \mfun{uses}(F, a),\\
          & \text{where\ } F \in \setL{Fact}, C \in \ZZ. \\
      t (F, a\text{\tt ++}) &= \{(a, \top)\} \cup (F\ \backslash\ \mfun{uses}(F, a)), \\
      & \text{where\ } F \in \setL{Fact}. \labeleq{eqn_back14} & \eqref{eqn_back14} \\\\
      \mfun{uses}(F, a) &= \{(a, x)\ |\ a \in \dom(F)\}, \\
      & \text{where\ } F \in \setL{Fact}, a \in \setL{Var}. \\\\

      \multicolumn{3}{c}{\emph{Direction}} \\

      \outBa &= t(\inBa, s), \labeleq{eqn_back3} & \eqref{eqn_back3} \\
      & \text{where $s$ a statement in block\ } B.\\
      \inBa &= \bigwedge\limits_{\mathclap{P \in \mathit{pred}(B)}} \outXa{P} \labeleq{eqn_back16} & \eqref{eqn_back16} \\\\ 
      \mfun{pred}(B) &= \text{Predecessors of block }\ B.
    \end{array}
  \end{math}
  \caption{The transfer function and associated definitions for the constant
  propagation analysis. Equation~\eqref{eqn_back3} shows how \out facts are
  created from \inE facts. \InBa facts, for some block $B$, are created from
  the \outBa facts of its predecessors. Facts are combined using the set-wise
  $\bigwedge$ operator.}
\label{fig_back10}
\end{myfig}
