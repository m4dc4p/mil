\begin{myfig}
\begin{align}
  \outBa &= \mathit{transfer}(\inBa). \label{eqn_back3} \\
  \inBa &= \bigwedge\limits_{\mathclap{P \in \mathit{pred}(B)}} \outXa{P} \label{eqn_back16}\\ 
  B \bigwedge\ \,\mathclap{\emptyset}\phantom{C} &= B \notag\\
  B \bigwedge\ \,\mathclap{C}\phantom{C} &= \bigcup\limits_{(a,x) \in B}
                    \left(\bigcup\limits_{(b,y) \in C} (a,x) \wedge (b,y)\right) \label{eqn_back12}\\ 
  (a,x) \wedge (b,y) &= 
  \begin{cases}
    (a,x \lub y) & \text{when}\ (a,x) \in B, (b,y) \in C,\ \text{\lub as in Table~\ref{tbl_back4}.}\\
    (a,x) & \text{when}\ b \not\in \mathit{var}(B). \\
    (b,y) & \text{when}\ a \not\in \mathit{var}(C). \\
  \end{cases} \label{eqn_back13}\\ 
  \mathit{transfer}(B) &= \bigcup\limits_{\mathclap{(a,x) \in B}} f(a,x) \label{eqn_back15}\\
  f (a,x) &= 
  \begin{cases}
    (a,x \lub C) & \text{when \texttt{a = \emph{C}}, where \texttt{\emph{C}} is an integer}. \\
    (a,\top) & \text{when \texttt{a} updated}. \\
    (a,x) & \text{otherwise}. \label{eqn_back14}\\
  \end{cases} \\
  \setLC &= \bot, 0, -1, 1, \dots \text{\it all integers} \dots, \top. \notag\\
  \setL{Var} &= \dots \text{\it set of variables used in the control-flow graph} \dots \notag\\
  B, C, \dots &= \dots \text{\it sets of pairs}\ (a, x)\ \text{such that}\ a \in 
    \setL{Var}\ \text{and}\ x \in \setLC. \notag\\
  \mathit{out}(B) &= \dots \text{\it \out facts from block}\ B\ \dots \notag\\
  \mathit{var}(B) &= \dots \text{\it variables used in block}\ B\ \dots \notag\\
  \mathit{pred}(B) &= \dots \text{\it predecessors for block}\ B\ \dots \notag
\end{align}
\caption{The transfer function and associated definitions for the constant
  propagation analysis. Equation~\eqref{eqn_back3} shows how \out facts are
  created from \inE facts. \InBa facts, for some block $B$, are created from
  the \outBa facts of its predecessors. Facts are combined using the set-wise
  $\bigwedge$ operator.}
\label{fig_back10}
\end{myfig}
