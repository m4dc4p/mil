\documentclass[12pt]{report}
%include polycode.fmt
\usepackage[T1]{fontenc}
\usepackage{calc}
\usepackage{palatino}
\usepackage{amsfonts}
\renewcommand\ttdefault{lmtt}
\usepackage{helvet}
\usepackage{xspace}
\usepackage{url}
\usepackage{fancyvrb}
\usepackage[doublespacing]{setspace}
%% below only necessary when using doublespacing -- corrects
%% the vertical space inserted when switching to singlespace
%% environment.
\def\correctspaceskip{\vskip-\baselineskip} 
\usepackage{amsmath}
\usepackage{booktabs}
\usepackage[margin=\parindent, format=hang,labelfont=bf]{caption}
%% \usepackage[subrefformat=parens]{subcaption}
%% The following makes sure we get parentheses around
%% subreferences. The newest version of the subcaption
%% package has an option for this, but that's not available
%% widely.
%%
%% From http://tex.stackexchange.com/questions/25644
\usepackage[labelformat=simple]{subcaption}
\makeatletter
  \def\thesubfigure{(\alph{subfigure})}
  \providecommand\thefigsubsep{~}
  \def\p@subfigure{\@nameuse{thefigure}\thefigsubsep}
\makeatother

\usepackage{ifthen}
\usepackage{stmaryrd}
\usepackage{longtable}
\usepackage{afterpage}
\usepackage{xifthen}
\usepackage{mathtools}
\usepackage[natbib=true,style=authoryear,backend=bibtex8]{biblatex}
\setlength{\bibitemsep}{\bigskipamount}
\addbibresource{thesis.bib}
\usepackage{microtype}

%% GSO margins.
\usepackage[left=1.5in, right=1in, top=1in, bottom=1in]{geometry}
\usepackage{abstract}

%% GSO requires 12 pt font for all headings
\usepackage[bf,sf,tiny,compact]{titlesec}
\titleformat{\chapter}[display]
            {}% format
            {\sffamily\bfseries\chaptertitlename\ \thechapter}
            {\baselineskip}
            {\sffamily\bfseries}
            {}

\hyphenation{data-flow mo-na-dic} 

%% Should unindent all haskell code set in a dispay (versus inline)
\makeatletter
  \@ifundefined{hscodestyle}
               {}
               {\renewcommand{\hscodestyle}{\advance\leftskip -\mathindent}}
\makeatother

% Used by included files to know they
% are NOT standalone
\newboolean{standaloneFlag}
\setboolean{standaloneFlag}{true}

\newlength{\rulefigmargin}
\setlength{\rulefigmargin}{2\parindent}

\newcommand\figbegin{\rule{\linewidth}{0.4pt}\\\vspace{12pt}}
\newcommand\figend{\rule{\linewidth}{0.4pt}}

%% Sets
\newcommand{\setL}[1]{\textsc{#1}\xspace}
\newcommand{\setLC}{\setL{Const}}

%% Lub, subset operators.
\protected\def\lub{\ifmmode\sqcap\else\raisebox{.1em}{\ensuremath{\sqcap}}\fi\xspace}
\newcommand{\sqlt}{\ensuremath{\sqsubset}\xspace}
\newcommand{\sqlte}{\ensuremath{\sqsubseteq}\xspace}

%% Subscripting with typewriter
\def\subtt#1{\ifmmode_{\ensurett{#1}}%%
  \else$_{\ensurett{#1}}$%%
  \fi}
%% Superscripting with typerwriter
\def\suptt#1{\ifmmode^{\ensurett{#1}}%%
  \else$^{\ensurett{#1}}$%%
  \fi}
%% Functional languages chapter commands
\newcommand{\lamA}{\ensuremath{\lambda}-calculus\xspace}
\newcommand{\LamA}{\ensuremath{\lambda}-Calculus\xspace}
\newcommand{\lamAbs}[2]{\ensuremath{\lambda#1.\ #2}}
\newcommand{\lamApp}[2]{\ensuremath{#1\ #2}}
\newcommand{\lamPApp}[2]{\ensuremath{(#1\ #2)}}
\newcommand{\lamAPp}[2]{\ensuremath{(#1)\ #2}}
\newcommand{\lamApP}[2]{\ensuremath{#1\ (#2)}}
\newcommand{\lamAPP}[2]{\ensuremath{(#1)\ (#2)}}
\let\lamApPp=\lamApP
\let\lamAppP=\lamAPp
%% LC definition
\newtoks\toksA
\protected\def\lcname#1/{\ensuremath{\mathit{#1}}}
\protected\def\lcdef#1(#2)=#3;{\def\arg{#2}%%
  \def\lcargs##1,##2/{\def\arg{##2}%%
    \ifx\empty\arg%%
    \lcname ##1/%%
    \else\lcname ##1/\ \lcargs ##2/%%
    \fi}%%
  \ifx\empty\arg\toksA={\ }%%
  \else\toksA={\ \lcargs #2,/\ }%%
  \fi%%
  \ensuremath{\lcname#1/\the\toksA =\ #3}}
%% Arbitary number of applied arguments, separated
%% by asterisks (*).
\protected\def\lcapp#1/{\def\lcappB##1*##2/{\def\arg{##2}%
    \ensuremath{\ifx\arg\empty%%
      \lcname ##1/%%
      \else%%
      \lcname##1/\ \lcappB##2/%%
      \fi}}%%
  %% Adding a star here makes
  %% sure our applicaitn always ends with an asterisks, ensuring
  %% #2 will be \empty at some point.
  \lcappB#1*/}
\protected\def\lcabs#1.{\ensuremath{\lambda#1.\ }}

\newcommand{\lamId}{\lamAbs{x}{x}}
\newcommand{\lamCompose}{\lamAbs{f}{\lamAbs{g}{\lamAbs{x}{\lamApp{f}{(\lamApp{g}{x})}}}}}
\newcommand{\machLam}{\ensuremath{M_\lambda}\xspace}
\newcommand{\compMach}[1]{\ensuremath{\left\llbracket #1 \right\rrbracket}}
\newcommand{\compRho}[1]{\ensuremath{\rho(#1)}}
\newcommand{\verSub}[2]{\ensuremath{#1_{#2}}}
\newcommand{\verSup}[2]{\ensuremath{#1^{#2}}}
\newcommand{\lamC}{\ensuremath{\lambda_C}\xspace}
\newcommand{\lamPlus}{\lamAbs{m}{\lamAbs{n}{\lamAbs{s}{\lamAbs{z}{\lamApp{m}{\lamApPp{s}{\lamApp{n}{\lamApp{s}{z}}}}}}}}}
%% Substitution notation -- [#1 -> #2]
\newcommand{\lamSubst}[2]{\ensuremath{[#1 \mapsto #2]}}
%% End functional languages chapter


%% MIL Chapter
\newcommand{\compMILE}[1]{\ensuremath{\left\llbracket #1 \right\rrbracket}}
\newcommand{\compMILV}[1]{\ensuremath{\left\llbracket #1 \right\rrbracket}}
\newcommand{\compMILQ}[2]{\ensuremath{\left\llbracket #2 \right\rrbracket}}
\newcommand{\milCtx}[1]{\ensuremath{\llfloor}#1\ensuremath{\rrfloor}}

%% This dimension makes sure the same amount of space
%% follows | and := in syntax rules like:
%%
%% term := var       (Variable)
%%      |  var var    (Application)
%%      |  \x. var    (Abstraction)
%%
\newdimen\termalign
\setbox0=\hbox{$:=$}
\termalign=\wd0 
\protected\def\term#1/{\ensuremath{\mathit{#1}}}
\protected\def\syntaxrule#1/{\hfil\text{\emph{#1}}}
\protected\long\def\termrule#1:#2:#3/{\term #1/ &\hbox{$:=$}\ensuremath{\ #2} & \syntaxrule #3/}
\protected\def\termcase#1:#2/{&\hbox to \termalign{$|$\hss}\ensuremath{\ #1} & \syntaxrule #2/}


%% End MIL chapter

%% Dataflow Chapter
% Domain function
\def\dom(#1){\ensuremath{\mfun{dom}(#1)}\xspace}
% Set of all integers.
\def\ZZ{\ensuremath{\mathbb{Z}}}
%%

%% Uncurrrying Chapter 
%% A space equal to a \thinspace, but we
%% can break a line at it.
\newskip\thinskipamt \thinskipamt=.16667em 
\protected\def\thinskip{\hskip \thinskipamt\relax}
\protected\def\thinnerskip{\hskip .5\thinskipamt\relax}
%% Capture a space token. Use a ``control-symbol'' (\. instead of \mksp)
%% to keep the trailing space from getting gobbled.
{\def\.{\global\let\sp= } \. }
%% Define \asp, which will capture the macro definition attached to space,
%% if one exists. Otherwise, \spa is relax after this.
{\catcode`\ =\active\gdef\asp{\ifx \relax\let\spa\relax\else\let\spa= \fi}}
\newtoks\foo
%% Removes spaces, implicit, active and explicit.
\protected\def\removespaces{\asp\afterassignment\removesp\let\next= }
\def\removesp{\foo={\next}\ifcat\noexpand\next\sp\foo={\removespace}%%
 \else\ifx\next\spa\foo={\removespaces}\fi%%
 \fi\the\foo}
%% MIL reserved word
\protected\def\milres#1/{\text{\ttfamily\bfseries #1}}
\protected\def\lab#1/{\textbf{\ensurett{\removespaces #1}}}
%% Constructs a closure: l { v1, ..., vN }
\protected\long\def\mkclo[#1:#2]{\lab #1/\ensuremath{\,\{\ensurett{#2}\}}\xspace}
%% Tuple version of closurs: {l: v1, ..., vN}.
\protected\long\def\clo[#1:#2]{\def\argA{#1}\def\argB{#2}\ensuremath{\{%%
      \ifx\argA\empty%%
      \else\lab #1/%%
        \ifx\argB\empty%%
        \else\ensurett{:\thinskip}%%
        \fi%%
      \fi\ensurett{#2}\}}\xspace}
%% Construct a thunk
\newbox\bracklbox \newbox\brackrbox
\setbox0=\hbox{$\{$} \setbox\bracklbox=\hbox to \wd0{\hfil[\kern0.25mm}
\setbox0=\hbox{$\}$} \setbox\brackrbox=\hbox to \wd0{\kern0.25mm]\hfil}
\protected\def\mkthunk[#1:#2]{\lab #1/%%
  \ensuremath{\,%%
    \mathopen{\copy\bracklbox}%%
    \ensurett{#2}%%
    \mathclose{\copy\brackrbox}\xspace}}
%% Binding statement: v <- {...}
\protected\def\binds#1<-#2;{\ensurett{\removespaces #1\texttt{<-}#2}\xspace}
%% In order to use \binds in verbatim environment, have to define
%% delimiters while they are active. The below defines \vbinds which
%% must be used in AVerb environments.  Notice the active space as
%% well - that is necessary so the space after \vbinds (and before the
%% first argument) in the verbatim environment gets eaten.
\begingroup\catcode`\!=\active \lccode`\!=`\< \lccode`\~=`\- 
  \catcode`\ =\active\lowercase{\endgroup\def\vbinds#1!~#2;}{\binds#1<-#2;}
%% Return statement: return ... ;
\protected\def\return#1;{\milres return/\ensurett{\ \removespaces #1}}
%% A closure capturing block. k {v1, ..., vN} x: ...
\protected\def\ccblock#1(#2)#3:{\lab#1/\ensuremath{\thinspace\{\ensurett{#2}\}}\ \ensurett{#3\hbox{:}}}
%% A normal block
\protected\def\block#1(#2):{\lab #1/\ensuremath{\thinspace(\ensurett{#2})}\ensurett{:}}
%% A goto expression
\protected\def\goto#1(#2){\lab #1/\thinspace\ensuremath{(\ensurett{#2})}}
%% An enter expression
\protected\def\app#1*#2/{\ensurett{\removespaces #1\ifmmode\ \fi{\text{\tt @}}\ifmmode\ \fi#2}}
\protected\def\bind{\texttt{<-}\xspace}
%% Primitive expression
\protected\def\prim#1(#2){\lab #1/\suptt*\ensuremath{(\ensurett{#2})}}
%% Program variable
\protected\def\var#1/{\ensurett{\removespaces #1}\xspace}
%% Case statement
\protected\def\case#1;{\milres case\ \ensuremath{\ensurett{\removespaces #1}}\ of/}
%% Case alternative
\protected\def\alt#1(#2)#3->#4;{\ensuremath{\ensurett{#1\ \ignorespaces#2\ \texttt{->}\ \ignorespaces #4}}}
%% Invoke
\protected\def\invoke#1/{\milres invoke/\ensurett{\ \removespaces #1}}
\def\rhs{right--hand side\xspace}
\def\lhs{left--hand side\xspace}
\def\enter{\texttt{@}\xspace}
\def\cc{closure--capturing\xspace}
\def\Cc{Closure--capturing\xspace}
%%

\newenvironment{myfig}[1][tbh]{\begin{figure}[#1]%%
\begin{singlespace}\centering%%
\figbegin}{\figend\end{singlespace}%%
\end{figure}}

%% Produce a sub-caption and label it.
\newcommand{\scap}[2][1in]{\begin{minipage}{#1}%%
\subcaption{}\label{#2}\end{minipage}}

%% Produce a sub-caption with text.
\newcommand{\lscap}[3][\hsize]{\begin{minipage}{#1}%%
\subcaption{#3}\label{#2}\end{minipage}}

% single-argument comment. Do not put
% a space before the command when used
% or the file will have two spaces.
\newcommand{\rem}[1]{}

%% A verbatim environment with active charactesr
%% so we can use math shortcuts and macros
\DefineVerbatimEnvironment{AVerb}{Verbatim}{commandchars=\\\{\},%% 
  codes={\catcode`\_8\catcode`\$3\catcode`\^7},%%
  numberblanklines=false}

\DefineVerbatimEnvironment{Verb}{Verbatim}{commandchars=\\\[\],%% 
  numberblanklines=false}

%% Turn on line numbers for Haskell code, 
%% and reset the line number counter.
\newcommand{\hsNumOn}{\numberson\numbersreset}
\newcommand{\hsNumOff}{\numbersoff}
%% Turn on line numbering in Haskell code within
%% the environment, then turn it off. The optional
%% argument specifies a prefix that \hslabel can
%% use to generate line number references. If no prefix
%% is givne, \hslabel will have no effect.
\newtoks\prefixtoks
\def\mkhslabel#1{\prefixtoks={#1}\let\prefix=a}
\def\hslabel#1{\ifx\prefix\relax%%
  \else\label{\the\prefixtoks_#1}%%
  \fi}
\def\unhslabel{\let\prefix=\relax}
\newenvironment{withHsNum}{\numberson\numbersreset}{\numbersoff}
\newenvironment{withHsLabeled}[1]{\numberson\numbersreset\mkhslabel{#1}}{\unhslabel\numbersoff}

%% Paragraph run-in
\newcommand{\runin}[1]{\begingroup\noindent\sffamily\textbf{#1}\qquad\endgroup}

%% Chapter bibliographies
\newcommand{\standaloneBib}{%%
  \ifthenelse{\boolean{standaloneFlag}}%%
             {\begin{singlespace}
                 \printbibliography
             \end{singlespace}}{}}

%% Adds an equation number on demand.
\newcommand\addtag{\refstepcounter{equation}\tag{\theequation}}

%% For typesetting set definitions like {x | x \in f(y)}
\newcommand\setdef[2]{\ensuremath{\{#1\ |\ #2\}}}

%% For typesetting function names like dom(f) or out(b).
\newcommand\mfun[1]{\ensuremath{\mathit{#1}}}

%% Marginal notes
\newcommand\margin[2]{\marginpar{\begin{singlespace}\emph{\footnotesize #2}\end{singlespace}}\relax #1}

%% Describe intent of a passage
\newcommand\intent[1]{{\begin{singlespace}\noindent\leftskip=-1in\emph{\footnotesize Intent: #1}\end{singlespace}}\nopagebreak[1]}

%% In aligned/alignedat/gathered environments, you don't et
%% automatice equation numbers. This command makes sure to
%% label them properly.
\newcommand\labeleq[1]{\refstepcounter{equation}\label{#1}}

%% Creates a hanging paragraph, where the first line is not
%% indented but all other lines are.
\def\itempar#1{\noindent\hangindent=\parindent\hangafter=1 #1\quad}

%% Disable overfull messages with ridiculous hfuzz value
\def\disableoverfull{\hfuzz=10in}

%% Set parfillskip so stretching does NOT occur at the end of
%% a paragraph (i.e., list of elements). Disable indent at beginning
%% of paragraph. Also turn off underfull hbox warnings.
%%
%% Intended to be used in a \vbox that forms part of a table or graphic,
%% which we want to be line-broken but not exactly like a normal paragraph.
\long\def\disableparspacing#1;{\def\arg{#1}\hbadness=100000\parindent=0pt\parfillskip=0pt\leftskip=0pt\rightskip=0pt%%
  \ifx\arg\empty\else\hsize=#1\relax\fi}
%% This stuff makes !+<text>+! write <text> in typewriter font.  

%% We make ! and + active characters early, then manipulate their
%% meaning to produce the right effect. Initially, + produces +. When
%% !  appears w/o a + following, it produces ``!''. When ``+''
%% follows, we start writing in teleteype (\ttfamily). The definition
%% of ``!'' changes to produce a bang. ``+'' changes such that it
%% looks for trailing ``!''. When no ``!'' appears, ``+'' produces ``+''. 
%% If a ``!'' appears, we shift out of \ttfamily (by ending the group) and
%% reset the meaning of ``!'' and ``+'' so we can start again.
\makeatletter
\let\mdplus=+\let\mdbang=!      %% Preserve meaning of + and ! so we can put them into document.
%% Turn off mark down for everyone
\outer\def\nomd{\catcode`!=12\catcode`+=12}
%% Turn mark down on for everyone
\outer\def\domd{\catcode`!=\active\catcode`+=\active %%
  \initialmd}
%% Use only with a group IMMEDIETALY following. Turns off
%% markdown for the group-to-come, without actually tokenizing the
%% group. If no group follows, this has no effect.
\protected\def\pausemd{\def\dopause{\catcode`!=12\catcode`+=12}%%
  \def\pausemdB{\ifx\next\bgroup%%
    %% A ``partial'' application of expandwith is used
    %% so we don't double up the group argument (which is what
    %% happens if we expand \next). This has the effect of 
    %% inserting \expandafter\dowith in front of the upcoming {. 
    %% If no brace is coming, \withmdC will have no effect.
    \def\pausemdC{\expandafter\dopause}
  \else
    \let\pausedmC=\relax
  \fi\pausemdC}
  %% \futurelet has to end the macro so it grabs the next token
  %% from the input file. Otherwise, it grabs it *from* this
  %% definition.
  \futurelet\next\pausemdB} %%
%% Turns markdown on for the group-to-come, without actually
%% tokenizing the group. Only has an effect when
%% used in front of a group, otherwise its a no-op.
\protected\def\withmd{\def\dowith{\catcode`!=\active\catcode`+=\active\initialmd}%%
  \def\withmdB{\ifx\next\bgroup %%
    %% A ``partial'' application of expandafter is used
    %% so we don't double up the group argument (which is what
    %% happens if we expand \next). This has the effect of 
    %% inserting \expandafter\dowith in front of the upcoming {. 
    %% If no brace is coming, \withmdC will have no effect.
      \def\withmdC{\expandafter\dowith} %%
    \else %%
      \let\withmdC=\relax %%
    \fi\withmdC}%%
  %% \futurelet has to end the macro so it grabs the next token
  %% from the input file. Otherwise, it grabs it *from* this
  %% definition.
  \futurelet\next\withmdB} %%
%% Make ! and + active in the following group so they have the right
%% catcode in the definitions to follow.
\catcode`!=\active\catcode`+=\active %%
%% Initial definitions associated with ! and +.
\def\initialmd{\protected\def!{\startTTA} %%
  \protected\def+{\stopTTA}} %%
%% Step 1 of startTT. Inital meaning of !; capture next token in \next, go to next step.
\def\startTTA{\futurelet\next\startTTB} %%
%% Step 2 of startTT. Compare captured token to + and go to step 3 if true. Otherwise
%% output a ! (since that started our macro), the argument captured and stop
%% processing.
\long\def\startTTB{\ifx\next+\expandafter\startTTC\expandafter\@gobble\else\mdbang\fi} %%
%% Step 3 of startTT. Shift into teletype mode and change definition of 
%% + and ! so we can stop processing.
\def\startTTC{\begingroup\ifmmode %%
  \let \math@bgroup \relax %%
  \def \math@egroup {\let \math@bgroup \@@math@bgroup %%
    \let \math@egroup \@@math@egroup} %%
  \mathtt\relax %%
  \else  %%
  \ttfamily\fi} %%
%% Step 1, 2  and 3 of stopTT follow the same pattern as startTT.
\def\stopTTA{\futurelet\next\stopTTB} %%
\long\def\stopTTB{\ifx\next!\expandafter\stopTTC\expandafter\@gobble\else\mdplus\fi} %%
\def\stopTTC{\endgroup}%%
\catcode`!=12\catcode`+=12
\makeatother

\domd

%% Place an input file on the next page
\def\onnextpage#1{\afterpage{\clearpage\input{#1}\clearpage}}

\begin{document}
\ifthenelse{\boolean{standaloneFlag}}
           {\VerbatimFootnotes
             \DefineShortVerb{\#}
             \doublespacing
             \setcounter{chapter}{0}}{}

%% Default float parameters. For case when
%% multiple chapters are included and
%% only one needs custom float settings.
\renewcommand{\textfraction}{0.2}
\renewcommand{\topfraction}{0.9}

\renewcommand{\textfraction}{0.1}
\renewcommand{\topfraction}{0.9}

\chapter{Dataflow Optimization}
\label{ref_chapter_background}

%% A short section giving the history of dataflow optimization techniques
%% and basic concepts.

The \emph{dataflow algorithm}, as described by Gary Kildall
\citep{Kildall1973}, provides a framework analyzing and optimizing
programs.  The algorithm \emph{iteratively} traverses the
\emph{control-flow graph} for a program either \emph{forwards} or
\emph{backwards}, computing \emph{facts} at each node using a
\emph{transfer function}, until the facts reach a \emph{fixed
  point}. The choice of facts, transfer function, and direction depend
on the particular analysis performed. In essence, the dataflow
\emph{algorithm} is parameterized by these choices; a dataflow
\emph{analysis} is a specific instance of the algorithm.

%% This chapter describes the concepts necessary to understand the
%% dataflow algorithm and gives an extended example demonstrating the use
%% of \emph{liveness analysis} to eliminate
%% \emph{dead-code}. Section~\ref{sec_back1} describes control-flow
%% graphs. We discuss facts, direction, the transfer function and the
%% meet operator in Section \ref{sec_back4}. We illustrate why dataflow
%% must be an iterative algorithm (and define what a fixed point means in
%% this context) in Section~\ref{sec_back6}. We treat rewriting in
%% Section~\ref{sec_back7}. To demonstrate these concepts together, we
%% give an extended example of \emph{dead-code elimination} in
%% Section~\ref{sec_back2}.

\section{Control-Flow Graphs}
\label{sec_back1}

%% Begin by placing the specific concept in the overall context of
%% dataflow. Give a small example highlighting the concept. Point
%% out fine points or subtleties that occur when generalizing the concept. End
%% by summarizing how the concept fits into dataflow (in a bit larger
%% sense than the first summary).

Figure~\ref{fig_back1} shows a simple C program and its
\emph{control-flow graph} (CFG). Each \emph{node} in
\subref{fig_back1_b} represents a statement or expression in the
original program. For example, \refNode{lst_back2_assigna} and
\refNode{lst_back2_assignb} represent the assignment statements on
line \ref{lst_back1_assign}. Notice that the declaration of #c# does
not appear in a corresponding node; because the declaration does not
cause a runtime effect, we do not represent it in the graph.  Nodes
\entryN and \exitN designate where program execution \emph{enters} and
\emph{leaves} the graph. If the graph represented the entire program,
we would say execution \emph{begins} at \entryN and \emph{terminates}
at \exitN. However, the CFG may be embedded in a larger program, for
which reason we say \emph{enters} and \emph{leaves}.

\begin{myfig}[th]
\begin{tabular}{cc}
\subfloat{\begin{minipage}[t]{2.5in}
\begin{AVerb}[numbers=left]
int a = 1, b = 2, c; \label{lst_back1_assign}
if(a > b) \label{lst_back1_test}
  c = 4; \label{lst_back1_test_true}
else     
  c = a + 3; \label{lst_back1_test_false}

printf(c); \label{lst_back1_print}
\end{AVerb}
\end{minipage}
%%
  \label{fig_back1_a}} \vline & 
\subfloat{\begin{minipage}[t]{2in}
\begin{Verbatim}
     E (1)
     |
     V
   -----(2)
  |a = 1|
  |b = 2|
   -----
     |
     V
 ---------(3)    -----(4)
|if(a > b)|---->|c = 4|
 ---------       -----
     |             |
     V             V
 ---------(5)  ---------(6)
|c = a + 3|-->|printf(c)|
 ---------     ---------
                   |
                   V
                   |
                   X (7)
\end{Verbatim}
\end{minipage}





%%
  \label{fig_back1_b}} \\
\subref{fig_back1_a} & \subref{fig_back1_b} 
\end{tabular}
\caption{\subref{fig_back1_a} A C-language program fragment. \subref{fig_back1_b} The
  \emph{control-flow graph} (CFG) for the program.}
\label{fig_back1}
\end{myfig}

Directed edges show the order in which nodes execute. The edges
leaving \refNode{lst_back2_test} (representing the test
``\verb=if(a > b)='' on line \ref{lst_back1_test}) show that
execution can branch to either \refNode{lst_back2_true} (when
$a > b$) or \refNode{lst_back2_false} (when
$a \leq b$). A node followed by multiple successors
(i.e., where multiple edges leave the node) represents a \emph{branch}
or \emph{conditional} statement. Any one of the successor nodes may
execute following the conditional statement.

In this particular example, it is obvious that
\refNode{lst_back2_false} will always execute after
\refNode{lst_back2_test}, because the test will always fail. However,
control-flow graphs show \emph{possible} execution paths. They do not
take into account the actual runtime values in a given graph. While in
this case it is easy to determine how the program will behave, in
general we cannot predict behavior without running the program. 

The dataflow algorithm approximates a program's runtime state by
analyzing the control flow of the program. Control-flow graphs show
the order in which expressions and statements in a program are
evaluated. It is the job of our \emph{dataflow analysis} to determine
how to make the program more efficient.

\section{Basic Blocks}
\label{sec_back3}

%% %% Begin by placing the specific concept in the overall context of
%% %% dataflow. Give a small example highlighting the concept. Point
%% %% out fine points or subtleties that occur when generalizing the concept. End
%% %% by summarizing how the concept fits into dataflow (in a bit larger
%% %% sense than the first summary).

%% Basic blocks
Consider the C-language fragment and control-flow graphs (CFG) in
Figure~\ref{fig_back4}.  Part~\subref{fig_back4_b} shows the CFG for
Part~\subref{fig_back4_a}: a long, straight sequence of nodes, one
after another. Part~\subref{fig_back4_c} represents the assignment statements on
lines~\ref{lst_back3_start} -- \ref{lst_back3_end} as a \emph{basic
  block}: a sequence of statements with one entry, one exit, and no
branches in-between. Execution cannot start in the ``middle'' of the
block, nor can it branch anywhere but at the end of the block. In fact,
Part~\ref{fig_back4_b} also shows four basic blocks -- they just happen
to consist of one statement each.

\begin{myfig}
\begin{tabular}{m{1.5in}m{1.5in}m{1.5in}}
  \begin{center}
    \subfloat{\begin{minipage}[t]{1.5in}
\begin{AVerb}[numbers=left]
int a = 1; \label{lst_back3_start}
int b = 2; 
int c = 3; 
int d = 4; \label{lst_back3_end}

if(a + d > b + c)
  \dots
else
  \dots
\end{AVerb}
\end{minipage}
\label{fig_back4_a}}
  \end{center} & %%
  \begin{center}
    \subfloat{\begin{tikzpicture}[>=stealth, node distance=.5in]
  \node[entex] (entry) {};

  \node[stmt, below of=entry] (assigna) {#a = 1#\labelNode{lst_back4_assigna}};
  \node[labelfor=assigna] () {\refNode{lst_back4_assigna}};

  \node[stmt, below of=assigna] (assignb) {#b = 2#\labelNode{lst_back4_assignb}};
  \node[labelfor=assignb] () {\refNode{lst_back4_assignb}};

  \node[stmt, below of=assignb] (assignc) {#c = 3#\labelNode{lst_back4_assignc}};
  \node[labelfor=assignc] () {\refNode{lst_back4_assignc}};

  \node[stmt, below of=assignc] (assignd) {#d = 4#\labelNode{lst_back4_assignd}};
  \node[labelfor=assignd] () {\refNode{lst_back4_assignd}};

  \node[entex, below of=assignd] (exit) {};

  \draw [->>] (entry) to (assigna);
  \draw [->] (assigna) to (assignb);
  \draw [->] (assignb) to (assignc);
  \draw [->] (assignc) to (assignd);
  \draw [->>] (assignd) to (exit);

\end{tikzpicture}
\label{fig_back4_b}}
  \end{center}
  & %%
  \begin{center}
    \subfloat{\begin{minipage}[t]{2in}
\begin{AVerb}
         E (1)
         |
         V
       -----(2)
      |a = 1|
      |b = 2|
      |c = 3|
      |d = 4|
       -----
         |
         V
 -----------------(3)
|if(a + d > b + c)|--|
 -----------------   V
         |          ---(4)
         V         |\dots|
     ---(5)         ---
    |\dots|--> X <---|
     ---
\end{AVerb}
\end{minipage}





\label{fig_back4_c}}
  \end{center} \\
  \vtop{\centering\subref{fig_back4_a}} & \vtop{\centering\subref{fig_back4_b}} & \vtop{\centering\subref{fig_back4_c}} \\
\end{tabular}
\caption{\subref{fig_back4_a}: A C-language fragment to illustrate
  \emph{basic blocks}.  \subref{fig_back4_b}: The CFG for
  \subref{fig_back4_a} without basic blocks. \subref{fig_back4_c}: The
  CFG for \subref{fig_back4_c} using basic blocks.}
\label{fig_back4}
\end{myfig}

The representation given in Part~\subref{fig_back4_c} has a number of
advantages. It tends to reduce both the number of nodes and the number
of edges in the graph. The dataflow algorithm maintains two sets of
\emph{facts} for every node -- reducing the number of nodes obviously
reduces the number of facts stored. The algorithm also iteratively
propagates facts along edges -- so reducing the number of edges
reduces the amount of work we need to do. When rewriting, blocks allow
us to move larger amounts of the program at once. It also can be shown
(see \citep{AhoXX}) that we do not lose any information by collapsing
statements into blocks. For efficiency and brevity, we will work with
basic blocks rather than statements from here forward.

\section{The Dataflow Algorithm}

Kildall's dataflow algorithm provides a general-purpose mechanism for
analyzing control-flow graphs of programs. The algorithm itself does
not mandate a specific analysis. Rather, it is parameterized by the
choice of \emph{facts}, \emph{lattice}, \emph{meet operator},
\emph{transfer function}, and \emph{direction}. Facts, the lattice,
and the meet operator are used to approximate some property of the
program that we wish to analyze. The transfer function creates facts
that mimic the flow of information in the control-flow graph. The
direction is dictacted by the type of analysis -- each particular
analysis runs \emph{forwards} or \emph{backwards}.

%% Constant-propagation
Consider Figure~\ref{fig_back7}, Part~\subref{fig_back7_initial}, which
shows a C function containing a loop that multiplies the argument by 10
some number of times. Line~\ref{fig_back7_m} declares #m# and assigns
it the value 10. The function uses #m# in the loop body on
Line~\ref{fig_back7_loop} to multiply the value passed in
repeatedly. 

\begin{myfig}[tbh]
  \begin{tabular}{cc}
    \subfloat{\begin{minipage}{3in}
\begin{AVerb}[gobble=2,numbers=left]
  int mult10(int cnt, int val) \{ 
    int m = 10, n = 1; \label{fig_back7_m}
    for(int i = 0; i < cnt; i++) \label{fig_back7_test}
      n = val * m; \label{fig_back7_loop}

    return n;
  \}
\end{AVerb}
\end{minipage}
\label{fig_back7_initial}} & %%
    \subfloat{\begin{minipage}{\hsize/2-0.1in}\raggedright\disableoverfull
\begin{AVerb}[gobble=2,numbers=left]
  int mult10(int cnt, int val) \{ 
    int m = 10, n = 0; 
    for(int i = 0; i < cnt; i++)
      n += val * 10; \label{fig_back7_opt_loop}

    return n;
  \}
\end{AVerb}
\end{minipage}
\label{fig_back7_opt}} \\

    \subref{fig_back7_initial} & \subref{fig_back7_opt} 
  \end{tabular}
  \caption{A C program which multiplies its argument, \texttt{val}, by
    10 \texttt{cnt} times. Part~\subref{fig_back7_initial} shows the
    original program. In Part~\subref{fig_back7_opt}, we have used
    \emph{constant propagation} to replace the use of \texttt{m} in
    the loop body with 10.}
  \label{fig_back7}
\end{myfig}

The program in Figure~\ref{fig_back7_initial} can be improved by
replacing the variable #m# with 10 in the loop body. We can use a
dataflow analysis known as \emph{constant propagation} to implement
this optimization. The constant propagation analysis recognizes when a
variable's value does not change in some context and then replaces
references to the variable with the constant
value. Figure~\ref{fig_back7}, Part~\subref{fig_back7_opt} shows the
optimized program, replacing #m# with 10 on
Line~\ref{fig_back7_opt_loop}.

\subsection{Facts and Lattices} 
\label{back_subsec_facts}

Constant propagation analyzes how each variable's value changes during
execution. The analysis \emph{approximates} the actual values of the
variables, as we cannot in general determine their exact value. We
will place the value of each variable into one of three categories at
each point in the control-flow graph: \emph{unknown}, a \emph{known
  constant}, or \emph{indeterminate}. \emph{Unknown}, represented by
$\bot$ (``bottom''), is the initial value for all variables in our
analysis. A \emph{known constant}, $C$, means our analysis identified
that the variable was assigned a specific value. \emph{Indeterminate},
indicated by $\top$ (``top''), means our analysis found that the
variable might have more than one value at a given point. The values
$\bot$, $C$, and $\top$ form a set which we will denote as \setLC.

\begin{myfig}[bth]
  \begin{tikzpicture}[>=stealth, node distance=.5in]
  \def\prefix{lst_back17_}
  \withmd{\pgfkeysifdefined{/tikz/incr}{}{\pgfkeys{/tikz/incr/.append style={}}}
\pgfkeysifdefined{/tikz/return}{}{\pgfkeys{/tikz/return/.append style={}}}
\pgfkeysifdefined{/tikz/assign}{}{\pgfkeys{/tikz/assign/.append style={}}}
\pgfkeysifdefined{/tikz/test}{}{\pgfkeys{/tikz/test/.append style={}}}
\pgfkeysifdefined{/tikz/mult}{}{\pgfkeys{/tikz/mult/.append style={}}}

  \node[invis] (entry) {};

  \node[stmt, assign, below=0.2in of entry] (assign) {\begin{minipage}{0.5in}
      \begin{AVerb} 
m = 10
n = 1
i = 0
      \end{AVerb}
    \end{minipage}
    \labelNode{\prefix assign}};
  \node[labelfor=assign] () {\refNode{\prefix assign}};

  \node[stmt, test, below=of assign] (test) {!+i < cnt+!\labelNode{\prefix test}};
  \node[labelfor=test] () {\refNode{\prefix test}};

  \node[stmt, mult, right=of test] (mult) {!+n += val * m+!\labelNode{\prefix mult}};
  \node[labelfor=mult] () {\refNode{\prefix mult}};

  \node[stmt, incr, below=of mult] (incr) {!+i+++!\labelNode{\prefix incr}};
  \node[labelfor=incr] () {\refNode{\prefix incr}};

  \node[stmt, return, below=of test] (return) {!+return n+!\labelNode{\prefix return}};
  \node[labelfor=return] () {\refNode{\prefix return}};

  \node[invis, below=0.2in of return] (exit) {};

  \draw [->>] (entry) to (assign);
  \draw [->] (assign) to (test);
  \draw [->] (test) to (mult);
  \draw [->] (mult) to (incr);
  \pausemd{\draw [->] (incr) -| ($(mult.east) + (5mm,0mm)$) |- ($(test.north)!.5!(assign.south)$) to (test.north);}
  \draw [->] (test) to (return);
  \draw [->>] (return) to (exit);
}

  \node[fact, below=5mm of assign, anchor=west] () {\outFactsM{lst_back17_assign}{i/0}{0.4in}};
  \node[fact, below=3mm of incr, anchor=west] () {\outFactsM{lst_back17_incr}{i/\top}{0.45in}};
  \node[fact, above=3mm of test, anchor=west] () {\inFactsM{lst_back17_test}{i/\top}{0.45in}};
\end{tikzpicture}

  \caption{Our program, annotated with facts partway through the
    analysis. Notice that \outB{lst_back17_assign} and
    \outB{lst_back17_incr} give differing values to $i$. We use a \emph{meet
      operator} when combining these two values while calculating
    \inB{lst_back17_test}.}
  \label{fig_back11}
\end{myfig}

Figure~\ref{fig_back11} shows the control-flow graph of our program
annotated with \emph{facts} before (\inE) and after (\out) select
program points. Each \emph{fact} indicates our knowledge of a
variable's value at that point in the program. For this analysis, our
facts are sets of pairs, $(a,x)$, where $a$ is the name of a variable
and $x$ a value in \setLC. 

Constant propagation is a \emph{forwards} analysis, meaning the values
for each \inE set are calculated based on the \out values of its
predecessors. Figure~\ref{fig_back11} shows the facts computed partway
through this analysis, concentrating on the nodes that reference $i$:
\refNode{lst_back17_assign}, \refNode{lst_back17_test} and
\refNode{lst_back17_incr}. \refNode{lst_back17_test} has two
predecessors: \refNode{lst_back17_assign} and
\refNode{lst_back17_incr}. Their \out sets,
\outB{lst_back17_assign} and \outB{lst_back17_incr}, give differing
values to $i$: 0 and $\top$. To combine these values when computing
\inB{lst_back17_test}, we use a \emph{meet operator}.

The \emph{meet operator}, \lub (pronounced ``least upper bound'' or
``lub''), defines how we will combine values in
\setLC. Table~\ref{tbl_back4} gives the definition of \lub. For any
value of $x \in \setLC$, $\bot \lub x$ results in $x$. Conversely, $x
\lub \top$ results in $\top$. Two differing constants, $C_1$ and
$C_2$, result in $\top$, while equal constants give the same constant. 

\begin{myfig}
  \begin{math}
    \begin{array}{cccc}
      v_1 & v_2 & v_1 \lub v_2 \\
      \cmidrule(r){1-1}\cmidrule(r){2-2}\cmidrule(r){3-3}
      \bot & x & x \\
      x & \top & \top & \\ 
      C_1 & C_2 & \top & \text{($C_1 \neq C_2$)} \\
      C_1 & C_2 & C_1 & \text{($C_1 = C_2$)}
    \end{array}
  \end{math}
  \caption{Definition of the \emph{meet operator}, lub, for the
    lattice used in our constant propagation analysis. $v_1$ and $v_2$
    are values in \setLC. The table shows how lub combines any two
    values.}
  \label{tbl_back4}
\end{myfig}

We have defined \lub on elements in \setLC, but our facts are set of
pairs $(a,x)$ where $a$ is a variable and $x$ a value in \setLC. To
compute \inBa, we apply \lub set-wise to the values for matching
variables in each \out set of $B$'s predecessors. We use the $\wedge$
(``wedge'') operator to represent our meet over sets and pairs:
\begin{align}\allowdisplaybreaks[0]
    \inBa &= \bigwedge\limits_{\mathclap{P \in \mathit{pred}(B)}} \outXa{P} \label{eqn_back8}\\ 
    B \bigwedge\ \,\mathclap{\emptyset}\phantom{C} &= B \notag\\*
    B \bigwedge\ \,\mathclap{C}\phantom{C} &= \bigcup\limits_{(a,x) \in B}
    \left(\bigcup\limits_{(b,y) \in C} (a,x) \wedge (b,y)\right) \label{eqn_back6}\\
  (a,x) \wedge (b,y) &= 
  \begin{cases}
    (a,x \lub y) & \text{when}\ (a,x) \in B, (b,y) \in C,\ \text{\lub as in Table~\ref{tbl_back4}.}\\
    (a,x) & \text{when}\ b \not\in \mathit{var}(B). \\
    (b,y) & \text{when}\ a \not\in \mathit{var}(C). \\
  \end{cases} \label{eqn_back7}
\end{align}


Using these equations, we can show how the \inB{lst_back17_test}
set in Figure~\ref{fig_back11} is derived:
\begin{flalign*}\allowdisplaybreaks[0]
    \inB{lst_back17_test} &= \bigwedge\limits_{\mathclap{P \in \mathit{pred}(\refNode{lst_back17_test})}} \outXa{P} 
    \intertext{Predecessors of \refNode{lst_back17_test}; Equation~\eqref{eqn_back8}.} 
    \inB{lst_back17_test} &= \outB{lst_back17_assign} \bigwedge \outB{lst_back17_incr} 
    \intertext{Equation~\eqref{eqn_back6}.}
    \inB{lst_back17_test} &= \bigcup\limits_{(a,x) \in \outB{lst_back17_assign}}
      \left(\bigcup\limits_{(b,y) \in \outB{lst_back17_incr}} (a,x) \wedge (b,y)\right) 
    \intertext{Definition of \outB{lst_back17_assign}, and \outB{lst_back17_incr} in Figure~\ref{fig_back11}.}
    \inB{lst_back17_test} &= \bigcup\limits_{(a,x) \in \{\factC{i}{0}\}}
      \left(\bigcup\limits_{(b,y) \in \{\factC{i}{\top}\}} (a,x) \wedge (b,y)\right) 
    \intertext{Equation~\eqref{eqn_back6}.}
    \inB{lst_back17_test} &= \{\factC{i}{0}\}\ \wedge \{\factC{i}{\top}\} 
    \intertext{Equation~\eqref{eqn_back7}.}
    \inB{lst_back17_test} &= \{\factC{i}{0 \lub \top}\}
    \intertext{Definition of \lub from Table~\ref{tbl_back4}.} 
    \inB{lst_back17_test} &= \{\factC{i}{\top}\} 
    \intertext{Definition of \inB{lst_back17_test}.}
    \{\factC{i}{\top}\} &= \{\factC{i}{\top}\} 
\end{flalign*}

Together, \setLC and \lub form a \emph{lattice}.\footnote{The lattice
  can also have a \emph{join} operator, but for our purposes we solely
  use the meet.}  The lattice precisely defines the facts computed in
our analysis. In this case, the lattice represents
knowledge about a variable's value. Each specific dataflow analysis
computes different facts, but those facts are always represented by a
lattice.

We have established that our analysis computes \emph{facts} at each
node in our programs control-flow graph. The facts are defined using a
\emph{lattice}. The lattice lets us assign the value $\bot$, $C$ (a
constant), or $\top$ to each variable in the program, at each node in
the graph. The lattice's \emph{meet operator} is used to combing
conflicting values when computing \inBa from the \out sets of $B$'s
predecessors. We next explore how \out facts are computed for each node.

%% Figure~\ref{fig_back11} demonstrates how the lattice computes facts
%% for constant propagation. The set \out{lst_back17_assign} indicates,
%% among other things, that $i$ is assigned 0: \factC{i}{0}. However,
%% \out{lst_back17_incr} indicates that the value of $i$ is indeterminate: 
%% \factC{i}{\top}. 

%% First, the values
%% computed for variables at each program point are in \setLC. Second,
%% the meet operator, \lub, is used to combine facts  The
%% lattice defines our facts. That is, the values in \setLC The lattice
%% defines how our facts will indicate if a variable has a constant
%% value.

%% Figure~\ref{fig_back10} shows our program with the final facts
%% computed. The sets \inB{lst_back13_mult} and \outB{lst_back13_mult}
%% show that #m# has the value 10 when block \refNode{lst_back13_mult}
%% executes. The variables #n# and #i# have the value $\top$, indicating
%% they could one of many different values. However, #cnt# and #val#
%% still have $\bot$, because their values are not modified anywhere in
%% the control-flow graph.

%% \begin{myfig}
%%   \begin{tikzpicture}[>=stealth, node distance=.75in]
  \def\prefix{lst_back14_}
  \withmd{\pgfkeysifdefined{/tikz/incr}{}{\pgfkeys{/tikz/incr/.append style={}}}
\pgfkeysifdefined{/tikz/return}{}{\pgfkeys{/tikz/return/.append style={}}}
\pgfkeysifdefined{/tikz/assign}{}{\pgfkeys{/tikz/assign/.append style={}}}
\pgfkeysifdefined{/tikz/test}{}{\pgfkeys{/tikz/test/.append style={}}}
\pgfkeysifdefined{/tikz/mult}{}{\pgfkeys{/tikz/mult/.append style={}}}

  \node[invis] (entry) {};

  \node[stmt, assign, below=0.2in of entry] (assign) {\begin{minipage}{0.5in}
      \begin{AVerb} 
m = 10
n = 1
i = 0
      \end{AVerb}
    \end{minipage}
    \labelNode{\prefix assign}};
  \node[labelfor=assign] () {\refNode{\prefix assign}};

  \node[stmt, test, below=of assign] (test) {!+i < cnt+!\labelNode{\prefix test}};
  \node[labelfor=test] () {\refNode{\prefix test}};

  \node[stmt, mult, right=of test] (mult) {!+n += val * m+!\labelNode{\prefix mult}};
  \node[labelfor=mult] () {\refNode{\prefix mult}};

  \node[stmt, incr, below=of mult] (incr) {!+i+++!\labelNode{\prefix incr}};
  \node[labelfor=incr] () {\refNode{\prefix incr}};

  \node[stmt, return, below=of test] (return) {!+return n+!\labelNode{\prefix return}};
  \node[labelfor=return] () {\refNode{\prefix return}};

  \node[invis, below=0.2in of return] (exit) {};

  \draw [->>] (entry) to (assign);
  \draw [->] (assign) to (test);
  \draw [->] (test) to (mult);
  \draw [->] (mult) to (incr);
  \pausemd{\draw [->] (incr) -| ($(mult.east) + (5mm,0mm)$) |- ($(test.north)!.5!(assign.south)$) to (test.north);}
  \draw [->] (test) to (return);
  \draw [->>] (return) to (exit);
}

  \node[fact, above=5mm of mult, anchor=west] () {\inFacts{lst_back14_mult}{m/{10},n/\top,i/\top,val/\top,cnt/\top}};
  \node[fact, below=3mm of mult, anchor=west] () {\outFacts{lst_back14_mult}{m/{10},n/\top,i/\top,val/\top,cnt/\top}};
\end{tikzpicture}

%%   \label{fig_back10}
%%   \caption{Our program, annotated with the final facts computed by the
%%     constant propagation analysis. Notice the \inB{lst_back14_mult}
%%     and \outB{lst_back14_mult} indicate that \texttt{m} has the value 10
%%     while \refNode{lst_back14_mult} executes.}
%% \end{myfig}

%% values. 
%% values $\bot$, an integer value, and $\top$ can be ordered such
%% that $\bot \le x \le \top$, for all integers $x$. The \emph{meet
%% operator} defines this ordering

%% Before
%% and after each block we will determine 

%% To track
%% the value of each variable, we use a \emph{lattice}. This particular
%% lattice encodes three types of values: 
%% Even so,
%% for this analysis all we care to know is if the value remains
%% the same or changes. 

%% approximate by determining, at each point in the control-flow graph,
%% whether a variable has one of three values: an \emph{unknown}, a
%% \emph{constant}, or

\subsection{Transfer Functions}
\label{back_subsec_transfer}

The dataflow algorithm calculates new facts using a \emph{transfer
  function}. The transfer function is specific to both the analysis
performed and the semantics of the programming language used to write
the program analyzed. Theoretically, each node in the graph can
have its own transfer function, but in practice the function is 
defined by cases for each statement or expression in the language. 

In a \emph{forwards} analysis, the transfer function computes the \out
set for a given node. In a backwards analysis, the transfer function
computes the \inE set. That is, a forwards analysis computes facts
that hold \emph{after} a node executes. A backwards analysis computes
facts that were true \emph{before} a node executed.  In both cases,
the transfer function also considers known facts (i.e., \inE facts for
forwards, \out for backwards) as well as the statements in the node.

Our constant propagation analysis determines, for each node in the
control-flow graph, what value a given variable has: $\bot$, $C$ (an
integer), or $\top$. We choose to use pairs $(a,x)$, where $a$ is
variable in the program and $x$ a value in \setLC, as the facts for
our analysis. We define our transfer function in terms of a single 
pair (i.e., fact):
\begin{equation}
  f (a,x) = 
  \begin{cases}
    (a,C) & \text{when \texttt{a = \emph{C}}, where \texttt{\emph{C}} is an integer.}. \\
    (a,\top) & \text{when \texttt{a} updated}. \\
    (a,x) & \text{otherwise}. \\
  \end{cases}
  \label{eqn_back4}
\end{equation}

Equation~\eqref{eqn_back4} considers two kinds of statements: constant
and non-constant updates. If the statement #a = C# appears in the
node, our new fact is $(a,C)$, meaning #a# was assigned a known
constant. We create the fact $(a,\top)$ for any other
update.\footnote{In practice this analysis is much more sophisticated,
  capable of analyzing complicated arithmetic expressions. Algebraic
  properties such as associativity and distributivity are also usually
  considered.} Finally, if neither type of statement appears in the
node, we just pass the initial fact through unchanged.

Though we have defined Equation~\eqref{eqn_back4} in terms of a single
fact, we can easily extend it to sets of facts. To create \outBa for
some node $B$, we apply Equation~\eqref{eqn_back4} element-wise to
each fact in \inBa and combine the results into a set:
\begin{equation}
  \outBa = \bigcup\limits_{\mathclap{(a,x) \in \inBa}} f(a,x)
  \label{eqn_back5}
\end{equation}

Figure~\ref{fig_back9}, Part \subref{fig_back9_initial}, shows our
program, annotated with initial \inE and \out
facts. Figure~\ref{fig_back9}, Part \subref{fig_back9_transfer}, shows
the same graph with annotations updated using
Equation~\eqref{eqn_back4}. The assignments in
\refNode{lst_back18_assign} create the facts \factC{m}{10},
\factC{n}{1}, and \factC{i}{0} in \outB{lst_back18_assign}. The
multiplication in \refNode{lst_back18_mult} is a non-constant update,
so \outB{lst_back18_mult} contains \factC{n}{\top}. However,
\outB{lst_back18_mult} also shows that #m# is not modified in
\refNode{lst_back18_mult}; the value from \inB{lst_back18_mult} is
just copied to \outB{lst_back18_mult}.

\afterpage{\clearpage\begin{myfig}[p]
  \begin{tabular}{c}
    \begin{minipage}{\hsize}\raggedright
      \begin{tikzpicture}[>=stealth, node distance=.75in and 2in]
  \def\prefix{lst_back15_}
  \newbox\fboxA
  \begin{pgfinterruptpicture}
    \global\setbox\fboxA=\hbox{\facts{m/\bot,n/\bot,i/\bot}}
  \end{pgfinterruptpicture}

  \withmd{\pgfkeysifdefined{/tikz/incr}{}{\pgfkeys{/tikz/incr/.append style={}}}
\pgfkeysifdefined{/tikz/return}{}{\pgfkeys{/tikz/return/.append style={}}}
\pgfkeysifdefined{/tikz/assign}{}{\pgfkeys{/tikz/assign/.append style={}}}
\pgfkeysifdefined{/tikz/test}{}{\pgfkeys{/tikz/test/.append style={}}}
\pgfkeysifdefined{/tikz/mult}{}{\pgfkeys{/tikz/mult/.append style={}}}

  \node[invis] (entry) {};

  \node[stmt, assign, below=0.2in of entry] (assign) {\begin{minipage}{0.5in}
      \begin{AVerb} 
m = 10
n = 1
i = 0
      \end{AVerb}
    \end{minipage}
    \labelNode{\prefix assign}};
  \node[labelfor=assign] () {\refNode{\prefix assign}};

  \node[stmt, test, below=of assign] (test) {!+i < cnt+!\labelNode{\prefix test}};
  \node[labelfor=test] () {\refNode{\prefix test}};

  \node[stmt, mult, right=of test] (mult) {!+n += val * m+!\labelNode{\prefix mult}};
  \node[labelfor=mult] () {\refNode{\prefix mult}};

  \node[stmt, incr, below=of mult] (incr) {!+i+++!\labelNode{\prefix incr}};
  \node[labelfor=incr] () {\refNode{\prefix incr}};

  \node[stmt, return, below=of test] (return) {!+return n+!\labelNode{\prefix return}};
  \node[labelfor=return] () {\refNode{\prefix return}};

  \node[invis, below=0.2in of return] (exit) {};

  \draw [->>] (entry) to (assign);
  \draw [->] (assign) to (test);
  \draw [->] (test) to (mult);
  \draw [->] (mult) to (incr);
  \pausemd{\draw [->] (incr) -| ($(mult.east) + (5mm,0mm)$) |- ($(test.north)!.5!(assign.south)$) to (test.north);}
  \draw [->] (test) to (return);
  \draw [->>] (return) to (exit);
}


  \node[fact, above=5mm of assign, anchor=west] () 
       {\inFactsM{\prefix assign}{m/\bot,n/\bot,i/\bot}{\wd\fboxA}};

  \node[fact, below=3mm of assign, anchor=west] () 
       {\outFactsM{\prefix assign}{m/\bot,n/\bot,i/\bot}{\wd\fboxA}};

  \node[fact, above=5mm of test, anchor=west] () 
       {\inFactsM{\prefix  test}{m/\bot,n/\bot,i/\bot}{\wd\fboxA}};

  \node[fact, below=3mm of test, anchor=west] () 
       {\outFactsM{\prefix test}{m/\bot,n/\bot,i/\bot}{\wd\fboxA}};

  \node[fact, above=5mm of mult, anchor=west] () 
       {\inFactsM{\prefix mult}{m/\bot,n/\bot,i/\bot}{\wd\fboxA}};

  \node[fact, below=3mm of mult, anchor=west] () 
       {\outFactsM{\prefix mult}{m/\bot,n/\bot,i/\bot}{\wd\fboxA}};

  \node[fact, above=5mm of incr, anchor=west] () 
       {\inFactsM{\prefix incr}{m/\bot,n/\bot,i/\bot}{\wd\fboxA}};

  \node[fact, below=3mm of incr, anchor=west] () 
       {\outFactsM{\prefix incr}{m/\bot,n/\bot,i/\bot}{\wd\fboxA}};

  \node[fact, above=5mm of return, anchor=west] () 
       {\inFactsM{\prefix return}{m/\bot,n/\bot,i/\bot}{\wd\fboxA}};

  \node[fact, below=3mm of return, anchor=west] () 
       {\outFactsM{\prefix return}{m/\bot,n/\bot,i/\bot}{\wd\fboxA}};

\end{tikzpicture}

    \end{minipage} \\
    \scap{fig_back9_initial} \\
    \begin{minipage}{\hsize}\raggedright
      \setcounter{nodeCounter}{0}\begin{tikzpicture}[>=stealth, node distance=.75in]
  \def\prefix{lst_back18_}
  \withmd{\pgfkeysifdefined{/tikz/incr}{}{\pgfkeys{/tikz/incr/.append style={}}}
\pgfkeysifdefined{/tikz/return}{}{\pgfkeys{/tikz/return/.append style={}}}
\pgfkeysifdefined{/tikz/assign}{}{\pgfkeys{/tikz/assign/.append style={}}}
\pgfkeysifdefined{/tikz/test}{}{\pgfkeys{/tikz/test/.append style={}}}
\pgfkeysifdefined{/tikz/mult}{}{\pgfkeys{/tikz/mult/.append style={}}}

  \node[invis] (entry) {};

  \node[stmt, assign, below=0.2in of entry] (assign) {\begin{minipage}{0.5in}
      \begin{AVerb} 
m = 10
n = 1
i = 0
      \end{AVerb}
    \end{minipage}
    \labelNode{\prefix assign}};
  \node[labelfor=assign] () {\refNode{\prefix assign}};

  \node[stmt, test, below=of assign] (test) {!+i < cnt+!\labelNode{\prefix test}};
  \node[labelfor=test] () {\refNode{\prefix test}};

  \node[stmt, mult, right=of test] (mult) {!+n += val * m+!\labelNode{\prefix mult}};
  \node[labelfor=mult] () {\refNode{\prefix mult}};

  \node[stmt, incr, below=of mult] (incr) {!+i+++!\labelNode{\prefix incr}};
  \node[labelfor=incr] () {\refNode{\prefix incr}};

  \node[stmt, return, below=of test] (return) {!+return n+!\labelNode{\prefix return}};
  \node[labelfor=return] () {\refNode{\prefix return}};

  \node[invis, below=0.2in of return] (exit) {};

  \draw [->>] (entry) to (assign);
  \draw [->] (assign) to (test);
  \draw [->] (test) to (mult);
  \draw [->] (mult) to (incr);
  \pausemd{\draw [->] (incr) -| ($(mult.east) + (5mm,0mm)$) |- ($(test.north)!.5!(assign.south)$) to (test.north);}
  \draw [->] (test) to (return);
  \draw [->>] (return) to (exit);
}

  
  \setfacts{m/{10},n/1,i/0}
  \node[fact, below right=5mm and 5mm of assign, anchor=west] ()
       {\outFactsM{lst_back18_assign}{m/{10},n/1,i/0}{\wd\factbox}};

  \setfacts{m/\bot,n/\bot,i/\bot}
  \node[fact, above right=5mm of mult, anchor=west] () 
       {\inFactsM{lst_back18_mult}{m/\bot,n/\bot,i/\bot}{\wd\factbox}};

  \setfacts{m/{10},n/\top,i/0}
  \node[fact, below left=3mm of mult, anchor=west] () 
       {\outFactsM{lst_back18_mult}{m/{10},n/\top,i/0}{\wd\factbox}};
\end{tikzpicture}

    \end{minipage} \\
    \scap{fig_back9_transfer} \\
  \end{tabular}
  \caption{Part~\subref{fig_back9_initial} shows our program annotated
    with initial facts. In Part~\subref{fig_back9_transfer}, we have
    updated each \outBa set using Equation~\eqref{eqn_back4}, our
    transfer function.}
  \label{fig_back9}
\end{myfig}
\clearpage}

Every dataflow analysis defines a \emph{transfer function} for
creating (or updating) facts. The function is specific to both the
analysis performed and the semantics of the underlying
program. Equation~\ref{eqn_back4}, the transfer function for our
constant propagation example, defines how we derive information about
a variable's value after the statements in the given node execute.
Equation~\ref{eqn_back5} extended Equation~\ref{eqn_back4} to sets,
showing how we can create \outBa from \inBa. In the next section, we
iteratively apply our transfer function and lattice to the
control-flow graph. We will show both how our analysis is guaranteed
to terminate and how we can be confident it has given us the best
possible answer.

\subsection{Iteration \& Fixed Points}
\label{back_subsec_iter}

Figure~\ref{fig_back9} hints that facts
develop over time during analysis. In fact, the transfer function is
applied to each node in turn, creating new facts from old, until the
facts stop changing. In other words, the control-flow graph is
analyzed \emph{iteratively} until all \out (in the case of a forwards
analysis) or \inE (in the case of a backwards analysis) sets reach a
\emph{fixed point}.

Figure~\ref{fig_back11}, Part~\subref{fig_back11_tbl} shows how the
\inE and \out sets associated with each node change with each
iteration of the analysis. The ``zero'' iteration corresponds to the
initial value for all facts: everything is $\bot$, i.e.,
unknown. Reading from left-to-right shows how the \inE and \out facts
for a given node change during analysis. The control-flow graph is
reproduced in Part~\subref{fig_back11_cfg}. Following the control-flow
between nodes shows which \out sets are used to calcuate \inE sets.

\begin{myfig}
  \setlength{\tabcolsep}{2pt}
  \hbox to \textwidth{\hss
  \begin{tabular}{cc}
    \subfloat{\begin{minipage}{5in}
\begin{math}
  \setlength{\arraycolsep}{2pt}
  \begin{array}{l@{\phantom{X}}ccc@{\phantom{X}}ccc@{\phantom{X}}ccc@{\phantom{X}}ccc@{\phantom{X}}ccc@{\phantom{X}}ccc}
    \rlap{\hfil\phantom{X}\textit{Iteration:}} &
    & 0 & & 
    & 1 & & 
    & 2 & & 
    & 3 & &
    & 4 & &
    & 5 & \\ 

    & %%
    m & n & i & %% 0
    m & n & i & %% 1
    m & n & i & %% 2
    m & n & i & %% 3
    m & n & i & %% 4
    m & n & i \\ %% 5

    \inB{lst_back15_assign} & %%
    \bot & \bot & \bot & %% 0 
    \bot & \bot & \bot & %% 1 
    \bot & \bot & \bot & %% 2
    \bot & \bot & \bot & %% 3
    \bot & \bot & \bot & %% 4
    \bot & \bot & \bot \\ %% 5
    \outB{lst_back15_assign} & %%
    \bot & \bot & \bot & %% 0 
    10 & 1 & 0 & %% 1 
    10 & 1 & 0 & %% 2
    10 & 1 & 0 & %% 3
    10 & 1 & 0 & %% 4
    10 & 1 & 0 \\\\ %% 5

    \inB{lst_back15_test} & %%
    \bot & \bot & \bot & %% 0 
    \bot & \bot & \bot & %% 1 
    10 & 1 & \top & %% 2
    10 & \top & \top & %% 3
    10 & \top & \top & %% 4
    10 & \top & \top\\ %% 5
    \outB{lst_back15_test} & %%
    \bot & \bot & \bot & %% 0 
    \bot & \bot & \bot & %% 1 
    10 & 1 & \top & %% 2
    10 & \top & \top & %% 3
    10 & \top & \top & %% 4
    10 & \top & \top \\\\ %% 5

    \inB{lst_back15_mult} & %%
    \bot & \bot & \bot & %% 0 
    \bot & \bot & \bot & %% 1 
    \bot & \bot & \bot & %% 2
    10 & 1 & \top & %% 3
    10 & \top & \top & %% 4
    10 & \top & \top \\ %%5  
    \outB{lst_back15_mult} & %%
    \bot & \bot & \bot & %% 0 
    \bot & \top & \bot & %% 1 
    \bot & \top & \bot & %% 2
    10 & \top & \top & %% 3
    10 & \top & \top & %% 4
    10 & \top & \top \\\\ %% 5

    \inB{lst_back15_incr} & %%
    \bot & \bot & \bot & %% 0 
    \bot & \bot & \bot & %% 1 
    \bot & \top & \bot & %% 2
    \bot & \top & \bot & %% 3
    10 & \top & \top & %% 4
    10 & \top & \top \\ %% 5
    \outB{lst_back15_incr} & %%
    \bot & \bot & \bot & %% 0 
    \bot & \bot & \top & %% 1 
    \bot & \top & \top & %% 2
    \bot & \top & \top & %% 3
    10 & \top & \top & %% 4
    10 & \top & \top \\\\ %% 5

    \inB{lst_back15_return} & %%
    \bot & \bot & \bot & %% 0 
    \bot & \bot & \bot & %% 1 
    \bot & \bot & \bot & %% 2
    10 & 1 & \top & %% 3
    10 & \top & \top & %% 4
    10 & \top & \top \\ %% 5
    \outB{lst_back15_return} & %%
    \bot & \bot & \bot & %% 0 
    \bot & \bot & \bot & %% 1 
    \bot & \bot & \bot & %% 2
    10 & 1 & \top & %% 3
    10 & \top & \top & %% 4
    10 & \top & \top \\\\ %% 5

  \end{array}
\end{math}
\end{minipage}
\label{fig_back11_tbl}} & 
    \subfloat{\begin{tikzpicture}[baseline=(mult.center)]
  \node[stmt] (assign) {\begin{minipage}{0.6in} %%
      \tt m = 10 \\n = 1 \\i = 0
    \end{minipage}\labelNode{fig_back11_assign}}; 
  \node[labelfor=assign] () {\refNode{fig_back11_assign}};

  \node[stmt, below=0.25in of assign] (test) {\texttt{i < cnt}\labelNode{fig_back11_test}}; 
  \node[labelfor=test] () {\refNode{fig_back11_test}};

  \node[stmt, below of=test] (mult) {\texttt{n = val * m}\labelNode{fig_back11_mult}};
  \node[labelfor=mult] (mult_label) {\refNode{fig_back11_mult}};

  \node[stmt, below of=mult] (incr) {\texttt{i++}\labelNode{fig_back11_incr}};
  \node[labelfor=incr] () {\refNode{fig_back11_incr}};
  
  \node[stmt, below of=incr] (return) {\texttt{return n}\labelNode{fig_back11_return}}; 
  \node[labelfor=return] () {\refNode{fig_back11_return}};
  
  \draw[->] (assign.south) -| (test.north);
  \draw[->] (test.south) -| (mult.north);
  \draw[->] (mult.south) -| (incr.north);
  \draw[->] (incr.west) -| ($(mult.west) - (1mm,0)$) |- (test.west);
  \draw[->] (test.east) -| ($(mult_label.east) + (1mm,0)$) |- (return.east);
\end{tikzpicture}
\label{fig_back11_cfg}} \\
    \subref{fig_back11_tbl} & \subref{fig_back11_cfg}
  \end{tabular}\hss}
  \caption{This figure shows the facts calculated for all nodes in our
    example program. Part~\subref{fig_back11_tbl} shows the \inE and
    \out facts associated with each node. Part~\subref{fig_back11_cfg}
    reproduces the control-flow graph for our program. After 5
    iterations the facts reach a fixed point (i.e., they stop
    changing) and we can see that \inB{lst_back15_mult} shows that $m$
    is always 10, proving we can rewrite the multiplication safely. }
  \label{fig_back11}
\end{myfig}

Consider the value for $i$ in \inB{lst_back15_test}, the node that
tests the condition #i < cnt#. In the first iteration,
\inB{lst_back15_test} still assigns $\bot$ to
$i$. Equation~\eqref{eqn_back8} states that \inB{lst_back15_test} is
derived from the \out sets of \refNode{lst_back15_test}'s
predecessors: \refNode{lst_back15_assign} and
\refNode{lst_back15_incr}. By Equations~\eqref{eqn_back8},
\eqref{eqn_back6}, and \eqref{eqn_back7} we can calculate the value of
$i$ in \inB{lst_back15_test}. Crucially, the \out set used comes from
the \emph{previous} iteration of the analysis, which we emphasize by
attaching the iteration number to each set:
\begin{align*}
  \inB{lst_back15_test}^1 &= \outB{lst_back15_assign}^0 \bigwedge \outB{lst_back15_incr}^0 \\
  &= {\factC{i}{\bot}} \wedge {\factC{i}{\bot}} \\
  &= {\factC{i}{\bot \lub \bot}} \\
  {\factC{i}{\bot}} &= {\factC{i}{\bot}}
\end{align*}
Now consider the second iteration, where \inB{lst_back15_test} assigns
$\top$ to $i$. \outB{lst_back15_assign} gives $i$ the value
0 (by #i = 0#). However, \outB{lst_back15_incr} assigns $i$ the value $\top$,
because #i++# is a non-constant update. We can see why by
Equations~\eqref{eqn_back8}, \eqref{eqn_back6}, and
\eqref{eqn_back7}. Again we attach the iteration number to each set,
emphasizing its origin:
\begin{align*}
  \inB{lst_back15_test}^2 &= \outB{lst_back15_assign}^1 \bigwedge \outB{lst_back15_incr}^1 \\
  &= {\factC{i}{0}} \wedge {\factC{i}{\top}} \\
  &= {\factC{i}{0 \lub \top}} \\
  {\factC{i}{\top}} &= {\factC{i}{\top}}
\end{align*}
Notice how the conflicting values for $i$ are resolved with the \lub
operator. By the third iteration, \inB{lst_back15_test} and
\outB{lst_back15_test} reach a fixed point,
${\facts{m/{10},n/\top,i/\top}\}$.

Figure~\ref{fig_back11} how we develop facts for this particular
analysis, for a particular program. In general, we would like our 
dataflow analysis to have two properties:
\begin{enumerate}
\item The analysis will terminate.
\item The answer given is the best possible.
\end{enumerate}


%% Monotonic
%% Finite height lattice
%% Partial orders
%% Maximum fixed point

%% \begin{myfig}
%%   \begin{tabular}[c]
%%     \subfloat{\begin{tikzpicture}[>=stealth, node distance=.75in and 2in]
  \def\prefix{lst_back15_}
  \newbox\fboxA
  \begin{pgfinterruptpicture}
    \global\setbox\fboxA=\hbox{\facts{m/\bot,n/\bot,i/\bot}}
  \end{pgfinterruptpicture}

  \withmd{\pgfkeysifdefined{/tikz/incr}{}{\pgfkeys{/tikz/incr/.append style={}}}
\pgfkeysifdefined{/tikz/return}{}{\pgfkeys{/tikz/return/.append style={}}}
\pgfkeysifdefined{/tikz/assign}{}{\pgfkeys{/tikz/assign/.append style={}}}
\pgfkeysifdefined{/tikz/test}{}{\pgfkeys{/tikz/test/.append style={}}}
\pgfkeysifdefined{/tikz/mult}{}{\pgfkeys{/tikz/mult/.append style={}}}

  \node[invis] (entry) {};

  \node[stmt, assign, below=0.2in of entry] (assign) {\begin{minipage}{0.5in}
      \begin{AVerb} 
m = 10
n = 1
i = 0
      \end{AVerb}
    \end{minipage}
    \labelNode{\prefix assign}};
  \node[labelfor=assign] () {\refNode{\prefix assign}};

  \node[stmt, test, below=of assign] (test) {!+i < cnt+!\labelNode{\prefix test}};
  \node[labelfor=test] () {\refNode{\prefix test}};

  \node[stmt, mult, right=of test] (mult) {!+n += val * m+!\labelNode{\prefix mult}};
  \node[labelfor=mult] () {\refNode{\prefix mult}};

  \node[stmt, incr, below=of mult] (incr) {!+i+++!\labelNode{\prefix incr}};
  \node[labelfor=incr] () {\refNode{\prefix incr}};

  \node[stmt, return, below=of test] (return) {!+return n+!\labelNode{\prefix return}};
  \node[labelfor=return] () {\refNode{\prefix return}};

  \node[invis, below=0.2in of return] (exit) {};

  \draw [->>] (entry) to (assign);
  \draw [->] (assign) to (test);
  \draw [->] (test) to (mult);
  \draw [->] (mult) to (incr);
  \pausemd{\draw [->] (incr) -| ($(mult.east) + (5mm,0mm)$) |- ($(test.north)!.5!(assign.south)$) to (test.north);}
  \draw [->] (test) to (return);
  \draw [->>] (return) to (exit);
}


  \node[fact, above=5mm of assign, anchor=west] () 
       {\inFactsM{\prefix assign}{m/\bot,n/\bot,i/\bot}{\wd\fboxA}};

  \node[fact, below=3mm of assign, anchor=west] () 
       {\outFactsM{\prefix assign}{m/\bot,n/\bot,i/\bot}{\wd\fboxA}};

  \node[fact, above=5mm of test, anchor=west] () 
       {\inFactsM{\prefix  test}{m/\bot,n/\bot,i/\bot}{\wd\fboxA}};

  \node[fact, below=3mm of test, anchor=west] () 
       {\outFactsM{\prefix test}{m/\bot,n/\bot,i/\bot}{\wd\fboxA}};

  \node[fact, above=5mm of mult, anchor=west] () 
       {\inFactsM{\prefix mult}{m/\bot,n/\bot,i/\bot}{\wd\fboxA}};

  \node[fact, below=3mm of mult, anchor=west] () 
       {\outFactsM{\prefix mult}{m/\bot,n/\bot,i/\bot}{\wd\fboxA}};

  \node[fact, above=5mm of incr, anchor=west] () 
       {\inFactsM{\prefix incr}{m/\bot,n/\bot,i/\bot}{\wd\fboxA}};

  \node[fact, below=3mm of incr, anchor=west] () 
       {\outFactsM{\prefix incr}{m/\bot,n/\bot,i/\bot}{\wd\fboxA}};

  \node[fact, above=5mm of return, anchor=west] () 
       {\inFactsM{\prefix return}{m/\bot,n/\bot,i/\bot}{\wd\fboxA}};

  \node[fact, below=3mm of return, anchor=west] () 
       {\outFactsM{\prefix return}{m/\bot,n/\bot,i/\bot}{\wd\fboxA}};

\end{tikzpicture}
\label{fig_back8_initial}} \\
%%     \subref{fig_back8_initial} \\
%%     \subfloat{\begin{tikzpicture}[>=stealth, node distance=.75in]
  \def\prefix{lst_back13}
  \withmd{\pgfkeysifdefined{/tikz/incr}{}{\pgfkeys{/tikz/incr/.append style={}}}
\pgfkeysifdefined{/tikz/return}{}{\pgfkeys{/tikz/return/.append style={}}}
\pgfkeysifdefined{/tikz/assign}{}{\pgfkeys{/tikz/assign/.append style={}}}
\pgfkeysifdefined{/tikz/test}{}{\pgfkeys{/tikz/test/.append style={}}}
\pgfkeysifdefined{/tikz/mult}{}{\pgfkeys{/tikz/mult/.append style={}}}

  \node[invis] (entry) {};

  \node[stmt, assign, below=0.2in of entry] (assign) {\begin{minipage}{0.5in}
      \begin{AVerb} 
m = 10
n = 1
i = 0
      \end{AVerb}
    \end{minipage}
    \labelNode{\prefix assign}};
  \node[labelfor=assign] () {\refNode{\prefix assign}};

  \node[stmt, test, below=of assign] (test) {!+i < cnt+!\labelNode{\prefix test}};
  \node[labelfor=test] () {\refNode{\prefix test}};

  \node[stmt, mult, right=of test] (mult) {!+n += val * m+!\labelNode{\prefix mult}};
  \node[labelfor=mult] () {\refNode{\prefix mult}};

  \node[stmt, incr, below=of mult] (incr) {!+i+++!\labelNode{\prefix incr}};
  \node[labelfor=incr] () {\refNode{\prefix incr}};

  \node[stmt, return, below=of test] (return) {!+return n+!\labelNode{\prefix return}};
  \node[labelfor=return] () {\refNode{\prefix return}};

  \node[invis, below=0.2in of return] (exit) {};

  \draw [->>] (entry) to (assign);
  \draw [->] (assign) to (test);
  \draw [->] (test) to (mult);
  \draw [->] (mult) to (incr);
  \pausemd{\draw [->] (incr) -| ($(mult.east) + (5mm,0mm)$) |- ($(test.north)!.5!(assign.south)$) to (test.north);}
  \draw [->] (test) to (return);
  \draw [->>] (return) to (exit);
}

\end{tikzpicture}
\label{fig_back8_final}} \\
%%     \subref{fig_back8_final} 
%%   \end{tabular}
%%   \caption{The control flow graph for our program. Part~\subref{fig_back8_initial}
%%     shows the initial facts associated with each node. Part~\subref{fig_back8_final}
%%     shows the final facts computed by our constant propagation analysis.}
%%   \label{fig_back8}
%% \end{myfig}


\subsection{Dataflow Equations}
\label{back_subsec_eq}

%% We can succinctly describe the constant propagation analysis using
%% \emph{dataflow equations}. These equations express our lattice and the
%% transfer function. From the equations we can derive the direction of
%% our analysis. Starting from initial values for each \inE and \out
%% set, we iteratively compute facts until reaching a fixed point. 

%% \runin{Facts} Our facts are a set; each element is a pair ($a$, $x$),
%% with $a \in \setL{Var}$ and $x \in \setLC$. \setL{Var} is the set of
%% all variables used in the controlf-flow graph (e.g., $m$, $cnt$, etc.)
%% \setLC contains all the integers, plus a bottom element ($\bot$) and a
%% top element ($\top$). 

%% We define the functions \inE and \out such that \inBa gives all the
%% facts prior to block $B$ executing. That is, \inBa takes a block, $B$,
%% and returns a set, \facts{a/x,b/y,\dots}, associating each variable in
%% the control-flow graph with the analysis' calculated value. Similarly,
%% \outBa gives all the facts after block $B$ executes. We can also say
%% \inBav{a} (as well as \outBav{a}) to mean the value associated with
%% variable $a$ in the set returned by \inBa (or \outBa, of course). When
%% discussing specific blocks, we will include the block number:
%% \inB{lst_back15_mult}, \inBv{lst_back15_mul}{m},
%% \outB{lst_back15_mult}, etc.

%% \runin{Direction} We define \inBa for block $B$ in terms of the \out
%% sets for each predecessor of $B$. The \inBa set will be the union of
%% all \out sets for each predecessor. If the same variable occurs in two
%% or more sets, then that variables' value will be calculated using the
%% \emph{meet}, \lub. Otherwise, the variable's value is copied from the
%% \out set to the \inBa set.

\afterpage{\clearpage\begin{math}
  \begin{array}{rlr}

    \multicolumn{3}{c}{\emph{Facts}} \\

    \setLC &= \{\bot, \top\} \cup \ZZ.\\
    \setL{Var} &= \text{Set of all variables.} \\
    \setL{Fact} &= \setL{Var} \times \setLC. \\\\

    \multicolumn{3}{c}{\emph{Meet}} \\

    F_1 \wedge\ F_2 &= \begin{array}{rl}
      \{(a, x \lub y)\ | & a \in \dom(F_1), a \in \dom(F_2)\}\ \cup \\
      \{(a, y)\ | & a \in \dom(F_1), a \not\in \dom(F_2)\ \text{or} \\
      & a \not\in \dom(F_1), a \in \dom(F_2)\},
    \end{array} \labeleq{eqn_back13} \labeleq{eqn_back12} & \eqref{eqn_back12} \\
    & \text{where\ } F_1, F_2 \in \setL{Fact}, \lub\ \text{as in Figure~\ref{tbl_back4}.} \\\\

    \multicolumn{3}{c}{\emph{Transfer}} \\
    t (F, a\ \text{\tt =}\ C) &= \{(a, x \lub C), \text{when\ } (a, x) \in F\ \text{or} \\
    & \phantom{= \{}(a, C), \text{when\ } a \not\in \dom(F)\}\ \cup \\
    & \phantom{=} F\ \backslash\ \mfun{uses}(F, a),\\
    & \text{where\ } F \in \setL{Fact}, C \in \ZZ. \labeleq{eqn_back14} & \eqref{eqn_back14} \\
    t (F, a\text{\tt ++}) &= \{(a, \top)\} \cup (F\ \backslash\ \mfun{uses}(F, a)), \\
    & \text{where\ } F \in \setL{Fact}. \\\\
    \mfun{uses}(F, a) &= \{(a, x)\ |\ a \in \dom(F)\}, \\
    & \text{where\ } F \in \setL{Fact}, a \in \setL{Var}. \\\\

    \multicolumn{3}{c}{\emph{Direction}} \\

    \outBa &= t(\inBa, s), \labeleq{eqn_back3} & \eqref{eqn_back3} \\
    & \text{where $s$ a statement in block\ } B.\\
    \inBa &= \bigwedge\limits_{\mathclap{P \in \mathit{pred}(B)}} \outXa{P} \labeleq{eqn_back16} & \eqref{eqn_back16} \\\\ 
    \mfun{pred}(B) &= \text{Predecessors of block }\ B.
  \end{array}
\end{math}
\clearpage}

%% \runin{Transfer Function} Our transfer function creates 

%% \section{Facts, Transfer Functions, Direction \& The Meet Operator}
%% \label{sec_back4}

%% Begin by placing the specific concept in the overall context of
%% dataflow. Give a small example highlighting the concept. Point
%% out fine points or subtleties that occur when generalizing the concept. End
%% by summarizing how the concept fits into dataflow (in a bit larger
%% sense than the first summary).
%% \begin{myfig}[th]
%% \centering
%% \begin{tikzpicture}
  \node[entex] (entry) {};

  \node[stmt,
    below of=entry] (assign) {
    \begin{minipage}{.5in}
      \begin{AVerb}
a = 1
b = 1
      \end{AVerb}
    \end{minipage}\labelNode{lst_back7_assign}};
  \node[labelfor=assign] {\refNode{lst_back7_assign}};
  \node[above=5mm of assign, anchor=west] {$\mathit{in:} \{a=\bot, b=\bot, c=\bot\}$};
  \node[below=3mm of assign, anchor=west] {$\mathit{out:} \{a=1, b=2, c=\bot\}$};

  \node[stmt,
    below=.75in of assign] (test) {#if(a > b)#\labelNode{lst_back7_test}};
  \node[labelfor=test] {\refNode{lst_back7_test}};
  \node[above=3mm of test, anchor=east] {$\mathit{in(\refNode{lst_back7_test})}: \{a=1, b=2, c=\bot\}$};
  \node[below=3mm of test, anchor=east] {$\mathit{out(\refNode{lst_back7_test})}: \{a=1, b=2, c=\bot\}$};
  \node[below right=0mm and 0mm of test, anchor=west] {$\mathit{out(\refNode{lst_back7_test})}: \{a=1, b=2, c=\bot\}$};
  
  \node[stmt,
    right=2in of test] (true) {#c = 4#\labelNode{lst_back7_true}};
  \node[labelfor=true] {\refNode{lst_back7_true}};
  \node[above=3mm of true, anchor=east] {$\mathit{in:} \{a=1, b=2, c=\bot\}$};
  \node[below=3mm of true, anchor=west] {$\mathit{out:} \{a=1, b=2, c=4\}$};

  \node[stmt,
    below=.75in of test] (false) {#c = a + 3#\labelNode{lst_back7_false}};
  \node[labelfor=false] {\refNode{lst_back7_false}};
  \node[above=5mm of false, anchor=west] {$\mathit{in:} \{a=1, b=2, c=\bot\}$};
  \node[below=3mm of false, anchor=west] {$\mathit{out:} \{a=1, b=2, c= a + 3\}$};

  \node[stmt,
    below=.75in of false] (print) {#print(c)#\labelNode{lst_back7_print}};
  \node[labelfor=print] {\refNode{lst_back7_print}};
  \node[above=3mm of print, anchor=east] {$\mathit{in:} \{a=1, b=2, c=a + 3\}$};
  \node[above right=-1mm and 3mm of print, anchor=west] {$\mathit{in:} \{a=1, b=2, c=4\}$};
  \node[below=3mm of print, anchor=east] {$\mathit{out:} \{a=1, b=2, c=\top\}$};

  \node[entex, below=.5in of print] (exit) {};

  \draw [->>] (entry) to (assign);
  \draw [->] (assign) to (test);
  \draw [->] (test) to (true);
  \draw [->] (test) to (false);
  \draw [->] (true) |- (print);
  \draw [->] (false) to (print);
  \draw [->>] (print) to (exit);

\end{tikzpicture}

%% \caption{The CFG for the C-language fragment from
%%   Figure~\ref{fig_back1_a}, annotated with \emph{facts} about the
%%   value of \texttt{a}, \texttt{b}, and \texttt{c} before (``\inBa'') and
%%   after (``\outBa'') each node.}
%% \label{fig_back5}
%% \end{myfig}

%% The dataflow algorithm computes two sets of \emph{facts} for every
%% node in the CFG. Facts are a data structure that describe the state of
%% the program before and after execution of the block represented by the
%% node. Figure~\ref{fig_back5} annotates the program fragment in
%% Figure~\ref{fig_back1} with facts about #a#, #b#, and #c# (the only
%% state we care about in this program). Each \inBa gives the variables'
%% values just prior to executing block $B$, while each \outBa gives
%% their values just after $B$ has executed.  

%% Figure~\ref{fig_back5} shows a \emph{forwards} analysis, where \outBa
%% is computed from \inBa, for each block. Facts are created by a
%% \emph{transfer function} that inspects the statements in each node and
%% updates values assigned to variables, if any. Sometimes a dataflow
%% analysis will run \emph{backwards}, computing \inBa from
%% \outBa. Section \ref{sec_back2} gives a detailed example illustrating
%% a \emph{backwards} analysis. In general, the transfer function and
%% direction vary depending on the particular analysis performed.

%% To help define our transfer function, we define |valueOf|,
%% which either returns the value assigned to a variable, or its value
%% from \inBa:
%% \begin{equation} |valueOf|(v) = 
%%   \begin{cases}
%%     |assign|(v) & \text{when $v$ is assigned a value in the node,} \\
%%     \text{\inBa}(v) & \text{when $v$ is not assigned.} 
%%   \end{cases}
%% \label{eqn_back2}
%% \end{equation}
%% In the above, $v$ represents a variable; |assign| retrieves the value
%% assigned to that variable, if any.  Our transfer function just needs
%% to apply |valueOf| to all variables in \inBa, as well as all
%% variable assignments in the node itself. If |assigned| is the set of
%% all assigned variables in the node, we can define how our transfer
%% function relates \inBa and \outBa using set notation:
%% \begin{equation}
%%   \text{\outBa} = [|valueOf|(v) || v \in (\text{\inBa} \cup |assigned|)].
%% \end{equation}

%% Our initial fact, \inB{lst_back7_assign}:~\facts{a/\top, b/\top,
%%   c/\top}, assigns the value ``$\top$'' (``top'') to all variables,
%% indicating that we do not know the value for the given variable. Our
%% transfer function determines that \outB{lst_back7_assign} should be
%% \facts{a/1, b/2, c/\top}, showing that we know #a# is 1, #b# is 2, and
%% that we still do not know the value of #c#. At each block we perform a
%% similar analysis, except \refNode{lst_back7_print}, where we need to
%%   take special action.

%% When a node has multiple predecessors, like \refNode{lst_back7_print},
%% we must combine multiple \outBa values into a single \inBa. The value
%% for #c# in \outB{lst_back7_true} is 4, while in \outB{lst_back7_false}
%% #c# is 3. We have two distinct values for #c# and no way to determine
%% which will be the case when \refNode{lst_back7_print} executes. We
%% must be conservative, so we assign the value $\top$ to #c# in
%% \inB{lst_back7_print}.

%% \begin{table}[tbh]
%%   \centering
%%   \figbegin
%%   \begin{math}
%%     \begin{array}{ccccc}
%%       & v_1 & v_2 & v_1 \lub v_2 \\
%%       \cmidrule(r){2-2}\cmidrule(r){3-3}\cmidrule(r){4-4}
%%       1. & \top & v_2 & \top & \\ 
%%       2. & v_1 & \top & \top & \\
%%       3. & v_1 & v_2 & \top & \text{($v_1 \neq v_2$)}\\
%%       4. & v_1 & v_1 & v_1 
%%     \end{array}
%%   \end{math}
%%   \caption{How the meet operator used in Figure \ref{fig_back5}
%%     combines facts. $v_1$ and $v_2$ are values given by separate
%%     \outBa facts to the same variable. The table shows how they are
%%     combined.}
%%   \label{tbl_back2}
%%   \figend
%% \end{table}

%% A \emph{meet operator} defines how we combine facts when values
%% conflict. Table~\ref{tbl_back2} defines ``\lub'' (``least upper
%% bound'' or ``lub''), which combines values as we did for
%% \outB{lst_back7_true} and \outB{lst_back7_false}. $v_1$~and $v_2$
%% represent values given to the same variable by different
%% facts. Lines~1 and 2 show that when either value is $\top$, the result
%% is $\top$. When values differ, as in Line~3, again the result is
%% $\top$. Only when values are equal, as shown in the last line, do we
%% preserve the value.

%% Facts define the state of the program that we are analyzing. The
%% transfer function transforms input facts into output facts. In a
%% forwards analysis, input facts come from predecessor nodes and output
%% facts flow to successors. For a backwards analysis, the opposite
%% occurs. When multiple facts need to be combined, we use a meet
%% operator. Each of these elements will vary depending on the specific
%% analysis performed.

%% \section{Iterative Analysis}
%% \label{sec_back6}
%% Begin by placing the specific concept in the overall context of
%% dataflow. Give a small example highlighting the concept. Point
%% out fine points or subtleties that occur when generalizing the concept. End
%% by summarizing how the concept fits into dataflow (in a bit larger
%% sense than the first summary).

%% \begin{equation}
%%   \begin{split}
%%     B \bigwedge\ \emptyset\ &= B \\
%%     B \bigwedge\ C &= [\{a=v_b\} \wedge\ \{a=v_c\} || \{a=v_b\}\ \in\ B, \{a=v_c\}\ \in\ C] \\%%
%%                    &\; \cup\ [\{b=v_b\} || \{b=v_b\}\ \in\ B, \{b=v_c\}\ \not\in\ C] \\%%
%%                    &\; \cup\ [\{c=v_c\} || \{c=v_b\}\ \not\in\ B, \{c=v_c\}\ \in\ C] \\
%%     \{a=\bot\} \wedge\ \{a=v\} &= \{a=v\} \\
%%     \{a=\top\} \wedge\ \{a=v\} &= \{a=\top\} \\
%%     \{a=v\} \wedge\ \{a=u\} &= \{a=\top\} (u \neq v) \\
%%     \{a=v\} \wedge\ \{a=v\} &= \{a=v\} \\
%%   \end{split}
%% \end{equation}

%% \begin{equation}
%%   \begin{split}
%%     f_B(In) &= [\mathit{assign}(v) || v \in\ In], \text{where $B$ is a block in the CFG} \\
%%     assign(v) &= %%
%%     \begin{cases}
%%       c & \text{when $v$ assigned $c$ in B.} \\
%%       In(v) & \text{when v not assigned in B.}
%%     \end{split}
%%   \end{align}
%% \end{equation}

%% \begin{tabular}{ll}
%%   \textbf{Lattice} & $\bot$, 0, 1, \ldots, and $\top$. \\
%%   \textbf{Meet} &  As above. \\
%%   \textbf{Transfer} & As above. \\
%%   \textbf{Direction} & Forward.
%% \end{tabular}


%% As we saw in Figure \ref{fig_back5}, facts can conflict when nodes
%% have multiple predecessors. Even more complicated situations arise
%% when a program contains loops. Consider the fragment in
%% \ref{fig_back6}. To compute \inB{lst_back9_test}, we need
%% \outB{lst_back9_assign} and and \outB{lst_back9_incr}. To compute
%% \inB{lst_back9_incr} (in order to find \outB{lst_back9_incr}, we need
%% \outB{lst_back9_test}. But to compute \outB{lst_back9_test} we need
%% \inB{lst_back9_test}.  How do we apply our |valueOf| function
%% (Equation \ref{eqn_back2}) to a \refNode{lst_back9_test} when
%% \inB{lst_back9_test} depends on \outB{lst_back9_test}?

%% \begin{myfig}
%% \begin{tabular}{cc}
%%   \subfloat{\begin{minipage}[t]{2in}
\begin{AVerb}[numbers=left]
int c = 0;
while(c < 10)
  c += 1;
print(c);
\end{AVerb}
\end{minipage}
%%
%%     \label{fig_back6_a}} \vline &%%
%%   \subfloat{\begin{minipage}[t]{2in}
\begin{AVerb}

\end{AVerb}
\end{minipage}
%%
%%     \label{fig_back6_b}} \\ 
%%   \subref{fig_back1_a} & \subref{fig_back1_b}
%% \end{tabular}
%% \caption{\subref{fig_back6_a}: A simple C-language program with a loop. \subref{fig_back6_b}: The CFG 
%% for the fragment.}
%% \label{fig_back6}
%% \end{myfig}

%% We solve this problem by applying our transfer function
%% \emph{iteratively}. In the case of Figure \ref{fig_back6}, we first
%% initialize each all \inBa and \outBa facts to some default. We then use
%% |valueOf| to compute each \outBa. Of course, facts will change over
%% the course of iteration -- especially in the case of node
%% \ref{lst_back9_test}. We keep iterating until we reach a \emph{fixed
%%   point}, meaning the facts stop changing.

%% \begin{table}
%%   \centering
%%   \begin{math}
%%     \begin{array}{lcccc}
%%       \mathit{Iteration} & \outB{lst_back9_assign} & \outB{lst_back9_incr} & \inB{lst_back9_test} & \outB{lst_back9_test} \\
%%       \cmidrule(r){1-1}\cmidrule(r){2-5} 
%%       0 & \bot & \bot & \bot & \bot  \\
%%       1 & 0 & 10 & \bot & \bot \\
%%       2 & 0 & 10 & \bot & \bot \\
%%       \multicolumn{5}{c}{\dots} \\
%%       \multicolumn{5}{l}{\inB{lst_back9_test} = \outB{lst_back9_assign} \lub \outB{lst_back9_incr}} \\
%%     \end{array}
%%   \end{math}
%%   \caption{Iterative analysis of the CFG from Figure
%%     \ref{fig_back6}. We how the inputs used to calculate
%%     \outB{lst_back9_test} change in one iteration. The zeroth
%%     iteration represents the initial values given to \inBa and \outBa
%%     for all nodes.}  
%%   \figend
%%   \label{tbl_back3}
%% \end{table}

%% Table \ref{tbl_back3} shows \inE and \out for
%% \refNode{lst_back9_test}. To compute \inB{lst_back9_test}, we combine
%% \outB{lst_back9_assign} and \outB{list_back9_incr} using the meet
%% operator from Section~\ref{sec_back4}:
%% \begin{equation}
%%   \inB{lst_back9_test} = \outB{lst_back9_assign} \lub \outB{lst_back9_incr}.
%% \end{equation}
%% The zeroth iteration shows the initial
%% value for all sets. On the first iteration, we can see \inB{lst_back9_test} is $\bot$:
%% \begin{equation}
%%   \begin{split}
%%     \inB{lst_back9_test} &= \outB{lst_back9_assign} \lub \outB{lst_back9_incr} \\
%%     &= \bot \lub \bot \\
%%     &= \bot.
%%   \end{split}
%% \end{equation}
%% When computing \inBa, we always use \outBa from the
%% \emph{previous} iteration. In the above we use $\bot$ for \outB{lst_back9_incr} and
%% \outB{lst_back9_assign}. 

%% When computing \inB{lst_back9_test} in the second iteration,
%% \outB{lst_back9_incr} is 10 and \outB{lst_back9_assign} is
%% 0. According to our meet operator, \inB{lst_back9_test} still equals
%% $\bot$:
%% \begin{equation}
%%   \begin{split}
%%     \inB{lst_back9_test} &= \outB{lst_back9_assign} \lub \outB{lst_back9_incr} \\
%%     &= 0 \lub 10 \\
%%     &= \bot.
%%   \end{split}
%% \end{equation}
%% At this point, our facts have stopped changing so we stop
%% iterating. Our final result $\bot$ for #c# in \outB{lst_back9_test}.

%% \section{Rewriting}
%% \label{sec_back7}

%% Begin by placing the specific concept in the overall context of
%% dataflow. Give a small example highlighting the concept. Point
%% out fine points or subtleties that occur when generalizing the concept. End
%% by summarizing how the concept fits into dataflow (in a bit larger
%% sense than the first summary).

%% Direction, the meet operator, facts, and the transfer function
%% together define a particular dataflow analysis. We can use the result
%% of the analysis to alter, or ``rewrite,'' the CFG of the program. The
%% meaning of the program should not change, but it should behave
%% differently: execute faster, use less memory, or whatever
%% characteristic the optimization should improve.  We do not have to
%% rewrite, of course. In some cases, we use the analysis to drive later
%% optimizations, or to report errors to the programmer. For example, a
%% \emph{reaching definitions} \citep{AhoXX} analysis can warn if
%% variables are used without being initialized. However, in most cases
%% we do want to rewrite the CFG.

%% \section{Example: Dead-Code Elimination}
%% \label{sec_back2}

%% Begin by placing the specific concept in the overall context of
%% dataflow. Give a small example highlighting the concept. Point
%% out fine points or subtleties that occur when generalizing the concept. End
%% by summarizing how the concept fits into dataflow (in a bit larger
%% sense than the first summary).

%% Consider Figure \ref{fig_back2}, again showing a C-language fragment.
%% The assignment to #b# on line~\ref{fig_back2_dead_line} has no visible
%% effect and can be removed without affecting the meaning of the
%% program. We call this optimization \emph{dead-code elimination}.

%% \begin{myfig}[ht]
%% \begin{minipage}{1in}
%%   \begin{AVerb}[numbers=left]
%%     a = 1;
%%     b = a + 1;\label{fig_back2_dead_line}
%%     return a + 1;
%%   \end{AVerb}
%% \end{minipage}
%% \caption{A C-language fragment illustrating \emph{dead code}. After
%% assignment on line \ref{fig_back2_dead_line}, \verb=b= is not used
%% and can be considered ``dead.''}
%% \label{fig_back2}
%% \end{myfig}

%% Of course, programmers do not normally write programs with such
%% obviously useless statements, but other compiler optimizations can
%% leave behind many such statements. \emph{Uncurrying}, described in
%% Chapter~\ref{ref_chapter_uncurrying}, in fact depends on dead-code
%% elimination.

%% The assignment on line~\ref{fig_back2_dead_line} can be eliminated
%% because #b# does not get referenced again. If a variable is referenced
%% after assignment, we say it is ``live.'' We call the dataflow
%% analysis that finds all live variables a ``liveness'' analysis. 

%% \begin{myfig}[th]
%% \begin{minipage}{2in}
%% %% \begin{AVerb}[commandchars=\\\{\}]
%%        E
%%        ||      
%%        v
%%      -----
%%     ||a = 1||    \emph{live:}  \ensuremath{\emptyset}
%%      -----
%%        ||      
%%        V
%%    ---------
%%   ||b = a + 1||  \emph{live:} \{a\}  
%%    ---------
%%        ||      
%%        V
%%   ------------
%%  ||return a + 1|| \emph{live:} \{a\}
%%   ------------
%%        ||      
%%        X          \emph{live:}  \ensuremath{\emptyset}
%% \end{AVerb}
\begin{tikzpicture}
  \node[entex] (entry) {};

  \node[stmt, below of=entry] (assigna) {#a = 1#\labelNode{lst_back10_assigna}};
  \node[labelfor=assigna] () {\refNode{lst_back10_assigna}};

  \node[stmt, below=.75in of assigna] (test) {#if(rnd() > 1)#\labelNode{lst_back10_test}};
  \node[labelfor=test] () {\refNode{lst_back10_test}};

  \node[stmt, right=.75in of test] (true) {#b = a + 1#\labelNode{lst_back10_true}};
  \node[labelfor=true] () {\refNode{lst_back10_true}};

  \node[stmt, below=.75in of test] (false) {#b = a - 1#\labelNode{lst_back10_false}};
  \node[labelfor=false] () {\refNode{lst_back10_false}};

  \node[stmt, below=.75in of false] (return) {#return a + 1#\labelNode{lst_back10_return}};
  \node[labelfor=return] () {\refNode{lst_back10_return}};

  \node[entex, below of=return] (exit) {};

  \draw [->>] (entry) to node[anchor=west] () {$\mathit{live}: \emptyset$} (assigna);
  \draw [->] (assigna) to node[anchor=west] () {$\mathit{live}: \lbrace a \rbrace$} (test);
  \draw [->] (test) to node [anchor=south] () {$\mathit{live}: \lbrace a \rbrace$} (true);
  \draw [->] (test) to node [anchor=east] () {$\mathit{live}: \lbrace a \rbrace$} (false);
  \draw [->] (true) |- node [anchor=west] () {$\mathit{live}: \lbrace a \rbrace$} (return);
  \draw [->] (false) to node [anchor=east] () {$\mathit{live}: \lbrace a \rbrace$} (return);
  \draw [->>] (return) to node[anchor=west] () {$\mathit{live}: \emptyset$} (exit);

\end{tikzpicture}

%% \end{minipage}
%% \caption{The CFG for our example program, annotated with the live
%% set for each node.}
%% \label{fig_back3}
%% \end{myfig}

%% Figure \ref{fig_back3} shows the CFG for this example. The annotations
%% on edges show the facts computed for this analysis: the set of live
%% variables. Recall from Section~\ref{sec_back4} that we can choose to
%% traverse the CFG forwards or backwards. Going forwards would require
%% us to track each assignment and then, after traversing the CFG,
%% determine if the variable was referenced again. The backwards case
%% requires that we add each variable reference to the live set. When an
%% assignment occurs, we can check if the variable appears in the live
%% set. If not, we know the variable was never referenced and is not
%% live. For simplicity and efficiency, we choose to go backwards rather
%% than forwards.

%% We define our transfer function such that, for some statement $B$,
%% \inBa represents the variables that are live in the statement. A live
%% variable is referenced (i.e., \emph{used}) in $B$ or one
%% of its successors. A variable that appears in \outBa but not \inBa must
%% not be live. The only way to remove a variable from \outBa is if it is 
%% assigned (i.e. or \emph{defined}) in $B$. We can then define our transfer
%% function from \outBa to \inBa in terms of the \emph{use} and \emph{def} sets:
%% \begin{align}
%%   & \inBa = (\outBa \cup |use|(B)) - |def|(B), \label{eqn_back1} \\
%% \text{where} \notag\\
%%   & B     & \text{Statement considered.} \notag\\
%%   & |use|(B) & \text{Variables referenced in $B$}. \notag\\
%%   & |def|(B) & \text{Variables assigned $B$}. \notag\\
%% \end{align}

%% Table \ref{tbl_back1} shows the |use|, |def|, \inBa and \outBa sets for
%% each statement. We include the exit node (``\exitN'') in the table to
%% show the initial value of \outBa for the last statement -- $\emptyset$,
%% the empty set. Our analysis then works backwards through the
%% program. Each \inBa becomes the input, \outBa, for the statement's
%% predecessor. If our program (and its CFG) contained any loops, we
%% would need to run this algorithm multiple times, until the live set
%% for each statement reached a fixed point.

%% \begin{table}
%%   \centering
%%   \begin{math}
%%     \begin{array}{lcccc}
%%       & |use|(B) & |def|(B) & \outBa & \inBa \\
%%       \cmidrule(r){2-5} %%\cmidrule(r){1-1}\cmidrule(r){2-2}\cmidrule(r){3-3}\cmidrule(r){4-4}\cmidrule(r){5-5}
%%       \exitN & \emptyset & \emptyset & \emptyset & \emptyset \\
%%       #return a + 1# & \{a\} & \emptyset & \emptyset & \{a\} \\
%%       #b = a + 1# & \{a\} & \{b\} & \{a\} & \{a\} \\
%%       #a = 1# & \emptyset & \{a\} & \{a\} & \emptyset \\
%%     \end{array}
%%   \end{math}
%%   \caption{The $|use|$, $|def|$, \inBa and \outBa sets computed using
%%     equation \ref{eqn_back1} for our example program.}
%%   \label{tbl_back1}
%%   \figend
%% \end{table}

%% With the live set computed for each statement, our analysis can now
%% determine which statements to eliminate. Only
%% \refNode{lst_back10_assigna} and \refNode{lst_back10_assignb} in
%% Figure~\ref{fig_back3} perform an assignment. The live set for
%% \refNode{lst_back10_assigna} (``#a = 1#'') contains #a#, so we do not
%% eliminate it. For \refNode{lst_back10_assignb} (``#b = a + 1#''), the
%% live set does \emph{not} contain #b#. Therefore, we can eliminate
%% \refNode{lst_back10_assignb}, giving us the new program in
%% Figure~\ref{fig_back6}.

%% \begin{myfig}[th]
%%   \centering
%%   \begin{minipage}{1in}
%%   \begin{AVerb}[numbers=left]
%% a = 1;
%% return a + 1;
%%   \end{AVerb}
%%   \end{minipage}
%%   \caption{The program from Figure~\ref{fig_back3} with the useless assignment to
%%     \verb=b= eliminated.}
%% \end{myfig}

%% In the
%% forwards case, we must track each assignment and determine, when we
%% exit the CFG, if the variable was ever used. We would need to track
%% every assignment until our traversal finished. However, if we traverse
%% backwards, we only need to add each reference to our live set. When we
%% see an assignment to a variable \emph{not} in our live set, we know it
%% has never been referenced.

%% \emph{live set} The \inE and \out sets show the facts computed for this
%% analysis. The computed show the live variables for that program point
%% \emph{live set}, annotates edges between each statement. The live set
%% is the \emph{fact} we compute for this analysis.

%% Annotations
%% show the facts we will compute
%% Recall from Section~\ref{sec_back4} that a dataflow analysis can
%% go \emph{forwards} or \emph{backwards}. 

%% To eliminate the assignment like that on
%% line~\ref{fig_back2_dead_line}, we need to determine which variables
%% are ``live'' -- that is, variables referenced after assignment. Such variables are ``live''; if a
%% variable is \emph{not} live, then it is dead. We use this ``liveness''
%% analysis to determine if a particular assignment is dead.

%% To determine if a variable is live, we need to know if it is
%% referenced after assignment. Such variables make up a \emph{live set}
%% that we can compute between each statement. To compute the live set,
%% we can choose to traverse the CFG for the program forwards or
%% backwards.  In the forwards case, we must track each assignment and
%% determine, when we exit the CFG, if the variable was ever used. We
%% would need to track every assignment until our traversal
%% finished. However, if we traverse backwards, we only need to add each
%% reference to our live set. When we see an assignment to a variable
%% \emph{not} in our live set, we know it has never been
%% referenced. Therefore we compute ``liveness'' using a backwards
%% traversal over the CFG.

%% \begin{myfig}[th]
%% \begin{minipage}{2in}
%% %% \begin{AVerb}[commandchars=\\\{\}]
%%        E
%%        ||      
%%        v
%%      -----
%%     ||a = 1||    \emph{live:}  \ensuremath{\emptyset}
%%      -----
%%        ||      
%%        V
%%    ---------
%%   ||b = a + 1||  \emph{live:} \{a\}  
%%    ---------
%%        ||      
%%        V
%%   ------------
%%  ||return a + 1|| \emph{live:} \{a\}
%%   ------------
%%        ||      
%%        X          \emph{live:}  \ensuremath{\emptyset}
%% \end{AVerb}
\begin{tikzpicture}
  \node[entex] (entry) {};

  \node[stmt, below of=entry] (assigna) {#a = 1#\labelNode{lst_back10_assigna}};
  \node[labelfor=assigna] () {\refNode{lst_back10_assigna}};

  \node[stmt, below=.75in of assigna] (test) {#if(rnd() > 1)#\labelNode{lst_back10_test}};
  \node[labelfor=test] () {\refNode{lst_back10_test}};

  \node[stmt, right=.75in of test] (true) {#b = a + 1#\labelNode{lst_back10_true}};
  \node[labelfor=true] () {\refNode{lst_back10_true}};

  \node[stmt, below=.75in of test] (false) {#b = a - 1#\labelNode{lst_back10_false}};
  \node[labelfor=false] () {\refNode{lst_back10_false}};

  \node[stmt, below=.75in of false] (return) {#return a + 1#\labelNode{lst_back10_return}};
  \node[labelfor=return] () {\refNode{lst_back10_return}};

  \node[entex, below of=return] (exit) {};

  \draw [->>] (entry) to node[anchor=west] () {$\mathit{live}: \emptyset$} (assigna);
  \draw [->] (assigna) to node[anchor=west] () {$\mathit{live}: \lbrace a \rbrace$} (test);
  \draw [->] (test) to node [anchor=south] () {$\mathit{live}: \lbrace a \rbrace$} (true);
  \draw [->] (test) to node [anchor=east] () {$\mathit{live}: \lbrace a \rbrace$} (false);
  \draw [->] (true) |- node [anchor=west] () {$\mathit{live}: \lbrace a \rbrace$} (return);
  \draw [->] (false) to node [anchor=east] () {$\mathit{live}: \lbrace a \rbrace$} (return);
  \draw [->>] (return) to node[anchor=west] () {$\mathit{live}: \emptyset$} (exit);

\end{tikzpicture}

%% \end{minipage}
%% \caption{The CFG for our example program, annotated with the live
%% set for each node.}
%% \label{fig_back3}
%% \end{myfig}

%% Figure \ref{fig_back3} shows the CFG for this example. Annotations
%% show the facts we will compute: the live set before and after. Though
%% execution follows the arrows in the CFG, our analysis proceeds
%% backwards. For example, the input to node 2 is the live set computed
%% for node 3 (``$\{a\}$'' in this case).

%% Our transfer function computes the live set based on \emph{uses} and
%% \emph{definitions} in a statement. Any reference (or use) of a
%% variable goes into the live set. Any assignment (or definition) of a
%% variable removes it from the live set. We can then define our transfer
%% function, |live|, for a statement as:

%% \begin{align}
%%   & |live|(s) = (\Varid{in}(s) \cup |use|(s)) - |def|(s), \label{eqn_back1} \\
%% \intertext{where}
%%   & s     & \text{Statement considered.} \notag\\
%%   & |use|(s) &  \text{Set of variables used in $s$}. \notag\\
%%   & |def|(s) & \text{Variable assigned to in $s$ (a singleton set)}. \notag\\
%%   & \Varid{in}(s) & \text{Live variables computed for $s$' successor}. \notag
%% \end{align}

%% Table \ref{tbl_back1} shows the |use| and |def| sets for each
%% statement. The live set computed, |live|, becomes the input, $\Varid{in}$, for
%% the statement's predecessor. We include the exit node (``#X#'') in the
%% table to show the initial value of $\Varid{in}$ for the last statement --
%% $\emptyset$, the empty set. Our analysis then works backwards through the
%% program. If our program (and its CFG) contained any loops, we would
%% need to run this algorithm multiple times, until the live set for each
%% statement reached a fixed point.

%% \begin{table}
%%   \centering
%%   \begin{tabular}{lcccc}
%%     $s$ & $|use|(s)$ & $|def|(s)$ & $\Varid{in}(s)$ &  $|live|(s)$ \\
%%     \cmidrule(r){1-1}\cmidrule(r){2-2}\cmidrule(r){3-3}\cmidrule(r){4-4}\cmidrule(r){5-5}
%%     #X# & & & & $\emptyset$ \\
%%     #return a + 1# & $\{a\}$ & $\emptyset$ & $\emptyset$ & $\{a\}$ \\
%%     #b = a + 1# & $\{a\}$ & $\{b\}$ & $\{a\}$ & $\{a\}$ \\
%%     #a = 1# & $\emptyset$ & $\{a\}$ & $\{a\}$ & $\emptyset$ \\
%%     \bottomrule
%%   \end{tabular}
%%   \caption{The $|use|$, $|def|$ and $|live|$ sets computed using equation \ref{eqn_back1} for our example program.}
%%   \label{tbl_back1}
%% \end{table}

%% With the live set computed for each statement, our analysis can now
%% determine which statements to eliminate. Only nodes 1 and 2 in Figure
%% \ref{fig_back3} perform an assignment. The live set for node 1 (``#a = 1#'')
%% contains #a#, so we do not eliminate it. In node 2 (``#b = a + 1#''),
%% the live set does \emph{not} contain #b#. Therefore, we can eliminate
%% node 2, giving us a new program without any dead code:

%% \begin{Verbatim}
%% a = 1;
%% return a + 1;
%% \end{Verbatim}

\section{Summary}
\label{sec_back9}

This chapter gave an overview of \emph{dataflow optimization}. The
dataflow \emph{algorithm} gives a general technique for applying an
\emph{optimizing function} to the \emph{control flow graph} (CFG)
representing a given program. The optimizing function computes
\emph{facts} about each node in the graph, using a \emph{transfer}
function. A given analysis can proceed \emph{forwards} (where \inBa
facts produce \outBa facts) or \emph{backwards} (where \outBa facts
produce \inBa facts). Each optimization defines a specific \emph{meet
  operator} that combines facts for nodes with multiple predecessors
(for forwards analysis) or successors (for backwards). We compute
facts\emph{iteratively}, stopping when they reach a \emph{fixed
  point}. Finally, we \emph{rewrite} the CFG using the facts computed. The 
meaning of our program does not change, but its behavior will be ``better,'' 
whatever that means for the particular optimization applied.


%% \subsection{Basic Blocks and Control-Flow Graphs}

%% A dataflow optimization operates over a ``control-flow graph'' (CFG)
%% of the program -- a directed graph where edges encode branches or
%% jumps and nodes represent statements. Programs run by entering a node
%% from a predecessor, executing the statements in turn, and exiting the
%% node to a successor. Multiple successors imply a conditional branch,
%% though the program can only choose one. A special ``entry'' node, with
%% no predecssors, exists to give the program a starting point.

%% The statements in each node must define a ``basic block,'' which means
%% there can only be one entry and one exit to the node. Each
%% predeccessor starts at the same statement; execution cannot start in
%% the ``middle'' of the statements in the node. Each successor also
%% leaves from the same instruction, so only one ``branch'' can exist in
%% each node.

%% For example, consider the ``fall-through'' implied by the use of #case#
%% statements in this C-language program fragment:

%% \begin{verbatim}
%%   switch(i) {
%%   case 1:
%%     printf("1");
%%     break;
%%   case 2:
%%     printf("2");
%%   case 3:
%%     printf("3");
%%   }
%% \end{verbatim}

%% \begin{figure}[h]
%% \begin{verbatim}
%%    A
%%   switch   ----<-
%%   | |  |  |      |
%%   | |  |  v C    ^
%%   | |   ->case 3 |
%%   | |     |      |
%%   | |      ->----_--
%%   | | B          |  |
%%   |  ->case 2 ->-   v
%%   |                 |
%%   |   D       ----<-
%%    ->case 1  |
%%      |       v
%%      v       |
%%    --+-----<-
%%   |
%%    -> ...
%% \end{verbatim}
%% \caption{CFG illustrating \emph{fall-through} allowed by the
%%   C-language \texttt{switch} statement.}
%% \label{switchCfgEg}
%% \end{figure}

%% Figure \ref{switchCfgEg} shows a CFG for this fragment. Execution
%% begins at node A. Node C has two predeccessors: A and B. The edge
%% between Node B and C represents fall-through from the second to third
%% case. They cannot be combined because the node would need two distinct
%% entry points. Encoding a program into basic blocks usually involves
%% inserting similar branches. The CFG makes explicit control--flow that
%% exists by implication in the source program.

%% \subsection{Direction, Facts and Rewrites}

%% \subsection{Example: Bind/Return Collapse}

%% Dataflow optimizations transform the CFG representation of a program,
%% with the goal of making a faster (or smaller, or more efficient, etc.)
%% program. Dataflow computes a set of ``entry'' assumptions and ``exit''
%% facts for each node in the graph. Facts for one node become
%% assumptions for the nodes' successors (thus the term
%% ``dataflow''). The algorithm iteratves over the entire graph until a
%% fixed point is reached -- that is, facts and assumptions no longer
%% change. The computed facts can then be used to transform the graph.

%% \emph{Constant propagation example -- or something more functional?}

%% \emph{Introduce forward and backwards dataflow.}

% What does dataflow mean?

% How do you use it?

% Example

\end{document}

% LocalWords:  Dataflow dataflow CFG printf variable's CFGs ccc Uncurrying lst
% LocalWords:  liveness Kildall AhoXX assigna assignb runtime valueOf ccccc lub
% LocalWords:  incr lcccc
