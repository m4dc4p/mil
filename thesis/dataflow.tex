\documentclass[12pt]{report}
%include polycode.fmt
\usepackage[T1]{fontenc}
\usepackage{calc}
\usepackage{palatino}
\usepackage{amsfonts}
\renewcommand\ttdefault{lmtt}
\usepackage{helvet}
\usepackage{xspace}
\usepackage{url}
\usepackage{fancyvrb}
\usepackage[doublespacing]{setspace}
%% below only necessary when using doublespacing -- corrects
%% the vertical space inserted when switching to singlespace
%% environment.
\def\correctspaceskip{\vskip-\baselineskip} 
\usepackage{amsmath}
\usepackage{booktabs}
\usepackage[margin=\parindent, format=hang,labelfont=bf]{caption}
%% \usepackage[subrefformat=parens]{subcaption}
%% The following makes sure we get parentheses around
%% subreferences. The newest version of the subcaption
%% package has an option for this, but that's not available
%% widely.
%%
%% From http://tex.stackexchange.com/questions/25644
\usepackage[labelformat=simple]{subcaption}
\makeatletter
  \def\thesubfigure{(\alph{subfigure})}
  \providecommand\thefigsubsep{~}
  \def\p@subfigure{\@nameuse{thefigure}\thefigsubsep}
\makeatother

\usepackage{ifthen}
\usepackage{stmaryrd}
\usepackage{longtable}
\usepackage{afterpage}
\usepackage{xifthen}
\usepackage{mathtools}
\usepackage[natbib=true,style=authoryear,backend=bibtex8]{biblatex}
\setlength{\bibitemsep}{\bigskipamount}
\addbibresource{thesis.bib}
\usepackage{microtype}

%% GSO margins.
\usepackage[left=1.5in, right=1in, top=1in, bottom=1in]{geometry}
\usepackage{abstract}

%% GSO requires 12 pt font for all headings
\usepackage[bf,sf,tiny,compact]{titlesec}
\titleformat{\chapter}[display]
            {}% format
            {\sffamily\bfseries\chaptertitlename\ \thechapter}
            {\baselineskip}
            {\sffamily\bfseries}
            {}

\hyphenation{data-flow mo-na-dic} 

%% Should unindent all haskell code set in a dispay (versus inline)
\makeatletter
  \@ifundefined{hscodestyle}
               {}
               {\renewcommand{\hscodestyle}{\advance\leftskip -\mathindent}}
\makeatother

% Used by included files to know they
% are NOT standalone
\newboolean{standaloneFlag}
\setboolean{standaloneFlag}{true}

\newlength{\rulefigmargin}
\setlength{\rulefigmargin}{2\parindent}

\newcommand\figbegin{\rule{\linewidth}{0.4pt}\\\vspace{12pt}}
\newcommand\figend{\rule{\linewidth}{0.4pt}}

%% Sets
\newcommand{\setL}[1]{\textsc{#1}\xspace}
\newcommand{\setLC}{\setL{Const}}

%% Lub, subset operators.
\protected\def\lub{\ifmmode\sqcap\else\raisebox{.1em}{\ensuremath{\sqcap}}\fi\xspace}
\newcommand{\sqlt}{\ensuremath{\sqsubset}\xspace}
\newcommand{\sqlte}{\ensuremath{\sqsubseteq}\xspace}

%% Subscripting with typewriter
\def\subtt#1{\ifmmode_{\ensurett{#1}}%%
  \else$_{\ensurett{#1}}$%%
  \fi}
%% Superscripting with typerwriter
\def\suptt#1{\ifmmode^{\ensurett{#1}}%%
  \else$^{\ensurett{#1}}$%%
  \fi}
%% Functional languages chapter commands
\newcommand{\lamA}{\ensuremath{\lambda}-calculus\xspace}
\newcommand{\LamA}{\ensuremath{\lambda}-Calculus\xspace}
\newcommand{\lamAbs}[2]{\ensuremath{\lambda#1.\ #2}}
\newcommand{\lamApp}[2]{\ensuremath{#1\ #2}}
\newcommand{\lamPApp}[2]{\ensuremath{(#1\ #2)}}
\newcommand{\lamAPp}[2]{\ensuremath{(#1)\ #2}}
\newcommand{\lamApP}[2]{\ensuremath{#1\ (#2)}}
\newcommand{\lamAPP}[2]{\ensuremath{(#1)\ (#2)}}
\let\lamApPp=\lamApP
\let\lamAppP=\lamAPp
%% LC definition
\newtoks\toksA
\protected\def\lcname#1/{\ensuremath{\mathit{#1}}}
\protected\def\lcdef#1(#2)=#3;{\def\arg{#2}%%
  \def\lcargs##1,##2/{\def\arg{##2}%%
    \ifx\empty\arg%%
    \lcname ##1/%%
    \else\lcname ##1/\ \lcargs ##2/%%
    \fi}%%
  \ifx\empty\arg\toksA={\ }%%
  \else\toksA={\ \lcargs #2,/\ }%%
  \fi%%
  \ensuremath{\lcname#1/\the\toksA =\ #3}}
%% Arbitary number of applied arguments, separated
%% by asterisks (*).
\protected\def\lcapp#1/{\def\lcappB##1*##2/{\def\arg{##2}%
    \ensuremath{\ifx\arg\empty%%
      \lcname ##1/%%
      \else%%
      \lcname##1/\ \lcappB##2/%%
      \fi}}%%
  %% Adding a star here makes
  %% sure our applicaitn always ends with an asterisks, ensuring
  %% #2 will be \empty at some point.
  \lcappB#1*/}
\protected\def\lcabs#1.{\ensuremath{\lambda#1.\ }}

\newcommand{\lamId}{\lamAbs{x}{x}}
\newcommand{\lamCompose}{\lamAbs{f}{\lamAbs{g}{\lamAbs{x}{\lamApp{f}{(\lamApp{g}{x})}}}}}
\newcommand{\machLam}{\ensuremath{M_\lambda}\xspace}
\newcommand{\compMach}[1]{\ensuremath{\left\llbracket #1 \right\rrbracket}}
\newcommand{\compRho}[1]{\ensuremath{\rho(#1)}}
\newcommand{\verSub}[2]{\ensuremath{#1_{#2}}}
\newcommand{\verSup}[2]{\ensuremath{#1^{#2}}}
\newcommand{\lamC}{\ensuremath{\lambda_C}\xspace}
\newcommand{\lamPlus}{\lamAbs{m}{\lamAbs{n}{\lamAbs{s}{\lamAbs{z}{\lamApp{m}{\lamApPp{s}{\lamApp{n}{\lamApp{s}{z}}}}}}}}}
%% Substitution notation -- [#1 -> #2]
\newcommand{\lamSubst}[2]{\ensuremath{[#1 \mapsto #2]}}
%% End functional languages chapter


%% MIL Chapter
\newcommand{\compMILE}[1]{\ensuremath{\left\llbracket #1 \right\rrbracket}}
\newcommand{\compMILV}[1]{\ensuremath{\left\llbracket #1 \right\rrbracket}}
\newcommand{\compMILQ}[2]{\ensuremath{\left\llbracket #2 \right\rrbracket}}
\newcommand{\milCtx}[1]{\ensuremath{\llfloor}#1\ensuremath{\rrfloor}}

%% This dimension makes sure the same amount of space
%% follows | and := in syntax rules like:
%%
%% term := var       (Variable)
%%      |  var var    (Application)
%%      |  \x. var    (Abstraction)
%%
\newdimen\termalign
\setbox0=\hbox{$:=$}
\termalign=\wd0 
\protected\def\term#1/{\ensuremath{\mathit{#1}}}
\protected\def\syntaxrule#1/{\hfil\text{\emph{#1}}}
\protected\long\def\termrule#1:#2:#3/{\term #1/ &\hbox{$:=$}\ensuremath{\ #2} & \syntaxrule #3/}
\protected\def\termcase#1:#2/{&\hbox to \termalign{$|$\hss}\ensuremath{\ #1} & \syntaxrule #2/}


%% End MIL chapter

%% Dataflow Chapter
% Domain function
\def\dom(#1){\ensuremath{\mfun{dom}(#1)}\xspace}
% Set of all integers.
\def\ZZ{\ensuremath{\mathbb{Z}}}
%%

%% Uncurrrying Chapter 
%% A space equal to a \thinspace, but we
%% can break a line at it.
\newskip\thinskipamt \thinskipamt=.16667em 
\protected\def\thinskip{\hskip \thinskipamt\relax}
\protected\def\thinnerskip{\hskip .5\thinskipamt\relax}
%% Capture a space token. Use a ``control-symbol'' (\. instead of \mksp)
%% to keep the trailing space from getting gobbled.
{\def\.{\global\let\sp= } \. }
%% Define \asp, which will capture the macro definition attached to space,
%% if one exists. Otherwise, \spa is relax after this.
{\catcode`\ =\active\gdef\asp{\ifx \relax\let\spa\relax\else\let\spa= \fi}}
\newtoks\foo
%% Removes spaces, implicit, active and explicit.
\protected\def\removespaces{\asp\afterassignment\removesp\let\next= }
\def\removesp{\foo={\next}\ifcat\noexpand\next\sp\foo={\removespace}%%
 \else\ifx\next\spa\foo={\removespaces}\fi%%
 \fi\the\foo}
%% MIL reserved word
\protected\def\milres#1/{\text{\ttfamily\bfseries #1}}
\protected\def\lab#1/{\textbf{\ensurett{\removespaces #1}}}
%% Constructs a closure: l { v1, ..., vN }
\protected\long\def\mkclo[#1:#2]{\lab #1/\ensuremath{\,\{\ensurett{#2}\}}\xspace}
%% Tuple version of closurs: {l: v1, ..., vN}.
\protected\long\def\clo[#1:#2]{\def\argA{#1}\def\argB{#2}\ensuremath{\{%%
      \ifx\argA\empty%%
      \else\lab #1/%%
        \ifx\argB\empty%%
        \else\ensurett{:\thinskip}%%
        \fi%%
      \fi\ensurett{#2}\}}\xspace}
%% Construct a thunk
\newbox\bracklbox \newbox\brackrbox
\setbox0=\hbox{$\{$} \setbox\bracklbox=\hbox to \wd0{\hfil[\kern0.25mm}
\setbox0=\hbox{$\}$} \setbox\brackrbox=\hbox to \wd0{\kern0.25mm]\hfil}
\protected\def\mkthunk[#1:#2]{\lab #1/%%
  \ensuremath{\,%%
    \mathopen{\copy\bracklbox}%%
    \ensurett{#2}%%
    \mathclose{\copy\brackrbox}\xspace}}
%% Binding statement: v <- {...}
\protected\def\binds#1<-#2;{\ensurett{\removespaces #1\texttt{<-}#2}\xspace}
%% In order to use \binds in verbatim environment, have to define
%% delimiters while they are active. The below defines \vbinds which
%% must be used in AVerb environments.  Notice the active space as
%% well - that is necessary so the space after \vbinds (and before the
%% first argument) in the verbatim environment gets eaten.
\begingroup\catcode`\!=\active \lccode`\!=`\< \lccode`\~=`\- 
  \catcode`\ =\active\lowercase{\endgroup\def\vbinds#1!~#2;}{\binds#1<-#2;}
%% Return statement: return ... ;
\protected\def\return#1;{\milres return/\ensurett{\ \removespaces #1}}
%% A closure capturing block. k {v1, ..., vN} x: ...
\protected\def\ccblock#1(#2)#3:{\lab#1/\ensuremath{\thinspace\{\ensurett{#2}\}}\ \ensurett{#3\hbox{:}}}
%% A normal block
\protected\def\block#1(#2):{\lab #1/\ensuremath{\thinspace(\ensurett{#2})}\ensurett{:}}
%% A goto expression
\protected\def\goto#1(#2){\lab #1/\thinspace\ensuremath{(\ensurett{#2})}}
%% An enter expression
\protected\def\app#1*#2/{\ensurett{\removespaces #1\ifmmode\ \fi{\text{\tt @}}\ifmmode\ \fi#2}}
\protected\def\bind{\texttt{<-}\xspace}
%% Primitive expression
\protected\def\prim#1(#2){\lab #1/\suptt*\ensuremath{(\ensurett{#2})}}
%% Program variable
\protected\def\var#1/{\ensurett{\removespaces #1}\xspace}
%% Case statement
\protected\def\case#1;{\milres case\ \ensuremath{\ensurett{\removespaces #1}}\ of/}
%% Case alternative
\protected\def\alt#1(#2)#3->#4;{\ensuremath{\ensurett{#1\ \ignorespaces#2\ \texttt{->}\ \ignorespaces #4}}}
%% Invoke
\protected\def\invoke#1/{\milres invoke/\ensurett{\ \removespaces #1}}
\def\rhs{right--hand side\xspace}
\def\lhs{left--hand side\xspace}
\def\enter{\texttt{@}\xspace}
\def\cc{closure--capturing\xspace}
\def\Cc{Closure--capturing\xspace}
%%

\newenvironment{myfig}[1][tbh]{\begin{figure}[#1]%%
\begin{singlespace}\centering%%
\figbegin}{\figend\end{singlespace}%%
\end{figure}}

%% Produce a sub-caption and label it.
\newcommand{\scap}[2][1in]{\begin{minipage}{#1}%%
\subcaption{}\label{#2}\end{minipage}}

%% Produce a sub-caption with text.
\newcommand{\lscap}[3][\hsize]{\begin{minipage}{#1}%%
\subcaption{#3}\label{#2}\end{minipage}}

% single-argument comment. Do not put
% a space before the command when used
% or the file will have two spaces.
\newcommand{\rem}[1]{}

%% A verbatim environment with active charactesr
%% so we can use math shortcuts and macros
\DefineVerbatimEnvironment{AVerb}{Verbatim}{commandchars=\\\{\},%% 
  codes={\catcode`\_8\catcode`\$3\catcode`\^7},%%
  numberblanklines=false}

\DefineVerbatimEnvironment{Verb}{Verbatim}{commandchars=\\\[\],%% 
  numberblanklines=false}

%% Turn on line numbers for Haskell code, 
%% and reset the line number counter.
\newcommand{\hsNumOn}{\numberson\numbersreset}
\newcommand{\hsNumOff}{\numbersoff}
%% Turn on line numbering in Haskell code within
%% the environment, then turn it off. The optional
%% argument specifies a prefix that \hslabel can
%% use to generate line number references. If no prefix
%% is givne, \hslabel will have no effect.
\newtoks\prefixtoks
\def\mkhslabel#1{\prefixtoks={#1}\let\prefix=a}
\def\hslabel#1{\ifx\prefix\relax%%
  \else\label{\the\prefixtoks_#1}%%
  \fi}
\def\unhslabel{\let\prefix=\relax}
\newenvironment{withHsNum}{\numberson\numbersreset}{\numbersoff}
\newenvironment{withHsLabeled}[1]{\numberson\numbersreset\mkhslabel{#1}}{\unhslabel\numbersoff}

%% Paragraph run-in
\newcommand{\runin}[1]{\begingroup\noindent\sffamily\textbf{#1}\qquad\endgroup}

%% Chapter bibliographies
\newcommand{\standaloneBib}{%%
  \ifthenelse{\boolean{standaloneFlag}}%%
             {\begin{singlespace}
                 \printbibliography
             \end{singlespace}}{}}

%% Adds an equation number on demand.
\newcommand\addtag{\refstepcounter{equation}\tag{\theequation}}

%% For typesetting set definitions like {x | x \in f(y)}
\newcommand\setdef[2]{\ensuremath{\{#1\ |\ #2\}}}

%% For typesetting function names like dom(f) or out(b).
\newcommand\mfun[1]{\ensuremath{\mathit{#1}}}

%% Marginal notes
\newcommand\margin[2]{\marginpar{\begin{singlespace}\emph{\footnotesize #2}\end{singlespace}}\relax #1}

%% Describe intent of a passage
\newcommand\intent[1]{{\begin{singlespace}\noindent\leftskip=-1in\emph{\footnotesize Intent: #1}\end{singlespace}}\nopagebreak[1]}

%% In aligned/alignedat/gathered environments, you don't et
%% automatice equation numbers. This command makes sure to
%% label them properly.
\newcommand\labeleq[1]{\refstepcounter{equation}\label{#1}}

%% Creates a hanging paragraph, where the first line is not
%% indented but all other lines are.
\def\itempar#1{\noindent\hangindent=\parindent\hangafter=1 #1\quad}

%% Disable overfull messages with ridiculous hfuzz value
\def\disableoverfull{\hfuzz=10in}

%% Set parfillskip so stretching does NOT occur at the end of
%% a paragraph (i.e., list of elements). Disable indent at beginning
%% of paragraph. Also turn off underfull hbox warnings.
%%
%% Intended to be used in a \vbox that forms part of a table or graphic,
%% which we want to be line-broken but not exactly like a normal paragraph.
\long\def\disableparspacing#1;{\def\arg{#1}\hbadness=100000\parindent=0pt\parfillskip=0pt\leftskip=0pt\rightskip=0pt%%
  \ifx\arg\empty\else\hsize=#1\relax\fi}
%% This stuff makes !+<text>+! write <text> in typewriter font.  

%% We make ! and + active characters early, then manipulate their
%% meaning to produce the right effect. Initially, + produces +. When
%% !  appears w/o a + following, it produces ``!''. When ``+''
%% follows, we start writing in teleteype (\ttfamily). The definition
%% of ``!'' changes to produce a bang. ``+'' changes such that it
%% looks for trailing ``!''. When no ``!'' appears, ``+'' produces ``+''. 
%% If a ``!'' appears, we shift out of \ttfamily (by ending the group) and
%% reset the meaning of ``!'' and ``+'' so we can start again.
\makeatletter
\let\mdplus=+\let\mdbang=!      %% Preserve meaning of + and ! so we can put them into document.
%% Turn off mark down for everyone
\outer\def\nomd{\catcode`!=12\catcode`+=12}
%% Turn mark down on for everyone
\outer\def\domd{\catcode`!=\active\catcode`+=\active %%
  \initialmd}
%% Use only with a group IMMEDIETALY following. Turns off
%% markdown for the group-to-come, without actually tokenizing the
%% group. If no group follows, this has no effect.
\protected\def\pausemd{\def\dopause{\catcode`!=12\catcode`+=12}%%
  \def\pausemdB{\ifx\next\bgroup%%
    %% A ``partial'' application of expandwith is used
    %% so we don't double up the group argument (which is what
    %% happens if we expand \next). This has the effect of 
    %% inserting \expandafter\dowith in front of the upcoming {. 
    %% If no brace is coming, \withmdC will have no effect.
    \def\pausemdC{\expandafter\dopause}
  \else
    \let\pausedmC=\relax
  \fi\pausemdC}
  %% \futurelet has to end the macro so it grabs the next token
  %% from the input file. Otherwise, it grabs it *from* this
  %% definition.
  \futurelet\next\pausemdB} %%
%% Turns markdown on for the group-to-come, without actually
%% tokenizing the group. Only has an effect when
%% used in front of a group, otherwise its a no-op.
\protected\def\withmd{\def\dowith{\catcode`!=\active\catcode`+=\active\initialmd}%%
  \def\withmdB{\ifx\next\bgroup %%
    %% A ``partial'' application of expandafter is used
    %% so we don't double up the group argument (which is what
    %% happens if we expand \next). This has the effect of 
    %% inserting \expandafter\dowith in front of the upcoming {. 
    %% If no brace is coming, \withmdC will have no effect.
      \def\withmdC{\expandafter\dowith} %%
    \else %%
      \let\withmdC=\relax %%
    \fi\withmdC}%%
  %% \futurelet has to end the macro so it grabs the next token
  %% from the input file. Otherwise, it grabs it *from* this
  %% definition.
  \futurelet\next\withmdB} %%
%% Make ! and + active in the following group so they have the right
%% catcode in the definitions to follow.
\catcode`!=\active\catcode`+=\active %%
%% Initial definitions associated with ! and +.
\def\initialmd{\protected\def!{\startTTA} %%
  \protected\def+{\stopTTA}} %%
%% Step 1 of startTT. Inital meaning of !; capture next token in \next, go to next step.
\def\startTTA{\futurelet\next\startTTB} %%
%% Step 2 of startTT. Compare captured token to + and go to step 3 if true. Otherwise
%% output a ! (since that started our macro), the argument captured and stop
%% processing.
\long\def\startTTB{\ifx\next+\expandafter\startTTC\expandafter\@gobble\else\mdbang\fi} %%
%% Step 3 of startTT. Shift into teletype mode and change definition of 
%% + and ! so we can stop processing.
\def\startTTC{\begingroup\ifmmode %%
  \let \math@bgroup \relax %%
  \def \math@egroup {\let \math@bgroup \@@math@bgroup %%
    \let \math@egroup \@@math@egroup} %%
  \mathtt\relax %%
  \else  %%
  \ttfamily\fi} %%
%% Step 1, 2  and 3 of stopTT follow the same pattern as startTT.
\def\stopTTA{\futurelet\next\stopTTB} %%
\long\def\stopTTB{\ifx\next!\expandafter\stopTTC\expandafter\@gobble\else\mdplus\fi} %%
\def\stopTTC{\endgroup}%%
\catcode`!=12\catcode`+=12
\makeatother

\domd

%% Place an input file on the next page
\def\onnextpage#1{\afterpage{\clearpage\input{#1}\clearpage}}

\begin{document}
\ifthenelse{\boolean{standaloneFlag}}
           {\VerbatimFootnotes
             \DefineShortVerb{\#}
             \doublespacing
             \setcounter{chapter}{0}}{}

%% Default float parameters. For case when
%% multiple chapters are included and
%% only one needs custom float settings.
\renewcommand{\textfraction}{0.2}
\renewcommand{\topfraction}{0.9}

\newcounter{nodeCounter}[subfigure]
%% Float parameters
\renewcommand{\textfraction}{0.1}
\renewcommand{\topfraction}{0.9}

\chapter{Dataflow Optimization}
\label{ref_chapter_background}

%% A short section giving the history of dataflow optimization techniques
%% and basic concepts.

% Describe dataflow analysis in general terms and defines key
% concepts: basic blocks, control flow, facts, and
% rewrites. Bind/Return elimination is used as an an example.

The term ``program optimization'' refers to the process of
transforming a program without changing its semantics (i.e., meaning),
while at the same time ``improving'' its behavior.  For example, an
optimized program may run faster, use less memory, consume less power,
or some sense perform ``better'' than the unoptimized
program. Optimizations can be performed ``by hand,'' while writing
the program, or automatically by a compiler. 

``Dataflow analysis'' (or ``dataflow optimization''), first introduced
by Gary Kildall \citep{Kildall1973}, refers to an algorithm for
applying an ``optimizing function'' to a given program. In itself it
does not give a specific optimization; rather, it gives a technique
for applying many different optimizations. In today's terms, dataflow
analysis treats a program as a ``control-flow graph'', applies a
``transfer'' function to compute ``facts'' about the execution of the
program, and then applies a ``rewriting'' function to transform the
program based on those facts. Dataflow analysis is now considered
standard technique and can be found in most compiler textbooks. 

\section{Control-Flow Graphs}
\label{sec_back1}
\newcommand{\inE}{\emph{in}\xspace}
\newcommand{\out}{\emph{out}\xspace}
\newcommand{\In}{\emph{In}\xspace}
\newcommand{\Out}{\emph{Out}\xspace}

%% Control-flow graph

The \emph{control-flow graph} (CFG) of a program shows how it
may be executed: which statements follow one another, branches that
can be taken, loops that may execute, and possible ways the program
can terminate.

\begin{myfig}[th]
\begin{tabular}{cc}
\subfloat{\begin{minipage}[t]{2.5in}
\begin{AVerb}[numbers=left]
int a = 1, b = 2, c; \label{lst_back1_assign}
if(a > b) \label{lst_back1_test}
  c = 4; \label{lst_back1_test_true}
else     
  c = a + 3; \label{lst_back1_test_false}

printf(c); \label{lst_back1_print}
\end{AVerb}
\end{minipage}
%%
  \label{fig_back1_a}} \vline & 
\subfloat{\begin{minipage}[t]{2in}
\begin{Verbatim}
     E (1)
     |
     V
   -----(2)
  |a = 1|
  |b = 2|
   -----
     |
     V
 ---------(3)    -----(4)
|if(a > b)|---->|c = 4|
 ---------       -----
     |             |
     V             V
 ---------(5)  ---------(6)
|c = a + 3|-->|printf(c)|
 ---------     ---------
                   |
                   V
                   |
                   X (7)
\end{Verbatim}
\end{minipage}





%%
  \label{fig_back1_b}} \\
\subref{fig_back1_a} & \subref{fig_back1_b} 
\end{tabular}
\caption{(\emph{a}): A C-language program fragment. (\emph{b}): The
  \emph{control-flow graph} (CFG) for the program.}
\label{fig_back1}
\end{myfig}

Figure \ref{fig_back1} shows a simple C program and its CFG. Each
\emph{node} in the graph represents a statement in the original
program. For example, nodes \ref{lst_back2_assigna} and
\ref{lst_back2_assignb} represent the assignment statements on line
\ref{lst_back1_assign}. Node \ref{lst_back2_entry}, the \emph{entry
  point}, designates where program execution begins. Nothing precedes
the entry point, and only one entry point exists in the graph. After
line \ref{lst_back1_print}, the program ends. Node
\ref{lst_back2_exit}, an \emph{exit point}, shows where execution
terminates. Unlike entry points, multiple exit points can exist in a
graph.

The \emph{directed edges} from node \ref{lst_back2_assigna} to node
\ref{lst_back2_assignb}, and from node \ref{lst_back2_assignb} to node
\ref{lst_back2_test} shows how the assignments on line
\ref{lst_back1_assign} precede the test on line
\ref{lst_back1_test}. Edges show the order in which nodes
execute. \emph{Predecessor} nodes always execute before
\emph{successor} nodes (except in the presence of loops).

The test on line \ref{lst_back1_test} can branch to line
\ref{lst_back1_test_true} or line \ref{lst_back1_test_false}. The
edges leaving node \ref{lst_back2_test} (representing the test
``\verb=if(a > b)='') show that execution can branch to either node
\ref{lst_back2_true} ($a > b$) or node \ref{lst_back2_false} ($a \leq
b$). A node followed by multiple successors (i.e., where multiple
edges leave the node) represents a \emph{branch} or \emph{conditional}
statement. Any one of the successor nodes may execute following the
conditional statement, depending on the condition tested.

Conversely, a node with multiple predecessors represents the
destination of multiple execution paths. The #printf# statement on
line \ref{lst_back1_print} always executes, regardless of the result
of the test on \ref{lst_back1_test}. Our CFG represents this by making
nodes \ref{lst_back2_true} and \ref{lst_back2_false} predecessors of
node \ref{lst_back2_print}.

\section{Basic Blocks}
%% Basic blocks
Consider the C-language fragment and CFGs in Figure
\ref{fig_back4}. Part \subref{fig_back4_b} shows the CFG for lines
\ref{lst_back3_start} -- \ref{lst_back3_end} in part
\subref{fig_back4_a}: a long, straight sequence of nodes, one after
another. Collapsing those nodes into one, as in part
\subref{fig_back4_c}, does not lose information and gives a more
compact representation. 

\afterpage{\clearpage{\begin{myfig}[th]
\begin{tabular}{ccc}
\subfloat{\begin{minipage}[t]{1.5in}
\begin{AVerb}[numbers=left]
int a = 1; \label{lst_back3_start}
int b = 2; 
int c = 3; 
int d = 4; \label{lst_back3_end}

if(a + d > b + c)
  \dots
else
  \dots
\end{AVerb}
\end{minipage}
\label{fig_back4_a}} \vline & %%
\subfloat{\begin{tikzpicture}[>=stealth, node distance=.5in]
  \node[entex] (entry) {};

  \node[stmt, below of=entry] (assigna) {#a = 1#\labelNode{lst_back4_assigna}};
  \node[labelfor=assigna] () {\refNode{lst_back4_assigna}};

  \node[stmt, below of=assigna] (assignb) {#b = 2#\labelNode{lst_back4_assignb}};
  \node[labelfor=assignb] () {\refNode{lst_back4_assignb}};

  \node[stmt, below of=assignb] (assignc) {#c = 3#\labelNode{lst_back4_assignc}};
  \node[labelfor=assignc] () {\refNode{lst_back4_assignc}};

  \node[stmt, below of=assignc] (assignd) {#d = 4#\labelNode{lst_back4_assignd}};
  \node[labelfor=assignd] () {\refNode{lst_back4_assignd}};

  \node[entex, below of=assignd] (exit) {};

  \draw [->>] (entry) to (assigna);
  \draw [->] (assigna) to (assignb);
  \draw [->] (assignb) to (assignc);
  \draw [->] (assignc) to (assignd);
  \draw [->>] (assignd) to (exit);

\end{tikzpicture}
\label{fig_back4_b}} \vline & %%
\subfloat{\begin{minipage}[t]{2in}
\begin{AVerb}
         E (1)
         |
         V
       -----(2)
      |a = 1|
      |b = 2|
      |c = 3|
      |d = 4|
       -----
         |
         V
 -----------------(3)
|if(a + d > b + c)|--|
 -----------------   V
         |          ---(4)
         V         |\dots|
     ---(5)         ---
    |\dots|--> X <---|
     ---
\end{AVerb}
\end{minipage}





\label{fig_back4_c}} \\
\subref{fig_back4_a} & \subref{fig_back4_b} & \subref{fig_back4_c} \\
\end{tabular}
\caption{\subref{fig_back4_a}: A C-language fragment to illustrate
  \emph{basic blocks}.  \subref{fig_back4_b}: The CFG for
  \subref{fig_back4_a} without basic blocks. \subref{fig_back4_c}: The
  CFG for \subref{fig_back4_c} using basic blocks.}
\label{fig_back4}
\end{myfig}
}\clearpage}

Each node in Figure \ref{fig_back4_c} represents a \emph{basic block}:
a sequence of statements with one entry, one exit, and no branches
in-between. Execution cannot start in the ``middle'' of the block, nor
can it branch anywhere but at the end of the block. Basic blocks are a
widely used representation for programs, found in standard compiler
textbooks (\citep{AhoXX}, \citep{MunchXX}, \citep{AppelXX}).

A basic block can consist of one statement -- a \emph{trivial}
block. For example, the nodes in Figures \ref{fig_back4_b} and
\ref{fig_back1_b} represent trivial basic blocks. We can always
convert a CFG of basic-blocks into a CFG of statements, if needed.
Dataflow optimization works as well with the ``collapsed'' nodes in
part \subref{fig_back4_c} as those in Figure
\subref{fig_back4_b}. Therefore, From this point forward we will work
exclusively with basic blocks, rather than individual statements.

\section{Facts \& The Meet Operator}
%% In & Out facts
Dataflow analysis associates every node in the CFG with two sets of
\emph{facts}. Facts describe the state of the machine before and after
execution of the statements represented by the node. \In facts
describe the machine's state beforehand, while \out facts
describe its state afterwards.

\afterpage{\clearpage\begin{myfig}[th]
\begin{tabular}{cc}
\subfloat{\begin{minipage}[t]{2.5in}
\begin{AVerb}[numbers=left]
int a = 1, b = 2, c;
if(a > b) 
  c = 4;
else     
  c = a + 3;

printf(c);
\end{AVerb}
\end{minipage}
%%
  \label{fig_back5_a}} \vline & 
\subfloat{\begin{tikzpicture}
  \node[entex] (entry) {};

  \node[stmt,
    below of=entry] (assign) {
    \begin{minipage}{.5in}
      \begin{AVerb}
a = 1
b = 1
      \end{AVerb}
    \end{minipage}\labelNode{lst_back7_assign}};
  \node[labelfor=assign] {\refNode{lst_back7_assign}};
  \node[above=5mm of assign, anchor=west] {$\mathit{in:} \{a=\bot, b=\bot, c=\bot\}$};
  \node[below=3mm of assign, anchor=west] {$\mathit{out:} \{a=1, b=2, c=\bot\}$};

  \node[stmt,
    below=.75in of assign] (test) {#if(a > b)#\labelNode{lst_back7_test}};
  \node[labelfor=test] {\refNode{lst_back7_test}};
  \node[above=3mm of test, anchor=east] {$\mathit{in(\refNode{lst_back7_test})}: \{a=1, b=2, c=\bot\}$};
  \node[below=3mm of test, anchor=east] {$\mathit{out(\refNode{lst_back7_test})}: \{a=1, b=2, c=\bot\}$};
  \node[below right=0mm and 0mm of test, anchor=west] {$\mathit{out(\refNode{lst_back7_test})}: \{a=1, b=2, c=\bot\}$};
  
  \node[stmt,
    right=2in of test] (true) {#c = 4#\labelNode{lst_back7_true}};
  \node[labelfor=true] {\refNode{lst_back7_true}};
  \node[above=3mm of true, anchor=east] {$\mathit{in:} \{a=1, b=2, c=\bot\}$};
  \node[below=3mm of true, anchor=west] {$\mathit{out:} \{a=1, b=2, c=4\}$};

  \node[stmt,
    below=.75in of test] (false) {#c = a + 3#\labelNode{lst_back7_false}};
  \node[labelfor=false] {\refNode{lst_back7_false}};
  \node[above=5mm of false, anchor=west] {$\mathit{in:} \{a=1, b=2, c=\bot\}$};
  \node[below=3mm of false, anchor=west] {$\mathit{out:} \{a=1, b=2, c= a + 3\}$};

  \node[stmt,
    below=.75in of false] (print) {#print(c)#\labelNode{lst_back7_print}};
  \node[labelfor=print] {\refNode{lst_back7_print}};
  \node[above=3mm of print, anchor=east] {$\mathit{in:} \{a=1, b=2, c=a + 3\}$};
  \node[above right=-1mm and 3mm of print, anchor=west] {$\mathit{in:} \{a=1, b=2, c=4\}$};
  \node[below=3mm of print, anchor=east] {$\mathit{out:} \{a=1, b=2, c=\top\}$};

  \node[entex, below=.5in of print] (exit) {};

  \draw [->>] (entry) to (assign);
  \draw [->] (assign) to (test);
  \draw [->] (test) to (true);
  \draw [->] (test) to (false);
  \draw [->] (true) |- (print);
  \draw [->] (false) to (print);
  \draw [->>] (print) to (exit);

\end{tikzpicture}
%%
  \label{fig_back5_b}} \\
\subref{fig_back5_a} & \subref{fig_back5_b} \\\rule{0pt}{24pt}
\end{tabular}
\caption{(\emph{a}): The same C-language fragment from Figure
  \ref{fig_back1_a}. (\emph{b}): The CFG
  for the program, annotated with \emph{in} and \emph{out} facts about the value of
  \texttt{a}, \texttt{b}, and \texttt{c} before and after
  each node.}
\label{fig_back5}
\end{myfig}
 
\clearpage}

Consider Figure \ref{fig_back5}, which annotates the program fragment
in Figure \ref{fig_back1} with the value of #a#, #b#, and #c# before
and after every node. The symbol ``$\bot$'' (``bottom'') indicates an
unknown value. Assigning $\bot$ to a variable means we do not know its
value. Prior to node \ref{lst_back7_assign}, none of the variables
values' can be known, which the fact $\{a : \bot,b : \bot,c : \bot\}$
shows.  They may be set to a compiler default, a random value in
memory, or even some previous value if this fragment is embedded in a
loop. After node \ref{lst_back7_assign}, our new fact, $\{a : 1,b : 2,c : \bot\}$
shows that we know that #a# equals 1, #b# equals 2, and that we stil do not know
the value of #c#.

At the entry node, we assign \inE facts a default value ($\bot$
in our example). Elsewhere, \inE facts come from predecessor
nodes. However, nodes with multiple predecessors can receive
conflicting facts. Consider the value for #c# given by nodes
\ref{lst_back7_true} and \ref{lst_back7_false}. Node
\ref{lst_back7_true} says #c# equals 4, while node
\ref{lst_back7_false} says #c# equals $a + 3$. We resolve the
situation by assigning #c# the value ``$\top$'' (``top''). As opposed
to $\bot$, $\top$ means we know a definite value for #c#, but we
cannot specify which.

The \emph{meet operator} defines how we combine facts as in the
previous situation. Table \ref{tbl_back2} shows how our meet operator,
$\sqcap$, combines facts. $v_1$ and $v_2$ represent values given to
the same variable facts by different facts. The meet operator replaces
$\bot$ with definite values, but replaces differing values with
$\top$. Though not illustrated in our example, the table also shows
that $\top$ values will always replace definite values.

\begin{table}[tbh]
  \centering
  \figbegin
  \begin{math}
    \begin{array}{cccc}
      v_1 & v_2 & v_1 \sqcap v_2 \\
      \cmidrule(r){1-1}\cmidrule(r){2-2}\cmidrule(r){3-3}
      \bot & v_2 & v_2 & \\ 
      v_1 & \bot & v_1 & \\
      \top & v_2 & \top & \\
      v_1 & \top & \top  &\\
      v_1 & v_2 & \top & \text{(when $v_1 \neq v_2$)}\\
      v_1 & v_2 & v_1 & \text{(when $v_1 = v_2$)}
    \end{array}
  \end{math}
  \caption{How the meet operator used in
    Figure \ref{fig_back5} combines facts. $v_1$ and $v_2$ are
    separate values given by separate facts to the same variable. The
    table shows how they are combined.}
  \label{tbl_back2}
  \figend
\end{table}

\section{Direction and The Transfer Function}
Returning to Figure \ref{fig_back5}, we need to define a transfer
function that can be applied to each node. The transfer function does
two things:
\begin{itemize}
\item When an assignment occurs, update the value of the assigned variable in the \out facts.
\item Otherwise, copy the variable's current value to the \out facts.
\end{itemize}
To help define our transfer function, we define the function |valueOf|,
which either returns the value assigned to a variable, or its value
from the \inE facts:
\begin{equation} |valueOf|(v) = 
  \begin{cases}
    |assign|(v) & \text{when $v$ is assigned a value in the node,} \\
    \text{\inE}(v) & \text{when $v$ is not assigned.} 
  \end{cases}
\label{eqn_back2}
\end{equation}
In the above, $v$ is a variable and |assign| retrieves the value assigned to
that variable, if any.
Our transfer function just needs to apply |valueOf| to all variables
in our \inE facts, as well as all variable assignments in the node
itself. If |assigned| is the set of all assigned variables in the
node, we can define how our transfer function relates \inE and \out using
set notation:
\begin{equation}
  \text{\out} = [|valueOf|(v) || v \in (\text{\inE} \cup |assigned|)].
\end{equation}

Dataflow analysis computes \inE and \out facts by repeatedly applyhing
a \emph{transfer function} in a particular \emph{direction}. In Figure
\ref{fig_back5}, we compute \out facts from by combining \inE facts
with any assignment statements in the node; we used a \emph{forwards}
analysis. A \emph{backwards} analysis computes \inE facts from \out
facts. Our example in Section \ref{sec_back2} describes one particular
backwards analysis in detail -- we will just illustrate a forwards
analysis here.

%% Our previous example showed \inE and \out facts on each
%% node, but did not specify how to compute those facts. For any given
%% analysis, we compute facts based on our \emph{analysis direction} and
%% \emph{transfer function}. The \emph{transfer function} specifies how we 
%% create facts from statements in the program. Our \emph{direction} can 
%% be \emph{forwards} or \emph{backwards}. A forwards analysis uses 
%% the transfer function to compute \out facts from \inE facts. Conversely,
%% a backwards analysis computes \inE facts using \out facts. 

%% Returning to Figure \ref{fig_back5}

%% \We used a forwards analysis to compute the \out facts in Figure
%% \ref{fig_back5}.

%% Dataflow analysis applies a ``transfer function'' to each node in the
%% CFG to compute ``facts'' about the program's state before and after
%% the execution of each node. The facts computed and the transfer
%% function used depend on the specific optimization, but dataflow
%% analysis always applies them in the same way. Each node has two sets
%% of facts -- ``in'' and ``out.'' The transfer function uses the ``in''
%% facts to compute ``out'' facts for a node. ``Out'' facts on a node
%% become ``in'' facts on the node analyzed next.  If the CFG for a
%% program contains loops, then ``in'' facts for a node may change based
%% on later ``out'' facts. The transfer function will be applied
%% repeatedly until the facts stop changing -- they reach a ``fixed
%% point.''

%% Because we represent the CFG for a program as a directed graph, we can
%% choose which direction to traverse the CFG -- forwards or backwards.
%% When traversing forward, we usually compute facts about program
%% execution past a certain point (e.g., does a variable's value
%% change?); a backwards analysis computes facts up to a certain point
%% (e.g., what variables will be referenced following a given
%% statement?). Where a forwards analysis begins at the entry point(s)
%% for the CFG, a backwards analysis begins at the exit points.

\section{Iterative Analysis}
%% Iterative Analysis & Fixed points
As we saw in Figure \ref{fig_back5}, facts can conflict
when nodes have multiple predecessors. Even more complicated
situations arise when a program contains loops. Consider
the fragment in \ref{fig_back6}. How do we apply
our |valueOf| function (Equation \ref{eqn_back2}) to a graph
where node X is both a successor and a predecessor?

Fortunately, \citep{SoAndSoXX} show that if the facts computed form a
\emph{lattice}, and our transfer function is \emph{monotonic} then we
can apply our transfer function over and over and know that, at some
point, the facts will stop changing.  A monotonic function always
increases. In our terms, it must always add information, and never
take any away. Our |valueOf| function always \emph{adds} to \out: a
variable will either have the value it did before (i.e., it was not
assigned) or it will be assigned a new value. Values that form a
lattice have some sort of ordering. In our case, we can say that $\bot
\leq v \leq \top$, for all values $v$. If facts conflict, the $\sqcup$
operator described in Table \ref{tbl_back2} ensures they will all
eventually equal $\top$, at which point they will no longer conflict. Values
will not endlessly cycle back and forth -- they eventually stop. 

\begin{myfig}
\begin{tabular}{cc}
  \subfloat{\begin{minipage}[t]{2in}
\begin{AVerb}[numbers=left]
int c = 0;
while(c < 10)
  c += 1;
print(c);
\end{AVerb}
\end{minipage}
%%
    \label{fig_back6_a}} \vline &%%
  \subfloat{\begin{minipage}[t]{2in}
\begin{AVerb}

\end{AVerb}
\end{minipage}
%%
    \label{fig_back6_b}} \\ 
  \subref{fig_back1_a} & \subref{fig_back1_b}
\end{tabular}
\caption{\subref{fig_back6_a}: A simple C-language program with a loop. \subref{fig_back6_b}: The CFG 
for the fragment.}
\label{fig_back6}
\end{myfig}

\section{Rewriting}

%% Rewrite based on analysis
Direction, the meet operator, facts, and the transfer function
together define the optimizing function applied by dataflow analysis
for a particular optimization. The result of the analysis is then used
to alter, or ``rewrite,'' the CFG. The meaning of the new program will
not be different than the old, but it will behave differently: execute
faster, use less memory, or whatever characteristic the optimization
should improve.

\section{Example: Dead-Code Elimination}
\label{sec_back2}

Consider Figure \ref{fig_back2}, again showing a C-language fragment.
After assignment on line \ref{fig_back2_dead_line}, #b# is not
referenced. Removing the #b# will not affect the program and,
if nothing else, will reduce the size of the program. It may even make
it run faster or use less memory. We call this optimization
\emph{dead-code elimination}.

\begin{myfig}[ht]
\begin{minipage}{1in}
  \begin{Verbatim}[numbers=left,commandchars=\\\{\}]
    a = 1;
    b = a + 1;\label{fig_back2_dead_line}
    return a + 1;
  \end{Verbatim}
\end{minipage}
\caption{A C-language fragment illustrating \emph{dead code}. After
assignment on line \ref{fig_back2_dead_line}, \verb=b= is not used
and can be considered ``dead.''}
\label{fig_back2}
\end{myfig}

Of course, people do not normally write programs with such obviously
useless statements, but other compiler optimizations can produce (or
leave behind) many such statements. \emph{Uncurrying}, described in
Chapter \ref{ref_chapter_uncurrying}, in fact depends on dead-code elimination.

To eliminate the assignment like that on line
\ref{fig_back2_dead_line}, we really need to determine which variables
are referenced after assignment. Such variables are ``live''; if a
variable is \emph{not} live, then it is dead. We use this ``liveness''
analysis to determine if a particular assignment is dead.

To determine if a variable is live, we need to know if it is
referenced after assignment.  Such variables make up the \emph{the
  live set} which we can compute between each statement. To compute
the live set, we can choose to traverse the CFG for the program forwards or
backwards.  In the forwards case, we must track each assignment and
determine, when we exit the fragment, if the variable was used
afterwards. In general we would need to track every assignment until
our traversal finished. However, if we traverse backwards, we only
need to note any reference to a variable. When we see an assignment to
a variable \emph{not} in our live set, we know it will not be
referenced afterwards. Therefore we compute ``liveness'' using a
backwards traversal over the CFG.

\begin{myfig}[th]
\begin{minipage}{2in}
\begin{Verbatim}[commandchars=\\\{\}]
       E
       ||      
       v
     -----
    ||a = 1||    \emph{live:}  \ensuremath{\emptyset}
     -----
       ||      
       V
   ---------
  ||b = a + 1||  \emph{live:} \{a\}  
   ---------
       ||      
       V
  ------------
 ||return a + 1|| \emph{live:} \{a\}
  ------------
       ||      
       X          \emph{live:}  \ensuremath{\emptyset}
\end{Verbatim}
\end{minipage}
\caption{The CFG for our example program, annotated with the live
set for each node.}
\label{fig_back3}
\end{myfig}

Figure \ref{fig_back3} shows the CFG for this example, with annotations
between each statement showing the live set. Though
execution follows the arrows in the CFG, our analysis proceeds
backwards. For example, the input to node 2 is the live set computed
for node 3 (``$\{a\}$'' in this case).

Our transfer function computes the live set based on \emph{uses} and
\emph{definitions} in a statement. Any reference (or use) of a
variable goes into the live set. Any assignment (or definition) of a
variable removes it from the live set. We can then define our transfer
function, |live|, for a statement as:

\begin{align}
  & |live|(s) = (\Varid{in}(s) \cup |use|(s)) - |def|(s), \label{eqn_back1} \\
\intertext{where}
  & s     & \text{Statement considered.} \notag\\
  & |use|(s) &  \text{Set of variables used in $s$}. \notag\\
  & |def|(s) & \text{Variable assigned to in $s$ (a singleton set)}. \notag\\
  & |in|(s) & \text{Live variables computed for $s$' successor}. \notag
\end{align}

Table \ref{tbl_back1} shows the |use| and |def| sets for each
statement. The live set computed, |live|, becomes the input, |in|, for
the statement's predecessor. We include the exit node (``#X#'') in the
table to show the initial value of |in| for the last statement --
$\emptyset$, the empty set. Our analysis then works backwards through the
program. If our program (and its CFG) contained any loops, we would
need to run this algorithm multiple times, until the live set for each
statement reached a fixed point.

\begin{table}
  \centering
  \begin{tabular}{lcccc}
    $s$ & $|use|(s)$ & $|def|(s)$ & $|in|(s)$ &  $|live|(s)$ \\
    \cmidrule(r){1-1}\cmidrule(r){2-2}\cmidrule(r){3-3}\cmidrule(r){4-4}\cmidrule(r){5-5}
    #X# & & & & $\emptyset$ \\
    #return a + 1# & $\{a\}$ & $\emptyset$ & $\emptyset$ & $\{a\}$ \\
    #b = a + 1# & $\{a\}$ & $\{b\}$ & $\{a\}$ & $\{a\}$ \\
    #a = 1# & $\emptyset$ & $\{a\}$ & $\{a\}$ & $\emptyset$ \\
    \bottomrule
  \end{tabular}
  \caption{The $|use|$, $|def|$ and $|live|$ sets computed using equation \ref{eqn_back1} for our example program.}
  \label{tbl_back1}
\end{table}

With the live set computed for each statement, our analysis can now
determine which statements to eliminate. Only nodes 1 and 2 in Figure
\ref{fig_back3} perform an assignment. The live set for node 1 (``#a = 1#'')
contains #a#, so we do not eliminate it. In node 2 (``#b = a + 1#''),
the live set does \emph{not} contain #b#. Therefore, we can eliminate
node 2, giving us a new program without any dead code:

\begin{Verbatim}
a = 1;
return a + 1;
\end{Verbatim}

\section{Conclusion}

This chapter gave an overview of \emph{dataflow optimization}, a
technique we used extensively in our work. The dataflow
\emph{algorithm} gives a general technique for applying an
\emph{optimizing function} to the \emph{control flow graph} (CFG)
representing a give program. The optimizing function computes
\emph{facts} about each node in the graph, using a \emph{transfer}
function to turn input facts into output facts. The CFG can be
traversed forwards or backwards (depending on the particular
optimization), and it may need to be traversed many times until the
computed facts reach a \emph{fixed point}.  Each optimization defines
a specific \emph{meet operator} that combines facts for nodes with
multiple inputs. Finally, the facts computed are used to
\emph{rewrite} the CFG, transforming the program so it still has the
same meaning, but behaves better, according to the optimization used.


%% \subsection{Basic Blocks and Control-Flow Graphs}

%% A dataflow optimization operates over a ``control-flow graph'' (CFG)
%% of the program -- a directed graph where edges encode branches or
%% jumps and nodes represent statements. Programs run by entering a node
%% from a predecessor, executing the statements in turn, and exiting the
%% node to a successor. Multiple successors imply a conditional branch,
%% though the program can only choose one. A special ``entry'' node, with
%% no predecssors, exists to give the program a starting point.

%% The statements in each node must define a ``basic block,'' which means
%% there can only be one entry and one exit to the node. Each
%% predeccessor starts at the same statement; execution cannot start in
%% the ``middle'' of the statements in the node. Each successor also
%% leaves from the same instruction, so only one ``branch'' can exist in
%% each node.

%% For example, consider the ``fall-through'' implied by the use of #case#
%% statements in this C-language program fragment:

%% \begin{verbatim}
%%   switch(i) {
%%   case 1:
%%     printf("1");
%%     break;
%%   case 2:
%%     printf("2");
%%   case 3:
%%     printf("3");
%%   }
%% \end{verbatim}

%% \begin{figure}[h]
%% \begin{verbatim}
%%    A
%%   switch   ----<-
%%   | |  |  |      |
%%   | |  |  v C    ^
%%   | |   ->case 3 |
%%   | |     |      |
%%   | |      ->----_--
%%   | | B          |  |
%%   |  ->case 2 ->-   v
%%   |                 |
%%   |   D       ----<-
%%    ->case 1  |
%%      |       v
%%      v       |
%%    --+-----<-
%%   |
%%    -> ...
%% \end{verbatim}
%% \caption{CFG illustrating \emph{fall-through} allowed by the
%%   C-language \texttt{switch} statement.}
%% \label{switchCfgEg}
%% \end{figure}

%% Figure \ref{switchCfgEg} shows a CFG for this fragment. Execution
%% begins at node A. Node C has two predeccessors: A and B. The edge
%% between Node B and C represents fall-through from the second to third
%% case. They cannot be combined because the node would need two distinct
%% entry points. Encoding a program into basic blocks usually involves
%% inserting similar branches. The CFG makes explicit control--flow that
%% exists by implication in the source program.

%% \subsection{Direction, Facts and Rewrites}

%% \subsection{Example: Bind/Return Collapse}

%% Dataflow optimizations transform the CFG representation of a program,
%% with the goal of making a faster (or smaller, or more efficient, etc.)
%% program. Dataflow computes a set of ``entry'' assumptions and ``exit''
%% facts for each node in the graph. Facts for one node become
%% assumptions for the nodes' successors (thus the term
%% ``dataflow''). The algorithm iteratves over the entire graph until a
%% fixed point is reached -- that is, facts and assumptions no longer
%% change. The computed facts can then be used to transform the graph.

%% \emph{Constant propagation example -- or something more functional?}

%% \emph{Introduce forward and backwards dataflow.}

% What does dataflow mean?

% How do you use it?

% Example

\end{document}

% LocalWords:  Dataflow dataflow CFG printf variable's CFGs ccc Uncurrying
% LocalWords:  liveness
