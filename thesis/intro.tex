%&preamble
%% Looking at LaTeX sources reveals \usepackage and
%% \RequirePackage are made equivalent when \documentclass
%% executes, so we can test if \documentclass has
%% been executed by comparing those two control sequences.
\def\ifnodocclass{\def\loaded{\iffalse}%
  \def\notloaded{\iftrue}%%
  \let\which=\loaded%%
  \ifx\usepackage\undefined\let\which=\notloaded%%
  \else%%
    \ifx\usepackage\RequirePackage%%
    \else\let\which=\notloaded%%
    \fi%%
  \fi\which}
\dodocclass
\begin{document}
\ifthenelse{\boolean{standaloneFlag}}
           {\VerbatimFootnotes
             \DefineShortVerb{\#}
             \setcounter{chapter}{0}}{}

%% Default float parameters. For case when
%% multiple chapters are included and
%% only one needs custom float settings.
\renewcommand{\textfraction}{0.2}
\renewcommand{\textfraction}{0.2}
\renewcommand{\topfraction}{0.9}


\chapter{Introduction}

\intent{Introduce the algorithm and what it does.}
The \emph{dataflow algorithm} treats programs as \emph{control-flow}
graphs, with edges representing execution paths and nodes representing
statements in the program. A particular dataflow analysis computes
some desired property for each node in the graph based on a static
approximation of the program's run-time behavior; the results of the
analysis can be used to transform the program. In most cases this
transformation optimizes the program according to some measure, such
as making it run faster, using less memory, or consuming less power.

\intent{Describe traditional application of dataflow analysis to imperative programs.}
Kildall, who first described the dataflow algorithm, used it to
optimize programs written in \textsc{algol 60},
an imperative programming language. For example, he implemented
\emph{constant propagation}, where program variables with constant values
are replaced by the values themselves, and \emph{common-subexpression
elimination}, which transforms the program so that it only computes
redundant expressions once. Research and refinements since then
continued to focus on imperative languages, as the algorithm applies
naturally to languages with explicit control-flow.

\intent{Contrast dataflow analysis with rewrite based transformations} In
contrast, much research on the analysis and optimization of programs
written in functional programming languages focuses on algebraic,
rewrite-based techniques. This approach searches for syntactic
patterns in the program's text and rewrites those patterns according
to some set of rules. The goal remains the same, namely producing a
program optimized according to some criteria. 

\intent{Introduce idea of dataflow analysis over MIL.}
Dataflow analysis on imperative programs arises naturally due to the
explicit flow-of-control from statement to statement. For functional
languages, with flow-of-control determined by evaluation order, the
fit seems more awkward. However, \emph{monadic} functional programs
embody the best of both styles: expression-based evaluation \emph{and}
explicit control-flow.

\intent{Thesis: Interesting things happen when we apply dataflow analysis to MIL.}
We propose that dataflow analysis can be used quite effectively on
programs written in our \emph{Monadic Intermediate Language} (\mil), a
pure, monadic, functional language of our own design. We will show
that dataflow analysis can be used to implement functional-language
specific optimizations and that our MIL lends itself to established
dataflow analyses, such as dead-code elimination.

\intent{Signposts for dataflow chapter.} Chapter~\ref{ref_chapter_background}  gives 
a thorough introduction to the dataflow algorithm in its traditional
setting. We explain the algorithm by
applying \emph{constant-propagation} to a simple C program. We
introduce fundamental definitions used throughout this thesis, such as
control-flow graphs and basic blocks. We discuss the theoretical basis
of the algorithm, including its correctness and quality of
solutions. Finally, we give the \emph{dataflow equations} that can be
used to describe any particular dataflow analysis.

\intent{Signposts for hoopl chapter.} In Chapter~\ref{ref_chapter_hoopl},
we describe how to use \hoopl \citep{Hoopl-3.8.7.0}, a Haskell library
for implementing dataflow-based optimizations that we used to
implement all dataflow analyses described in this document. We use the
same structure for this Chapter as the previous, making the connection
between the more abstract description of the dataflow algorithm and
its use explicit. Chapter~\ref{ref_chapter_hoopl} uses \emph{dead-code
elimination} for a subset of the C programming language as a running
example. In Appendix~\ref{ref_appendix_deadcode}, we give the complete source
code for the program described.

\intent{Signposts for MIL chapter.}

\intent{Signposts for Uncurrying chapter.}

\intent{conclude}

Compilers for imperative languages implement many optimizations using
\emph{dataflow analysis}. This method treats the program as a graph,
where edges represent execution paths and nodes represent statements
in the program. Dataflow analysis computes facts about each node and
then transforms the graph into an equivalent, yet faster (or smaller,
or more efficient, etc.) program. Optimizations which use dataflow
analysis include constant propagation, dead-code elimination,
common-subexpression elimination and many others.

Dataflow analysis on imperative programs arises naturally due to the
explicit flow-of-control from statement to statement. For pure
functional languages, with flow-of-control determined by evaluation
order, the fit seems more awkward. However, call-by-value, pure,
\emph{monadic} functional programs embody the best of both
styles: expression-based evaluation \emph{and} explicit
control-flow. 

Our work, then, defines a monadic language and optimizes programs in
it using dataflow analysis. We implement a number of optimizations
common to imperative and functional languages, including constant
propagation and dead-code elimination. We implement an optimization
which eliminates intermediate closure construction, showing that this
technique can be used for optimizations specific to functional
languages.


\end{document}
