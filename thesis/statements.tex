\documentclass[11pt]{article}
\usepackage{url}
\usepackage{fancyvrb}
\usepackage{setspace}
\newcommand{\Ques}[1]{\noindent%
\textbf{Question:} #1}
\begin{document}
\VerbatimFootnotes
\DefineShortVerb{\#}
\doublespacing

\Ques{Can we use a monadic intermediate language to compile
  ``bare-metal'' programs such that no heap allocation occurs?}

\bigskip

I will explore efficient compilation techniques for Habit programs
using a mondaic intermediate language. My research will focus on
minimizing or eliminating the allocation of two types of data:
algebraic data types and function closures. The #prioset# example from
the Habit language report will motivate my research, as well as other
low-level core operating system algorithms.

A successful project will show that the program can be compiled such
that no heap memory gets allocated --- all data will be stored in
registers or on the stack. The code will not require special
annotations; that is, the optimizations implemented will be general
purpose. Other motivating examples will need to be found to validate
that the optimizations are truly general.

\bigskip

\Ques{How efficiently can we compile \Verb=bitdata= expressions?}

#bitdata# provides a new way to express bit-level operations using
high-level, abstract expressions. It replaces traditional masking,
shifting and ``bit-twiddling'' with pattern-matching and data
constructors similar to those in Haskell. 

This research will explore efficient compilation techniques for #bitdata#
expressions. I will research well-known bit-twiddling algorithms and implement
them as compiler transformations. A successful project will present novel 
techniques for automatically applying known algorithms and algebraic properties
to #bitdata# expressions.

\end{document}
