\usepackage[T1]{fontenc}
\usepackage{fourier}
%% \usepackage{palatino}
%% \renewcommand\ttdefault{cmtt}
\usepackage{helvet}
\usepackage{inconsolata}
\usepackage{comment}
\usepackage{calc}
\usepackage{xspace}
\usepackage{verbatim}
\usepackage{url}
\usepackage{fancyvrb}
\usepackage{setspace}
\usepackage{amsmath}
\usepackage{booktabs}
\usepackage[margin=\parindent, format=hang,labelfont=bf]{caption}
\usepackage{ifthen}
\usepackage{stmaryrd}
\usepackage{longtable}
\usepackage{afterpage}
\usepackage{subfig}

\usepackage{tikz}
\usetikzlibrary{arrows,automata,positioning}
%% Used for CFGs.
\tikzset{
  >=stealth, 
  node distance=.5in,
  stmt/.style={rectangle,
    draw=black, thick,        
    minimum height=2em,
    %% inner sep=2pt,
    %% text centered,
    %% node distance=.5in,
  },
  entex/.style={
    minimum height=2em,
    %% inner sep=2pt,
    %% text centered,
  },
  labelfor/.style={circle, 
    draw=black, thin,
    font={\footnotesize},
    inner sep=0,
    fill=white,
    above right=-1.5mm and -1.5mm of #1,
  },
}

%% GSO margins.
\usepackage[left=1.5in, right=1in, top=1in, bottom=1in]{geometry}
\usepackage{abstract}

%% GSO requires 12 pt font for all headings
\usepackage[sf, bf, tiny]{titlesec}
\titleformat{\chapter}[hang]{}% format
 {\sffamily\bfseries\thechapter}
 {1em}
 {\sffamily\bfseries}
\titlespacing{\chapter}{}{}{2ex}

\hyphenation{data-flow}

\newboolean{lhs2tex}
\setboolean{lhs2tex}{true}

% Used by included files to know they
% are NOT standalone
\newboolean{standaloneFlag}
\setboolean{standaloneFlag}{true}

\newlength{\rulefigmargin}
\setlength{\rulefigmargin}{2\parindent}

\newcommand\figbegin{\rule{\linewidth-\rulefigmargin}{0.4pt}\\\vspace{12pt}}
\newcommand\figend{\rule{\linewidth-\rulefigmargin}{0.4pt}}

\newcommand{\citep}[1]{(\emph{#1})\xspace}
\renewcommand{\cite}[1]{\emph{#1}\xspace}

%% Functional languages chapter commands
\newcommand{\lamA}{\ensuremath{\lambda}-calculus\xspace}
\newcommand{\LamA}{\ensuremath{\lambda}-Calculus\xspace}
\newcommand{\lamAbs}[2]{\ensuremath{\lambda#1.\ #2}}
\newcommand{\lamApp}[2]{\ensuremath{#1\ #2}}
\newcommand{\lamPApp}[2]{\ensuremath{(#1\ #2)}}
\newcommand{\lamAppP}[2]{\ensuremath{(#1)\ #2}}
\newcommand{\lamCompose}{\lamAbs{f}{\lamAbs{g}{\lamAbs{x}{\lamApp{f}{\lamApp{g}{x}}}}}}
\newcommand{\machLam}{\ensuremath{M_\lambda}\xspace}
\newcommand{\compMach}[1]{\ensuremath{\llbracket #1 \rrbracket}}
\newcommand{\compRho}[1]{\ensuremath{\rho(#1)}}
\newcommand{\verSub}[2]{\enusremath{#1_{#2}}}
\newcommand{\verSup}[2]{\enusremath{#1_{#2}}}
%% End functional languages chapter

%% Dataflow chapter commands
\newcounter{nodeCounter}[figure]
\newcommand{\inE}{\emph{in}\xspace}
\newcommand{\out}{\emph{out}\xspace}
\newcommand{\In}{\emph{In}\xspace}
\newcommand{\Out}{\emph{Out}\xspace}
\newcommand{\inB}{\ensuremath{\mathit{in}(B)}\xspace}
\newcommand{\outB}{\ensuremath{\mathit{out}(B)}\xspace}
\newcommand{\entryN}{\emph{E}\xspace}
\newcommand{\exitN}{\emph{X}\xspace}
\newcommand{\refNode}[1]{\ensuremath{B_{\ref{#1}}}\xspace}
\newcommand{\labelNode}[1]{\refstepcounter{nodeCounter}\label{#1}}
%% End dataflow

\newenvironment{myfig}[1][h]{\begin{figure}[#1]%%
\centering%%
\figbegin}{\figend%%
\end{figure}}

% single-argument comment. Do not put
% a space before the command when used
% or the file will have two spaces.
\newcommand{\rem}[1]{}

%% A verbatim environment with active charactesr
%% so we can use math shortcuts and macros
\DefineVerbatimEnvironment{AVerb}{Verbatim}{commandchars=\\\{\},%% 
  codes={\catcode`\_8\catcode`\$3\catcode`\^7},%%
  numberblanklines=false}

