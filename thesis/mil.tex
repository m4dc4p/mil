\documentclass[12pt]{report}
%include polycode.fmt
\usepackage[T1]{fontenc}
\usepackage{calc}
\usepackage{palatino}
\usepackage{amsfonts}
\renewcommand\ttdefault{lmtt}
\usepackage{helvet}
\usepackage{xspace}
\usepackage{url}
\usepackage{fancyvrb}
\usepackage[doublespacing]{setspace}
%% below only necessary when using doublespacing -- corrects
%% the vertical space inserted when switching to singlespace
%% environment.
\def\correctspaceskip{\vskip-\baselineskip} 
\usepackage{amsmath}
\usepackage{booktabs}
\usepackage[margin=\parindent, format=hang,labelfont=bf]{caption}
%% \usepackage[subrefformat=parens]{subcaption}
%% The following makes sure we get parentheses around
%% subreferences. The newest version of the subcaption
%% package has an option for this, but that's not available
%% widely.
%%
%% From http://tex.stackexchange.com/questions/25644
\usepackage[labelformat=simple]{subcaption}
\makeatletter
  \def\thesubfigure{(\alph{subfigure})}
  \providecommand\thefigsubsep{~}
  \def\p@subfigure{\@nameuse{thefigure}\thefigsubsep}
\makeatother

\usepackage{ifthen}
\usepackage{stmaryrd}
\usepackage{longtable}
\usepackage{afterpage}
\usepackage{xifthen}
\usepackage{mathtools}
\usepackage[natbib=true,style=authoryear,backend=bibtex8]{biblatex}
\setlength{\bibitemsep}{\bigskipamount}
\addbibresource{thesis.bib}
\usepackage{microtype}

%% GSO margins.
\usepackage[left=1.5in, right=1in, top=1in, bottom=1in]{geometry}
\usepackage{abstract}

%% GSO requires 12 pt font for all headings
\usepackage[bf,sf,tiny,compact]{titlesec}
\titleformat{\chapter}[display]
            {}% format
            {\sffamily\bfseries\chaptertitlename\ \thechapter}
            {\baselineskip}
            {\sffamily\bfseries}
            {}

\hyphenation{data-flow mo-na-dic} 

%% Should unindent all haskell code set in a dispay (versus inline)
\makeatletter
  \@ifundefined{hscodestyle}
               {}
               {\renewcommand{\hscodestyle}{\advance\leftskip -\mathindent}}
\makeatother

% Used by included files to know they
% are NOT standalone
\newboolean{standaloneFlag}
\setboolean{standaloneFlag}{true}

\newlength{\rulefigmargin}
\setlength{\rulefigmargin}{2\parindent}

\newcommand\figbegin{\rule{\linewidth}{0.4pt}\\\vspace{12pt}}
\newcommand\figend{\rule{\linewidth}{0.4pt}}

%% Sets
\newcommand{\setL}[1]{\textsc{#1}\xspace}
\newcommand{\setLC}{\setL{Const}}

%% Lub, subset operators.
\protected\def\lub{\ifmmode\sqcap\else\raisebox{.1em}{\ensuremath{\sqcap}}\fi\xspace}
\newcommand{\sqlt}{\ensuremath{\sqsubset}\xspace}
\newcommand{\sqlte}{\ensuremath{\sqsubseteq}\xspace}

%% Subscripting with typewriter
\def\subtt#1{\ifmmode_{\ensurett{#1}}%%
  \else$_{\ensurett{#1}}$%%
  \fi}
%% Superscripting with typerwriter
\def\suptt#1{\ifmmode^{\ensurett{#1}}%%
  \else$^{\ensurett{#1}}$%%
  \fi}
%% Functional languages chapter commands
\newcommand{\lamA}{\ensuremath{\lambda}-calculus\xspace}
\newcommand{\LamA}{\ensuremath{\lambda}-Calculus\xspace}
\newcommand{\lamAbs}[2]{\ensuremath{\lambda#1.\ #2}}
\newcommand{\lamApp}[2]{\ensuremath{#1\ #2}}
\newcommand{\lamPApp}[2]{\ensuremath{(#1\ #2)}}
\newcommand{\lamAPp}[2]{\ensuremath{(#1)\ #2}}
\newcommand{\lamApP}[2]{\ensuremath{#1\ (#2)}}
\newcommand{\lamAPP}[2]{\ensuremath{(#1)\ (#2)}}
\let\lamApPp=\lamApP
\let\lamAppP=\lamAPp
%% LC definition
\newtoks\toksA
\protected\def\lcname#1/{\ensuremath{\mathit{#1}}}
\protected\def\lcdef#1(#2)=#3;{\def\arg{#2}%%
  \def\lcargs##1,##2/{\def\arg{##2}%%
    \ifx\empty\arg%%
    \lcname ##1/%%
    \else\lcname ##1/\ \lcargs ##2/%%
    \fi}%%
  \ifx\empty\arg\toksA={\ }%%
  \else\toksA={\ \lcargs #2,/\ }%%
  \fi%%
  \ensuremath{\lcname#1/\the\toksA =\ #3}}
%% Arbitary number of applied arguments, separated
%% by asterisks (*).
\protected\def\lcapp#1/{\def\lcappB##1*##2/{\def\arg{##2}%
    \ensuremath{\ifx\arg\empty%%
      \lcname ##1/%%
      \else%%
      \lcname##1/\ \lcappB##2/%%
      \fi}}%%
  %% Adding a star here makes
  %% sure our applicaitn always ends with an asterisks, ensuring
  %% #2 will be \empty at some point.
  \lcappB#1*/}
\protected\def\lcabs#1.{\ensuremath{\lambda#1.\ }}

\newcommand{\lamId}{\lamAbs{x}{x}}
\newcommand{\lamCompose}{\lamAbs{f}{\lamAbs{g}{\lamAbs{x}{\lamApp{f}{(\lamApp{g}{x})}}}}}
\newcommand{\machLam}{\ensuremath{M_\lambda}\xspace}
\newcommand{\compMach}[1]{\ensuremath{\left\llbracket #1 \right\rrbracket}}
\newcommand{\compRho}[1]{\ensuremath{\rho(#1)}}
\newcommand{\verSub}[2]{\ensuremath{#1_{#2}}}
\newcommand{\verSup}[2]{\ensuremath{#1^{#2}}}
\newcommand{\lamC}{\ensuremath{\lambda_C}\xspace}
\newcommand{\lamPlus}{\lamAbs{m}{\lamAbs{n}{\lamAbs{s}{\lamAbs{z}{\lamApp{m}{\lamApPp{s}{\lamApp{n}{\lamApp{s}{z}}}}}}}}}
%% Substitution notation -- [#1 -> #2]
\newcommand{\lamSubst}[2]{\ensuremath{[#1 \mapsto #2]}}
%% End functional languages chapter


%% MIL Chapter
\newcommand{\compMILE}[1]{\ensuremath{\left\llbracket #1 \right\rrbracket}}
\newcommand{\compMILV}[1]{\ensuremath{\left\llbracket #1 \right\rrbracket}}
\newcommand{\compMILQ}[2]{\ensuremath{\left\llbracket #2 \right\rrbracket}}
\newcommand{\milCtx}[1]{\ensuremath{\llfloor}#1\ensuremath{\rrfloor}}

%% This dimension makes sure the same amount of space
%% follows | and := in syntax rules like:
%%
%% term := var       (Variable)
%%      |  var var    (Application)
%%      |  \x. var    (Abstraction)
%%
\newdimen\termalign
\setbox0=\hbox{$:=$}
\termalign=\wd0 
\protected\def\term#1/{\ensuremath{\mathit{#1}}}
\protected\def\syntaxrule#1/{\hfil\text{\emph{#1}}}
\protected\long\def\termrule#1:#2:#3/{\term #1/ &\hbox{$:=$}\ensuremath{\ #2} & \syntaxrule #3/}
\protected\def\termcase#1:#2/{&\hbox to \termalign{$|$\hss}\ensuremath{\ #1} & \syntaxrule #2/}


%% End MIL chapter

%% Dataflow Chapter
% Domain function
\def\dom(#1){\ensuremath{\mfun{dom}(#1)}\xspace}
% Set of all integers.
\def\ZZ{\ensuremath{\mathbb{Z}}}
%%

%% Uncurrrying Chapter 
%% A space equal to a \thinspace, but we
%% can break a line at it.
\newskip\thinskipamt \thinskipamt=.16667em 
\protected\def\thinskip{\hskip \thinskipamt\relax}
\protected\def\thinnerskip{\hskip .5\thinskipamt\relax}
%% Capture a space token. Use a ``control-symbol'' (\. instead of \mksp)
%% to keep the trailing space from getting gobbled.
{\def\.{\global\let\sp= } \. }
%% Define \asp, which will capture the macro definition attached to space,
%% if one exists. Otherwise, \spa is relax after this.
{\catcode`\ =\active\gdef\asp{\ifx \relax\let\spa\relax\else\let\spa= \fi}}
\newtoks\foo
%% Removes spaces, implicit, active and explicit.
\protected\def\removespaces{\asp\afterassignment\removesp\let\next= }
\def\removesp{\foo={\next}\ifcat\noexpand\next\sp\foo={\removespace}%%
 \else\ifx\next\spa\foo={\removespaces}\fi%%
 \fi\the\foo}
%% MIL reserved word
\protected\def\milres#1/{\text{\ttfamily\bfseries #1}}
\protected\def\lab#1/{\textbf{\ensurett{\removespaces #1}}}
%% Constructs a closure: l { v1, ..., vN }
\protected\long\def\mkclo[#1:#2]{\lab #1/\ensuremath{\,\{\ensurett{#2}\}}\xspace}
%% Tuple version of closurs: {l: v1, ..., vN}.
\protected\long\def\clo[#1:#2]{\def\argA{#1}\def\argB{#2}\ensuremath{\{%%
      \ifx\argA\empty%%
      \else\lab #1/%%
        \ifx\argB\empty%%
        \else\ensurett{:\thinskip}%%
        \fi%%
      \fi\ensurett{#2}\}}\xspace}
%% Construct a thunk
\newbox\bracklbox \newbox\brackrbox
\setbox0=\hbox{$\{$} \setbox\bracklbox=\hbox to \wd0{\hfil[\kern0.25mm}
\setbox0=\hbox{$\}$} \setbox\brackrbox=\hbox to \wd0{\kern0.25mm]\hfil}
\protected\def\mkthunk[#1:#2]{\lab #1/%%
  \ensuremath{\,%%
    \mathopen{\copy\bracklbox}%%
    \ensurett{#2}%%
    \mathclose{\copy\brackrbox}\xspace}}
%% Binding statement: v <- {...}
\protected\def\binds#1<-#2;{\ensurett{\removespaces #1\texttt{<-}#2}\xspace}
%% In order to use \binds in verbatim environment, have to define
%% delimiters while they are active. The below defines \vbinds which
%% must be used in AVerb environments.  Notice the active space as
%% well - that is necessary so the space after \vbinds (and before the
%% first argument) in the verbatim environment gets eaten.
\begingroup\catcode`\!=\active \lccode`\!=`\< \lccode`\~=`\- 
  \catcode`\ =\active\lowercase{\endgroup\def\vbinds#1!~#2;}{\binds#1<-#2;}
%% Return statement: return ... ;
\protected\def\return#1;{\milres return/\ensurett{\ \removespaces #1}}
%% A closure capturing block. k {v1, ..., vN} x: ...
\protected\def\ccblock#1(#2)#3:{\lab#1/\ensuremath{\thinspace\{\ensurett{#2}\}}\ \ensurett{#3\hbox{:}}}
%% A normal block
\protected\def\block#1(#2):{\lab #1/\ensuremath{\thinspace(\ensurett{#2})}\ensurett{:}}
%% A goto expression
\protected\def\goto#1(#2){\lab #1/\thinspace\ensuremath{(\ensurett{#2})}}
%% An enter expression
\protected\def\app#1*#2/{\ensurett{\removespaces #1\ifmmode\ \fi{\text{\tt @}}\ifmmode\ \fi#2}}
\protected\def\bind{\texttt{<-}\xspace}
%% Primitive expression
\protected\def\prim#1(#2){\lab #1/\suptt*\ensuremath{(\ensurett{#2})}}
%% Program variable
\protected\def\var#1/{\ensurett{\removespaces #1}\xspace}
%% Case statement
\protected\def\case#1;{\milres case\ \ensuremath{\ensurett{\removespaces #1}}\ of/}
%% Case alternative
\protected\def\alt#1(#2)#3->#4;{\ensuremath{\ensurett{#1\ \ignorespaces#2\ \texttt{->}\ \ignorespaces #4}}}
%% Invoke
\protected\def\invoke#1/{\milres invoke/\ensurett{\ \removespaces #1}}
\def\rhs{right--hand side\xspace}
\def\lhs{left--hand side\xspace}
\def\enter{\texttt{@}\xspace}
\def\cc{closure--capturing\xspace}
\def\Cc{Closure--capturing\xspace}
%%

\newenvironment{myfig}[1][tbh]{\begin{figure}[#1]%%
\begin{singlespace}\centering%%
\figbegin}{\figend\end{singlespace}%%
\end{figure}}

%% Produce a sub-caption and label it.
\newcommand{\scap}[2][1in]{\begin{minipage}{#1}%%
\subcaption{}\label{#2}\end{minipage}}

%% Produce a sub-caption with text.
\newcommand{\lscap}[3][\hsize]{\begin{minipage}{#1}%%
\subcaption{#3}\label{#2}\end{minipage}}

% single-argument comment. Do not put
% a space before the command when used
% or the file will have two spaces.
\newcommand{\rem}[1]{}

%% A verbatim environment with active charactesr
%% so we can use math shortcuts and macros
\DefineVerbatimEnvironment{AVerb}{Verbatim}{commandchars=\\\{\},%% 
  codes={\catcode`\_8\catcode`\$3\catcode`\^7},%%
  numberblanklines=false}

\DefineVerbatimEnvironment{Verb}{Verbatim}{commandchars=\\\[\],%% 
  numberblanklines=false}

%% Turn on line numbers for Haskell code, 
%% and reset the line number counter.
\newcommand{\hsNumOn}{\numberson\numbersreset}
\newcommand{\hsNumOff}{\numbersoff}
%% Turn on line numbering in Haskell code within
%% the environment, then turn it off. The optional
%% argument specifies a prefix that \hslabel can
%% use to generate line number references. If no prefix
%% is givne, \hslabel will have no effect.
\newtoks\prefixtoks
\def\mkhslabel#1{\prefixtoks={#1}\let\prefix=a}
\def\hslabel#1{\ifx\prefix\relax%%
  \else\label{\the\prefixtoks_#1}%%
  \fi}
\def\unhslabel{\let\prefix=\relax}
\newenvironment{withHsNum}{\numberson\numbersreset}{\numbersoff}
\newenvironment{withHsLabeled}[1]{\numberson\numbersreset\mkhslabel{#1}}{\unhslabel\numbersoff}

%% Paragraph run-in
\newcommand{\runin}[1]{\begingroup\noindent\sffamily\textbf{#1}\qquad\endgroup}

%% Chapter bibliographies
\newcommand{\standaloneBib}{%%
  \ifthenelse{\boolean{standaloneFlag}}%%
             {\begin{singlespace}
                 \printbibliography
             \end{singlespace}}{}}

%% Adds an equation number on demand.
\newcommand\addtag{\refstepcounter{equation}\tag{\theequation}}

%% For typesetting set definitions like {x | x \in f(y)}
\newcommand\setdef[2]{\ensuremath{\{#1\ |\ #2\}}}

%% For typesetting function names like dom(f) or out(b).
\newcommand\mfun[1]{\ensuremath{\mathit{#1}}}

%% Marginal notes
\newcommand\margin[2]{\marginpar{\begin{singlespace}\emph{\footnotesize #2}\end{singlespace}}\relax #1}

%% Describe intent of a passage
\newcommand\intent[1]{{\begin{singlespace}\noindent\leftskip=-1in\emph{\footnotesize Intent: #1}\end{singlespace}}\nopagebreak[1]}

%% In aligned/alignedat/gathered environments, you don't et
%% automatice equation numbers. This command makes sure to
%% label them properly.
\newcommand\labeleq[1]{\refstepcounter{equation}\label{#1}}

%% Creates a hanging paragraph, where the first line is not
%% indented but all other lines are.
\def\itempar#1{\noindent\hangindent=\parindent\hangafter=1 #1\quad}

%% Disable overfull messages with ridiculous hfuzz value
\def\disableoverfull{\hfuzz=10in}

%% Set parfillskip so stretching does NOT occur at the end of
%% a paragraph (i.e., list of elements). Disable indent at beginning
%% of paragraph. Also turn off underfull hbox warnings.
%%
%% Intended to be used in a \vbox that forms part of a table or graphic,
%% which we want to be line-broken but not exactly like a normal paragraph.
\long\def\disableparspacing#1;{\def\arg{#1}\hbadness=100000\parindent=0pt\parfillskip=0pt\leftskip=0pt\rightskip=0pt%%
  \ifx\arg\empty\else\hsize=#1\relax\fi}
%% This stuff makes !+<text>+! write <text> in typewriter font.  

%% We make ! and + active characters early, then manipulate their
%% meaning to produce the right effect. Initially, + produces +. When
%% !  appears w/o a + following, it produces ``!''. When ``+''
%% follows, we start writing in teleteype (\ttfamily). The definition
%% of ``!'' changes to produce a bang. ``+'' changes such that it
%% looks for trailing ``!''. When no ``!'' appears, ``+'' produces ``+''. 
%% If a ``!'' appears, we shift out of \ttfamily (by ending the group) and
%% reset the meaning of ``!'' and ``+'' so we can start again.
\makeatletter
\let\mdplus=+\let\mdbang=!      %% Preserve meaning of + and ! so we can put them into document.
%% Turn off mark down for everyone
\outer\def\nomd{\catcode`!=12\catcode`+=12}
%% Turn mark down on for everyone
\outer\def\domd{\catcode`!=\active\catcode`+=\active %%
  \initialmd}
%% Use only with a group IMMEDIETALY following. Turns off
%% markdown for the group-to-come, without actually tokenizing the
%% group. If no group follows, this has no effect.
\protected\def\pausemd{\def\dopause{\catcode`!=12\catcode`+=12}%%
  \def\pausemdB{\ifx\next\bgroup%%
    %% A ``partial'' application of expandwith is used
    %% so we don't double up the group argument (which is what
    %% happens if we expand \next). This has the effect of 
    %% inserting \expandafter\dowith in front of the upcoming {. 
    %% If no brace is coming, \withmdC will have no effect.
    \def\pausemdC{\expandafter\dopause}
  \else
    \let\pausedmC=\relax
  \fi\pausemdC}
  %% \futurelet has to end the macro so it grabs the next token
  %% from the input file. Otherwise, it grabs it *from* this
  %% definition.
  \futurelet\next\pausemdB} %%
%% Turns markdown on for the group-to-come, without actually
%% tokenizing the group. Only has an effect when
%% used in front of a group, otherwise its a no-op.
\protected\def\withmd{\def\dowith{\catcode`!=\active\catcode`+=\active\initialmd}%%
  \def\withmdB{\ifx\next\bgroup %%
    %% A ``partial'' application of expandafter is used
    %% so we don't double up the group argument (which is what
    %% happens if we expand \next). This has the effect of 
    %% inserting \expandafter\dowith in front of the upcoming {. 
    %% If no brace is coming, \withmdC will have no effect.
      \def\withmdC{\expandafter\dowith} %%
    \else %%
      \let\withmdC=\relax %%
    \fi\withmdC}%%
  %% \futurelet has to end the macro so it grabs the next token
  %% from the input file. Otherwise, it grabs it *from* this
  %% definition.
  \futurelet\next\withmdB} %%
%% Make ! and + active in the following group so they have the right
%% catcode in the definitions to follow.
\catcode`!=\active\catcode`+=\active %%
%% Initial definitions associated with ! and +.
\def\initialmd{\protected\def!{\startTTA} %%
  \protected\def+{\stopTTA}} %%
%% Step 1 of startTT. Inital meaning of !; capture next token in \next, go to next step.
\def\startTTA{\futurelet\next\startTTB} %%
%% Step 2 of startTT. Compare captured token to + and go to step 3 if true. Otherwise
%% output a ! (since that started our macro), the argument captured and stop
%% processing.
\long\def\startTTB{\ifx\next+\expandafter\startTTC\expandafter\@gobble\else\mdbang\fi} %%
%% Step 3 of startTT. Shift into teletype mode and change definition of 
%% + and ! so we can stop processing.
\def\startTTC{\begingroup\ifmmode %%
  \let \math@bgroup \relax %%
  \def \math@egroup {\let \math@bgroup \@@math@bgroup %%
    \let \math@egroup \@@math@egroup} %%
  \mathtt\relax %%
  \else  %%
  \ttfamily\fi} %%
%% Step 1, 2  and 3 of stopTT follow the same pattern as startTT.
\def\stopTTA{\futurelet\next\stopTTB} %%
\long\def\stopTTB{\ifx\next!\expandafter\stopTTC\expandafter\@gobble\else\mdplus\fi} %%
\def\stopTTC{\endgroup}%%
\catcode`!=12\catcode`+=12
\makeatother

\domd

%% Place an input file on the next page
\def\onnextpage#1{\afterpage{\clearpage\input{#1}\clearpage}}

\begin{document}
\ifthenelse{\boolean{standaloneFlag}}
           {\VerbatimFootnotes
             \DefineShortVerb{\#}
             \doublespacing
             \setcounter{chapter}{0}}{}

%% Default float parameters. For case when
%% multiple chapters are included and
%% only one needs custom float settings.
\renewcommand{\textfraction}{0.2}
\renewcommand{\topfraction}{0.9}


\chapter{A Monadic Intermediate Language}
\label{ref_chapter_mil}

%% Compiling the lambda calculus - fundamentals

%% MIL and Three Address Code

%% * Motivate intermediate languages; motivate MIL thorugh three-address
%%     code.
%%    a. Describe details specific to functional languages

%% What makes MIL cool and interesting? How is it unique? Why did
%% we pick the features it has?
%%
%%  * Make curried function application explicit.
%%  * Make allocation explicit.
%%  * Make basic-blocks the default.
%%

%% Syntax/Examples of MIL

%% Compiling LC to MIL
%% Compiling lambda-c to MIL. 
%%    a. Closures, environments.

%% Evaluating MIL (?)  

Most compilers do not generate executable machine code directly from a
program source file. Rather, the compiler translates programs into one
or more \emph{intermediate languages}. Each intermediate language aids
the compiler in some specific way (otherwise, why use it?). For
example, an intermediate language may simplify register allocation,
expose intermediate values, or make memory management explicit. The
compiler may implement a pipeline of translations, each transforming
the program into a more detailed (i.e., lower-level) representation.

A number of intermediate languages have been described for both
imperative and functional language compilers. A register transfer
language (RTL) makes data movement between memory and processor
registers explicit. The RTL aids in optimizing the use of registers,
typically a scarce resource on most processors. There is no one RTL
``language'' -- rather, RTL covers a family of languages and is
described very well by \cite{Torczon2007}
\citep{Torczon2007}. Single-static assignment form (described in
detail by \cite{Muchnick1998} \citep{Muchnick1998}) appears similar to
an RTL, but never re-uses a register assignment (thus, ``single-static
assignment''). It is particularly useful for discovering constant
values and for untangling register usage in the presence of complex
control-flow. Administrative-Normal form (ANF), first described by
\cite{Flanagan1993} \citep{Flanagan1993}, is an intermediate form for
functional languages which makes all intermediate values explicit. It
is useful for showing the exact order of evaluation for expressions.

Our intermediate language, MIL (``monadic intermediate language''), is
designed for functional languages but also has imperative-looking
features. MIL directly supports function application and
abstraction. It also specifies evaluation order and side-effects using
a monadic programming style, making MIL programs look
imperative. MIL's syntax also enforces basic-block structure on
programs, making them ideal for dataflow analysis.

We introduce MIL by first describing fundamental concepts necessary to
compile the \lamA; namely, closures, function application, and the
relationship between the two. We then introduce MIL by comparing and
contrasting it with another intermediate form, ``three-address code.''
MIL syntax and examples follow, to give a flavor of what MIL programs
look like. We then show how to translate \lamC to MIL, without any
optimizations. We sketch how MIL programs can be evaluated, using the
same structure-operations semantics (SOS) style as used in
Chapter~\ref{ref_chapter_languages}.

%% Examples of intermediate forms: SSA, RTL, a-normal, bytecode

%% MIL overture

%% Plan of the chapter

%% This chapter describes our intermediate language, MIL

%% Our intermediate language, MIL (``monadic intermediate language''),
%% directly supports several concepts fundamental to functional language
%% compilers. 

%% We describe our intermediate language, MIL (``monadic intermediate language''). Before
%% introducing MIL, however, we first discuss  

\section{Compiling the \LamA}
\label{lang_sec1}

%% Define which steps in compilation we're going to worry about
Compiling even a language as simple as the \lamA involves a number of
steps, such as defining a concrete syntax, parsing source programs
into an abstract syntax tree (AST), and producing an executable
program from the AST. For our purposes, however, we just focus on the
\lamA' three fundamental operations:

\begin{itemize}
\item Naming values (\emph{variables}).
\item Apply a function to an argument (\emph{application}).
\item Create a new function (\emph{abstraction}). 
\end{itemize}

Any compiler for the \lamA must be able to produce executable programs
which implement these operations. 

\subsection{Variables}
\label{mil_subsec1}

A variable names a value -- in essence, it associates some storage
location with a name, allowing our program to use a consistent label
for some location. Our compiler must be able to associate names with
locations.For example, consider this function (a fragment of the
|compose| function given in Expression~\ref{lang_eq_const1}):
\begin{equation}
  \lamAbs{x}{\lamApP{f}{\lamApp{g}{x}}}.  \label{mil_eq1}
\end{equation}
We see three variables: $f$, $g$, and $x$. We say $x$ is \emph{bound},
because it is given as an argument, and that $f$ and $g$ are
\emph{free} because, in this context, they are not arguments in a 
$\lambda$-abstraction. To evaluate this expression, though, we need
a way to find the values of these terms.  

We can describe where to find $f, g$ and $x$ in terms of memory
locations. We can say that $x$ will appear in a special location,
|arg|, because it is the argument to the function and we will always
put arguments in the same place. We can further say that another
special location, |clo|, will have two
slots. The first will contain $g$ and the second will contain
$f$. Conceptually, then, our expression can be represented as:
\begin{center}
  \begin{tabular}{c}
    \begin{math}\begin{aligned}[b]
      |arg| &= x, \\
      |clo|[0] &= g, \\
      |clo|[1] &= f 
    \end{aligned}\text{\ in}\end{math} \\
    \lamAbs{|arg|}{\lamApp{|clo|[1]}{\lamPApp{|clo|[0]}{arg}}}.
  \end{tabular}
\end{center}

\par
In general, the $|clo|$ location holds the \emph{environment} for our
expression. For any given expression, we will be able to find all the
free variables (i.e., all those except the argument) in the
environment. The compiler will be responsible for ensuring the correct
environment is available whenever a given expression is evaluated.

Our machine, then, must have instructions for storing and retrieving
values. #Store# and #Load# (from Table \ref{tbl_mil1}) serve this
purpose. 

\subsection{Function Application}
\label{mil_subsec2}

Associating locations with names is not enough, however. Looking again
at expression \ref{mil_eq1}, $g$ clearly represents a function to
which we pass the argument $x$. To compute the value of
$\lamPApp{g}{x}$, we must be able to execute the code representing
$g$. We already assigned a storage location for $g$ ($|clo|[0]$) -- now
we just say that the value in $|clo|[0]$ is a \emph{label} that tells
us where to find the code representing $g$. However, $g$ will need
an environment of its own, to hold any free variables for $g$. Therefore,
we pair the label indicating where to find $g$ with a list of free
variables. We call this structure a \emph{closure}.

Closures are the fundamental data structures used to compile
functional languages. They may not have the exact form described here
but they always have the same purpose: they pair a label with the free
variables used in the function represented. 

\subsection{Abstraction}
\label{mil_subsec3}
The \lamA lets us define functions which return new functions. We have
seen how to access variables in the environment and how to execute
unknown functions using closures. Now we come to the final element
needed to compile the \lamA\ -- creating closures.

Consider the following expression, where we apply the $|const|$ function (expression 
\ref{mil_eq4}) to an argument:
\begin{equation}
  \begin{split}
    |main| &= \lamApp{|const|}{s} \\
         &= \lamAppP{\lamAbs{a}{\lamAbs{b}{a}}}{s}.
  \end{split}
\end{equation}
In order to evaluate $|main|$, we need to apply the $|const|$ function
to $s$. From the previous section we know that a closure is required to
implement function application. It follows that
\lamAbs{a}{\lamAbs{b}{a}} must create a closure which will
then be used to execute the body of the $\lambda$-abstraction with the
argument $s$. In fact, the ``value'' created by a
$\lambda$-abstraction is always a closure. The closure will point to
the body of the $\lambda$-abstraction and will hold the free variables
necessary to evaluate it.

\section{Three-address Code}

Intermediate forms typically expose more detail about the
implementation of a program, while at the same time making some
optimization or transformation easier or even possible. 
\emph{Three-address code}, one such intermediate form, translates the
program into assembly-language like form, using registers to
hold values. Infinitely many registers can be named, making registers
more like memory locations than registers in real hardware. Each
instruction in the translated program has two operand registers and one
destination register, thus the name ``three-address.'' 

Three-address code makes all intermediate expression values explicit, 
by reducing complicated expressions to a series of assignments. 
For example, the expression:
\begin{equation}
  a = \frac{(b * c + d)}{2},
\end{equation}
would be expressed in three-address code as:
\begin{AVerb}
  s = b * c;
  t = s + d;
  a = t / 2;
\end{AVerb}
where #s# and #t# are new temporaries created by the compiler. This 
representation makes it easier for the compiler to re-order expressions,
unravel complex control-flow, and manipulate intermediate values. 

\section{Monadic Intermediate Language}

Our intermediate language, MIL, serves the same purpose as
three-address code and other intermediate forms: it exposes more
detail about the implementation of a program, while making some
optimizations simpler or even possible. In contrast to three-address
code, however, our language supports features unique to functional
languages: the ability to treat functions as first-class values, and
the representation of stateful computations in a monad.

\subsection{Monads \& Functional Programming}
As described by Wadler \citep{Wadler1990}, \emph{monads} can be used
distinguish \emph{pure} and \emph{impure} functions. A \emph{pure}
function has no side-effects: it will not print to the screen, throw
an exception, write to disk, or in any other way change the obversable
state of the machine. An \emph{impure} function may change the
machine's state.

%% Presentation drawn from http://en.wikipedia.org/wiki/Monad_%28functional_programming%29, 
%% accessed April 6 2010.
A \emph{monad} provides the abstraction that separates pure and impure
functions. Impure (or ``monadic'') functions execute ``inside'' the
monad. Values returned from a monadic function are not directly
accessible -- they are ``wrapped'' in the monad. The only way
to ``unwrap'' a monadic value is to execute the computation -- inside
the monad! 

\subsection{The Monad in MIL}

When designing MIL we wished to make all memory allocation
explicit. Besides the obvious effect of reducing free memory
available, allocation can also cause two other effects: the allocation
may fail, or a garbage-collection may occur. A monad allows us to
separate computations which (potentially) allocate memory from those
that do not.

\subsection{MIL Example: $compose$}

To give a sense of MIL, consider the definition of $compose$ given in
Figure~\ref{mil_fig1a}. Figure~\ref{mil_fig1b} shows a fragment of this 
expression in MIL. The \emph{block declaration}
on Line~\ref{mil_block_decl_fig1b} gives the name of
the block (#compose#) and arguments that will be passed in (#f#, #g#,
and #x#). Line~\ref{mil_gofx_fig1b} applies #g# to #x# and assigns
the result to #t1#. The ``enter'' operator (#@@#), represents function application.
\footnote{So called because in the expression #g @@ x#, we ``enter''
  function #g# with the argument #x#.}  We assume #g# refers to a
function (or, more precisely, a \emph{closure}). The ``bind'' operator
(#<-#) assigns the result of the operation on its right-hand side to
the location on the left. In turn, Line~\ref{mil_fofx_fig1b} applies
#f# to #t1# and assigns the result to #t2#. The last line returns
#t2#. Thus, the #compose# block returns the value of
\lamApPp{f}{\lamApp{g}{x}}, just as in our original \lamA expression.

\begin{myfig}[t]
  \begin{tabular}{cc}
    \subfloat{$compose = \lamCompose$%%
      \label{mil_fig1a}} & 
    \subfloat{\begin{minipage}{2in}%%
\begin{center}%%
\begin{minipage}{1in}%%
\begin{AVerb}[numbers=left]
compose(f,g,x) = \label{mil_block_decl_fig1b}
  t1 <- g @ x \label{mil_gofx_fig1b}
  t2 <- f @ t1 \label{mil_fofx_fig1b}
  return t2 \label{mil_result_fig1b}
\end{AVerb}
\end{minipage}%%
\end{center}%%
\end{minipage}%%
\label{mil_fig1b}} \\
    \subref{mil_fig1a} & \subref{mil_fig1b}
  \end{tabular} 
  \caption{\subref{mil_fig1a} gives a \lamA definition of the composition
    function; \subref{mil_fig1b} shows a fragment of the MIL program
    for $compose$.}
  \label{mil_fig1}
\end{myfig}

%% Closures

However, according to rules in Figure~\ref{lang_fig6},
Chapter~\ref{ref_chapter_languages} on page~\pageref{lang_fig6},
evaluating an expression which applies $compose$ actually involves the
creation of several intermediate values. Consider the expression
\begin{equation}
  main = \lamApp{\lamApp{\lamApp{compose}{a}}{b}}{c}, \label{mil_eqn4}
\end{equation}
where $a$, $b$ and $c$ are given values elsewhere. Using the
rules for call-by-value evaluation order from Figure~\ref{lang_fig6} in 
Chapter \ref{ref_chapter_languages}, we can compute the value of the expression
as follows:
\begin{align*}
  main &= \lamApp{\lamApp{\lamApp{compose}{a}}{b}}{c} \\
  &= \lamApp{\lamApp{\lamAPp{\lamCompose}{a}}{b}}{c} & \text{\emph{Definition of |compose|.}} \\
  &= \lamApp{\lamAPp{\lamAbs{g}{\lamAbs{x}{\lamApP{a}{\lamApp{g}{x}}}}}{b}}{c} & \text{\emph{E-App.}} \\
  &= \lamAPp{\lamAbs{x}{\lamApP{a}{\lamApp{b}{x}}}}{c} & \text{\emph{E-App.}} \\
  &= \lamApP{a}{\lamApp{b}{c}}. & \text{\emph{E-App.}} 
\end{align*}

We can capture each intermediate value created when evaluating this
expression by assigning each result to a new variable. 

\begin{align*}
  main &= \lamApp{\lamApp{\lamApp{compose}{a}}{b}}{c} \\
  &= \lamApp{\lamApp{\lamAPp{\lamCompose}{a}}{b}}{c} & \text{\emph{Definition of |compose|.}} \\
  t_1 &\leftarrow \lamAbs{g}{\lamAbs{x}{\lamApP{a}{\lamApp{g}{x}}}} & \text{\emph{Result of E-App.}}\\
  &= \lamApp{t_1}{\lamApp{b}{c}} \\
  t_2 &\leftarrow \lamAbs{x}{\lamApP{a}{\lamApp{b}{x}}} & \text{\emph{Result of E-App.}} \\
  &= \lamApp{t_2}{c} \\
  t_3 &\leftarrow \lamApP{a}{\lamApp{b}{c}} & \text{\emph{Result of E-App.}} \\
  &= t_3.
\end{align*}

We apply $t_1$ to $b$ to create our next intermediate value, $t_2$:
\begin{equation}
  t_2 = \lamApp{t_1}{b} = \lamAbs{x}{\lamApp{a}{\lamApp{b}{x}}}. \label{mil_eqn2}
\end{equation}
Finally, we compute our final value, $main$, by applying $t_2$ to $c$:
\begin{equation}
  main = \lamApp{t_2}{c} = \lamApp{a}{\lamApp{b}{c}}. \label{mil_eqn3}  
\end{equation}

Both $t_1$ and $t_2$ will hold \emph{closures} when evaluating
expression \eqref{mil_eqn4}. As detailed in Section \ref{mil_subsec2}, a closure
holds a pointer to a body of code and any \emph{free variables}. In this case,
$t_1$ holds $a$ and points to the code that evaulates to $t_2$. In turn, $t_2$
holds $a$ and $b$, and points to the code which evaluates to $main$. The
\lamA does not make this explicit, but our MIL does. 

\begin{myfig}[t]
  \begin{minipage}{5in}%%
\begin{center}%%
\begin{minipage}{4in}%%
\begin{AVerb}[numbers=left]
\block main(a, b, c): \label{mil_main_fig2}
  \vbinds t0 <- \goto k0();; \label{mil_t0_fig2}
  \vbinds t1 <- \app t0*a/; \label{mil_t1_fig2}
  \vbinds t2 <- \app t1*b/; \label{mil_t2_fig2}
  \vbinds t3 <- \app t2*c/; \label{mil_t3_fig2}
  \return t3;

\block k0(): \mkclo[k1:] \label{mil_k0_fig2}
\ccblock k1()f: \mkclo[k2:f] \label{mil_k1_fig2}
\ccblock k2(f)g: \mkclo[k3:f, g] \label{mil_k2_fig2}
\ccblock k3(f, g)x: \goto compose(f, g, x) \label{mil_k3_fig2}

\block compose(f, g, x): \dots {\rm\emph{as in Figure \ref{mil_fig1b}}} \dots 
\end{AVerb}
\end{minipage}%%
\end{center}%%
\end{minipage}%%

  \caption{The MIL program which computes $main = \lamApp{\lamApp{\lamApp{compose}{a}}{b}}{c}$. Note that $a$, $b$, and $c$ are assumed to be arguments given
    outside the program.}
  \label{mil_fig2}
\end{myfig}

Figure \ref{mil_fig2} shows the complete MIL program for $main =
\lamApp{\lamApp{\lamApp{compose}{a}}{b}}{c}$. #k1#, #k2# and #k3#
(lines \ref{mil_k1_fig2} -- \ref{mil_fig2_k3}) represent
\emph{closure-capturing} blocks. As opposed to #main#, these blocks
create new closures. In the definition #k1 {} f = k2 {f}#, the braces
on the left-hand side represent variables expected in the closure
given to this function. In this case, #k1# does not expect to find any
variables. #f# names the argument given to #k1#. The right-hand side,
#k2 {f}#, shows the creation of a new closure. The closure points to
#k2# and captures the value of #f#. In other words, evaluating #k1#
returns a closure which can be used to execute #k2#. #k2# behaves
similarly. It expects to find one value in its closure (#{f}#) and
returns a closure pointing to #k3# that copies the value #f# from the
existing closure and adds the argument, #g# (#k3 {f,g}#). #k3#,
however, does something new. Instead of returning a closure, it
executes the #compose# block (defined in Figure \ref{mil_fig1b}) with
three arguments: #f#, #g#, and #x#. This does \emph{not} return a
closure or ``enter'' a function. Instead, we jump directly to the
block. The value returned by #k3# will be the value computed by
#compose# with the arguments given.

Returning to #main# on line \ref{mil_main_fig2} in Figure
\ref{mil_fig2}, we can now see how MIL makes explicit the intemediate
closures created while evaluating
\lamApp{\lamApp{\lamApp{compose}{a}}{b}}{c}. On line
\ref{mil_t1_fig2}, we enter #k1# with the first argument, #a#. #t1#
holds the closure returned. On the next line, we enter #t1# (which
will point to #k2#) with the second argument, #b#. #t2# then holds the
closure returned. Finally, on line \ref{mil_t3_fig2}, we enter #t2#
(which will point to #k3#) with the final argument, #c#. #k3# will directly
execute #compose# with our specific arguments. #t3# holds the result returned
by #compose#. On the last line of #main# we return the value computed, #t3#.

%% Syntax of MIL
\subsection{MIL Syntax}

Figure \ref{mil_fig3} gives the syntax for MIL.  A MIL program
consists of a number of \emph{blocks}: \emph{closure} blocks (line
\ref{mil_k1_fig3}), basic blocks (line \ref{mil_b_fig3}) and top-level
blocks (line \ref{mil_t_fig3}). Though the syntax for closure blocks
seems to allow any tail, in practice they can only do one of two
things: either return a closure (\texttt{k \{\dots\}}) or jump to a
basic block (\texttt{b(\dots)}). Top-level blocks (line
\ref{mil_t_fig3}) provide an entry point for top-level functions --
they provide a closure which can be used to initially ``enter'' the
function.

\afterpage{\clearpage{\begin{myfig}
\begin{centering}\begin{tabular}{r@{}lrl}
  \termrule variable:\var v_1/, \dots, \var v_n/:Variables/\\\\
  %% The kern below makes the space between "v" and ":" look better.
  \termrule block:{\ccblock k(v_1, \dots, v_n)v\kern.07em:\ \term tail/}:Closure-Capturing Block/\\
  \termcase {\block b(v_1, \dots, v_n):\ \rlap{\begin{minipage}[t]{\widthof{\quad\term stmt_1/}}\disableparspacing;%%
      \term stmt/\endgraf%%
      $\dots$\endgraf%%
      \term tail/\end{minipage}}}:Basic Block/\\

  \termrule stmt:{\binds v\ <-\ \term tail/;}:Bind/ \\
  \termcase {\begin{minipage}[t]{\widthof{\quad\case v;}}\disableparspacing;%%
      \case v;\endgraf%%
      \quad \term alt_1/\endgraf%%
      \quad $\dots$\endgraf%%
      \quad \term alt_n/%%
  \end{minipage}}:Case Discrimination/ \\
  \termcase \term tail/:/ \\\\

  \termrule alt:\alt C(v_1\ \dots\ v_n) -> \goto b(v_1, \dots, v_n);:Case Alternative/ \\\\

  \termrule tail:{\return v/}:Return/ \\
  \termcase \app v_1 * \ v_2/:Enter/ \\
  \termcase \invoke v/:Execute Thunk/ \\
  \termcase \goto b(v_1, \dots, v_n):Goto Block/ \\
  \termcase \prim p(v_1, \dots, v_n):Goto Primitive/ \\
  \termcase {\mkclo[k:v_1, \dots, v_n]}:Allocate Closure/ \\
  \termcase {\mkthunk[m:v_1, \dots, v_n]}:Allocate Monadic Thunk/ \\
  \termcase \ensurett{C\ v_1\ \dots\ v_n}:Allocate Data/ 
\end{tabular}
\end{centering}
\caption{Complete syntax for MIL.}
\label{mil_fig3}
\end{myfig}
}\clearpage}

Basic blocks (line \ref{mil_body_fig3}) consist of a sequence of statements that
execute in order without any intra-block jumps or conditional
branches. Each basic block ends with a branch: either they return a
value (#done#) or take conditional branch (#case#). Conditional
branches can specify multiple destinations, though at any given time
only one will be taken.

The #case# statement (line \ref{mil_case_fig3}) specifies a list of
\emph{alternatives}, each of which matches a \emph{constructor} and
binds new variables to the values held by the constructor. #case#
examines the variable given (note, this cannot be an expression) and
selects the alternative that matches the constructor
found. Alternatives always branch immediately to some block -- they do
not allow any other statement. The result of block called becomes the
result of the #case#, which in turn becomese the result of the calling
block.

Only the binding statement (line \ref{mil_bind_fig3}) can appear multiple
times in a block. Each binding assigns the result of the \emph{tail}
on the right-hand side to a variable on the left. If a variable is
bound more than once, later bindings will ``shadow'' previous
bindings.

The #done# statement (line \ref{mil_done_fig3}) ends a block and returns
the value of tail expression specified.

\emph{Tail} expressions represent effects -- they create monadic
values. #return# (line \ref{mil_return_fig3}) takes a variable and
makes its value monadic. Notice it can only take a variable, not an
expression.  The ``enter'' operator, #@@#, expects a closure on its
left and some value on the right. It will enter the function pointed
to by the closure, with the argument given, and will evaluate to the
result of that function. #k#, the ``capture'' operator, creates a
closure from a block name and a list of variables. The name given is
not an arbitrary code pointer -- it is a location determined during
compilation. The ``goto'' expression, \texttt{b(\dots)}, jumps to the
particular block with the arguments given. Again, this is not a
computed value -- #b# represents a known location for the block. The
variables mentioned in the #goto# do not have to have the same names
as those given in the block's declaration. The constructor expression,
``C'', will create a data value with the given tag (``C'') and
variables. Primitives, which are not implemented in MIL, have the form
#p*# and are treated the same as ``goto'' expressions. They are not 
implemented in MIL, however. 

\section{Compiling \lamA to MIL}

\emph{\dots Lots of text \dots}

\section{Intermediate Languages, MIL, and Three-Address Code}

Intermediate langauges, and three-address code in particular, have at
least two purposes: making certain optimizations simpler, and exposing
more details about implementation. The intermediate language does
\emph{not} expose all details about implementation -- only those the
designers considered relevant. Three-address code accomplishes this
first goal by reducing the complexity of expresssions that need to be
analyzed. The second goal is achieved by deferring decisions about the
actual location of values to some later stage of compilation.
Finally, while not required, three-address code also easily adapts to
organizing code into basic blocks, which makes control-flow analysis
much simpler.

Our MIL shares some of the same goals as three-address code, and
accomplishes them in similar ways. Blocks do not have complex
structure -- they either return a closure, jump to another block, or
execute a series of statements followed by a return or branch. Tail
expressions ensure that all intermediate values are named, and also 
isolate monadic effects to one area of the language. Finally, the limited
number of statements ensure control-flow is straightforward.

\section{Conclusion}

This chapter presented our Monadic Intermediate Language (MIL). Our
MIL resembles three-address code in sevarl ways: infinitely many
registers can be named, nested expressions are not allowed, and
implementatino details are made explicit. The MIL's unique features
include separate representations for \emph{closure-capturing} and
basic blocks, and the use of monadic \emph{tail} expressions. We 
presented a simple scheme for compiling the \lamA given in
Chapter \ref{ref_chapter_languages} to our MIL. Later will be devoted
to optimizing those MIL programs using dataflow techniques.

%% Compiling the lambda-calculus to MIL

%% \section{Monadic Intermediate Language}

%% %% What does the language support?

%% Our monadic language takes its inspiration from Haskell's @do@
%% notation. It is a pure functional language, making allocation of data
%% structures and closures explicit via monadic syntax. Functions in MIL
%% define computations which, when run, can affect heap memory. Figure
%% \ref{figMILDef} gives the syntax of the language.

%% %% TODO: Mention that v restricts the term to variables
%% %% only.

%% MIL programs consist of a series of definitions (@defM@). Each
%% definition can be any of the following.

%% \begin{description}
%%   \item[Closure-capturing] (@k {v1, ..., vN} v = k1 {v1, ..., vN, v}@) -- This function
%%     expects to find the variables @v1, ..., vN@ in its own closure. It constructs
%%     a new closure containing the existing variables plus the newly captured variable
%%     @v@. The new closure refers to @k1@, another closure-capturing function.
%%   \item[Block-calling] (@k {v1, ..., vN} v = b(v1, ..., vN, v)@) -- This function immediately
%%     jumps to block @b@ with arguments @v1, ..., vN@ and @v@. No closure value needs to
%%     be constructed. 
%%   \item[Function block] (@b(v1, ..., vN) = bodyM@) -- This function executes the statements
%%     in the body. 
%%   \item[Top-level] (@t <- k {}@) -- This special case ensures top-level definitions in the program
%%     can be accessed like any other function. The notation indicates that @t@ holds a closure
%%     structure, referring to the definition @k@. 
%% \end{description}

%% Notice that we can distinguish syntatically between functions that
%% merely create a closure (@k { ... }@) and those that do actual work
%% (@b(...)@). The body of a @k@ functin can only allocate another
%% closure or jump to a block. A block, on the other hand, can do other
%% work, but it cannot directly return a closure. As will be described in
%% chapter \ref{ref_chapter_uncurrying} this makes it much easier to
%% recognize and elminate intermediate closures.

%% The body of each block consists of statements followed by a
%% \emph{tail}. Tails can only
%% appear as the last statement in a block or on the right-hand side of
%% the monadic arrow (``@<-@''). Tail instructions, in other words, cause 
%% effects. The three tail statements follow:

%% \begin{description}
%% \item[Return a computation] (@return v@) -- Returns the result of a computation
%%   to the caller.

%% \item[Create a closure] (@k {v1, ..., vN}@) -- Creates a closure pointing to
%%   function @k@, capturing variables @v1@ through @vN@.

%% \item[Enter a function] (@v1 @@ v2@) -- Enter the closure referred to by @v1@, with
%%   argument @v2@. In other words, function application. Note that @v1@ represents an
%%   \emph{unknown} function -- one for which we compute the address at run-time.

%% \item[Call a block] (@f(v1, ..., vN)@) -- Jump to the block labeled @f@ with the arguments
%%   given. In this case we know the function @f@ refers to and do not need to examine
%%   a closure in order to execute it.
%% \item[Create a value] (@C v1 ... vN@) -- Create a data value with tag @C@, holding
%%   the values found in variables @v1 ... vN@.
%% \end{description}

%% %% TODO: Describe alt syntax.

%% Statements in a block either bind the result of a tail statement 
%% (@v <- tailM@) or branch conditionally (@case v of ... @). Binding ``runs''
%% a computation and ``dereferences'' the result, placing
%% the value in a variable (e.g., @v@). That same variable can be bound
%% again later, but that does not affect previous uses of @v@. In essence, the old
%% name becomes hidden and its value inaccessible.

%% Though the syntax allows multiple @case@ statements in a function
%% body, only one can appear and it must be the last statement in the
%% body. The arms of the @case@ statement can only match on constructor
%% tags (@C@) and can only bind the constructor arguments to variables
%% (@v1 ... vN@). Each arm then jumps to a known block with those
%% variables as arguments. This choice makes compilation simpler.


%% %% \emph{Defines our monadic language and explains the terms in
%% %%   it. Example programs are given which illustrate closure construction
%% %%   and data allocation. The use of ``tail'' vs. statements is motivated
%% %%   and described. }

%% \emph{Need to talk about the monad we work in as well - what 
%% do bind and return mean?}

%% \section{Compiling to Our MIL}
%% \emph{A compilation scheme which uses Hoopls ``shapes'' is
%% described. This scheme will give use our initial, unoptimized
%% MIL program. An example (possibly |compose|, or |const3|) illustrates 
%% our scheme.}

\standaloneBib

\end{document}
