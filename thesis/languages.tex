\documentclass[12pt]{report}
%include polycode.fmt
%include lineno\lineno.fmt
\usepackage[T1]{fontenc}
\usepackage{calc}
%% \usepackage{fourier}
\usepackage{palatino}
\renewcommand\ttdefault{lmtt}
\usepackage{helvet}
%% \usepackage{inconsolata}
\usepackage{comment}
\usepackage{calc}
\usepackage{xspace}
\usepackage{verbatim}
\usepackage{url}
\usepackage{fancyvrb}
\usepackage{setspace}
\usepackage{amsmath}
\usepackage{booktabs}
\usepackage[margin=\parindent, format=hang,labelfont=bf]{caption}
%% \usepackage[subrefformat=parens]{subcaption}
%% The following makes sure we get parentheses around
%% subreferences. The newest version of the subcaption
%% package has an option for this, but that's not available
%% widely.
%%
%% From http://tex.stackexchange.com/questions/25644
\usepackage[labelformat=simple]{subcaption}
\makeatletter
  \def\thesubfigure{(\alph{subfigure})}
  \providecommand\thefigsubsep{~}
  \def\p@subfigure{\@nameuse{thefigure}\thefigsubsep}
\makeatother

\usepackage{ifthen}
\usepackage{stmaryrd}
\usepackage{longtable}
\usepackage{afterpage}
\usepackage{xifthen}
\usepackage{mathtools}
\usepackage{xparse}
\usepackage[natbib=true,style=authoryear,backend=bibtex8]{biblatex}
\setlength{\bibitemsep}{\bigskipamount}
\addbibresource{thesis.bib}
\usepackage{microtype}

\usepackage{tikz}
\usetikzlibrary{arrows,automata,positioning,calc}
%% Used for CFGs.
\tikzset{
  >=stealth, 
  node distance=.5in,
  stmt/.style={rectangle,
    draw=black, thick,        
    minimum height=2em,
    %% inner sep=2pt,
    %% text centered,
    %% node distance=.5in,
  },
  entex/.style={
    minimum height=2em,
    %% inner sep=2pt,
    %% text centered,
  },
  labelfor/.style={circle, 
    draw=black, thin,
    font={\footnotesize},
    inner sep=0,
    fill=white,
    above right=-1.5mm and -1.5mm of #1,
  },
  fact/.style={overlay},
  %% Invisible node
  invis/.style={inner sep=0pt, 
    minimum height=0em}, 
  table/.style={circle, fill=white,height=2mm}
}

%% GSO margins.
\usepackage[left=1.5in, right=1in, top=1in, bottom=1in]{geometry}
\usepackage{abstract}

%% GSO requires 12 pt font for all headings
\usepackage[bf,sf,tiny,compact]{titlesec}
\titleformat{\chapter}[display]
            {}% format
            {\sffamily\bfseries\chaptertitlename\ \thechapter}
            {\baselineskip}
            {\sffamily\bfseries}
            {}

\hyphenation{data-flow mo-na-dic} 

\newboolean{lhs2tex}
\setboolean{lhs2tex}{true}

% Used by included files to know they
% are NOT standalone
\newboolean{standaloneFlag}
\setboolean{standaloneFlag}{true}

\newlength{\rulefigmargin}
\setlength{\rulefigmargin}{2\parindent}

\newcommand\figbegin{\rule{\linewidth-\rulefigmargin}{0.4pt}\\\vspace{12pt}}
\newcommand\figend{\rule{\linewidth-\rulefigmargin}{0.4pt}}

%\providecommand{\citep}[1]{(\emph{#1})\xspace}
%\renewcommand{\cite}[1]{\emph{#1}\xspace}

%% Functional languages chapter commands
\newcommand{\lamA}{\ensuremath{\lambda}-calculus\xspace}
\newcommand{\LamA}{\ensuremath{\lambda}-Calculus\xspace}
\newcommand{\lamAbs}[2]{\ensuremath{\lambda#1.\ #2}}
\newcommand{\lamApp}[2]{\ensuremath{#1\ #2}}
\newcommand{\lamPApp}[2]{\ensuremath{(#1\ #2)}}
\newcommand{\lamAPp}[2]{\ensuremath{(#1)\ #2}}
\newcommand{\lamApP}[2]{\ensuremath{#1\ (#2)}}
\newcommand{\lamAPP}[2]{\ensuremath{(#1)\ (#2)}}
\let\lamApPp=\lamApP
\let\lamAppP=\lamAPp

\newcommand{\lamId}{\lamAbs{x}{x}}
\newcommand{\lamCompose}{\lamAbs{f}{\lamAbs{g}{\lamAbs{x}{\lamApp{f}{(\lamApp{g}{x})}}}}}
\newcommand{\machLam}{\ensuremath{M_\lambda}\xspace}
\newcommand{\compMach}[1]{\ensuremath{\left\llbracket #1 \right\rrbracket}}
\newcommand{\compRho}[1]{\ensuremath{\rho(#1)}}
\newcommand{\verSub}[2]{\ensuremath{#1_{#2}}}
\newcommand{\verSup}[2]{\ensuremath{#1^{#2}}}
\newcommand{\lamC}{\ensuremath{\lambda_C}\xspace}
\newcommand{\lamPlus}{\lamAbs{m}{\lamAbs{n}{\lamAbs{s}{\lamAbs{z}{\lamApp{m}{\lamApPp{s}{\lamApp{n}{\lamApp{s}{z}}}}}}}}}
%% Substitution notation -- [#1 -> #2]
\newcommand{\lamSubst}[2]{\ensuremath{[#1 \mapsto #2]}}
%% End functional languages chapter

%% Dataflow chapter commands
\newcounter{nodeCounter}[figure]
\newcommand{\inE}{\ensuremath{\mathit{in}}\xspace}
\newcommand{\out}{\ensuremath{\mathit{out}}\xspace}
\newcommand{\In}{\ensuremath{\mathit{In}}\xspace}
\newcommand{\InBa}{\ensuremath{\mathit{In}(B)}\xspace}
\newcommand{\Out}{\ensuremath{\mathit{Out}}\xspace}
%% Out(B_x) -- fact function for an named block.
\newcommand{\OutB}[1]{\ensuremath{\mathit{Out}(B_{\ref{#1}})}\xspace}
\newcommand{\OutBa}{\ensuremath{\mathit{Out}(B)}\xspace}
%% in(B) -- fact function for an anonymous block.
\newcommand{\inBa}{\ensuremath{\mathit{in}(B)}\xspace}
%% in(X) -- fact function for an anonymous block, but using a different variable.
\newcommand{\inXa}[1]{\ensuremath{\mathit{in}(#1)}\xspace}
%% in(B,v) -- fact function for an anonymous block and some variable.
\newcommand{\inBav}[1]{\ensuremath{\mathit{in}(B, #1)}\xspace}
%% in(B_x) -- fact function for an named block.
\newcommand{\inB}[1]{\ensuremath{\mathit{in}(B_{\ref{#1}})}\xspace}
%% in(B_x,v) -- fact function for an named block and some variable.
\newcommand{\inBv}[2]{\ensuremath{\mathit{in}(B_{\ref{#1}}, #2)}\xspace}
%% out(B) -- fact function for an anonymous block.
\newcommand{\outBa}{\ensuremath{\mathit{out}(B)}\xspace}
%% out(X) -- fact function for an anonymous block, but using a different variable.
\newcommand{\outXa}[1]{\ensuremath{\mathit{out}(#1)}\xspace}
%% out(B,v) -- fact function for an anonymous block and some variable.
\newcommand{\outBav}[1]{\ensuremath{\mathit{out}(B, #1)}\xspace}
%% out(B_x) -- fact function for an named block.
\newcommand{\outB}[1]{\ensuremath{\mathit{out}(B_{\ref{#1}})}\xspace}
%% out(B_x,v) -- fact function for an named block and some variable.
\newcommand{\outBv}[2]{\ensuremath{\mathit{out}(B_{\ref{#1}}, #2)}\xspace}
\newcommand{\entryN}{\emph{E}\xspace}
\newcommand{\exitN}{\emph{X}\xspace}
\newcommand{\refNode}[1]{\ensuremath{B_{\ref{#1}}}\xspace}
\newcommand{\labelNode}[1]{\refstepcounter{nodeCounter}\label{#1}}
\newcommand{\setL}[1]{\textsc{#1}\xspace}
\newcommand{\setLC}{\setL{Const}}

%% Formats a list of facts
%% Argument should be like \facts{a/1, b/2, foobar/\bot, baz/\top}.
%% 
\newcounter{factctr}
\newtoks\varVal
\newtoks\varName
\newcommand{\facts}[1]{\begingroup%%
  %% Test if the argument given contains a forward slash (/). Expands
  %% slashTest with argument such that if a slash is NOT present the 
  %% token \noSlash will be given as argument 2 to slashTest. Otherwise
  %% there must be slash.
  \def\hasSlash##1{\expandafter\slashTest##1/\noslash\endslash}%%
  \def\slashTest##1/##2##3\endslash{\ifx\noslash##2 N\else Y\fi}%%
  \def\getArgs##1/##2{\varName={##1}%%
    \varVal={##2}}
  \ensuremath{%%
    \setcounter{factctr}{0}%%
    \foreach \var in {#1}{%%
      %% Separate list with a comma
      \ifthenelse{\value{factctr}>0}{,\allowbreak}{}%%
      %% \tracingmacros=1%%
      %% If key/val arguments, use first form. Otherwise
      %% use second.
      \ifthenelse{\equal{\hasSlash{\var}}{Y}}%%
                  {\expandafter\getArgs\var \factC{\the\varName}{\the\varVal}}%%
                  {\var}%%
      %% \tracingmacros=0%%
      \stepcounter{factctr}%%
    }}%%
\endgroup}
\newcommand{\factC}[2]{{\ensuremath{(\mathit{#1},#2)}}}
\newcommand{\doFacts}[4]{\ensuremath{#3{#1}: %%
    \left\{ %%
    \begin{minipage}[c]{#4}%%
      \facts{#2} %%
  \end{minipage}\kern -0.23em\right\}}}

\ExplSyntaxOn
\DeclareDocumentCommand \inFactsM {m m m} {\doFacts{#1}{#2}{\inB}{#3}}
\DeclareDocumentCommand \inFacts {m m O{1in}} {\doFacts{#1}{#2}{\inB}{#3}}
\DeclareDocumentCommand \outFactsM {m m m} {\doFacts{#1}{#2}{\outB}{#3}}
\DeclareDocumentCommand \outFacts {m m O{1in}} {\doFacts{#1}{#2}{\outB}{#3}}
\ExplSyntaxOff

\newcommand{\lub}{\ifthenelse{\boolean{mmode}}{\sqcap}{\raisebox{.1em}{\ensuremath{\sqcap}}}\xspace}
\newcommand{\sqlt}{\ensuremath{\sqsubset}\xspace}
\newcommand{\sqlte}{\ensuremath{\sqsubseteq}\xspace}

%% End dataflow

%% MIL Chapter
\newcommand{\compMILE}[1]{\ensuremath{\left\llbracket #1 \right\rrbracket}}
\newcommand{\compMILV}[1]{\ensuremath{\left\llbracket #1 \right\rrbracket}}
\newcommand{\compMILQ}[2]{\ensuremath{\left\llbracket #2 \right\rrbracket}}
\newcommand{\milCtx}[1]{\ensuremath{\llfloor}#1\ensuremath{\rrfloor}}
%% End MIL chapter

\newenvironment{myfig}[1][tbh]{\begin{figure}[#1]%%
\centering%%
\figbegin}{\figend%%
\end{figure}}

%% Produce a sub-caption and label it.
\newcommand{\scap}[2][1in]{\begin{minipage}{#1}%%
\subcaption{}\label{#2}\end{minipage}}

%% Produce a sub-caption with text.
\newcommand{\lscap}[3][1in]{\begin{minipage}{#1}%%
\subcaption{#3}\label{#2}\end{minipage}}

% single-argument comment. Do not put
% a space before the command when used
% or the file will have two spaces.
\newcommand{\rem}[1]{}

%% A verbatim environment with active charactesr
%% so we can use math shortcuts and macros
\DefineVerbatimEnvironment{AVerb}{Verbatim}{commandchars=\\\{\},%% 
  codes={\catcode`\_8\catcode`\$3\catcode`\^7},%%
  numberblanklines=false}

%% Turn on line numbers for Haskell code, 
%% and reset the line number counter.
\newcommand{\hsNumOn}{\numberson\numbersreset}
\newcommand{\hsNumOff}{\numbersoff}
%% Turn on line numbering in Haskell code within
%% the environment, then turn it off.
\newenvironment{withHsNum}{\numberson\numbersreset}{\numbersoff}

%% Paragraph run-in
\newcommand{\runin}[1]{\begingroup\noindent\sffamily\textbf{#1}\qquad\endgroup}

%% Chapter bibliographies
\newcommand{\standaloneBib}{%%
  \ifthenelse{\boolean{standaloneFlag}}%%
             {\begin{singlespace}
                \printbibliography
             \end{singlespace}}{}}

%% Adds an equation number on demand.
\newcommand\addtag{\refstepcounter{equation}\tag{\theequation}}

%% For typesetting set definitions like {x | x \in f(y)}
\newcommand\setdef[2]{\ensuremath{\{#1\ |\ #2\}}}

%% For typesetting function names like dom(f) or out(b).
\newcommand\mfun[1]{\ensuremath{\mathit{#1}}}

%% Marginal notes
\newcommand\margin[2]{\marginpar{\begin{singlespace}\emph{\footnotesize #2}\end{singlespace}}\relax #1}

%% Describe intent of a passage
\newcommand\intent[1]{{\leftskip = -1in\begin{singlespace}\emph{\noindent\footnotesize Intent: #1}\end{singlespace}}}

\begin{document}
\ifthenelse{\boolean{standaloneFlag}}
           {\VerbatimFootnotes
             \DefineShortVerb{\#}
             \setcounter{chapter}{0}}{}

%% Default float parameters. For case when
%% multiple chapters are included and
%% only one needs custom float settings.
\renewcommand{\textfraction}{0.2}
\renewcommand{\textfraction}{0.2}
\renewcommand{\topfraction}{0.9}


\chapter{The \LamA \& Functional Languages}
\label{ref_chapter_languages}

%% Overall: Justify why the lambda-calculus matters
%%   * Give syntax and evaluation rules for pure lambda calculus 
%%   * Extend for ADTs and case statements
%%   * Set the foundation for LC-to-MIL later

John McCarthy created LISP, the first ``functional'' language, in 196X
\citep{McCarthyXX}. Other functional languages created since then
include Scheme, ML, Haskell, and many more. While syntax, semantics
and capabilities differ widely between all these languages, they share
a common characteristic: \emph{functions are values}.  This chapter
explains what that means and introduces the language we will use to
write our programs. Section~\ref{lang_sec2} introduces the basic idea
of treating functions as values through examples in four different
functional languages: Haskell, Scheme, ML and
JavaScript. Section~\ref{lang_sec1} introduces the syntax and
semantics of the \lamA -- in some ways the simplest possible
functional programming language. We then describe \lamC in
Section~\ref{lang_sec4}, a variant of the \lamA that we will be using
throughout the remainder of this thesis. We summarize our chapter in
Section~\ref{lang_sec5}.

\section{Introductory Example}
\label{lang_sec2}
A function that returns a function can be hard to get used to, so we
first look at a function that just computes a value -- it does not
return a function or anything fancy.  Figure~\ref{lang_fig1} shows
``#mag#,'' a function that doubles its value, written in four
different functional languages.

\begin{myfig}[bth]
  \begin{tabular}{cc}
  \subfloat{%%
    \begin{minipage}{2in}\begin{withHsNum}%%
> mag :: Float -> Float {-"\label{lang_fig1_haskell_sig}"-}
> mag a = 2 * a {-"\label{lang_fig1_haskell_impl}"-}
    \end{withHsNum}\end{minipage}%%
    \label{lang_fig1_haskell}} & %%
  \subfloat{\begin{minipage}{2in}
  \begin{AVerb}[gobble=4,numbers=left]
    mag : float -> float \label{lang_fig1_ml_sig}
    mag a = 2 *. a \label{lang_fig1_ml_impl}
  \end{AVerb}
\end{minipage}
\label{lang_fig1_ml}} \\

  \subref{lang_fig1_haskell} & \subref{lang_fig1_ml} \\

  \subfloat{\begin{minipage}{2in}
  \begin{AVerb}[gobble=4,numbers=left]
    (def double a (* 2 a))
  \end{AVerb}
\end{minipage}
\label{lang_fig1_scheme}} & %%
  \subfloat{\begin{minipage}{2in}
  \begin{AVerb}[gobble=4,numbers=left]
    function mag(a) \{ \label{lang_fig1_js_def}
      return a * 2; \label{fig_lang1_js_impl}
    \}
  \end{AVerb}
\end{minipage}
\label{lang_fig1_js}} \\

  \subref{lang_fig1_scheme} & \subref{lang_fig1_js} 
  \end{tabular}
  \caption{Definitions of a function that doubles its argument in
    \subref{lang_fig1_haskell} Haskell, \subref{lang_fig1_ml} ML, 
    \subref{lang_fig1_scheme} Scheme, and \subref{lang_fig1_js} JavaScript.}
  \label{lang_fig1}
\end{myfig}

Part \subref{lang_fig1_haskell} gives the Haskell version. Haskell is
a statically typed language, so we begin with a type signature on
Line~\ref{lang_fig1_haskell_sig}: |mag :: Float -> Float|. This
signature indicates that |mag| takes an argument of type |Float| and
returns a result, also of type
|Float|. Line~\ref{lang_fig1_haskell_impl} implements |mag|. The
function name comes first, followed by the argument (|a|). The
right-hand side of the equals sign defines the \emph{body} of the
function: |2 * a|. The body is evaluated when the function is applied
to an argument and a result must be computed.\footnote{Haskell is a
  \emph{lazy} language, meaning no computations are performed until
  demanded. Therefore, we say the body is evaluated only when a
  ``result must be computed.''}

Figure~\ref{lang_fig1}, Part~\subref{lang_fig1_ml}, gives the ML
implementation. ML is also statically typed, so we start with the type
signature on Line~\ref{lang_fig1_ml_sig}: #float -> float#. This
signature has much the same meaning as the Haskell
version. Line~\ref{lang_fig1_ml_impl} gives the implementation of
#mag#. The ``#*.#'' operator represents floating-point
multiplication. Otherwise, the implementation is much the same as the
Haskell version.

Figure~\ref{lang_fig1}, Part \subref{lang_fig1_scheme}, gives the
Scheme definition. Scheme is a dynamically typed language, so no
signature can be given -- just the implementation. The expression
``\texttt{define mag}\ \emph{expr}'' associates \emph{expr} with the
name ``#mag#.'' The expression in this case, ``\texttt{lambda (a)
  \ldots)},'' creates a new function that takes one argument, ``#a#.''
The body of the function, ``#(* 2 a)#,'' shows that the argument will
be doubled when the function is applied.

%% ``#define#'' keyword
%% associates a name with a value. The
%% ``#lambda#'' keyword indicates that a function will be created. The
%% funciton defined takes only one argument, designated ``#a#.'' The
%% postfix expression, ``#(* 2 a)#,'' defines the body of the function
%% and will be evaluated when the function is applied.

Figure~\ref{lang_fig1}, Part \subref{lang_fig1_js}, shows the
JavaScript version. Line~\ref{lang_fig1_js_def} gives the
\emph{signature} of the function -- the name of the function and any
named arguments. JavaScript is also a dynamically typed language, so
this is \emph{not} a type signature, but rather a specification of how
to call the function.\footnote{We say ``named'' here because JavaScript functions
can take more arguments than declared.}  Function definitions always start with the
``#function#'' keyword, followed by the function name and any
named arguments in parentheses: #mag (a)#. The body of the function,
on Line~\ref{fig_lang1_js_impl}, uses the ``#return#'' keyword to
indicate the function doubles its argument and returns the resulting
value: #return 2 * a#.

The functions defined in Figure~\ref{lang_fig1} all have one thing in
common: they are limited to doubling their argument. If we want to
triple our argument, halve it, zero it or perform any other
multiplication, then we need to write a new function.

Of course, we can write a function that takes two arguments, the
multiplier and the argument. For example, in JavaScript:
\begin{AVerb}
function magBy(multiple, a) \{
  return multiple * a;
\}
\end{AVerb}
But this limits us from re-using ``#magBy#'' in certain ways. 

Imagine a function that wants to apply ``#magBy#'' to all items in a
list:\footnote{In this fragment, #items# is an array of values,
  accessed by index. We enumerate it using a #for# loop much like the
  that found in the C language.}
\begin{AVerb}
function magAll(items, multiple) \{
  for(var i = 0; i < items.length(); i++)
    items[i] = magBy(multiple, items[i]);
\}
\end{AVerb}
This definition creates two problems:
\begin{enumerate}
\item Every call to ``#magAll#'' requires us to specify a value for
  ``#multiple#.'' 
\item Our function is limited to using ``#magBy#.'' If ``#magBy#''
  is not appropriate for some situation, we need to write a new
  ``#magAll#'' that uses a different version.
\end{enumerate}
We solve these two problems by making ``#magBy#'' a \emph{parameter}
to ``#magAll#.'' It is something like creating a ``hole'' in the
definition of ``#magAll#'', where we can put code that is passed in as
an argument:
\begin{AVerb}
function magAll(items, \emph{<code>}) \{
  for(var i = 0; i < items.length(); i++)
    items[i] = \emph{<code>};
\}
\end{AVerb}
The ``\emph{<code>}'' argument illustrates how functional languages
treat ``functions as values.''

Figure~\ref{lang_fig2} shows the definition of |mag| in terms of a
new function, |multiplier|.  When |multiplier| is evaluated, it
produces a value like any function; that value just happens to be a
function itself! The function returned takes an argument and
multiplies it by the original multiple given to |multiplier|.

\begin{myfig}
  \begin{tabular}{cc}
    \subfloat{\begin{minipage}{3.5in}\begin{withHsNum} %%
> multiplier :: Float -> (Float -> Float)
> multiplier multiple = 
>   \a -> multiple * a {-"\label{lang_fig2_hs_fun}"-}
>
> mag :: Float -> Float
> mag = multiplier 2 {-"\label{lang_fig2_hs_mag}"-}
        \end{withHsNum}
      \end{minipage}\label{lang_fig2_hs}} & %%
    \subfloat{\begin{minipage}{3in}
  \begin{AVerb}[gobble=4,numbers=left]
    multiplier : float -> 
      (float -> float)
    multiplier multiple = 
      let f a = a *. multiple \label{lang_fig2_ml_fun}
      in f

    mag : float -> float
    mag = multiplier 2 \label{lang_fig2_ml_mag}
  \end{AVerb}
\end{minipage}
\label{lang_fig2_ml}} \\

    \subref{lang_fig2_hs} & \subref{lang_fig2_ml} \\

    \subfloat{\begin{minipage}{3in}
  \begin{AVerb}[gobble=4,numbers=left]
    (define multiplier \label{lang_fig2_scheme_fun}
      (lambda (multiple)  
        (lambda (a) (* multiple a))))

    (define mag \label{lang_fig2_scheme_mag}
      (multiplier 2))
  \end{AVerb}
\end{minipage}
\label{lang_fig2_scheme}} & %%
    \subfloat{\begin{minipage}{3in}
  \begin{AVerb}[gobble=4,numbers=left]
    function multiplier(multiple) \{
      return function(a) \{ \label{lang_fig2_js_fun}
        return multiple * a;
      \};
    \}
    
    var double = multiplier(2); \label{lang_fig2_js_double}
  \end{AVerb}
\end{minipage}
\label{lang_fig2_js}} \\

    \subref{lang_fig2_scheme} & \subref{lang_fig2_js} \\

  \end{tabular}
  \caption{The |multiplier| function and how it can be used to define
    |mag|. When evaluated, |multiplier| returns a function that
    will multiply its argument by |multiple|. We give
    \subref{lang_fig2_hs} Haskell, \subref{lang_fig2_ml} ML,
    \subref{lang_fig2_scheme} Scheme, and \subref{lang_fig2_js}
    JavaScript versions.}
  \label{lang_fig2}
\end{myfig}

Figure~\ref{lang_fig2}, Part~\subref{lang_fig2_hs} gives the Haskell
version of |multiplier|. The signature, |Float -> (Float -> Float)|,
shows that |multiplier| takes one argument, a |Float| value, and
returns a functio,n (|Float -> Float)|. The implementation on Line~\ref{lang_fig2_hs_fun}
usea an \emph{anonymous} function:

> \a -> multiple * a

The anonymous function is introduced with the |\| (``lambda'') symbol,
followed by the function's argument, |a|. The body of the function
follows the arrow.  Notice that |multiple| is \emph{not} an
argument to this function. Instead, it is an argument to
|multiplier|. We say |multiple| is \emph{captured} by the anonymous
function. The anonymous function is the value returned by
|multiplier|. When that value is itself applied to an argument, it
will use the value of |multiple| originally given to |multiplier|.

On Line~\ref{lang_fig2_hs_mag} we use |multiplier| to define the
|mag| function from Figure~\ref{lang_fig2_hs}. The function has
the same signature, |Float -> Float|, but no argument:

> mag :: Float -> Float
> mag = multiplier 2

If we substitute the definition of
|multiplier| in |mag|, we can see the function |mag| represents:

\begin{math}
  \begin{array}{cc}
    |mag| &= |multiplier 2| \\
    &= |\a -> 2 * a | 
  \end{array}
\end{math}

Notice that the argument |a| appears on the right-hand
side here, for which reason |mag| does not specify an argument
in Figure~\ref{lang_fig2_hs}.

Figure~\ref{lang_fig2}, Part \subref{lang_fig2_ml,} shows the ML
definition for #multiplier# and #mag#. \texttt{multiplier} returns
the value #f#, which is again a function. Line~\ref{lang_fig2_ml_fun}
defines #f# as a local, named function:

\begin{AVerb}
  let f a = a *. multiple
  in f
\end{AVerb}

Again, we capture the value of #multiple# when defining #f#. When #f#
is evaluated, it will multiply its argument by the #multiple#
given. The definition of #mag# on Line~\ref{lang_fig2_ml_mag} in terms
of #multiplier# looks almost exactly the same as the Haskell version.

In Figure~\ref{lang_fig2}, Part \subref{lang_fig2_scheme}, we give the
Scheme version of #multiplier# and #mag#. As in
Figure~\ref{lang_fig1_scheme}, the body of #multiplier# is a function,
defined using #lambda#. However, this function returns a function,
again defined with #lambda#:

\begin{AVerb}
  (lambda (a) (* multiple a))))
\end{AVerb}

As in the Haskell and ML versions, the inner function captures the
value of #multiple# given to the outer function. On
Line~\ref{lang_fig2_scheme_mag} we evaluate the expression
\texttt{({multiplier} 2)} and assign the result to #mag#:
\begin{AVerb}
  (define mag 
    (multiplier 2))
\end{AVerb}

Figure~\ref{lang_fig2}, Part \subref{lang_fig2_js} shows the JavaScript
version of #multiplier#. The body of #multiplier# returns an anonmous
function, defined by using the #function# keyword without a name:
\begin{AVerb}
  return function(a) \{ 
    return multiple * a;
  \};
\end{AVerb}

Once again, the #multiple# argument is captured and used by the
returned function. Line~\ref{lang_fig2_js_mag} shows how #mag#
is defined in terms of #multiplier#:

\begin{AVerb}
  var mag = multiplier(2);
\end{AVerb}

The #var# keyword introduces an identifier, to which we assign the
function returned by #multiplier(2)#. In some ways this syntax makes
it most obvious that we are treating functions as values.

Returning to #magAll#, we can redefine it to take a function argument:
\begin{AVerb}
function magAll(items, magnifier) \{
  for(var i = 0; i < items.length(); i++)
    items[i] = magnifier(i);
\}
\end{AVerb}
Here, #magnifier# is a function, passed as an argument. If 
we wish to double the items in the array, we just pass #double#
to #magAll#:
\begin{AVerb}
  magAll(items, double);
\end{AVerb}
To multiply the items however we need, we just create appropraite
\emph{function values} and pass them to #magAll#:
\begin{AVerb}
  var halve = multiplier(0.5);
  var quadruple = multiplier(4);
  magAll(items, quadruple);
  magAll(items, halve);
\end{AVerb}
We can even pass an \emph{anonymous function} directly to 
#magAll#, as here where we halve the values again:
\begin{AVerb}
  magAll(items, function (i) \{ return i * 0.5; \});
\end{AVerb}

\emph{\ldots Signposts and Transition \ldots}
%%  However, that value is itself a 
%% function. |multilp
%% evaluated, creates a new function. in our four languages

%% We will be demonstrating a number of dataflow optimizations over
%% our intermediate language programs, but all of our source programs will
%% be written in a variant of the \lamA. Any variante

%% A compilation technique demonstrated for
%% some variant of the \lamA can be translated into any other functional
%% programming language. Making the translation work well with the syntax
%% and semantics of the target language is still hard work, but
%% absolutely possible -- a result developed for the \lamA really is
%% universal (as far as you want to make use of that result, of course). 

%% It is these three reasons that make the \lamA such a popular
%% language for showing theoretic (or practical) results

%% This chapter introduces the \lamA calcus, giving its 

%% Most importantly, results obtained
%% using the \lamA are guaranteed to translate to other Turing-complete
%% languages -- and usually with better syntax! 

%% Being Turing-complete, the \lamA is capable of exeucting any program
%% you could write on a modern computer.  

%%  . What it does mean is that anything possible in the \lamA Being
%% Turing-complete, it can be

%% \emph{\ldots transition
%%   \ldots}. Most importantly, the \lamA serves as the \emph{lingua
%%   franca} for functional programmers. It provides a way to translate
%% between functional programming languages, and a way to carry
%% developments from one language to another. 

%% so its capabilities are
%% as powerful as any other Turing-complete programming language. 

%% Being Turing-complete, it can emulate any other
%% Turing-complete language. Its direct support for manipulating
%% functions-as-values makes it a good choice for emulating higher-level
%% functional languages.

%% It is not a language that you want to write many 
%% programs in, 

%% Its sparse syntax
%% and straightforward evaluation rules means their is less to worry about
%% when trying out new design ideas or theories.  . Most importantly, though, 


%% Figure~\ref{lang_fig1} defines
%% a function for adding two numbers in Scheme, ML, Haskell and JavaScript,
%% some of the more prominent functional languages in use today. 




 
%% 1. Relate lambda-calculus to functional langauges in general
%% 2. Define the lambda-calculus
%%    * syntax, semantics, evaluation rules
%% 3. Compliling - or better to move that to MIL chapter?

%% Informally, a \emph{function} is a mathematical definition that takes
%% arguments and computes some result. For example, \emph{plus1} just adds 1 
%% to its argument, using the normal rules of arithmetic:
%% \begin{equation}
%%   |plus1|\ x = x + 1.
%% \end{equation}
%% We are not limited to defining functions that add 1. Functions can
%% also be \emph{values} -- just as 1 or $x$ are in the expression
%% above. Here, we define a function, that returns a function, that always
%% adds 1:
%% \begin{equation}
%%   |adder1| = \lambda\ x = x + 1.
%% \end{equation}
%% In |adder1|, ``$\lambda\ x$'' indicates we return a function that takes one
%% argument. We can go further and define a function that, given an
%% argument, returns a function which always adds that amount:
%% \begin{equation}
%%   |adder|\ n = \lambda\ x = x + n.
%% \end{equation}
%% Notice how the outer argument $n$ gets ``captured'' by the body $x +
%% n$. Using $|adder|$, we can now re-define $|adder1|$ above:
%% \begin{equation}
%%   |adder1| = |adder|\ 1.
%% \end{equation}

%% Other functional languages created since then include 
%% Scheme, ML, Haskell, and many more. While syntax, semantics and capabilities
%% differ widely between all these languages, they all share the characteristic
%% shown above: \emph{the ability to manipulate functions as first-class values}.

\section{The \LamA}
\label{lang_sec1}
%%\emph{Why is it important}

%%\emph{What is the \lamA}

Alonzo Church defined his \lamA (``lambda calculus'') in 19XX
\citep{ChurchXX} to study systems of recursive equations. Being
Turing-complete, it can be used to model the behavior of any
computational system. Like Turing machines, the \lamA is \emph{not}
a ``programming language'' in any practical sense, but it has proved
incredibly useful when modelling the behavior of other functional
programming languages.

\subsection{Syntax of the \LamA}
\label{lang_sec1_syntax}

Figure~\ref{lang_fig3} shows the three types of terms used in the
\lamA: \emph{variables}, \emph{abstractions} and
\emph{applications}. \emph{Variables} have no further structure --
they are just names. An \emph{abstraction} defines a function that has
one parameter, $x$, and a body given by the term $t$. The parameter
$x$ does not stand for a term -- it can only be used as a variable in
the body of $t$.  An \emph{application} applies the term $t_2$ to the
term $t_1$.

\begin{myfig}[th]
  \begin{minipage}{5in}
    \begin{align*}
      \mathit{term} &= a, b, \ldots & \hfill\text{\emph{(Variables)}} \\
      &= \lamAbs{x}{t} & \hfill\text{\emph{(Abstraction)}} \\ 
      &= \lamApp{t}{t} & \hfill\text{\emph{(Application)}}
    \end{align*}
  \end{minipage}
  \caption{The \lamA' syntax.}
  \label{lang_fig3}
\end{myfig}

Using this syntax, we can define some basic functions. |Identity|
returns its argument:
\begin{align}
  |identity| &= \lamAbs{x}{x}. \label{lang_eq1} \\
  \intertext{|Const| takes two arguments but always returns the first:}
  |const| &= \lamAbs{a}{\lamAbs{b}{a}}. \label{lang_eq_const1} \\
  \intertext{|Compose| takes two functions and an argument. The result of
    applying the second function to the argument is passed to the first:}
  |compose| &= \lamCompose. \label{lang_eq_compose1} 
\end{align}
Note that application is left-associative, meaning
\lamApp{const}{\lamApp{a}{b}} is the same as \lamAppP{\lamApp{const}{a}}{b}
but \emph{not} the same as \lamApp{const}{(\lamApp{a}{b})}. Abstractions
also extend as far right as possible, unless parentheses are used to
delimit scope. That is, \lamAbs{x}{\lamApp{x}{\lamAbs{y}{y}}} is the
same as $(\lamAbs{x}{\lamApp{x}{(\lamAbs{y}{y})}})$, but not the
same as $\lamApp{(\lamAbs{x}{x})}{(\lamAbs{y}{y})}$.

\subsection{Evaluating \LamA Terms}
\label{lang_sec1_eval}

When an abstraction, \lamAbs{x}{t_1}, is applied to a term, $t_2$, we
can \emph{substitute} $t_2$ for each occurrence of $x$ in $t_1$. We say
that $t_1$ ``maps to'' or ``substitues for'' each $x$ in $t_2$. Using
Pierce's notation \citep{PierceXX}, we write this as:
\begin{equation}
  \lamAppP{\lamAbs{x}{t_1}}{t_2} = [t_2 \mapsto x] t_1.
\end{equation}
Though substituation may appear simple, it is surprisingly hard to get
right. When $t_2$ mentions variables that are bound by $t_1$, we can
unintenionally capture values that we did not expect. For example,
consider the following substitution, where we replace $x$ with $y$
blindly:
\begin{equation}
  \begin{split}
    & \lamAppP{\lamAbs{x}{\lamAbs{y}{\lamApp{x}{y}}}}{y} \\
    & \quad = [y \mapsto x] \lamAbs{y}{\lamApp{x}{y}} \\
    & \quad = \lamAbs{y}{\lamApp{y}{y}}
  \end{split}
\label{lang_eq_capt1}
\end{equation}
Clearly, ``doubling'' the $y$ term in the last abstraction is not what
we mean here -- we are really referring to two different
$y$'s. Solving this issue algorithmically is very tricky -- but we can
avoid it altogether by assuming we can \emph{rename} variables
correctly and at any time, to avoid unintentional capture. That is, we
imagine we can rewrite Equation~\ref{lang_eq_capt1} using new
variables as needed:
\begin{equation}
  \begin{split}
    & \lamAppP{\lamAbs{x}{\lamAbs{y}{\lamApp{x}{y}}}}{y} \\
    & \qquad = [y \mapsto x] \lamAbs{w}{\lamApp{x}{w}} \\
    & \qquad = \lamAbs{w}{\lamApp{y}{w}}
  \end{split}
\label{lang_eq_capt2}
\end{equation}
For a full treatment of this issue, see Pierce \citep{PierceXX},
Chapter X, Section Y.

Following Pierce \citep{PierceXX}, who in turn follows Church
\citep{ChurchXX}, we call any term where we can use substitution
(i.e., an abstraction applied to term) a \emph{reducible expression}
or \emph{redex}. We call the act of reducing
\emph{beta-reduction}. Evaluating (or ``executing'') a \lamA
expression means finding these redexes and \emph{beta-reducing} them.

For example, consider the following term:
\begin{equation}
  \lamAppP{\lamAbs{y}{\lamApp{\lamAbs{x}{y}}{y}}} %%
         {\lamAbs{z}{\lamAppP{\lamAbs{x}{x}}{z}}}
\end{equation}
There are three possible redexes (i.e., substitutions) we can make in 
this program:
\begin{center}
  \begin{tabular}{lccc}
    & \emph{term} & \emph{substitutes for} & \emph{in} \\
    \cmidrule(r){2-2} \cmidrule(r){3-3} \cmidrule(r){4-4}
    1. & $y$ & $x$ & $(\lamAbs{x}{y})$ \\
    2. & $z$ & $x$ & $(\lamAbs{x}{x})$ \\
    3. & $(\lamAbs{z}{\lamAppP{\lamAbs{x}{x}}{z}})$ & $y$ & $(\lamAbs{y}{\lamApp{\lamAbs{x}{y}}{y}})$
  \end{tabular}
\end{center}
The means by which we choose what redex to beta-reduce is called our
\emph{evaluation strategy}.

There are a number of possible strategies we can use. We follow
Pierce's taxonomy \citep{PierceXX} here. \emph{Full beta-reduction}
lets us choose any of the possible redexes, in any order we wish,
until no more redexes remain. \emph{Normal-order} requires us to
choose the leftmost, outermost redex, again until none
remain. \emph{Call-by-name} also requires us to choose the leftmost,
outermost redex first, but we cannot reduce any redexes inside (or
``under'') abstractions. \emph{Call-by-value}, like call-by-need,
restricts us from reducing inside abstractions and requires that we
reduce the outermost redex first, and also requires that we reduce the
right side of each application to a value before applying it to the
left.

\emph{\ldots Talk about normal form and values here \ldots}

We can view evaulation of lambda-terms as a sequence of
\emph{rewrites}. That is, at each step of evaluation we replace some
portion of the term with another term, reduced according to a
rule. Plotkin \citep{PlotkinXX} first gave this presentation as
``structural operational semantics.'' Figure~\ref{lang_fig4} presents
the rewrite rules for call-by-name and call-by-value strategies. We
apply the rules by finding a match for our term on the left-hand side
of a rule, and then rewriting according the right-hand side. Each
rewrite is called a \emph{transition}, and the strategy as a whole a
\emph{transition function}.

\begin{myfig}[th]
  \begin{tabular}{cc}
    \subfloat{\begin{minipage}{2.5in}
  \begin{math}
    \begin{array}{lclr}
      \lamApp{t_1}{t_2} & \rightarrow & \lamApp{t_1}{v} & \text{\sc E-App1} \\
      \lamApp{t_1}{v_2} & \rightarrow & \lamApp{v_1}{v_2} & \text{\sc E-App2} \\
      \lamAppP{\lamAbs{x}{t}}{v} & \rightarrow & [v \mapsto x] t & \text{\sc E-Abs}
    \end{array}
  \end{math}
\end{minipage}
\label{lang_fig4_eval}} & %%
    \subfloat{\begin{minipage}{2.5in}
  \begin{math}
    \begin{array}{lclr}
      \lamApp{t_1}{t_2} & \rightarrow & \lamApp{v_1}{t_2} & \text{\sc CBN-App} \\
      \lamAppP{\lamAbs{x}{t_1}}{t_2} & \rightarrow & [t_1 \mapsto x] t_2 & \text{\sc CBN-Abs}
    \end{array}
  \end{math}
\end{minipage}
\label{lang_fig4_eval_need}} \\
    \subref{lang_fig4_eval} & \subref{lang_fig4_eval_need}
  \end{tabular}
  \caption{Two different strategies for evaluating \lamA terms: \subref{lang_fig4_eval}
    shows \emph{call-by-value}; \subref{lang_fig4_eval_need} gives \emph{call-by-need}.
  In both cases, terms are evaluated by matching a rule on the left and rewriting
  according to the right-hand side. The $t$'s represent arbitrary terms, while
  the $v$'s represent values.}
  \label{lang_fig4}
\end{myfig}

%% as \emph{rewrite} rules. 
%% Interestingly, these strategies are not strictly equivalent: some will
%% fail to terminate where others do not. The SuchAndSuch theorem states that,
%% if a term can 
%% The call-by-value strategy is used by many 
%% programming languages today: Java, C, PHP, and more. Call-by-name 
%% (under the variant call-by-need) is not very widely used -- the Haskell
%% programming language is one of the most prominent to employ it. 



%% using
%%   a variety Part \subref{lang_fig4_eval} shows to evaluate \lamA
%%   terms. Each rule should be read top-to-bottom. When a term matches
%%   the top of the rule, we rewrite according to the bottom
%%   portion. Each \texttt{\emph{t}} stands for an arbitrarily large term
%%   -- meaning we can ``match'' the portions in typewriter font, while
%%   ``parameterizing'' over \texttt{\emph{t}}. These rules show
%%   ``call-by-value'' style, where arguments are evaluated before
%%   functions. Other styles include ``call-by-need'' and
%%   ``call-by-name,'' where arguments are only evaluated as needed, but
%%   we do not address them here.

%% The {\sc E-App1} and {\sc E-App2} rules show that terms must be
%% reduced to values before application can be applied. {\sc E-App1}
%% ensures that the argument ($t_2$) is reduce to a value before the
%% function ($t_1$) is applied.  Each $v$ represents a value -- a term
%% that cannot be reduced any further. {\sc E-Abs} shows how arguments
%% are substituted in the body of an abstraction: the notation $[v/x] t$
%% means we replace all occurrences of $x$ in $t$ with $v$. There are
%% many subtle issues that arise when arguments have the same names as
%% parameters, but we do not need to worry about them here. {\sc E-Val}
%% lets us continue to reduce any terms found inside a $\lambda$ if
%% needed.  Notice, however, that no rule will match $\lambda$ terms
%% (i.e., abstractions) except when applied to an argument or when their
%% body is not fully evaulated: $\lambda$ terms are our values.


%% Every abstraction, \lamAbs{x}{t}, \emph{binds} its argument, $x$, in
%% the context of its body, $t$. Alternatively, we can say $x$ is
%% \emph{bound} by the abstraction. Our syntax allows us to write terms
%% containing \emph{free} variables -- variables not bound by any
%% enclosing abstraction. For example, $b$ is free in \lamAbs{a}{\lamApp{a}{b}},
%% but bound in \lamAbs{b}{\lamApp{a}{b}}. A term containing free variables
%% is \emph{open}; otherwise, it is \emph{closed}.


%%\emph{What does it look like?}
Simple as it is, the pure \lamA can be used to represent a number of
common data structures. For example, the natural numbers (0, 1, 2,
\ldots) can be represented by a series of ``successors'' of zero. Note
that function application is right-associative, meaning
\lamPApp{s}{\lamPApp{s}{z}} is the same as \lamApp{\lamApp{s}{s}}{z},
but \emph{not} the same as \lamPApp{\lamPApp{s}{s}}{z}:
\begin{equation*}
  \begin{split}
    0 &= \lamAbs{s}{\lamAbs{z}{z}} \\
    1 &= \lamAbs{s}{\lamAbs{z}{\lamApp{s}{z}}} \\
    2 &= \lamAbs{s}{\lamAbs{z}{\lamApp{s}{\lamApp{s}{z}}}} \\
    \ldots
  \end{split}
\end{equation*}
In other words, the number of applications of $s$ gives the natural
number represented by the function. 

This representation allows
us to define a number of common arithmetic operations. We start 
with the \emph{successor} function, which always adds 1 to its
argument:
\begin{equation}
  \mathit{succ} = \lamAbs{n}{\lamAbs{s}{\lamAbs{z}{\lamApp{s}{\lamApp{n}{z}}}}}
\end{equation}

%% \s. \z. z
%% \s. \z. s z
%% \s. \z. s s z

%% \n. \s. \z. s (n z)

%% succ (\s. \z. z)
%%   = \s. \z. s ((\s. \z. z) z)
%%   = \s. \z. s ((\z. z))
%%   = \s. \z. s z

%% Nothing is built-in,
%% not even the natural numbers. There is almost nothing to get in the
%% way of studying your particular domain, rather than fiddling with the
%% programming language. However, the \lamA is inherently functional --
%% it can only define functions, and all values computed are really
%% functions. Translating from the \lamA to another functional language
%% is usually straightforward.\footnote{And, if not, maybe your target
%%   language isn't very functional.}

%% Its power, simplicity, and functional nature of the \lamA motivates
%% our own choice in using it. We do not show how to compile a
%% ``real-world'' functional language to our intermediate language, but
%% by showing how to compile the \lamA to our language, we show our
%% technique could be used by ``real'' functional languages (with some
%% adaptions, of course).

%% Using this syntax, we can define some common functions. |Identity|
%% returns its argument:
%% \begin{align}
%%   |identity| &= \lamAbs{x}{x}. \label{eq_lang2} \\
%%   \intertext{|Const| takes two arguments but always returns the first:}
%%   |const| &= \lamAbs{a}{\lamAbs{b}{a}}. \label{eq_lang4} \\
%%   \intertext{|Compose| takes two functions and an argument. The result of
%%     applying the second function to the argument is passed to the first:}
%%   |compose| &= \lamCompose. \label{eq_lang3} 
%% \end{align}
%% Note that function application is right-associative, meaning
%% \lamPApp{f}{\lamPApp{g}{x}} is the same as \lamApp{\lamApp{f}{g}}{x},
%% but \emph{not} the same as \lamPApp{\lamPApp{f}{g}}{x}.

%% \begin{myfig}[bt]
%% \begin{minipage}{2in}
%% \begin{Verbatim}
%%                 ##  
%%                  #  
%%  ##  ## ## ###   #  
%% ####  # ## ###   #  
%% #     ###  # #   #  
%%  ###   #   ## # ### 
%% \end{Verbatim}
%% \end{minipage}
%%   \caption{Evaluation rules for \lamA. These rules show 
%%     \emph{call-by-value}, where arguments are evaluated
%%     before functions.}
%%   \label{fig_lang2}
%% \end{myfig}

%% A \lamA term executes by rewriting the expression according to the
%% rules in Figure \ref{fig_lang2}. We match our term to each of the
%% patterns above the line. If we have a match, we rewrite according to
%% the pattern below the line. When no more matches can be made, we say
%% the term is in \emph{normal form}: we have finished executing.

%% The rules given implement \emph{call-by-value} evaluation order,
%% meaning arguments to a function are evaluated before the function
%% itself. Other variants include \emph{call-by-need} and
%% \emph{call-by-name}, where arguments are not evaluated until
%% needed. We do not considers those variants further, however.

\section{\lamC -- Extending the \lamA}
\label{lang_sec4}

%% \section{Why \LamA?}
%% \label{lang_sec1}

%% \section{Compiling the \LamA}
%% \label{sec_lang1}

%% %% Define which steps in compilation we're going to worry about
%% Compiling even a language as simple as the \lamA involves a number of
%% steps, such as defining a concrete syntax, parsing source programs
%% into an abstract syntax tree (AST), and producing an executable
%% program from the AST. For our purposes, however, we just focus on the
%% \lamA' three fundamental operations:

%% \begin{itemize}
%% \item Naming values (\emph{variables}).
%% \item Apply a function to an argument (\emph{application}).
%% \item Create a new function (\emph{abstraction}). 
%% \end{itemize}

%% Any compiler for the \lamA must be able to produce executable programs
%% which implement these operations. 

%% \subsection{The Target Machine}
%% We begin by defining a \emph{target machine}, |M|, for our compiler. To
%% reduce complexity we do not target an actual computer, but one of our
%% own design. Our machine will have an infinite number of
%% \emph{registers} (i.e., storage locations) that we can refer to by
%% name. It will have an unlimited supply of memory (called the
%% \emph{heap}) in which we can allocate structured values. However, we
%% will not refer to memory locations directly. Instead, we will always
%% store references to heap values in registers. Finally, the machine
%% will execute a list of instructions (our \emph{program}), starting at
%% the beginning and proceeding in sequential order (unless otherwise
%% instructed), until reaching the end of the list. Each instruction will
%% have a definite location, but we will only refer to certain special
%% locations using named labels.

%% \subsection{M's Language: \machLam}
%% Table \ref{tbl_lang1} gives the language that our machine will
%% execute, \machLam. A benefit of defining our own machine is that we
%% can also define the language it executes -- and the language we need
%% to compile to! We cannot make it too dissimilar from a ``real''
%% machine, but at this stage it helps to keep things simple. 

%% \begin{table}[th]
%%   \centering
%%   \begin{tabular}{lp{3.5in}}
%%     \emph{Instruction} & \emph{Description} \\
%%     \cmidrule(r){1-1}\cmidrule(r){2-2}
%%     \texttt{Store \emph{R} (\emph{F}, \emph{M})} & Store the value found in register #R# to field %%
%%     #F# of the value in register #M#. \\
%%     \texttt{Load (\emph{F}, \emph{M}) \emph{R}} & Load field #F# of the value in register #M# to register #R#. \\
%%     \texttt{Set \emph{v} \emph{R}} & Sets the register #R# to name of the variable $v$. \\
%%     \texttt{Copy \emph{R} \emph{M}} & Copies the contents of register #R# to register #M#. \\
%%     #Enter# & Jump to the location indicated by the closure in
%%     register #clo#, assuming an argument in register #arg#. The next #Return# executed
%%     will return to this location, with a result in register #res#.\\
%%     #Return# & Jump to the instruction following the most recently 
%%     executed #Enter# instruction and begin executing.  \\
%%     \texttt{MkClo \emph{L} [\emph{R}, \emph{S}, \dots]} &  Create a closure pointing to the 
%%     label #L# and holding the values in registers #R#, #S#, etc. The closure will be stored in 
%%     the #res# register.
%%   \end{tabular}
%%   \caption{\machLam, the ``machine language'' executed by our machine |M|.}
%%   \label{tbl_lang1}
%%   \figend
%% \end{table}

%% Each instruction supports an some aspect of the \lamA. In brief:
%% \begin{description}
%% \item[Variables] -- #Store# and #Load# help access variables and
%%   function arguments.
%% \item[Function Application] -- #Enter# and #Return# allow us to execute a function with arguments.
%% \item[Abstraction] -- #MkClo# lets us create functions as values.
%% \end{description}
%% The following sections describe each aspect in detail.

%% \subsection{Variables}
%% \label{subsec_lang1}

%% Free variables and environment
%% Consider how to find a value by its name. For example, consider
%% the |compose| function (expression \ref{eq_lang3}):
%% \begin{equation}
%%   \lamCompose.  \label{eq_lang1}
%% \end{equation}
%% We see three variables: $f$, $g$, and $x$. We say $x$ is \emph{bound},
%% because it is given as an argument, and that $f$ and $g$ are
%% \emph{free} because, in this context, they are not arguments in a 
%% $\lambda$-abstraction. To evaluate this expression, though, we need
%% a way to find the values of these terms.  

%% We can describe where to find $f, g$ and $x$ in terms of memory
%% locations. We can say that $x$ will appear in a special location,
%% |arg|, because it is the argument to the function and we will always
%% put arguments in the same place. We can further say that another
%% special location, |clo|, will have two
%% slots. The first will contain $g$ and the second will contain
%% $f$. Conceptually, then, our expression can be represented as:
%% \begin{center}
%%   \begin{tabular}{c}
%%     \begin{math}\begin{aligned}[b]
%%       |arg| &= x, \\
%%       |clo|[0] &= g, \\
%%       |clo|[1] &= f 
%%     \end{aligned}\text{\ in}\end{math} \\
%%     \lamAbs{|arg|}{\lamApp{|clo|[1]}{\lamPApp{|clo|[0]}{arg}}}.
%%   \end{tabular}
%% \end{center}

%% \par
%% In general, the $|clo|$ location holds the \emph{environment} for our
%% expression. For any given expression, we will be able to find all the
%% free variables (i.e., all those except the argument) in the
%% environment. The compiler will be responsible for ensuring the correct
%% environment is available whenever a given expression is evaluated.

%% Our machine, then, must have instructions for storing and retrieving
%% values. #Store# and #Load# (from Table \ref{tbl_lang1}) serve this
%% purpose. 

%% \subsection{Function Application}
%% \label{subsec_lang2}

%% Application & closures
%% Associating locations with names is not enough, however. Looking again
%% at expression \ref{eq_lang1}, $g$ clearly represents a function to
%% which we pass the argument $x$. To compute the value of
%% $\lamPApp{g}{x}$, we must be able to execute the code representing
%% $g$. We already assigned a storage location for $g$ ($|clo|[0]$) -- now
%% we just say that the value in $|clo|[0]$ is a \emph{label} that tells
%% us where to find the code representing $g$. However, $g$ will need
%% an environment of its own, to hold any free variables for $g$. Therefore,
%% we pair the label indicating where to find $g$ with a list of free
%% variables. We call this structure a \emph{closure}.

%% Closures are the fundamental data structures used to compile
%% functional languages. They may not have the exact form described here
%% but they always have the same purpose: they pair a label with the free
%% variables used in the function represented. 

%% \subsection{Abstraction}
%% \label{subsec_lang3}
%% The \lamA lets us define functions which return new functions. We have
%% seen how to access variables in the environment and how to execute
%% unknown functions using closures. Now we come to the final element
%% needed to compile the \lamA\ -- creating closures.

%% Consider the following expression, where we apply the $|const|$ function (expression 
%% \ref{eq_lang4}) to an argument:
%% \begin{equation}
%%   \begin{split}
%%     |main| &= \lamApp{|const|}{s} \\
%%          &= \lamAppP{\lamAbs{a}{\lamAbs{b}{a}}}{s}.
%%   \end{split}
%% \end{equation}
%% In order to evaluate $|main|$, we need to apply the $|const|$ function
%% to $s$. From the previous section we know that a closure is required to
%% implement function application. It follows that
%% \lamAbs{a}{\lamAbs{b}{a}} must create a closure which will
%% then be used to execute the body of the $\lambda$-abstraction with the
%% argument $s$. In fact, the ``value'' created by a
%% $\lambda$-abstraction is always a closure. The closure will point to
%% the body of the $\lambda$-abstraction and will hold the free variables
%% necessary to evaluate it.

%% \subsection{Compiling from \lamA to \machLam}

%% Table \ref{tbl_lang2} gives our algorithm to compile from \lamA to
%% \machLam. We present it in in three parts, \emph{a} - \emph{c},
%% corresponding to the syntax of \lamA terms given in Figure
%% \ref{lang_fig2}. The ``fat brackets,'' \compMach{t}, represent our
%% compiler, with the term being compiled given as the argument, $t$.
%% Each term compiles to a given sequence of instructions. We also assume
%% a function $\rho$, maintained by the compiler, that knows which
%% register holds a given variable.

%% %% Compilation rules ...
%% \afterpage{\clearpage{%% Used in the languages chapter, this
%% table is placed in its own file so we can use
%% it with the afterpage command.
\begin{singlespace}
  \begin{longtable}{p{2in}p{3.5in}}
    \caption{Compilation rules from \lamA to \machLam.} \\
    \hline \\
    \endfirsthead
    \caption{Compilation rules from \lamA to \machLam \emph{(cont'd)}} \\
    \hline \\
    \endhead
    \\ \hline \multicolumn{2}{r}{\emph{Continued on next page}}
    \endfoot 
    \\ \hline
    \endlastfoot
    %% Variables
    \multicolumn{2}{c}{\emph{(a) Variable Reference}} \\ 
    \begin{minipage}[t]{2in}
      \begin{Verbatim}[commandchars=\\\{\}]
\compMach{v} = 
      \end{Verbatim}
    \end{minipage} \\

    \begin{minipage}[t]{2in}
      \begin{Verbatim}[commandchars=\\\{\}]
  Set ``v'' ``res''
  Return
      \end{Verbatim}
    \end{minipage} &  Because our \lamA does not have any real ``values'', we just
    set the #res# register to the variable name. \\ \\

    %% Application
    \multicolumn{2}{c}{\emph{(b) Function Application}} \\ 
    \begin{minipage}[t]{2in}
      \begin{Verbatim}[commandchars=\\\{\}]
\compMach{\lamApp{f}{g}} = 
      \end{Verbatim}
    \end{minipage} \\

    \begin{minipage}[t]{2in}
      \begin{Verbatim}[commandchars=\\\{\}, codes={\catcode`\_8\catcode`\$3}]
  Copy ``arg'' $r$
  Copy ``clo'' $s$
      \end{Verbatim}
    \end{minipage} & $r$ and $s$ are ``fresh'' registers. \\ \\[-.5em]

    \begin{minipage}[t]{2in}
      \begin{Verbatim}[commandchars=\\\{\}]
  Copy \compRho{g} arg
  Copy \compRho{f} clo
  Enter
      \end{Verbatim}
    \end{minipage} & $\rho$ associates variables
    to the register that they will be found in. This lookup occurs
    during compilation, not while the program executes. Here we copy
    $f$ and $g$ to the #clo# and #arg# registers, respectively. \\ \\[-.5em]

    \begin{minipage}[t]{2in}
      \begin{Verbatim}[commandchars=\\\{\}, codes={\catcode`\_8\catcode`\$3}]
  Copy $r$ ``arg''
  Copy $s$ ``clo''
      \end{Verbatim}
    \end{minipage} & Restore the previous #arg# and #clo# registers. \\ \\

    %% Abstraction
    \multicolumn{2}{c}{\emph{(c) Abstraction with an Abstraction}} \\ 
    \begin{minipage}[t]{2in}
      \begin{Verbatim}[commandchars=\\\{\}]
\compMach{\lamAbs{x}{\lamAbs{y}{t}}} = 
      \end{Verbatim}
    \end{minipage} \\

    \begin{minipage}[t]{2in}
      \begin{Verbatim}[commandchars=\\\{\}]
m : 
      \end{Verbatim}
    \end{minipage} & We mark the function body with a fresh label, #m#. \\ \\[-.5em]
    
    \begin{minipage}[t]{2in}
      \begin{Verbatim}[commandchars=\\\{\}, codes={\catcode`\_8\catcode`\$3}]
  Store r1 (``clo'', 0) 
  Store r2 (``clo'', 1) 
  \dots
  Store r$N$ (``clo'', 
            $N$)
      \end{Verbatim}
    \end{minipage} & Copy current values out of the closure. $N$
    equals the number of fields in the closure. \\ \\[-.5em]

    \begin{minipage}[t]{2in}
      \begin{Verbatim}[commandchars=\\\{\}, codes={\catcode`\_8\catcode`\$3}]
  MkClo l [r1, \dots, r$N$, 
           arg]
  Return
      \end{Verbatim}
    \end{minipage} & We call ourselves recursively and 
    retrieve a label, #l#, holding the location of the compiled body:
    \[ l = \compMach{\lamAbs{y}{t}} \].
    We then create a new closure which points to #l#. We put
    all values from the current closure into the new, and add our 
    argument, $x$, using the #arg# register. Because #MkClo# puts
    the closure created in #res#, we can immediately return. \\ \\

    %% Abstraction 2
    \multicolumn{2}{c}{\emph{(d) Abstraction with a Term}} \\* 
    \begin{minipage}[t]{2in}
      \begin{Verbatim}[commandchars=\\\{\}]
\compMach{\lamAbs{x}{t}} = 
      \end{Verbatim}
    \end{minipage} \\*

    \begin{minipage}[t]{2in}
      \begin{Verbatim}[commandchars=\\\{\}]
m : 
      \end{Verbatim}
    \end{minipage} & Again, we mark the function body with a fresh label, #m#. \\* \\*[-.5em]

    \begin{minipage}[t]{2in}
      \begin{Verbatim}[commandchars=\\\{\}, codes={\catcode`\_8\catcode`\$3}]
  Load \compRho{v_1} (0, ``clo'')
  Load \compRho{v_2} (1, ``clo'')
  \dots
  Load \compRho{v_n} ($N$, 
             ``clo'')
  Copy ``arg'' \compRho{x} 
      \end{Verbatim}
    \end{minipage} & We load all free variables $v_1, \dots, v_n$ from our
    closure into the appropriate registers, using the $\rho$ function. We also
    copy the argument $x$ from the #arg# register to the location given by
    $\rho$. \\ \\[-.5em]

    \begin{minipage}[t]{2in}
      \begin{Verbatim}[commandchars=\\\{\}]
  \compMach{t}
      \end{Verbatim}
    \end{minipage} & Now that variables are set up correctly, we compile the body
    of the abstraction and place it inline here.
  \label{tbl_lang2}
  \end{longtable}
\end{singlespace}
}\clearpage}

%% Table \ref{tbl_lang2}, part \emph{a}, shows the compilation
%% scheme for variables. Variable refrences that are not used
%% in function application can only be the body of an expression, so we
%% just copy the variable's name to the #res#
%% register and return.

%% Function application, \lamPApp{f}{g}, is shown in part
%% \emph{b}. To apply a function, we must save the current #clo#
%% and #arg# registers. The compiler creates \emph{fresh} registers,
%% guaranteed to be unused anywhere else in the program, to store #clo#
%% and #arg#. We then use $\rho$ to find the registers holding $f$ and
%% $g$. Remember that $f$ will be a closure, while $g$ will be some
%% value. We copy those values into #clo# and #arg#. The #Enter#
%% instruction will execute the code pointed to by #clo#. When that
%% function returns, we restore #clo# and #arg# from the fresh registers
%% created earlier.

%% Abstractions, such as \lamAbs{x}{t}, return a closure pointing to the
%% code implementing $t$. Therefore, our compiler needs to generate code
%% that returns a closure, which in turn points to the code generated for
%% the body of the abstraction. To accomplish this, our compiler
%% recursively calls itself on the body. We get a label back, which is the
%% location of the just compiled code. In parts \emph{c} and \emph{d}
%% the expression $l = \compMach{\lamAbs{y}{t}}$ shows this
%% recursive call, and the label that results. That label can then be used in the 
%% closure returned by the abstraction.

%% We separate compilation of abstractions into two cases, depending if
%% the body is an abstraction or not. In the first case, as shown in part
%% \emph{c}, we begin by marking the location of this code with a new label,
%% #m#. We prepare to create a new closure by copying all values out of
%% the current closure into fresh registers. We then create a closure that
%% points to the body of our abstraction, contains all the values found
%% in the current closure, and ``captures'' our argument in the new
%% closure. 

%% For example, consider compiling this expression:

%% \begin{equation}
%%   \lamAbs{x}{\lamAbs{y}{\lamApp{f}{\lamPApp{y}{x}}}}. 
%% \end{equation}

%% $f$ and $x$ must be available when the body
%% \lamPApp{f}{\lamPApp{y}{x}} executes. Therefore, the closure returned
%% by \lamAbs{x}{(\dots)} must copy all values in the existing
%% closure as well as add the argument, $x$.

%% Part \emph{d} shows the code generated when the body of an abstraction
%% is \emph{not} another abstraction. We first mark the location of the
%% start of the body with a new label, #m#.  We then find the free
%% variables in the body, calling them $v_1, \dots, v_n$. This is a
%% compile-time operation, not something the program will do when
%% executing.  We assume that value of each free variables can be found
%% in the corresponding closure slot. For example, $v_0$ will be found in
%% $clo[0]$, $v_1$ in $clo[1]$, and so on. We also copy the $arg$
%% register to the corresponding register for our argument, as determined
%% by the $\rho$ function. Now that we have placed all variables in the
%% registers expected by our function, we generate the code for our body
%% and place it inline.

\section{Conclusion}
\label{lang_sec5}
%% Functional languages distinguish themselves by their ability to treat
%% \emph{functions} as \emph{first-class values}. The \lamA, invented
%% some time before the first functional language, turned out to be a
%% simple but effective way to model (and experiment with) the behavior
%% of functional languages. Therefore, understanding how to compile the
%% \lamA can effectively show us how to compile functional languages in
%% general.

%% This chapter gave the basic mechanisms needed to understand the \lamA:
%% \emph{variables}, \emph{application}, and
%% \emph{abstraction}. Understanding how to compile the \lamA means
%% understanding how to compile these three mechanisms. Variables become
%% \emph{locations}. Application means evaluating a function with a given
%% \emph{environment} for any \emph{free variables}. Abstractions create
%% \emph{closures} that carry two pieces of information: the location of
%% the compiled function body and the value of free variables to be used
%% when evaluating the function.

\end{document}
