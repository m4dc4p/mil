\documentclass[12pt]{report}
 %include polycode.fmt
\usepackage[T1]{fontenc}
\usepackage{calc}
\usepackage{palatino}
\usepackage{amsfonts}
\renewcommand\ttdefault{lmtt}
\usepackage{helvet}
\usepackage{xspace}
\usepackage{url}
\usepackage{fancyvrb}
\usepackage[doublespacing]{setspace}
%% below only necessary when using doublespacing -- corrects
%% the vertical space inserted when switching to singlespace
%% environment.
\def\correctspaceskip{\vskip-\baselineskip} 
\usepackage{amsmath}
\usepackage{booktabs}
\usepackage[margin=\parindent, format=hang,labelfont=bf]{caption}
%% \usepackage[subrefformat=parens]{subcaption}
%% The following makes sure we get parentheses around
%% subreferences. The newest version of the subcaption
%% package has an option for this, but that's not available
%% widely.
%%
%% From http://tex.stackexchange.com/questions/25644
\usepackage[labelformat=simple]{subcaption}
\makeatletter
  \def\thesubfigure{(\alph{subfigure})}
  \providecommand\thefigsubsep{~}
  \def\p@subfigure{\@nameuse{thefigure}\thefigsubsep}
\makeatother

\usepackage{ifthen}
\usepackage{stmaryrd}
\usepackage{longtable}
\usepackage{afterpage}
\usepackage{xifthen}
\usepackage{mathtools}
\usepackage[natbib=true,style=authoryear,backend=bibtex8]{biblatex}
\setlength{\bibitemsep}{\bigskipamount}
\addbibresource{thesis.bib}
\usepackage{microtype}

%% GSO margins.
\usepackage[left=1.5in, right=1in, top=1in, bottom=1in]{geometry}
\usepackage{abstract}

%% GSO requires 12 pt font for all headings
\usepackage[bf,sf,tiny,compact]{titlesec}
\titleformat{\chapter}[display]
            {}% format
            {\sffamily\bfseries\chaptertitlename\ \thechapter}
            {\baselineskip}
            {\sffamily\bfseries}
            {}

\hyphenation{data-flow mo-na-dic} 

%% Should unindent all haskell code set in a dispay (versus inline)
\makeatletter
  \@ifundefined{hscodestyle}
               {}
               {\renewcommand{\hscodestyle}{\advance\leftskip -\mathindent}}
\makeatother

% Used by included files to know they
% are NOT standalone
\newboolean{standaloneFlag}
\setboolean{standaloneFlag}{true}

\newlength{\rulefigmargin}
\setlength{\rulefigmargin}{2\parindent}

\newcommand\figbegin{\rule{\linewidth}{0.4pt}\\\vspace{12pt}}
\newcommand\figend{\rule{\linewidth}{0.4pt}}

%% Sets
\newcommand{\setL}[1]{\textsc{#1}\xspace}
\newcommand{\setLC}{\setL{Const}}

%% Lub, subset operators.
\protected\def\lub{\ifmmode\sqcap\else\raisebox{.1em}{\ensuremath{\sqcap}}\fi\xspace}
\newcommand{\sqlt}{\ensuremath{\sqsubset}\xspace}
\newcommand{\sqlte}{\ensuremath{\sqsubseteq}\xspace}

%% Subscripting with typewriter
\def\subtt#1{\ifmmode_{\ensurett{#1}}%%
  \else$_{\ensurett{#1}}$%%
  \fi}
%% Superscripting with typerwriter
\def\suptt#1{\ifmmode^{\ensurett{#1}}%%
  \else$^{\ensurett{#1}}$%%
  \fi}
%% Functional languages chapter commands
\newcommand{\lamA}{\ensuremath{\lambda}-calculus\xspace}
\newcommand{\LamA}{\ensuremath{\lambda}-Calculus\xspace}
\newcommand{\lamAbs}[2]{\ensuremath{\lambda#1.\ #2}}
\newcommand{\lamApp}[2]{\ensuremath{#1\ #2}}
\newcommand{\lamPApp}[2]{\ensuremath{(#1\ #2)}}
\newcommand{\lamAPp}[2]{\ensuremath{(#1)\ #2}}
\newcommand{\lamApP}[2]{\ensuremath{#1\ (#2)}}
\newcommand{\lamAPP}[2]{\ensuremath{(#1)\ (#2)}}
\let\lamApPp=\lamApP
\let\lamAppP=\lamAPp
%% LC definition
\newtoks\toksA
\protected\def\lcname#1/{\ensuremath{\mathit{#1}}}
\protected\def\lcdef#1(#2)=#3;{\def\arg{#2}%%
  \def\lcargs##1,##2/{\def\arg{##2}%%
    \ifx\empty\arg%%
    \lcname ##1/%%
    \else\lcname ##1/\ \lcargs ##2/%%
    \fi}%%
  \ifx\empty\arg\toksA={\ }%%
  \else\toksA={\ \lcargs #2,/\ }%%
  \fi%%
  \ensuremath{\lcname#1/\the\toksA =\ #3}}
%% Arbitary number of applied arguments, separated
%% by asterisks (*).
\protected\def\lcapp#1/{\def\lcappB##1*##2/{\def\arg{##2}%
    \ensuremath{\ifx\arg\empty%%
      \lcname ##1/%%
      \else%%
      \lcname##1/\ \lcappB##2/%%
      \fi}}%%
  %% Adding a star here makes
  %% sure our applicaitn always ends with an asterisks, ensuring
  %% #2 will be \empty at some point.
  \lcappB#1*/}
\protected\def\lcabs#1.{\ensuremath{\lambda#1.\ }}

\newcommand{\lamId}{\lamAbs{x}{x}}
\newcommand{\lamCompose}{\lamAbs{f}{\lamAbs{g}{\lamAbs{x}{\lamApp{f}{(\lamApp{g}{x})}}}}}
\newcommand{\machLam}{\ensuremath{M_\lambda}\xspace}
\newcommand{\compMach}[1]{\ensuremath{\left\llbracket #1 \right\rrbracket}}
\newcommand{\compRho}[1]{\ensuremath{\rho(#1)}}
\newcommand{\verSub}[2]{\ensuremath{#1_{#2}}}
\newcommand{\verSup}[2]{\ensuremath{#1^{#2}}}
\newcommand{\lamC}{\ensuremath{\lambda_C}\xspace}
\newcommand{\lamPlus}{\lamAbs{m}{\lamAbs{n}{\lamAbs{s}{\lamAbs{z}{\lamApp{m}{\lamApPp{s}{\lamApp{n}{\lamApp{s}{z}}}}}}}}}
%% Substitution notation -- [#1 -> #2]
\newcommand{\lamSubst}[2]{\ensuremath{[#1 \mapsto #2]}}
%% End functional languages chapter


%% MIL Chapter
\newcommand{\compMILE}[1]{\ensuremath{\left\llbracket #1 \right\rrbracket}}
\newcommand{\compMILV}[1]{\ensuremath{\left\llbracket #1 \right\rrbracket}}
\newcommand{\compMILQ}[2]{\ensuremath{\left\llbracket #2 \right\rrbracket}}
\newcommand{\milCtx}[1]{\ensuremath{\llfloor}#1\ensuremath{\rrfloor}}

%% This dimension makes sure the same amount of space
%% follows | and := in syntax rules like:
%%
%% term := var       (Variable)
%%      |  var var    (Application)
%%      |  \x. var    (Abstraction)
%%
\newdimen\termalign
\setbox0=\hbox{$:=$}
\termalign=\wd0 
\protected\def\term#1/{\ensuremath{\mathit{#1}}}
\protected\def\syntaxrule#1/{\hfil\text{\emph{#1}}}
\protected\long\def\termrule#1:#2:#3/{\term #1/ &\hbox{$:=$}\ensuremath{\ #2} & \syntaxrule #3/}
\protected\def\termcase#1:#2/{&\hbox to \termalign{$|$\hss}\ensuremath{\ #1} & \syntaxrule #2/}


%% End MIL chapter

%% Dataflow Chapter
% Domain function
\def\dom(#1){\ensuremath{\mfun{dom}(#1)}\xspace}
% Set of all integers.
\def\ZZ{\ensuremath{\mathbb{Z}}}
%%

%% Uncurrrying Chapter 
%% A space equal to a \thinspace, but we
%% can break a line at it.
\newskip\thinskipamt \thinskipamt=.16667em 
\protected\def\thinskip{\hskip \thinskipamt\relax}
\protected\def\thinnerskip{\hskip .5\thinskipamt\relax}
%% Capture a space token. Use a ``control-symbol'' (\. instead of \mksp)
%% to keep the trailing space from getting gobbled.
{\def\.{\global\let\sp= } \. }
%% Define \asp, which will capture the macro definition attached to space,
%% if one exists. Otherwise, \spa is relax after this.
{\catcode`\ =\active\gdef\asp{\ifx \relax\let\spa\relax\else\let\spa= \fi}}
\newtoks\foo
%% Removes spaces, implicit, active and explicit.
\protected\def\removespaces{\asp\afterassignment\removesp\let\next= }
\def\removesp{\foo={\next}\ifcat\noexpand\next\sp\foo={\removespace}%%
 \else\ifx\next\spa\foo={\removespaces}\fi%%
 \fi\the\foo}
%% MIL reserved word
\protected\def\milres#1/{\text{\ttfamily\bfseries #1}}
\protected\def\lab#1/{\textbf{\ensurett{\removespaces #1}}}
%% Constructs a closure: l { v1, ..., vN }
\protected\long\def\mkclo[#1:#2]{\lab #1/\ensuremath{\,\{\ensurett{#2}\}}\xspace}
%% Tuple version of closurs: {l: v1, ..., vN}.
\protected\long\def\clo[#1:#2]{\def\argA{#1}\def\argB{#2}\ensuremath{\{%%
      \ifx\argA\empty%%
      \else\lab #1/%%
        \ifx\argB\empty%%
        \else\ensurett{:\thinskip}%%
        \fi%%
      \fi\ensurett{#2}\}}\xspace}
%% Construct a thunk
\newbox\bracklbox \newbox\brackrbox
\setbox0=\hbox{$\{$} \setbox\bracklbox=\hbox to \wd0{\hfil[\kern0.25mm}
\setbox0=\hbox{$\}$} \setbox\brackrbox=\hbox to \wd0{\kern0.25mm]\hfil}
\protected\def\mkthunk[#1:#2]{\lab #1/%%
  \ensuremath{\,%%
    \mathopen{\copy\bracklbox}%%
    \ensurett{#2}%%
    \mathclose{\copy\brackrbox}\xspace}}
%% Binding statement: v <- {...}
\protected\def\binds#1<-#2;{\ensurett{\removespaces #1\texttt{<-}#2}\xspace}
%% In order to use \binds in verbatim environment, have to define
%% delimiters while they are active. The below defines \vbinds which
%% must be used in AVerb environments.  Notice the active space as
%% well - that is necessary so the space after \vbinds (and before the
%% first argument) in the verbatim environment gets eaten.
\begingroup\catcode`\!=\active \lccode`\!=`\< \lccode`\~=`\- 
  \catcode`\ =\active\lowercase{\endgroup\def\vbinds#1!~#2;}{\binds#1<-#2;}
%% Return statement: return ... ;
\protected\def\return#1;{\milres return/\ensurett{\ \removespaces #1}}
%% A closure capturing block. k {v1, ..., vN} x: ...
\protected\def\ccblock#1(#2)#3:{\lab#1/\ensuremath{\thinspace\{\ensurett{#2}\}}\ \ensurett{#3\hbox{:}}}
%% A normal block
\protected\def\block#1(#2):{\lab #1/\ensuremath{\thinspace(\ensurett{#2})}\ensurett{:}}
%% A goto expression
\protected\def\goto#1(#2){\lab #1/\thinspace\ensuremath{(\ensurett{#2})}}
%% An enter expression
\protected\def\app#1*#2/{\ensurett{\removespaces #1\ifmmode\ \fi{\text{\tt @}}\ifmmode\ \fi#2}}
\protected\def\bind{\texttt{<-}\xspace}
%% Primitive expression
\protected\def\prim#1(#2){\lab #1/\suptt*\ensuremath{(\ensurett{#2})}}
%% Program variable
\protected\def\var#1/{\ensurett{\removespaces #1}\xspace}
%% Case statement
\protected\def\case#1;{\milres case\ \ensuremath{\ensurett{\removespaces #1}}\ of/}
%% Case alternative
\protected\def\alt#1(#2)#3->#4;{\ensuremath{\ensurett{#1\ \ignorespaces#2\ \texttt{->}\ \ignorespaces #4}}}
%% Invoke
\protected\def\invoke#1/{\milres invoke/\ensurett{\ \removespaces #1}}
\def\rhs{right--hand side\xspace}
\def\lhs{left--hand side\xspace}
\def\enter{\texttt{@}\xspace}
\def\cc{closure--capturing\xspace}
\def\Cc{Closure--capturing\xspace}
%%

\newenvironment{myfig}[1][tbh]{\begin{figure}[#1]%%
\begin{singlespace}\centering%%
\figbegin}{\figend\end{singlespace}%%
\end{figure}}

%% Produce a sub-caption and label it.
\newcommand{\scap}[2][1in]{\begin{minipage}{#1}%%
\subcaption{}\label{#2}\end{minipage}}

%% Produce a sub-caption with text.
\newcommand{\lscap}[3][\hsize]{\begin{minipage}{#1}%%
\subcaption{#3}\label{#2}\end{minipage}}

% single-argument comment. Do not put
% a space before the command when used
% or the file will have two spaces.
\newcommand{\rem}[1]{}

%% A verbatim environment with active charactesr
%% so we can use math shortcuts and macros
\DefineVerbatimEnvironment{AVerb}{Verbatim}{commandchars=\\\{\},%% 
  codes={\catcode`\_8\catcode`\$3\catcode`\^7},%%
  numberblanklines=false}

\DefineVerbatimEnvironment{Verb}{Verbatim}{commandchars=\\\[\],%% 
  numberblanklines=false}

%% Turn on line numbers for Haskell code, 
%% and reset the line number counter.
\newcommand{\hsNumOn}{\numberson\numbersreset}
\newcommand{\hsNumOff}{\numbersoff}
%% Turn on line numbering in Haskell code within
%% the environment, then turn it off. The optional
%% argument specifies a prefix that \hslabel can
%% use to generate line number references. If no prefix
%% is givne, \hslabel will have no effect.
\newtoks\prefixtoks
\def\mkhslabel#1{\prefixtoks={#1}\let\prefix=a}
\def\hslabel#1{\ifx\prefix\relax%%
  \else\label{\the\prefixtoks_#1}%%
  \fi}
\def\unhslabel{\let\prefix=\relax}
\newenvironment{withHsNum}{\numberson\numbersreset}{\numbersoff}
\newenvironment{withHsLabeled}[1]{\numberson\numbersreset\mkhslabel{#1}}{\unhslabel\numbersoff}

%% Paragraph run-in
\newcommand{\runin}[1]{\begingroup\noindent\sffamily\textbf{#1}\qquad\endgroup}

%% Chapter bibliographies
\newcommand{\standaloneBib}{%%
  \ifthenelse{\boolean{standaloneFlag}}%%
             {\begin{singlespace}
                 \printbibliography
             \end{singlespace}}{}}

%% Adds an equation number on demand.
\newcommand\addtag{\refstepcounter{equation}\tag{\theequation}}

%% For typesetting set definitions like {x | x \in f(y)}
\newcommand\setdef[2]{\ensuremath{\{#1\ |\ #2\}}}

%% For typesetting function names like dom(f) or out(b).
\newcommand\mfun[1]{\ensuremath{\mathit{#1}}}

%% Marginal notes
\newcommand\margin[2]{\marginpar{\begin{singlespace}\emph{\footnotesize #2}\end{singlespace}}\relax #1}

%% Describe intent of a passage
\newcommand\intent[1]{{\begin{singlespace}\noindent\leftskip=-1in\emph{\footnotesize Intent: #1}\end{singlespace}}\nopagebreak[1]}

%% In aligned/alignedat/gathered environments, you don't et
%% automatice equation numbers. This command makes sure to
%% label them properly.
\newcommand\labeleq[1]{\refstepcounter{equation}\label{#1}}

%% Creates a hanging paragraph, where the first line is not
%% indented but all other lines are.
\def\itempar#1{\noindent\hangindent=\parindent\hangafter=1 #1\quad}

%% Disable overfull messages with ridiculous hfuzz value
\def\disableoverfull{\hfuzz=10in}

%% Set parfillskip so stretching does NOT occur at the end of
%% a paragraph (i.e., list of elements). Disable indent at beginning
%% of paragraph. Also turn off underfull hbox warnings.
%%
%% Intended to be used in a \vbox that forms part of a table or graphic,
%% which we want to be line-broken but not exactly like a normal paragraph.
\long\def\disableparspacing#1;{\def\arg{#1}\hbadness=100000\parindent=0pt\parfillskip=0pt\leftskip=0pt\rightskip=0pt%%
  \ifx\arg\empty\else\hsize=#1\relax\fi}
%% This stuff makes !+<text>+! write <text> in typewriter font.  

%% We make ! and + active characters early, then manipulate their
%% meaning to produce the right effect. Initially, + produces +. When
%% !  appears w/o a + following, it produces ``!''. When ``+''
%% follows, we start writing in teleteype (\ttfamily). The definition
%% of ``!'' changes to produce a bang. ``+'' changes such that it
%% looks for trailing ``!''. When no ``!'' appears, ``+'' produces ``+''. 
%% If a ``!'' appears, we shift out of \ttfamily (by ending the group) and
%% reset the meaning of ``!'' and ``+'' so we can start again.
\makeatletter
\let\mdplus=+\let\mdbang=!      %% Preserve meaning of + and ! so we can put them into document.
%% Turn off mark down for everyone
\outer\def\nomd{\catcode`!=12\catcode`+=12}
%% Turn mark down on for everyone
\outer\def\domd{\catcode`!=\active\catcode`+=\active %%
  \initialmd}
%% Use only with a group IMMEDIETALY following. Turns off
%% markdown for the group-to-come, without actually tokenizing the
%% group. If no group follows, this has no effect.
\protected\def\pausemd{\def\dopause{\catcode`!=12\catcode`+=12}%%
  \def\pausemdB{\ifx\next\bgroup%%
    %% A ``partial'' application of expandwith is used
    %% so we don't double up the group argument (which is what
    %% happens if we expand \next). This has the effect of 
    %% inserting \expandafter\dowith in front of the upcoming {. 
    %% If no brace is coming, \withmdC will have no effect.
    \def\pausemdC{\expandafter\dopause}
  \else
    \let\pausedmC=\relax
  \fi\pausemdC}
  %% \futurelet has to end the macro so it grabs the next token
  %% from the input file. Otherwise, it grabs it *from* this
  %% definition.
  \futurelet\next\pausemdB} %%
%% Turns markdown on for the group-to-come, without actually
%% tokenizing the group. Only has an effect when
%% used in front of a group, otherwise its a no-op.
\protected\def\withmd{\def\dowith{\catcode`!=\active\catcode`+=\active\initialmd}%%
  \def\withmdB{\ifx\next\bgroup %%
    %% A ``partial'' application of expandafter is used
    %% so we don't double up the group argument (which is what
    %% happens if we expand \next). This has the effect of 
    %% inserting \expandafter\dowith in front of the upcoming {. 
    %% If no brace is coming, \withmdC will have no effect.
      \def\withmdC{\expandafter\dowith} %%
    \else %%
      \let\withmdC=\relax %%
    \fi\withmdC}%%
  %% \futurelet has to end the macro so it grabs the next token
  %% from the input file. Otherwise, it grabs it *from* this
  %% definition.
  \futurelet\next\withmdB} %%
%% Make ! and + active in the following group so they have the right
%% catcode in the definitions to follow.
\catcode`!=\active\catcode`+=\active %%
%% Initial definitions associated with ! and +.
\def\initialmd{\protected\def!{\startTTA} %%
  \protected\def+{\stopTTA}} %%
%% Step 1 of startTT. Inital meaning of !; capture next token in \next, go to next step.
\def\startTTA{\futurelet\next\startTTB} %%
%% Step 2 of startTT. Compare captured token to + and go to step 3 if true. Otherwise
%% output a ! (since that started our macro), the argument captured and stop
%% processing.
\long\def\startTTB{\ifx\next+\expandafter\startTTC\expandafter\@gobble\else\mdbang\fi} %%
%% Step 3 of startTT. Shift into teletype mode and change definition of 
%% + and ! so we can stop processing.
\def\startTTC{\begingroup\ifmmode %%
  \let \math@bgroup \relax %%
  \def \math@egroup {\let \math@bgroup \@@math@bgroup %%
    \let \math@egroup \@@math@egroup} %%
  \mathtt\relax %%
  \else  %%
  \ttfamily\fi} %%
%% Step 1, 2  and 3 of stopTT follow the same pattern as startTT.
\def\stopTTA{\futurelet\next\stopTTB} %%
\long\def\stopTTB{\ifx\next!\expandafter\stopTTC\expandafter\@gobble\else\mdplus\fi} %%
\def\stopTTC{\endgroup}%%
\catcode`!=12\catcode`+=12
\makeatother

\domd

%% Place an input file on the next page
\def\onnextpage#1{\afterpage{\clearpage\input{#1}\clearpage}}

\begin{document}
\ifthenelse{\boolean{standaloneFlag}}
           {\VerbatimFootnotes
             \DefineShortVerb{\#}
             \doublespacing
             \setcounter{chapter}{0}}{}

%% Default float parameters. For case when
%% multiple chapters are included and
%% only one needs custom float settings.
\renewcommand{\textfraction}{0.2}
\renewcommand{\topfraction}{0.9}


\chapter{Dead-code elimination}

Dead-code elimination seeks to remove program statements that will not
be executed or which have no visible effect. It can be applied at multiple
points during compilation, as other transformations frequently introduce 
dead code. Without dead-code elimination, many other optimizations have
minimal effect, since they leave behind code which still executes even if
it is no longer needed.

\emph{Uncurrying} (descrbed in Chapter \ref{ref_chapter_uncurrying}),
usually produces many unused bindings. For example, consider the
following program which applies @const3@ to return 3 from @main@:

%% If only we actually supported local declarations ...
\begin{code}
main = 
  let const3 a b c = c 
  in const3 1 2 3
\end{code}

\noindent
and which compiles (in part) to this monadic program:

%% -- Not strictly accurate, main
%% -- really is main <- mainK {}
%%
%% Do I need to implement inter-procedural
%% optimization since this actually won't 
%% get reduced?
\begin{code}
main = do
  t1 <- const3 @ 1
  t2 <- t1 @ 2
  ...
 
const3 <- k1 {}
k1 {} a = k2 {a} 
k2 {a} b = k3 {a, b}
...
\end{code}

Uncurrying will rewrite @t1 @@ 2@ as @k2 {1} @@ 2@:

%% Does syntax allow a closure value to be entered here?
\begin{code}
main = do
 t1 <- const3 @ 1
 t2 <- k2 {a} @ 2 
 t3 <- t2 @ 3
 return t3
\end{code}

This changes makes @t1@ useless --- it causes an allocation but otherwise 
has no effect. Our dead-code elimination optimization should remove that binding.

%% \emph{The bulk of our work. Each optimization implemented is
%%   described. Each subsection should follow a common recipe: describe
%%   the optimization, give an example program and show how we want it
%%   changed, show salient points about the optimization (with code
%%   snippets and references to the full library), and reflect on the
%%   implementation.  Lazy Code Motion may get its own section.}

%% \emph{One question: should optimizations be categorized by direction? We could
%% describe all forward optimizations, then all backward ones. That may help
%% set up the concepts necessary to describe LCM.}

%% \emph{This portion gives an overview of the optimization, without
%% code or (much) notation. We motivate the optimization by 
%% showing an example.}

\section{Implementation}
We seek to remove two types of dead code: unused bindings (and,
therefore, allocation) and unused blocks. Each begins with a
``liveness'' analysis. Bindings (or blocks) that are referenced are ``live''
and therefore cannot be eliminated. Everything else is ``dead.'' 

\subsection{Eliminating Bindings}
\label{ref_sub_elim_bindings}

A binding can be eliminated if no references to the bound variable
occur except the initial binding. We must traverse each block
backwards, noting when a variable is used. We do not need to worry
about \emph{how} the variable is used, so we can just maintain a set
of variables. When we see a binding, we eliminate the bound variable
from the set, since variables can be bound more than once (and they
cannot be used before being bound).

%% TODO - leave out woTops?  

Figure \ref{ref_fig_liveTransfer} shows the algorithm for determining
liveness. We update the live set @f@ as we examine each
statement. \emph{Tail} (@Done@) and \emph{case} (@CaseM@) statements
add all referenced variables. Additionally, case statements must
inspect all alternatives and add the union of those sets (@tailVars@
collects the all variables used in a tail expression). \emph{Binding}
(@Bind@) statements remove the bound variable (@v@) and add any
variables in the tail (@t@).

\begin{figure}[h]
\begin{code}
live (Bind v t) f = Set.delete v f  `Set.union` tailVars mapEmpty t
live (CaseM v alts) f = Set.insert v (Set.unions (map (setAlt f) alts))
live (Done t) f = tailVars f t
\end{code}
\caption{Our transfer function for determing live variables within a block. In
each case @f@ represents the set of live variables. @CaseM@ and @Done@ statements
add all variables referenced (in @alts@ and @t@, respectively). @Bind@ statements
eliminate the newly bound variable @v@, while adding all referenced variables on the
right-hand side of the binding (@t@).}
\label{ref_fig_liveTransfer}
\end{figure}

With the facts collected by @live@, we can then eliminate useless
bindings by looking up the bound variable in the set and removing the
binding if the variable is not there. The @deadRewrite@ function,
shown in Figure \ref{ref_fig_deadRewrite}, gives our implementation of
this process. Recall that Hoopl interleaves analysis and rewriting.
Therefore, at each statement we know what variables will be live. When
a bind is seen, we check if the bound variable (@v@) is absent from
the live set (@f@). We also take a conservative view and only
eliminate safe operations from the tail @t@: data allocation, closure
construction and @returns@ (@safe@ implements these checks). We then
eliminate the binding when both are true. Otherwise, we leave the
program unchanged.

\begin{figure}[h]
\begin{code}
rewrite (Bind v t) f 
  | safe t && not (v `Set.member` f) = return (Just emptyGraph)
rewrite _ _ = return Nothing
\end{code}
\caption{Our rewrite function that elminates useless bindings. We eliminate a
binding when it does not appear in the live set (@f@) and when the right-hand side
(@t@) is safe: a data allocation, closure construction or return.}
\label{ref_fig_deadRewrite}
\end{figure}

\subsection{Eliminating Blocks}

%% TODO - talk about direction of analysis, and more about
%% dataflow analysis.
We eliminate blocks that will never execute in order to decrease code
size and memory footprint. We proceed by analyzing the program and
finding all @Goto@ and @Closure@ statements, which in turn gives us
all blocks which may be executed. Since we do not have computed
labels in our language, we know that if a label is never referenced in
one of these two statements then it will never be called.

Figure \ref{ref_figure_liveBlocks} gives our code that finds all
blocks called in the program. @tailDest@ looks for @Goto@ and
@Closure@ statements in tail expressions. @Bind@, @CaseM@ and @Done@
add labels found by @tailDest@ to @liveBlocks@, the set of known live
blocks. @BlockEntry@ and @CloEntry@ statements do not play a role in
the analysis, except to pass information about live blocks to their
predecessors.

\begin{figure}[h]
\begin{code}
live (Bind v tail) liveBlocks = 
  Set.fromList (tailDest tail) `Set.union` liveBlocks
live (CaseM _ alts) liveBlocks = 
  Set.unions (map (Set.fromList . tailDest . altExpr) alts 
    ++ mapElems liveBlocks)
live (Done tail) liveBlocks = 
  Set.unions (Set.fromList (tailDest tail) 
    : mapElems liveBlocks)
live (BlockEntry _ l _) liveBlocks = liveBlocks
live (CloEntry _ l _ _) liveBlocks = liveBlocks
\end{code}
\caption{Our transfer function the finds all blocks called by the
  program. @tailDest@ finds referenced blocks in tail
  expressions. @liveBlocks@ refers to the set of blocks found
  already. As each statement is examined, we add the blocks found
  to @liveBlocks@.}
\label{ref_figure_liveBlocks}
\end{figure}

After collecting this information we look at the set of blocks in the
program and eliminate any not in @liveBlocks@. Figure
\ref{ref_figure_remove} shows the function which we fold over all the
blocks in the graph. Any blocks in the @live@ set are retained
(@blockGraph block |*><*| prog@ keeps them in the graph). Otherwise,
the block is dropped. We continue to run dead-block elimination on the
altered program until the flag indicates no blocks were dropped. This ensures
that as we drop blocks, others which now cannot be called will also be dropped.

\begin{figure}[h]
  \begin{code}
remove live (flag, prog) (dest, block) 
  | dest `Set.member` live = (flag, blockGraph block |*><*| prog)
  | otherwise = (SomeChange, prog)
  \end{code}
  \caption{Our function that removes any block not referenced
    elsewhere in the program. When a block is dropped, @SomeChange@
    will be returned. Dead-block elimination will be run until
    @remove@ returns @NoChange@, the initial value of the flag.}
  \label{ref_figure_remove}
\end{figure}

\section{Reflection}
%% \emph{What was good, what didn't work so well, and how Hoopl helped
%% or hindered the implementation}

The Hoopl library made the implementation of dead-code elimination
(Section \ref{ref_sub_elim_bindings}) straightforward. Because it
interleaves analysis and rewriting, removing one useless binding 
can make other bindings useless, which will in turn be removed. 

Unfortunately, the same cannot be said for dead block
elimination. Hoopl's interface does not allow a block to be removed
during rewrite, and this limitation comes from the types used by
the library. We implement Hoopl rewrites over @StatM@ nodes -- i.e., we rewrite
graphs of StatM values, representing programs. This signature shows the type for all  rewrite functions (using our nodes):

\begin{code}
(FuelMonad m) => (forall e x.
  StatM e x
  -> Fact x f
  -> m (Maybe (Graph StatM e x)))
\end{code}

The @Maybe@ value returned tells Hoopl if the statement changed or
not. @Nothing@ indicates that the statement did not change. A
@Just@ value can update the graph, it must have at least one
statement. There isn't a way to express that the block should be
eliminated altogether. At best, we can replace a block with one that
immediately returns to the caller or that consists of a special
``marker'' statement which we eliminate in a second pass.

\end{document}
