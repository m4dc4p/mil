\documentclass[12pt]{report}
 %include polycode.fmt
\usepackage[T1]{fontenc}
\usepackage{calc}
%% \usepackage{fourier}
\usepackage{palatino}
\renewcommand\ttdefault{lmtt}
\usepackage{helvet}
%% \usepackage{inconsolata}
\usepackage{comment}
\usepackage{calc}
\usepackage{xspace}
\usepackage{verbatim}
\usepackage{url}
\usepackage{fancyvrb}
\usepackage{setspace}
\usepackage{amsmath}
\usepackage{booktabs}
\usepackage[margin=\parindent, format=hang,labelfont=bf]{caption}
%% \usepackage[subrefformat=parens]{subcaption}
%% The following makes sure we get parentheses around
%% subreferences. The newest version of the subcaption
%% package has an option for this, but that's not available
%% widely.
%%
%% From http://tex.stackexchange.com/questions/25644
\usepackage[labelformat=simple]{subcaption}
\makeatletter
  \def\thesubfigure{(\alph{subfigure})}
  \providecommand\thefigsubsep{~}
  \def\p@subfigure{\@nameuse{thefigure}\thefigsubsep}
\makeatother

\usepackage{ifthen}
\usepackage{stmaryrd}
\usepackage{longtable}
\usepackage{afterpage}
\usepackage{xifthen}
\usepackage{mathtools}
\usepackage{xparse}
\usepackage[natbib=true,style=authoryear,backend=bibtex8]{biblatex}
\setlength{\bibitemsep}{\bigskipamount}
\addbibresource{thesis.bib}
\usepackage{microtype}

\usepackage{tikz}
\usetikzlibrary{arrows,automata,positioning,calc}
%% Used for CFGs.
\tikzset{
  >=stealth, 
  node distance=.5in,
  stmt/.style={rectangle,
    draw=black, thick,        
    minimum height=2em,
    %% inner sep=2pt,
    %% text centered,
    %% node distance=.5in,
  },
  entex/.style={
    minimum height=2em,
    %% inner sep=2pt,
    %% text centered,
  },
  labelfor/.style={circle, 
    draw=black, thin,
    font={\footnotesize},
    inner sep=0,
    fill=white,
    above right=-1.5mm and -1.5mm of #1,
  },
  fact/.style={overlay},
  %% Invisible node
  invis/.style={inner sep=0pt, 
    minimum height=0em}, 
  table/.style={circle, fill=white,height=2mm}
}

%% GSO margins.
\usepackage[left=1.5in, right=1in, top=1in, bottom=1in]{geometry}
\usepackage{abstract}

%% GSO requires 12 pt font for all headings
\usepackage[bf,sf,tiny,compact]{titlesec}
\titleformat{\chapter}[display]
            {}% format
            {\sffamily\bfseries\chaptertitlename\ \thechapter}
            {\baselineskip}
            {\sffamily\bfseries}
            {}

\hyphenation{data-flow mo-na-dic} 

\newboolean{lhs2tex}
\setboolean{lhs2tex}{true}

% Used by included files to know they
% are NOT standalone
\newboolean{standaloneFlag}
\setboolean{standaloneFlag}{true}

\newlength{\rulefigmargin}
\setlength{\rulefigmargin}{2\parindent}

\newcommand\figbegin{\rule{\linewidth-\rulefigmargin}{0.4pt}\\\vspace{12pt}}
\newcommand\figend{\rule{\linewidth-\rulefigmargin}{0.4pt}}

%\providecommand{\citep}[1]{(\emph{#1})\xspace}
%\renewcommand{\cite}[1]{\emph{#1}\xspace}

%% Functional languages chapter commands
\newcommand{\lamA}{\ensuremath{\lambda}-calculus\xspace}
\newcommand{\LamA}{\ensuremath{\lambda}-Calculus\xspace}
\newcommand{\lamAbs}[2]{\ensuremath{\lambda#1.\ #2}}
\newcommand{\lamApp}[2]{\ensuremath{#1\ #2}}
\newcommand{\lamPApp}[2]{\ensuremath{(#1\ #2)}}
\newcommand{\lamAPp}[2]{\ensuremath{(#1)\ #2}}
\newcommand{\lamApP}[2]{\ensuremath{#1\ (#2)}}
\newcommand{\lamAPP}[2]{\ensuremath{(#1)\ (#2)}}
\let\lamApPp=\lamApP
\let\lamAppP=\lamAPp

\newcommand{\lamId}{\lamAbs{x}{x}}
\newcommand{\lamCompose}{\lamAbs{f}{\lamAbs{g}{\lamAbs{x}{\lamApp{f}{(\lamApp{g}{x})}}}}}
\newcommand{\machLam}{\ensuremath{M_\lambda}\xspace}
\newcommand{\compMach}[1]{\ensuremath{\left\llbracket #1 \right\rrbracket}}
\newcommand{\compRho}[1]{\ensuremath{\rho(#1)}}
\newcommand{\verSub}[2]{\ensuremath{#1_{#2}}}
\newcommand{\verSup}[2]{\ensuremath{#1^{#2}}}
\newcommand{\lamC}{\ensuremath{\lambda_C}\xspace}
\newcommand{\lamPlus}{\lamAbs{m}{\lamAbs{n}{\lamAbs{s}{\lamAbs{z}{\lamApp{m}{\lamApPp{s}{\lamApp{n}{\lamApp{s}{z}}}}}}}}}
%% Substitution notation -- [#1 -> #2]
\newcommand{\lamSubst}[2]{\ensuremath{[#1 \mapsto #2]}}
%% End functional languages chapter

%% Dataflow chapter commands
\newcounter{nodeCounter}[figure]
\newcommand{\inE}{\ensuremath{\mathit{in}}\xspace}
\newcommand{\out}{\ensuremath{\mathit{out}}\xspace}
\newcommand{\In}{\ensuremath{\mathit{In}}\xspace}
\newcommand{\InBa}{\ensuremath{\mathit{In}(B)}\xspace}
\newcommand{\Out}{\ensuremath{\mathit{Out}}\xspace}
%% Out(B_x) -- fact function for an named block.
\newcommand{\OutB}[1]{\ensuremath{\mathit{Out}(B_{\ref{#1}})}\xspace}
\newcommand{\OutBa}{\ensuremath{\mathit{Out}(B)}\xspace}
%% in(B) -- fact function for an anonymous block.
\newcommand{\inBa}{\ensuremath{\mathit{in}(B)}\xspace}
%% in(X) -- fact function for an anonymous block, but using a different variable.
\newcommand{\inXa}[1]{\ensuremath{\mathit{in}(#1)}\xspace}
%% in(B,v) -- fact function for an anonymous block and some variable.
\newcommand{\inBav}[1]{\ensuremath{\mathit{in}(B, #1)}\xspace}
%% in(B_x) -- fact function for an named block.
\newcommand{\inB}[1]{\ensuremath{\mathit{in}(B_{\ref{#1}})}\xspace}
%% in(B_x,v) -- fact function for an named block and some variable.
\newcommand{\inBv}[2]{\ensuremath{\mathit{in}(B_{\ref{#1}}, #2)}\xspace}
%% out(B) -- fact function for an anonymous block.
\newcommand{\outBa}{\ensuremath{\mathit{out}(B)}\xspace}
%% out(X) -- fact function for an anonymous block, but using a different variable.
\newcommand{\outXa}[1]{\ensuremath{\mathit{out}(#1)}\xspace}
%% out(B,v) -- fact function for an anonymous block and some variable.
\newcommand{\outBav}[1]{\ensuremath{\mathit{out}(B, #1)}\xspace}
%% out(B_x) -- fact function for an named block.
\newcommand{\outB}[1]{\ensuremath{\mathit{out}(B_{\ref{#1}})}\xspace}
%% out(B_x,v) -- fact function for an named block and some variable.
\newcommand{\outBv}[2]{\ensuremath{\mathit{out}(B_{\ref{#1}}, #2)}\xspace}
\newcommand{\entryN}{\emph{E}\xspace}
\newcommand{\exitN}{\emph{X}\xspace}
\newcommand{\refNode}[1]{\ensuremath{B_{\ref{#1}}}\xspace}
\newcommand{\labelNode}[1]{\refstepcounter{nodeCounter}\label{#1}}
\newcommand{\setL}[1]{\textsc{#1}\xspace}
\newcommand{\setLC}{\setL{Const}}

%% Formats a list of facts
%% Argument should be like \facts{a/1, b/2, foobar/\bot, baz/\top}.
%% 
\newcounter{factctr}
\newtoks\varVal
\newtoks\varName
\newcommand{\facts}[1]{\begingroup%%
  %% Test if the argument given contains a forward slash (/). Expands
  %% slashTest with argument such that if a slash is NOT present the 
  %% token \noSlash will be given as argument 2 to slashTest. Otherwise
  %% there must be slash.
  \def\hasSlash##1{\expandafter\slashTest##1/\noslash\endslash}%%
  \def\slashTest##1/##2##3\endslash{\ifx\noslash##2 N\else Y\fi}%%
  \def\getArgs##1/##2{\varName={##1}%%
    \varVal={##2}}
  \ensuremath{%%
    \setcounter{factctr}{0}%%
    \foreach \var in {#1}{%%
      %% Separate list with a comma
      \ifthenelse{\value{factctr}>0}{,\allowbreak}{}%%
      %% \tracingmacros=1%%
      %% If key/val arguments, use first form. Otherwise
      %% use second.
      \ifthenelse{\equal{\hasSlash{\var}}{Y}}%%
                  {\expandafter\getArgs\var \factC{\the\varName}{\the\varVal}}%%
                  {\var}%%
      %% \tracingmacros=0%%
      \stepcounter{factctr}%%
    }}%%
\endgroup}
\newcommand{\factC}[2]{{\ensuremath{(\mathit{#1},#2)}}}
\newcommand{\doFacts}[4]{\ensuremath{#3{#1}: %%
    \left\{ %%
    \begin{minipage}[c]{#4}%%
      \facts{#2} %%
  \end{minipage}\kern -0.23em\right\}}}

\ExplSyntaxOn
\DeclareDocumentCommand \inFactsM {m m m} {\doFacts{#1}{#2}{\inB}{#3}}
\DeclareDocumentCommand \inFacts {m m O{1in}} {\doFacts{#1}{#2}{\inB}{#3}}
\DeclareDocumentCommand \outFactsM {m m m} {\doFacts{#1}{#2}{\outB}{#3}}
\DeclareDocumentCommand \outFacts {m m O{1in}} {\doFacts{#1}{#2}{\outB}{#3}}
\ExplSyntaxOff

\newcommand{\lub}{\ifthenelse{\boolean{mmode}}{\sqcap}{\raisebox{.1em}{\ensuremath{\sqcap}}}\xspace}
\newcommand{\sqlt}{\ensuremath{\sqsubset}\xspace}
\newcommand{\sqlte}{\ensuremath{\sqsubseteq}\xspace}

%% End dataflow

%% MIL Chapter
\newcommand{\compMILE}[1]{\ensuremath{\left\llbracket #1 \right\rrbracket}}
\newcommand{\compMILV}[1]{\ensuremath{\left\llbracket #1 \right\rrbracket}}
\newcommand{\compMILQ}[2]{\ensuremath{\left\llbracket #2 \right\rrbracket}}
\newcommand{\milCtx}[1]{\ensuremath{\llfloor}#1\ensuremath{\rrfloor}}
%% End MIL chapter

\newenvironment{myfig}[1][tbh]{\begin{figure}[#1]%%
\centering%%
\figbegin}{\figend%%
\end{figure}}

%% Produce a sub-caption and label it.
\newcommand{\scap}[2][1in]{\begin{minipage}{#1}%%
\subcaption{}\label{#2}\end{minipage}}

%% Produce a sub-caption with text.
\newcommand{\lscap}[3][1in]{\begin{minipage}{#1}%%
\subcaption{#3}\label{#2}\end{minipage}}

% single-argument comment. Do not put
% a space before the command when used
% or the file will have two spaces.
\newcommand{\rem}[1]{}

%% A verbatim environment with active charactesr
%% so we can use math shortcuts and macros
\DefineVerbatimEnvironment{AVerb}{Verbatim}{commandchars=\\\{\},%% 
  codes={\catcode`\_8\catcode`\$3\catcode`\^7},%%
  numberblanklines=false}

%% Turn on line numbers for Haskell code, 
%% and reset the line number counter.
\newcommand{\hsNumOn}{\numberson\numbersreset}
\newcommand{\hsNumOff}{\numbersoff}
%% Turn on line numbering in Haskell code within
%% the environment, then turn it off.
\newenvironment{withHsNum}{\numberson\numbersreset}{\numbersoff}

%% Paragraph run-in
\newcommand{\runin}[1]{\begingroup\noindent\sffamily\textbf{#1}\qquad\endgroup}

%% Chapter bibliographies
\newcommand{\standaloneBib}{%%
  \ifthenelse{\boolean{standaloneFlag}}%%
             {\begin{singlespace}
                \printbibliography
             \end{singlespace}}{}}

%% Adds an equation number on demand.
\newcommand\addtag{\refstepcounter{equation}\tag{\theequation}}

%% For typesetting set definitions like {x | x \in f(y)}
\newcommand\setdef[2]{\ensuremath{\{#1\ |\ #2\}}}

%% For typesetting function names like dom(f) or out(b).
\newcommand\mfun[1]{\ensuremath{\mathit{#1}}}

%% Marginal notes
\newcommand\margin[2]{\marginpar{\begin{singlespace}\emph{\footnotesize #2}\end{singlespace}}\relax #1}

%% Describe intent of a passage
\newcommand\intent[1]{{\leftskip = -1in\begin{singlespace}\emph{\noindent\footnotesize Intent: #1}\end{singlespace}}}

\begin{document}
\ifthenelse{\boolean{standaloneFlag}}
           {\VerbatimFootnotes
             \DefineShortVerb{\#}
             \setcounter{chapter}{0}}{}

%% Default float parameters. For case when
%% multiple chapters are included and
%% only one needs custom float settings.
\renewcommand{\textfraction}{0.2}
\renewcommand{\textfraction}{0.2}
\renewcommand{\topfraction}{0.9}


\chapter{Dead-code elimination}

Dead-code elimination seeks to remove program statements that will not
be executed or which have no visible effect. It can be applied at multiple
points during compilation, as other transformations frequently introduce 
dead code. Without dead-code elimination, many other optimizations have
minimal effect, since they leave behind code which still executes even if
it is no longer needed.

\emph{Uncurrying} (descrbed in Chapter \ref{ref_chapter_uncurrying}), especially
when repeatedly applied, can leave behind many unusued bindings. Consider
this program, which executes @const3@ to return 3:

\begin{code}
const3 a b c = c
main = const3 1 2 3
\end{code}

@main@ compiles to this monadic program:

\begin{code}
-- Not strictly accurate, main
-- really is main <- mainK {}
main = do
 t1 <- const3 @ 1
 t2 <- t1 @ 2
 t3 <- t2 @ 3
 return t3

const3 <- k1 {}
k1 {} a = k2 {a} 
k2 {a} b = k3 {a, b}
k3 {a, b} c = b(a,b,c)
b(a,b,c) = return c
\end{code}

Uncurrying will rewrite @t1 @@ 2@ as @k2 {1} @@ 2@:

\begin{code}
main = do
 t1 <- const3 @ 1
 t2 <- k2 {a} @ 2 -- TODO: Does syntax allow a closure value to be entered here?
 t3 <- t2 @ 3
 return t3
\end{code}

This changes makes @t1@ useless --- it causes an allocation but otherwise 
has no effect. Our dead-code elimination optimization will remove the binding.

%% \emph{The bulk of our work. Each optimization implemented is
%%   described. Each subsection should follow a common recipe: describe
%%   the optimization, give an example program and show how we want it
%%   changed, show salient points about the optimization (with code
%%   snippets and references to the full library), and reflect on the
%%   implementation.  Lazy Code Motion may get its own section.}

%% \emph{One question: should optimizations be categorized by direction? We could
%% describe all forward optimizations, then all backward ones. That may help
%% set up the concepts necessary to describe LCM.}

%% \emph{This portion gives an overview of the optimization, without
%% code or (much) notation. We motivate the optimization by 
%% showing an example.}

\section{Implementation}
We seek to remove two types of dead code: unused bindings (and,
therefore, allocation) and unused blocks. Each begins with a
``liveness'' analysis. Bindings (or blocks) that are referenced are ``live''
and therefore cannot be eliminated. Everything else is ``dead.'' 

\subsection{Eliminating Bindings}

A binding can be eliminated if no references to the bound variable
occur except the initial binding. We must travese each block backwards, noting when
a variable is used. We do not need to worry about \emph{how} the variable is used, so
we can just maintain a set of variables. When we see a binding, we eliminate the bound
variable from the set, since variables can be boudn more than once (and they cannot be used before being bound). 

%% TODO - leave out woTops?  

Figure \ref{ref_fig_liveTransfer} shows the algorithm for determining
liveness.  Top-level definitions are excluded from the set by the
@woTops@ function and are not considered here. 
At each statement, we add referenced or remove variables from
the live set @f@. \emph{Tail} (@Done@) and \emph{case} (@CaseM@)
statements add all referenced variables. Additionally, case statements
must inspect all alternatives and add the union of those
sets. @tailVars@ collects the variables used in a tail expression and
its definition is not shown here. \emph{Binding} (@Bind@) statements
remove the bound variable (@v@) and add any variables in the tail
(@t@).

\begin{figure}[h]
\begin{code}
live (Bind v t) f = Set.delete v f  `Set.union` tailVars mapEmpty t
live (CaseM v alts) f = Set.insert v (Set.unions (map (setAlt f) alts))
live (Done t) f = tailVars f t
\end{code}
\caption{Our transfer function for determing live variables within a block.}
\label{ref_fig_liveTransfer}
\end{figure}

With the facts collected by @live@, we can then eliminate useless
bindngs by looking up the bound variable in the set and removing the
binding if the variable is not there. The @deadRewrite@ functin,
shown in Figure \ref{ref_fig_deadRewrite}, gives our implementation
of this process. Recall that Hoopl interleaves analysis and
rewriting. %% TODO: Better mention this!  
Therefore, at each statement
we know what variables will be live. When a bind is seen, we check if
the bound variable (@v@) is absent from the live set (@f@). We also
take a conservative view and make sure the tail portion (@t@) is an
allocation and not an application ($@@$) or block call. We then eliminate
the binding when both are true. Otherwise, we leave the program unchanged.

\begin{figure}[h]
\begin{code}
rewrite (Bind v t) f 
  | safeTail t && not (v `Set.member` f) = return (Just emptyGraph)
rewrite _ _ = return Nothing
\end{code}
\caption{Our rewrite function that elminates useless bindings.}
\label{ref_fig_deadRewrite}
\end{figure}

%% \emph{``Interesting'' pieces of the implementation are described.}

\section{Reflection}
\emph{What was good, what didn't work so well, and how Hoopl helped
or hindered the implementation}

\end{document}
