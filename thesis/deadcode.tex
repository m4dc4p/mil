\documentclass[12pt]{report}
 %include polycode.fmt
\usepackage[T1]{fontenc}
\usepackage{calc}
%% \usepackage{fourier}
\usepackage{palatino}
\renewcommand\ttdefault{lmtt}
\usepackage{helvet}
%% \usepackage{inconsolata}
\usepackage{comment}
\usepackage{calc}
\usepackage{xspace}
\usepackage{verbatim}
\usepackage{url}
\usepackage{fancyvrb}
\usepackage{setspace}
\usepackage{amsmath}
\usepackage{booktabs}
\usepackage[margin=\parindent, format=hang,labelfont=bf]{caption}
%% \usepackage[subrefformat=parens]{subcaption}
%% The following makes sure we get parentheses around
%% subreferences. The newest version of the subcaption
%% package has an option for this, but that's not available
%% widely.
%%
%% From http://tex.stackexchange.com/questions/25644
\usepackage[labelformat=simple]{subcaption}
\makeatletter
  \def\thesubfigure{(\alph{subfigure})}
  \providecommand\thefigsubsep{~}
  \def\p@subfigure{\@nameuse{thefigure}\thefigsubsep}
\makeatother

\usepackage{ifthen}
\usepackage{stmaryrd}
\usepackage{longtable}
\usepackage{afterpage}
\usepackage{xifthen}
\usepackage{mathtools}
\usepackage{xparse}
\usepackage[natbib=true,style=authoryear,backend=bibtex8]{biblatex}
\setlength{\bibitemsep}{\bigskipamount}
\addbibresource{thesis.bib}
\usepackage{microtype}

\usepackage{tikz}
\usetikzlibrary{arrows,automata,positioning,calc}
%% Used for CFGs.
\tikzset{
  >=stealth, 
  node distance=.5in,
  stmt/.style={rectangle,
    draw=black, thick,        
    minimum height=2em,
    %% inner sep=2pt,
    %% text centered,
    %% node distance=.5in,
  },
  entex/.style={
    minimum height=2em,
    %% inner sep=2pt,
    %% text centered,
  },
  labelfor/.style={circle, 
    draw=black, thin,
    font={\footnotesize},
    inner sep=0,
    fill=white,
    above right=-1.5mm and -1.5mm of #1,
  },
  fact/.style={overlay},
  %% Invisible node
  invis/.style={inner sep=0pt, 
    minimum height=0em}, 
  table/.style={circle, fill=white,height=2mm}
}

%% GSO margins.
\usepackage[left=1.5in, right=1in, top=1in, bottom=1in]{geometry}
\usepackage{abstract}

%% GSO requires 12 pt font for all headings
\usepackage[bf,sf,tiny,compact]{titlesec}
\titleformat{\chapter}[display]
            {}% format
            {\sffamily\bfseries\chaptertitlename\ \thechapter}
            {\baselineskip}
            {\sffamily\bfseries}
            {}

\hyphenation{data-flow mo-na-dic} 

\newboolean{lhs2tex}
\setboolean{lhs2tex}{true}

% Used by included files to know they
% are NOT standalone
\newboolean{standaloneFlag}
\setboolean{standaloneFlag}{true}

\newlength{\rulefigmargin}
\setlength{\rulefigmargin}{2\parindent}

\newcommand\figbegin{\rule{\linewidth-\rulefigmargin}{0.4pt}\\\vspace{12pt}}
\newcommand\figend{\rule{\linewidth-\rulefigmargin}{0.4pt}}

%\providecommand{\citep}[1]{(\emph{#1})\xspace}
%\renewcommand{\cite}[1]{\emph{#1}\xspace}

%% Functional languages chapter commands
\newcommand{\lamA}{\ensuremath{\lambda}-calculus\xspace}
\newcommand{\LamA}{\ensuremath{\lambda}-Calculus\xspace}
\newcommand{\lamAbs}[2]{\ensuremath{\lambda#1.\ #2}}
\newcommand{\lamApp}[2]{\ensuremath{#1\ #2}}
\newcommand{\lamPApp}[2]{\ensuremath{(#1\ #2)}}
\newcommand{\lamAPp}[2]{\ensuremath{(#1)\ #2}}
\newcommand{\lamApP}[2]{\ensuremath{#1\ (#2)}}
\newcommand{\lamAPP}[2]{\ensuremath{(#1)\ (#2)}}
\let\lamApPp=\lamApP
\let\lamAppP=\lamAPp

\newcommand{\lamId}{\lamAbs{x}{x}}
\newcommand{\lamCompose}{\lamAbs{f}{\lamAbs{g}{\lamAbs{x}{\lamApp{f}{(\lamApp{g}{x})}}}}}
\newcommand{\machLam}{\ensuremath{M_\lambda}\xspace}
\newcommand{\compMach}[1]{\ensuremath{\left\llbracket #1 \right\rrbracket}}
\newcommand{\compRho}[1]{\ensuremath{\rho(#1)}}
\newcommand{\verSub}[2]{\ensuremath{#1_{#2}}}
\newcommand{\verSup}[2]{\ensuremath{#1^{#2}}}
\newcommand{\lamC}{\ensuremath{\lambda_C}\xspace}
\newcommand{\lamPlus}{\lamAbs{m}{\lamAbs{n}{\lamAbs{s}{\lamAbs{z}{\lamApp{m}{\lamApPp{s}{\lamApp{n}{\lamApp{s}{z}}}}}}}}}
%% Substitution notation -- [#1 -> #2]
\newcommand{\lamSubst}[2]{\ensuremath{[#1 \mapsto #2]}}
%% End functional languages chapter

%% Dataflow chapter commands
\newcounter{nodeCounter}[figure]
\newcommand{\inE}{\ensuremath{\mathit{in}}\xspace}
\newcommand{\out}{\ensuremath{\mathit{out}}\xspace}
\newcommand{\In}{\ensuremath{\mathit{In}}\xspace}
\newcommand{\InBa}{\ensuremath{\mathit{In}(B)}\xspace}
\newcommand{\Out}{\ensuremath{\mathit{Out}}\xspace}
%% Out(B_x) -- fact function for an named block.
\newcommand{\OutB}[1]{\ensuremath{\mathit{Out}(B_{\ref{#1}})}\xspace}
\newcommand{\OutBa}{\ensuremath{\mathit{Out}(B)}\xspace}
%% in(B) -- fact function for an anonymous block.
\newcommand{\inBa}{\ensuremath{\mathit{in}(B)}\xspace}
%% in(X) -- fact function for an anonymous block, but using a different variable.
\newcommand{\inXa}[1]{\ensuremath{\mathit{in}(#1)}\xspace}
%% in(B,v) -- fact function for an anonymous block and some variable.
\newcommand{\inBav}[1]{\ensuremath{\mathit{in}(B, #1)}\xspace}
%% in(B_x) -- fact function for an named block.
\newcommand{\inB}[1]{\ensuremath{\mathit{in}(B_{\ref{#1}})}\xspace}
%% in(B_x,v) -- fact function for an named block and some variable.
\newcommand{\inBv}[2]{\ensuremath{\mathit{in}(B_{\ref{#1}}, #2)}\xspace}
%% out(B) -- fact function for an anonymous block.
\newcommand{\outBa}{\ensuremath{\mathit{out}(B)}\xspace}
%% out(X) -- fact function for an anonymous block, but using a different variable.
\newcommand{\outXa}[1]{\ensuremath{\mathit{out}(#1)}\xspace}
%% out(B,v) -- fact function for an anonymous block and some variable.
\newcommand{\outBav}[1]{\ensuremath{\mathit{out}(B, #1)}\xspace}
%% out(B_x) -- fact function for an named block.
\newcommand{\outB}[1]{\ensuremath{\mathit{out}(B_{\ref{#1}})}\xspace}
%% out(B_x,v) -- fact function for an named block and some variable.
\newcommand{\outBv}[2]{\ensuremath{\mathit{out}(B_{\ref{#1}}, #2)}\xspace}
\newcommand{\entryN}{\emph{E}\xspace}
\newcommand{\exitN}{\emph{X}\xspace}
\newcommand{\refNode}[1]{\ensuremath{B_{\ref{#1}}}\xspace}
\newcommand{\labelNode}[1]{\refstepcounter{nodeCounter}\label{#1}}
\newcommand{\setL}[1]{\textsc{#1}\xspace}
\newcommand{\setLC}{\setL{Const}}

%% Formats a list of facts
%% Argument should be like \facts{a/1, b/2, foobar/\bot, baz/\top}.
%% 
\newcounter{factctr}
\newtoks\varVal
\newtoks\varName
\newcommand{\facts}[1]{\begingroup%%
  %% Test if the argument given contains a forward slash (/). Expands
  %% slashTest with argument such that if a slash is NOT present the 
  %% token \noSlash will be given as argument 2 to slashTest. Otherwise
  %% there must be slash.
  \def\hasSlash##1{\expandafter\slashTest##1/\noslash\endslash}%%
  \def\slashTest##1/##2##3\endslash{\ifx\noslash##2 N\else Y\fi}%%
  \def\getArgs##1/##2{\varName={##1}%%
    \varVal={##2}}
  \ensuremath{%%
    \setcounter{factctr}{0}%%
    \foreach \var in {#1}{%%
      %% Separate list with a comma
      \ifthenelse{\value{factctr}>0}{,\allowbreak}{}%%
      %% \tracingmacros=1%%
      %% If key/val arguments, use first form. Otherwise
      %% use second.
      \ifthenelse{\equal{\hasSlash{\var}}{Y}}%%
                  {\expandafter\getArgs\var \factC{\the\varName}{\the\varVal}}%%
                  {\var}%%
      %% \tracingmacros=0%%
      \stepcounter{factctr}%%
    }}%%
\endgroup}
\newcommand{\factC}[2]{{\ensuremath{(\mathit{#1},#2)}}}
\newcommand{\doFacts}[4]{\ensuremath{#3{#1}: %%
    \left\{ %%
    \begin{minipage}[c]{#4}%%
      \facts{#2} %%
  \end{minipage}\kern -0.23em\right\}}}

\ExplSyntaxOn
\DeclareDocumentCommand \inFactsM {m m m} {\doFacts{#1}{#2}{\inB}{#3}}
\DeclareDocumentCommand \inFacts {m m O{1in}} {\doFacts{#1}{#2}{\inB}{#3}}
\DeclareDocumentCommand \outFactsM {m m m} {\doFacts{#1}{#2}{\outB}{#3}}
\DeclareDocumentCommand \outFacts {m m O{1in}} {\doFacts{#1}{#2}{\outB}{#3}}
\ExplSyntaxOff

\newcommand{\lub}{\ifthenelse{\boolean{mmode}}{\sqcap}{\raisebox{.1em}{\ensuremath{\sqcap}}}\xspace}
\newcommand{\sqlt}{\ensuremath{\sqsubset}\xspace}
\newcommand{\sqlte}{\ensuremath{\sqsubseteq}\xspace}

%% End dataflow

%% MIL Chapter
\newcommand{\compMILE}[1]{\ensuremath{\left\llbracket #1 \right\rrbracket}}
\newcommand{\compMILV}[1]{\ensuremath{\left\llbracket #1 \right\rrbracket}}
\newcommand{\compMILQ}[2]{\ensuremath{\left\llbracket #2 \right\rrbracket}}
\newcommand{\milCtx}[1]{\ensuremath{\llfloor}#1\ensuremath{\rrfloor}}
%% End MIL chapter

\newenvironment{myfig}[1][tbh]{\begin{figure}[#1]%%
\centering%%
\figbegin}{\figend%%
\end{figure}}

%% Produce a sub-caption and label it.
\newcommand{\scap}[2][1in]{\begin{minipage}{#1}%%
\subcaption{}\label{#2}\end{minipage}}

%% Produce a sub-caption with text.
\newcommand{\lscap}[3][1in]{\begin{minipage}{#1}%%
\subcaption{#3}\label{#2}\end{minipage}}

% single-argument comment. Do not put
% a space before the command when used
% or the file will have two spaces.
\newcommand{\rem}[1]{}

%% A verbatim environment with active charactesr
%% so we can use math shortcuts and macros
\DefineVerbatimEnvironment{AVerb}{Verbatim}{commandchars=\\\{\},%% 
  codes={\catcode`\_8\catcode`\$3\catcode`\^7},%%
  numberblanklines=false}

%% Turn on line numbers for Haskell code, 
%% and reset the line number counter.
\newcommand{\hsNumOn}{\numberson\numbersreset}
\newcommand{\hsNumOff}{\numbersoff}
%% Turn on line numbering in Haskell code within
%% the environment, then turn it off.
\newenvironment{withHsNum}{\numberson\numbersreset}{\numbersoff}

%% Paragraph run-in
\newcommand{\runin}[1]{\begingroup\noindent\sffamily\textbf{#1}\qquad\endgroup}

%% Chapter bibliographies
\newcommand{\standaloneBib}{%%
  \ifthenelse{\boolean{standaloneFlag}}%%
             {\begin{singlespace}
                \printbibliography
             \end{singlespace}}{}}

%% Adds an equation number on demand.
\newcommand\addtag{\refstepcounter{equation}\tag{\theequation}}

%% For typesetting set definitions like {x | x \in f(y)}
\newcommand\setdef[2]{\ensuremath{\{#1\ |\ #2\}}}

%% For typesetting function names like dom(f) or out(b).
\newcommand\mfun[1]{\ensuremath{\mathit{#1}}}

%% Marginal notes
\newcommand\margin[2]{\marginpar{\begin{singlespace}\emph{\footnotesize #2}\end{singlespace}}\relax #1}

%% Describe intent of a passage
\newcommand\intent[1]{{\leftskip = -1in\begin{singlespace}\emph{\noindent\footnotesize Intent: #1}\end{singlespace}}}

\begin{document}
\ifthenelse{\boolean{standaloneFlag}}
           {\VerbatimFootnotes
             \DefineShortVerb{\#}
             \setcounter{chapter}{0}}{}

%% Default float parameters. For case when
%% multiple chapters are included and
%% only one needs custom float settings.
\renewcommand{\textfraction}{0.2}
\renewcommand{\textfraction}{0.2}
\renewcommand{\topfraction}{0.9}


\chapter{Dead-code elimination}

Dead-code elimination seeks to remove program statements that will not
be executed or which have no visible effect. It can be applied at
multiple points during compilation, as other transformations
frequently introduce dead code. Without dead-code elimination, many
other optimizations have minimal effect, since they leave behind code
which still executes even if it is no longer needed.

For our MIL, we want to eliminate two types of dead code: bindings and
blocks. Dead bindings represent allocations we do not need to
make. Dead blocks increase the size of our program without providing
any benefit. To illustrate these two types of dead code, consider
the following program:

\begin{code}
main = do
  a <- 1 + 2
  b <- block1(a)
  c <- a + a
  return c

block1 (a) = do
  v1 <- a + a
  return v1
\end{code}

In @main2@, @a@ and @c@ will be used, but @b@ will not. The binding to
@b@ is therefore dead and can be eliminated.\footnote{If @block1@ had
  a side effect this would not be the case, but we assume it is does
  not for now} With @b@ gone, our program becomes:

\begin{code}
main = do
  a <- 1 + 2
  c <- a + a
  return c

block1 (a) = do
  v1 <- a + a
  return v1
\end{code}

Now we can see that @block1@ is not used and can also be
eliminated. This not only demonstrates how dead-code can be removed,
it also shows that iteratively applying the optimization can result in
multiple eliminations.

\section{Implementation}
We seek to remove two types of dead code: unused bindings (and,
therefore, allocation) and unused blocks. Each begins with a
``liveness'' analysis. Bindings (or blocks) that are referenced are ``live''
and therefore cannot be eliminated. Everything else is ``dead.''

\subsection{Eliminating Bindings}
\label{ref_sub_elim_bindings}

A binding can be eliminated if no references to the bound variable
occur after the initial binding. Within each block, we find all the
variables which are ``live'' (i.e., used after being bound) and then
eliminate any binding which is not in the live set. 
Hoopl applies the transfer function in turn to each statement
in the block. Variables used are collected in the set of live
variables, while new bindings remove variables from the set. 

To illustrate, consider the program fragement in Figure
\ref{ref_table_liveAnn}, where the live set is annotated between each
statement. At the end of the block (where analysis begins), no
variables are live (#5). After @return a@, ``a'' is in live set
(#4). However, the next statement, @a <- block3(c)@, \emph{removes}
@a@ from the live set and adds @c@ (#3). At point #2, a new @a@ (different than
the one at point 4) is added. That @a@ is removed by the next statement, leaving 
only @c@ in the live set (#1). 

\begin{figure}{h}
  \begin{code}
    -- 1. live: [c] 
    a <- block1(c)
    -- 2. live: [c, a]
    b <- block2(a)
    -- 3. live: [c]
    a <- block3(c)
    -- 4. live: [a] 
    return a
    -- 5. live: []
  \end{code}
  \caption{A program fragment with the ``live'' variable set annoted between each fragment.}
  \label{ref_fig_liveAnn}
\end{figure}

Table
\ref{ref_tbl_liveBindings} summarizes the facts, transfer function
and rewriting used to elminate bindings.
\begin{figure}
\begin{tabular}{ll}

Facts & 
\begin{code}
  type LiveFact = Set Name
\end{code}
\\

Direction & \emph{Backwards} \\

Transfer & 
\begin{minipage}{3in}
\begin{code}
live :: StmtM e x -> Fact x LiveFact -> LiveFact
live (BlockEntry _ _ _) f = woTops f
live (CloEntry _ _ _ _) f = woTops f
live (Bind v t) f = woTops
    (Set.delete v f  `Set.union` tailVars t )
live (CaseM v alts) f =
  woTops (Set.insert v
    (Set.unions (map (setAlt f) alts)))
live (Done t) f = woTops (tailVars t) 
\end{code} 
\end{minipage} \\

Rewrite & 
\begin{minipage}{3in}
\begin{code}
rewrite :: FuelMonad m => forall e x. StmtM e x -> 
           Fact x LiveFact ->
           m (Maybe (ProgM e x))
rewrite (Bind v t) f
  | safe t && not (v `Set.member` f) = return (Just emptyGraph)
rewrite _ _ = return Nothing 
\end{code} 
\end{minipage}
\\

\end{tabular}
\caption{The facts, transfer function and rewrite function
implemented for eliminating useless bindings within blocks.}
\label{ref_tbl_liveBindings}
\end{figure}

% \setlength\leftskip{\oldleftskip}

We analyze each block independently, as
bindings in one block cannot affect another. We the block
backwards, noting when a variable is used. We do not need to worry
about \emph{how} the variable is used, so we can just maintain a set
of variables. Any appearance of a variable in @Tail@ expressions constitues
use. When we see a binding, we eliminate the bound variable
from the set, since variables can be bound more than once (and they
cannot be used before being bound).

%% TODO - leave out woTops?

Figure \ref{ref_fig_liveTransfer} shows the algorithm for determining
liveness. We update the live set @f@ as we examine each
statement. \emph{Tail} (@Done@) and \emph{case} (@CaseM@) statements
add all referenced variables. Additionally, case statements must
inspect all alternatives and add the union of those sets (@tailVars@
collects the all variables used in a tail expression). \emph{Binding}
(@Bind@) statements remove the bound variable (@v@) and add any
variables in the tail (@t@).

\begin{figure}[h]
\begin{code}
live (Bind v t) f = Set.delete v f  `Set.union` tailVars mapEmpty t
live (CaseM v alts) f = Set.insert v (Set.unions (map (setAlt f) alts))
live (Done t) f = tailVars f t
\end{code}
\caption{Our transfer function for determing live variables within a block. In
each case @f@ represents the set of live variables. @CaseM@ and @Done@ statements
add all variables referenced (in @alts@ and @t@, respectively). @Bind@ statements
eliminate the newly bound variable @v@, while adding all referenced variables on the
right-hand side of the binding (@t@).}
\label{ref_fig_liveTransfer}
\end{figure}

With the facts collected by @live@, we can then eliminate useless
bindings by looking up the bound variable in the set and removing the
binding if the variable is not there. The @deadRewrite@ function,
shown in Figure \ref{ref_fig_deadRewrite}, gives our implementation of
this process. Recall that Hoopl interleaves analysis and rewriting.
Therefore, at each statement we know what variables will be live. When
a bind is seen, we check if the bound variable (@v@) is absent from
the live set (@f@). We also take a conservative view and only
eliminate safe operations from the tail @t@: data allocation, closure
construction and @returns@ (@safe@ implements these checks). We then
eliminate the binding when both are true. Otherwise, we leave the
program unchanged.

\begin{figure}[h]
\begin{code}
rewrite (Bind v t) f
  | safe t && not (v `Set.member` f) = return (Just emptyGraph)
rewrite _ _ = return Nothing
\end{code}
\caption{Our rewrite function that elminates useless bindings. We eliminate a
binding when it does not appear in the live set (@f@) and when the right-hand side
(@t@) is safe: a data allocation, closure construction or return.}
\label{ref_fig_deadRewrite}
\end{figure}

\subsection{Eliminating Blocks}

%% TODO - talk about direction of analysis, and more about
%% dataflow analysis.
We eliminate blocks that will never execute in order to decrease code
size and memory footprint. We proceed by analyzing the program and
finding all @Goto@ and @Closure@ statements, which in turn gives us
all blocks which may be executed. Since we do not have computed
labels in our language, we know that if a label is never referenced in
one of these two statements then it will never be called.

Figure \ref{ref_figure_liveBlocks} gives our code that finds all
blocks called in the program. @tailDest@ looks for @Goto@ and
@Closure@ statements in tail expressions. @Bind@, @CaseM@ and @Done@
add labels found by @tailDest@ to @liveBlocks@, the set of known live
blocks. @BlockEntry@ and @CloEntry@ statements do not play a role in
the analysis, except to pass information about live blocks to their
predecessors.

\begin{figure}[h]
\begin{code}
live (Bind v tail) liveBlocks =
  Set.fromList (tailDest tail) `Set.union` liveBlocks
live (CaseM _ alts) liveBlocks =
  Set.unions (map (Set.fromList . tailDest . altExpr) alts
    ++ mapElems liveBlocks)
live (Done tail) liveBlocks =
  Set.unions (Set.fromList (tailDest tail)
    : mapElems liveBlocks)
live (BlockEntry _ l _) liveBlocks = liveBlocks
live (CloEntry _ l _ _) liveBlocks = liveBlocks
\end{code}
\caption{Our transfer function the finds all blocks called by the
  program. @tailDest@ finds referenced blocks in tail
  expressions. @liveBlocks@ refers to the set of blocks found
  already. As each statement is examined, we add the blocks found
  to @liveBlocks@.}
\label{ref_figure_liveBlocks}
\end{figure}

After collecting this information we look at the set of blocks in the
program and eliminate any not in @liveBlocks@. Figure
\ref{ref_figure_remove} shows the function which we fold over all the
blocks in the graph. Any blocks in the @live@ set are retained
(@blockGraph block |*><*| prog@ keeps them in the graph). Otherwise,
the block is dropped. We continue to run dead-block elimination on the
altered program until the flag indicates no blocks were dropped. This ensures
that as we drop blocks, others which now cannot be called will also be dropped.

\begin{figure}[h]
  \begin{code}
remove live (flag, prog) (dest, block)
  | dest `Set.member` live = (flag, blockGraph block |*><*| prog)
  | otherwise = (SomeChange, prog)
  \end{code}
  \caption{Our function that removes any block not referenced
    elsewhere in the program. When a block is dropped, @SomeChange@
    will be returned. Dead-block elimination will be run until
    @remove@ returns @NoChange@, the initial value of the flag.}
  \label{ref_figure_remove}
\end{figure}

\section{Reflection}
%% \emph{What was good, what didn't work so well, and how Hoopl helped
%% or hindered the implementation}

The Hoopl library made the implementation of dead-code elimination
(Section \ref{ref_sub_elim_bindings}) straightforward. Because it
interleaves analysis and rewriting, removing one useless binding
can make other bindings useless, which will in turn be removed.

Unfortunately, the same cannot be said for dead block
elimination. Hoopl's interface does not allow a block to be removed
during rewrite, and this limitation comes from the types used by
the library. We implement Hoopl rewrites over @StatM@ nodes -- i.e., we rewrite
graphs of StatM values, representing programs. This signature shows the type for all  rewrite functions (using our nodes):

\begin{code}
(FuelMonad m) => (forall e x.
  StatM e x
  -> Fact x f
  -> m (Maybe (Graph StatM e x)))
\end{code}

The @Maybe@ value returned tells Hoopl if the statement changed or
not. @Nothing@ indicates that the statement did not change. A
@Just@ value can update the graph, it must have at least one
statement. There isn't a way to express that the block should be
eliminated altogether. At best, we can replace a block with one that
immediately returns to the caller or that consists of a special
``marker'' statement which we eliminate in a second pass.

\end{document}
