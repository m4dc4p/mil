\documentclass[11pt]{article}
%include polycode.fmt
\usepackage{palatino}
\usepackage[scaled=0.92]{helvet}
\usepackage{graphicx}
\usepackage{fullpage}
\usepackage{url}
\usepackage{fancyvrb}
\usepackage{qtree}
\usepackage{setspace}
\renewcommand\ttdefault{cmtt}
\begin{document}
\doublespacing
\fvset{fontfamily=cmtt}
%if False
\CustomVerbatimEnvironment{code}{Verbatim}{}
\DefineShortVerb{\|}
%endif
\VerbatimFootnotes
\DefineShortVerb{\#}

\section*{Hoopl}

\subsection*{Caffeine IR; Hoopl IR}

\subsection*{Control Flow Analysis}
\subsection*{No-op Optimization}
\subsection*{Copy Propagation}
\subsection*{Liveness Analysis}
\subsection*{Closure Capture}

\section*{Compiling to Closures and Blocks}

Mark and I spent the last half of the term talking about how to eliminate
the intermediate closures that Caffeine creates when compiling Habit code. Every function
with more than one argument gets compiled into $n - 1$ ``intermediate'' procedures which
build closures. Each intermediate procedure creates a closure which is used to call
the next procedure. Only the final procedure (which receives a closure with all but one of
the arguments needed for the function) actually does ``work.'' 

Mark came up with concept of ``k'' (``capture'') and ``b'' (``block'')
procedures to describe this process. Capture ($k$) procedures create a
new closure pointing at the next capture procedure. The new closure
holds one more argument. The last capture procedure creates a closure
which will call the block ($b$) procedure for the argument. The block
procedure actually executes the code defined by the original function.

For example, consider the |const3| function:

\begin{code}
const3 a b c = a
\end{code}

|const3| is translated into two $k$ procedures:

\begin{code}
  const3_k0 a = const3_k1 a
  const3_k1 a b= const3_b a b
\end{code}

|const3_k0| takes one argument (|a|) and calls |const3_k1| with it. |const3_k1| takes
two arguments and calls |const3_b|, which does the actual work of |const3|:

\begin{code}
  const3_b a b c = a
\end{code}

Without optimization, this scheme does not perform well. For example, the |main| program:

\begin{code}
  main = const3 1 2 3 
\end{code}

creates 2 intermediate closures before calling the body of |const3|!
The steps to evaluate |main| would be:

\begin{code}
main = const3 1 2 3
     = (const3_k0 1) 2 3 -- closure
     = (const3_k1 1 2) 3 -- closure
     = (const3_b 1 2 3)  -- work
     = 1
\end{code}

Each intermediate $k$ procedure expects two arguments: a closure and a
new argument. Though the notation shows multiple arguments, all but
the last are packaged into a closure. The $k$ procedure creates a new
closure, copies all the values from the existing closure, copies the
new argument into the new closure, and returns that closure. A procedure
with $n$ arguments will have $n - 1$ $k$ procedures. For example, |const3|
has two $k$ procedures. 

Optimizing this scheme will be described in the next section.

\subsection*{Optimizing Captures}

All function applications are treated uniformly, but it is inefficient
when all arguments are present. In the definition of $main$, for
example, three function applications occur (see Figure
\ref{fig_main}). Each intermediate $k$ procedure creates a closure and
copies values to it.  It would be more efficient to create one closure
with all arguments in the beginning.

\begin{figure}[h]
  
\Tree [.@@
         [.@@
           [.@@
             |const3|
             |1|
           ]
           |2|
         ]
         |3|
      ]
\caption{The unoptimized represention of |main|. Three application nodes (@) are present.}
\label{fig_main}
\end{figure}

When a function application calls a $k$ procedure, we can effectively
inline the call. If the result of the application is another $k$
procedure, we can inline again. After optimization, te $b$
procedure (i.e., the one that actually does work) will be called directly if
all the arguments were given.

Consider |main|. First, we'll rewrite the $k$ procedures:

\begin{code}
const3_k0 = \a -> const3_k1 a
const3_k1 = \a b -> const3_b a b
\end{code}

This makes it clear that the $k$ procedures really return functions, rather than doing work. Now
we can rewrite $main$ as follows:

\begin{code}
  main = const3 1 2 3
       = (const3_k0 a) 1 2 3
       = (\a -> const3_k1 a) 1 2 3
       = (\a -> (\a b -> const3_b a b) a) 1 2 3 
\end{code}

Now that all the intermediate procedures are inlined, we can substitute their
arguments when compiling, rather than when executing. These rewrites look like:

\begin{code}
  main = (\a -> (\a b -> const3_b a b) a) 1 2 3 
       = ((\a b -> const3_b a b) 1) 2 3
       = (const3_b 1 2) 3
\end{code}

\section*{Code Improvements}

\section*{Next Steps}


\end{document}
