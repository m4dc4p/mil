\documentclass[11pt]{article}
\begin{document}

\section*{Hoopl}

\subsection*{Caffeine IR; Hoopl IR}

\subsection*{Control Flow Analysis}
\subsection*{No-op Optimization}
\subsection*{Copy Propagation}
\subsection*{Liveness Analysis}
\subsection*{Closure Capture}

\section*{Compiling to Closures and Blocks}

Mark and I spent the last half of the term talking about how to eliminate
the intermediate closures that Caffeine creates when compiling Habit code. Every function
with more than one argument gets compiled into $n - 1$ ``intermediate'' procedures which
build closures. Each intermediate procedure creates a closure which is used to call
the next procedure. Only the final procedure (which receives a closure with all but one of
the arguments needed for the function) actually does ``work.'' 

Mark came up with concept of ``k'' (``capture'') and ``b'' (``block'')
procedures to describe this process. Capture ($k$) procedures create a
new closure pointing at the next capture procedure. The new closure
holds one more argument. The last capture procedure creates a closure
which will call the block ($b$) procedure for the argument. The block
procedure actually executes the code defined by the original function.

For example, consider the |const3| function:

> const3 a b c = a

|const3| is translated into two $k$ procedures:

> const3_k0 a = const3_k1 a
> const3_k1 a b= const3_b a b

|const3_k0| takes one argument (|a|) and calls |const3_k1| with it. |const3_k1| takes
two arguments and calls |const3_b|, which does the actual work of |const3|:

> const3_b a b c = a

Without optimization, this scheme does not perform well. For example, the |main| program:

> main = const3 1 2 3 

creates 2 intermediate closures before calling the body of |const3|!
The steps to evaluate |main| would be:

> main = const3 1 2 3 ==>
>      = (const3_k0 1) 2 3 <-- closure
>      = (const3_k1 1 2) 3 <-- closure
>      = (const3_b 1 2 3)  <-- work
>      = 1

Optimizing this scheme will be described in the next section.

\subsection*{Optimizing Captures}



\section*{Code Improvements}

\section*{Next Steps}


\end{document}
