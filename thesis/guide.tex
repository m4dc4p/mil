%&preamble
%% Looking at LaTeX sources reveals \usepackage and
%% \RequirePackage are made equivalent when \documentclass
%% executes, so we can test if \documentclass has
%% been executed by comparing those two control sequences.
\def\ifnodocclass{\def\loaded{\iffalse}%
  \def\notloaded{\iftrue}%%
  \let\which=\loaded%%
  \ifx\usepackage\undefined\let\which=\notloaded%%
  \else%%
    \ifx\usepackage\RequirePackage%%
    \else\let\which=\notloaded%%
    \fi%%
  \fi\which}
\dodocclass
%include polycode.fmt
%include lineno.fmt
%include subst.fmt
\begin{document}
\fancyhf{}
\numbersoff
\ifthenelse{\boolean{standaloneFlag}}
           {\VerbatimFootnotes
             \DefineShortVerb{\#}
             \setcounter{chapter}{0}}{}

%% Default float parameters. For case when
%% multiple chapters are included and
%% only one needs custom float settings.
\renewcommand{\textfraction}{0.2}
\renewcommand{\textfraction}{0.2}
\renewcommand{\topfraction}{0.9}


\section{Thesis Style Guide}

This document specifies the typographic guidelines followed for various
elements in the thesis. 

\section*{Abbreviations}
Abbreviations such as \mil and \cfg appear in small caps. When an
acronym starts a sentence, its first letter is capitalized (e.g.,
``\Mil'').

New terms are written in italic type when they are first
introduced. For example, ``The \emph{dataflow algorithm} defines a
general approach to program analysis\dots''. Abbreviations
are introduced by writing them after the term: ``A \emph{control-flow graph}
(\cfg) \dots''

\section*{Numbers}
Chapter, Section, Page and Figure references use old-style numerals:
0, 1, 2, 3, etc.  Line numbers are also old-style.

Superscripts, subscripts, and footnotes use ``normal'' numerals: $0,
1, 2, 3,$ etc.  Numbers in mathematical expressions are normal (e.g.,
``$x = 10$, '' not ``$x = $\ 10'').  Indexed variables (e.g., $B_1$,
$t_1$, etc.) also use normal numerals.

\section*{Syntax}
\Mil program fragments appear in typewriter font: \binds x <- \goto readChar();,
\lab main/, \block print(x):, etc. \lamC program fragments use italic
font: \lcdef compose (f,g,x)=\lcapp f * (g * x)/;.

\end{document}
