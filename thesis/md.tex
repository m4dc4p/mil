%% This stuff makes !+<text>+! write <text> in typewriter font.  

%% We make ! and + active characters early, then manipulate their
%% meaning to produce the right effect. Initially, + produces +. When
%% !  appears w/o a + following, it produces ``!''. When ``+''
%% follows, we start writing in teleteype (\ttfamily). The definition
%% of ``!'' changes to produce a bang. ``+'' changes such that it
%% looks for trailing ``!''. When no ``!'' appears, ``+'' produces ``+''. 
%% If a ``!'' appears, we shift out of \ttfamily (by ending the group) and
%% reset the meaning of ``!'' and ``+'' so we can start again.
\makeatletter
\let\mdplus=+\let\mdbang=!      %% Preserve meaning of + and ! so we can put them into document.
%% Turn off mark down for everyone
\outer\def\nomd{\catcode`!=12\catcode`+=12}
%% Turn mark down on for everyone
\outer\def\domd{\catcode`!=\active\catcode`+=\active %%
  \initialmd}
%% Use only with a group IMMEDIETALY following. Turns off
%% markdown for the group-to-come, without actually tokenizing the
%% group. If no group follows, this has no effect.
\protected\def\pausemd{\def\dopause{\catcode`!=12\catcode`+=12}%%
  \def\pausemdB{\ifx\next\bgroup%%
    %% A ``partial'' application of expandwith is used
    %% so we don't double up the group argument (which is what
    %% happens if we expand \next). This has the effect of 
    %% inserting \expandafter\dowith in front of the upcoming {. 
    %% If no brace is coming, \withmdC will have no effect.
    \def\pausemdC{\expandafter\dopause}
  \else
    \let\pausedmC=\relax
  \fi\pausemdC}
  %% \futurelet has to end the macro so it grabs the next token
  %% from the input file. Otherwise, it grabs it *from* this
  %% definition.
  \futurelet\next\pausemdB} %%
%% Turns markdown on for the group-to-come, without actually
%% tokenizing the group. Only has an effect when
%% used in front of a group, otherwise its a no-op.
\protected\def\withmd{\def\dowith{\catcode`!=\active\catcode`+=\active\initialmd}%%
  \def\withmdB{\ifx\next\bgroup %%
    %% A ``partial'' application of expandafter is used
    %% so we don't double up the group argument (which is what
    %% happens if we expand \next). This has the effect of 
    %% inserting \expandafter\dowith in front of the upcoming {. 
    %% If no brace is coming, \withmdC will have no effect.
      \def\withmdC{\expandafter\dowith} %%
    \else %%
      \let\withmdC=\relax %%
    \fi\withmdC}%%
  %% \futurelet has to end the macro so it grabs the next token
  %% from the input file. Otherwise, it grabs it *from* this
  %% definition.
  \futurelet\next\withmdB} %%
%% Make ! and + active in the following group so they have the right
%% catcode in the definitions to follow.
\catcode`!=\active\catcode`+=\active %%
%% Initial definitions associated with ! and +.
\def\initialmd{\protected\def!{\startTTA}\protected\def+{\mdplus}} %%
%% Step 1 of startTT. Inital meaning of !; capture next token in \next, go to next step.
\def\startTTA{\futurelet\next\startTTB} %%
%% Step 2 of startTT. Compare captured token to + and go to step 3 if true. Otherwise
%% output a ! (since that started our macro), the argument captured and stop
%% processing.
\long\def\startTTB{\ifx\next+\expandafter\startTTC\expandafter\@gobble\else\mdbang\fi} %%
%% Step 3 of startTT. Shift into teletype mode and change definition of 
%% + and ! so we can stop processing.
\def\startTTC{\begingroup\ttfamily\protected\def+{\stopTTA}\protected\def!{\mdbang}} %%
%% Step 1, 2  and 3 of stopTT follow the same pattern as startTT.
\def\stopTTA{\futurelet\next\stopTTB} %%
\long\def\stopTTB{\ifx\next!\expandafter\stopTTC\expandafter\@gobble\else\mdplus\fi} %%
\def\stopTTC{\endgroup}%%
\catcode`!=12\catcode`+=12
\makeatother
